%%%preamble.
\documentclass[12pt]{article}

\usepackage[english]{babel}
\usepackage{enumerate}
\usepackage{fullpage}
\usepackage[colorlinks=true,linkcolor=cyan]{hyperref}
\usepackage{amsmath,amsfonts,amssymb}
\usepackage[nottoc, notlof, notlot]{tocbibind}

\usepackage{amsthm,thmtools}
\declaretheorem[style=plain]{definition}
\declaretheorem[sibling=definition]{theorem}
\declaretheorem[sibling=definition]{corollary}
\declaretheorem[sibling=definition]{proposition}
\declaretheorem[sibling=definition]{lemma}
%\declaretheorem[style=remark,sibling=definition,qed={\Coffeecup}]{remark}
%\declaretheorem[style=remark,sibling=definition,qed={\WritingHand}]{example}
%\declaretheoremstyle[headpunct={},notebraces={\textbf{--}~}{}]{algorithm}
%\declaretheorem[style=algorithm]{problem}
%\declaretheorem[style=algorithm]{algorithm}

\usepackage[capitalize]{cleveref}

\title{Matrix Berlekamp-Massey}

\author{}
\date{\today}

%%%notation
%misc
\newcommand{\storeArg}{} % aux, not to be used in document!!
\newcounter{notationCounter}
%integers
\renewcommand{\ge}{\geqslant} % greater or equal
\renewcommand{\le}{\leqslant} % lesser or equal
%spaces
\newcommand{\NN}{\mathbb{Z}_{\ge 0}} % nonnegative integers
\newcommand{\var}{X} % variable for univariate polynomials
\newcommand{\field}{\mathbb{K}} % base field
\newcommand{\polRing}{\field[\var]} % polynomial ring
\newcommand{\Pox}{[\mkern-3mu[ \var ]\mkern-3.2mu]}
\newcommand{\psRing}{\field\Pox}
\newcommand{\matSpace}[1][\rdim]{\renewcommand\storeArg{#1}\matSpaceAux} % polynomial matrix space, 2 opt args
\newcommand{\matSpaceAux}[1][\storeArg]{\field^{\storeArg \times #1}} % not to be used in document
\newcommand{\polMatSpace}[1][\rdim]{\renewcommand\storeArg{#1}\polMatSpaceAux} % polynomial matrix space, 2 opt args
\newcommand{\polMatSpaceAux}[1][\storeArg]{\polRing^{\storeArg \times #1}} % not to be used in document
\newcommand{\psMatSpace}[1][\rdim]{\renewcommand\storeArg{#1}\psMatSpaceAux} % polynomial matrix space, 2 opt args
\newcommand{\psMatSpaceAux}[1][\storeArg]{\psRing^{\storeArg \times #1}} % not to be used in document
\newcommand{\mat}[1]{\mathbf{\MakeUppercase{#1}}} % for a matrix
\newcommand{\row}[1]{\mathbf{\MakeLowercase{#1}}} % for a matrix
\newcommand{\col}[1]{\mathbf{\MakeLowercase{#1}}} % for a matrix
\newcommand{\matCoeff}[1]{\MakeLowercase{#1}} % for a coefficient in a matrix
\newcommand{\rdim}{m} % row dimension
\newcommand{\cdim}{n} % column dimension
\newcommand{\diag}[1]{\mathrm{Diag}(#1)}  % diagonal matrix with diagonal entries #1
\newcommand{\seqelt}[1]{S_{#1}} % element of sequence of matrices
\newcommand{\seq}{\mathcal{S}} % sequence of matrices
\newcommand{\seqpm}{\mat{S}} % power series matrix from a sequence
\newcommand{\rel}{\col{p}} % linear relation
\newcommand{\relbas}{\mat{p}} % linear relation
\newcommand{\relSpace}{\polMatSpace[\cdim][1]} % space for linear relations
\newcommand{\relbasSpace}{\polMatSpace[\cdim][\cdim]} % space for linear relations

\begin{document}
  \maketitle


\section{Definition of recurrence relations}
\label{sec:relations}

First thing: what are linearly recurrent sequences, when talking about matrix
sequences? The definition is essentially the same as in the scalar case. There
is a similar notion of generator of a sequence (which is a matrix): it is related to the
denominator in some minimal fraction description of the generating series of
the sequence. More precisely, this depends on how the generating series is defined:
\begin{itemize}
  \item if $\sum_{k \ge 0} \seqelt{k} \var^k$ then the matrix generator is the
    reversed denominator;
  \item if $\sum_{k>0} \seqelt{k} \var^{-k}$, it is exactly the denominator.
\end{itemize}

Let us see two corresponding definitions from \cite{KalYuh13} and from
\cite{Thome02}.

\begin{definition}[\cite{Thome02}]
  \label{dfn:relation_thome}
  Let $\seq = (\seqelt{k})_{k\in\NN} \subset \matSpace[\rdim][\cdim]$ be a
  sequence of $\rdim\times\cdim$ matrices over $\field$. We define the power
  series matrix $\seqpm = \sum_{k\ge 0} \seqelt{k} \var^k \in
  \psMatSpace[\rdim][\cdim]$. Then, a vector $\rel \in \relSpace$ is said to be
  a \emph{(linear recurrence) relation for $\seq$} if the product $\seqpm \rel$
  has polynomial entries, that is, $\seqpm \rel \in \relSpace$.
\end{definition}

For some reason (which is unclear to me), in \cite{Thome02} the word
``generator'' is used for such relations. Here we will reserve this word for
sets of vectors that indeed generate the set of all relations.

Assume there exists a nonzero relation for $\seq$, and let $\rel$ be such a
relation. Writing $\rel = \sum_k p_k \var^k$ for matrices
$p_k \in \matSpace[\cdim][1]$, then we have
\begin{equation}
  \label{eqn:relation_thome}
  \sum_{k=0}^{d} \seqelt{\delta - k} p_{k} = 0 \quad \text{ for all } d \ge
  \deg(\rel) \text{ and } \delta \ge \max(d,\deg(\seqpm \rel)+1).
\end{equation}
One may have in mind that $\deg(\seqpm \rel) < \deg(\rel)$ since this often
holds in interesting examples; of course in general this does not necessary
hold. For example, if $\seq$ has only finitely many nonzero terms, and thus
$\seqpm$ already has polynomial entries, any coordinate vector is a relation
$\rel$ such that $\seqpm \rel$ has degree larger than $\deg(\rel)$.

\begin{definition}[\cite{KalYuh13}]
  \label{dfn:relation_ky}
  Let $\seq = (\seqelt{k})_{k\in\NN} \subset \matSpace[\rdim][\cdim]$ be a
  sequence of $\rdim\times\cdim$ matrices over $\field$. We define the power
  series matrix $\seqpm = \sum_{k\ge 0} \seqelt{k} \var^k \in
  \psMatSpace[\rdim][\cdim]$. Then, a vector $\rel \in \relSpace$ of degree at
  most $d$ is said to be a \emph{(linear recurrence) relation for $\seq$} if
  \[
    \sum_{k=0}^{d} \seqelt{\delta + k} p_{k} = 0
    \quad\text{ for all } \delta \ge 0.
  \]
  where $(p_k)_{k}$ are the matrices in $\matSpace[\cdim][1]$ such that $\rel =
  \sum_{0\le k\le d} p_k \var^k$.
\end{definition}

\begin{lemma}
  \label{lem:link_defs}
  For a given sequence $\seq \subset \matSpace[\rdim][\cdim]$, a nonzero vector
  $\rel \in \relSpace$ is a relation for \cref{dfn:relation_thome} if and only
  if there exists $d \ge \deg(\rel)$ such that the reverse $\var^{d}
  \rel(\var^{-1})$ is a relation for \cref{dfn:relation_ky}.
\end{lemma}
\begin{proof}
  First, we assume that $\var^{d} \rel(\var^{-1}) = \sum_{k=0}^{d} p_{d-k}
  \var^k$ is a relation for \cref{dfn:relation_ky}, for some integer $d \ge
  \deg(\rel)$. This means that, for all $\delta \ge 0$, we have $0 =
  \sum_{k=0}^{d} \seqelt{\delta + k} p_{d-k} = \sum_{k=0}^{d} \seqelt{\delta+d
  - k} p_{k}$. This implies that $\seqpm\rel$ has polynomial entries (and
  $\deg(\seqpm\rel) \le d$).

  Now, we assume that $\rel$ is a relation for \cref{dfn:relation_thome}.
  Taking $d = \max(\deg(\rel),\deg(\seqpm \rel)+1)$ in
  \cref{eqn:relation_thome}, we obtain $\sum_{k=0}^{d} \seqelt{\delta - k}
  p_{k} = 0$ for all $\delta \ge d$. This implies $\sum_{k=0}^{d}
  \seqelt{\delta-d + k} p_{d-k} = 0$ for all $\delta\ge d$, or equivalently,
  $\sum_{k=0}^{d} \seqelt{\delta+k} p_{d-k} = 0$ for all $\delta\ge 0$.
  Therefore the reverse $\var^{d} \rel(\var^{-1})$ is a relation for
  \cref{dfn:relation_ky}.
\end{proof}

\section{Matrix Berlekamp-Massey, or computing matrix generators}
\label{sec:matrix_berlekamp_massey_or_computing_matrix_generators}



\bibliographystyle{plain}
\bibliography{biblio.bib}

\end{document}
