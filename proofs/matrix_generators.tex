%%%preamble.
\documentclass[12pt]{article}

\usepackage[english]{babel}
\usepackage{enumerate}
\usepackage{fullpage}
\usepackage[colorlinks=true,linkcolor=cyan]{hyperref}
\usepackage{color}
\newcommand{\todo}[1]{\textcolor{red}{#1}}
\newcommand{\fixme}[1]{\textcolor{blue}{#1}}

% for enumerate / itemize: define reasonable margins
\usepackage[shortlabels]{enumitem}
\setlist{topsep=0.25\baselineskip,partopsep=0pt,itemsep=1pt,parsep=0pt}

% math and theorem names
\usepackage{amsmath,amsfonts,amssymb,amsthm,thmtools}
\declaretheorem[style=plain,parent=section]{definition}
\declaretheorem[sibling=definition]{theorem}
\declaretheorem[sibling=definition]{corollary}
\declaretheorem[sibling=definition]{proposition}
\declaretheorem[sibling=definition]{lemma}
\declaretheorem[style=remark,sibling=definition,qed={\qedsymbol}]{remark}
\declaretheorem[style=remark,sibling=definition,qed={\qedsymbol}]{example}
\declaretheoremstyle[headpunct={},notebraces={\textbf{--}~}{}]{algorithm}
\declaretheorem[style=algorithm]{problem}
\declaretheorem[style=algorithm]{algorithm}

% for our algorithms & problems
\usepackage{mdframed}

\usepackage[capitalize]{cleveref}

\title{Matrix Berlekamp-Massey}

\author{}
\date{\today}

%%%notation
%misc
\newcommand{\storeArg}{} % aux, not to be used in document!!
\newcounter{notationCounter}
%integers
\renewcommand{\ge}{\geqslant} % greater or equal
\renewcommand{\le}{\leqslant} % lesser or equal
%spaces
\newcommand{\NN}{\mathbb{Z}_{\ge 0}} % nonnegative integers
\newcommand{\var}{X} % variable for univariate polynomials
\newcommand{\field}{\mathbb{K}} % base field
\newcommand{\polRing}{\field[\var]} % polynomial ring
\newcommand{\Pox}{[\mkern-3mu[ \var ]\mkern-3.2mu]}
\newcommand{\Poxi}{[\mkern-3mu[ \var^{-1} ]\mkern-3.2mu]}
\newcommand{\psRing}{\field\Pox}
\newcommand{\matSpace}[1][\rdim]{\renewcommand\storeArg{#1}\matSpaceAux} % polynomial matrix space, 2 opt args
\newcommand{\matSpaceAux}[1][\storeArg]{\field^{\storeArg \times #1}} % not to be used in document
\newcommand{\polMatSpace}[1][\rdim]{\renewcommand\storeArg{#1}\polMatSpaceAux} % polynomial matrix space, 2 opt args
\newcommand{\polMatSpaceAux}[1][\storeArg]{\polRing^{\storeArg \times #1}} % not to be used in document
\newcommand{\psMatSpace}[1][\rdim]{\renewcommand\storeArg{#1}\psMatSpaceAux} % polynomial matrix space, 2 opt args
\newcommand{\psMatSpaceAux}[1][\storeArg]{\psRing^{\storeArg \times #1}} % not to be used in document
\newcommand{\mat}[1]{\mathbf{\MakeUppercase{#1}}} % for a matrix
\newcommand{\row}[1]{\mathbf{\MakeLowercase{#1}}} % for a matrix
\newcommand{\col}[1]{\mathbf{\MakeLowercase{#1}}} % for a matrix
\newcommand{\matCoeff}[1]{\MakeLowercase{#1}} % for a coefficient in a matrix
\newcommand{\rdim}{m} % row dimension
\newcommand{\cdim}{n} % column dimension
\newcommand{\diag}[1]{\mathrm{Diag}(#1)}  % diagonal matrix with diagonal entries #1
\newcommand{\seqelt}[1]{S_{#1}} % element of sequence of matrices
\newcommand{\seqeltSpace}{\matSpace[\rdim][\cdim]} % element of sequence of matrices
\newcommand{\seq}{\mathcal{S}} % sequence of matrices
\newcommand{\seqpm}{\mat{S}} % power series matrix from a sequence
\newcommand{\rel}{\col{p}} % linear relation
\newcommand{\relbas}{\mat{P}} % linear relation
\newcommand{\relSpace}{\polMatSpace[1][\rdim]} % space for linear relations
\newcommand{\relbasSpace}{\polMatSpace[\rdim][\rdim]} % space for linear relations
\newcommand{\num}{\row{q}} % numerator for relation
\newcommand{\nummat}{\mat{Q}} % numerator for relation basis
\newcommand{\degBd}{d} % bound on degree of minimal generator
\newcommand{\degDet}[1][\seq]{\operatorname{\Delta}(#1)}
\newcommand{\rdeg}[2][]{\mathrm{rdeg}_{{#1}}(#2)} % shifted row degree
\newcommand{\cdeg}[2][]{\mathrm{cdeg}_{{#1}}(#2)} % shifted column degree
\newcommand{\leadingMat}[2][\unishift]{\mathrm{lm}_{#1}(#2)} % leading matrix of polynomial matrix, default shifts = 0 (uniform)

%% ------------------------------------------------
%% --------- For problems and algorithms ----------
%% ------------------------------------------------
\newcommand{\argfig}[1]{\begin{figure}[#1]} % to be able to feed an optional argument to the inside figure

\newenvironment{algobox}[1][htbp]{
  \newcommand{\algoInfo}[3]{
    \begin{algorithm}[{name=[\algoname{##2}:~##1]\algoname{##2}}]
    \label{##3}
    ~ \hfill
    {\small\emph{(##1)}}
    \smallskip

  }
  \newcommand{\dataInfos}[2]{
    \algoword{##1:}
      \begin{itemize}[topsep=0pt]
          ##2
      \end{itemize}
    \smallskip
  }
  \newcommand{\dataInfo}[2]{
    \algoword{##1:} ##2 
    %\smallskip
  }
  \newcommand{\algoSteps}[1]{
    %\smallskip
    \addtolength{\leftmargini}{-0.35cm}
    \begin{enumerate}[{\bf 1.}]
        ##1
    \end{enumerate}
    \smallskip
  }

  \expandafter\argfig\expandafter{#1}
    \centering
    \begin{minipage}{0.99\textwidth}
    \begin{mdframed}
      \smallskip
    }
    {
    \end{algorithm}
    \end{mdframed}
    \end{minipage}
  \end{figure}
}

\newenvironment{problembox}[1][htbp]{
  \newcommand{\problemInfo}[3]{
    \begin{problem}[{name=[\emph{##2}\ifx&##1&\else##1\fi]\emph{##2}}]
    \label{##3}
    ~\smallskip

  }
  \newcommand{\dataInfos}[2]{
    \emph{##1:}
    \begin{itemize}[topsep=0pt]
      ##2
    \end{itemize}
    \smallskip
  }
  \newcommand{\dataInfo}[2]{
    \emph{##1:}  ##2
  }

  \expandafter\argfig\expandafter{#1}
    \centering
    \begin{minipage}{0.75\textwidth}
    \begin{mdframed}
    }
    {
    \end{problem}
    \end{mdframed}
    \end{minipage}
  \end{figure}
}


\begin{document}
  \maketitle


\section{Definitions}
\label{sec:relations}

First thing: what are linearly recurrent sequences and minimal generators, when
talking about matrix sequences? The definition is essentially the same as in
the scalar case. There is a similar notion of generator of a sequence (which is
a matrix): it is related to the denominator in some minimal fraction
description of the generating series of the sequence.

We consider the following definition which extends the scalar case. It can be
found in \cite[Sec.\,3]{KalVil01}, and in \cite[Def.\,4.2]{Turner02}.

\begin{definition}
  \label{dfn:recurrence_relation}
  Let $\seq = (\seqelt{k})_{k\in\NN} \subset \matSpace[\rdim][\cdim]$. A vector
  $\rel = \sum_{0\le k\le \degBd} p_k \var^k \in \relSpace$ of degree at most
  $\degBd$ is said to be a \emph{(linear recurrence) relation for $\seq$} if
  $\sum_{k=0}^{\degBd} p_{k} \seqelt{\delta + k} = 0$ for all $\delta \ge 0$.
  Then, $\seq$ is said to be linearly recurrent if there exists a nontrivial
  relation for $\seq$.
\end{definition}

The set of relations for $\seq$ is a $\polRing$-submodule of $\relSpace$, which
\fixme{has rank $\rdim$ if $\seq$ is linearly recurrent}. This is showed in
\cite[Fact\,1]{KalYuh13} but only for sequences having a \emph{scalar}
recurrence relation. Let us now link this with the property of being linearly
recurrent as in the above definition.

\begin{lemma}
  The sequence $\seq$ is linearly recurrent if and only if there exists a
  polynomial $P(\var) = \sum_{0\le k\le \degBd} p_k \var^k \in \polRing$ such
  that $\sum_{k=0}^{\degBd} p_{k} \seqelt{\delta + k} = 0$ for all $\delta \ge
  0$.
\end{lemma}
\begin{proof}
  \todo{I cannot find this result in the literature..!} (but we don't care since
  in our case the sequence is obviously cancelled by a polynomial)
\end{proof}

Then, the fact that the relation module has rank $\rdim$ is straightforward: since
the sequence is linearly recurrent, it has a scalar recurrence polynomial
$P(\var)$, hence for each $i$ the vector $[0 \; \cdots \; 0 \; P(\var) \; 0 \;
\cdots \; 0]$ with $P(\var)$ at position $i$ is a relation for $\seq$.

A \emph{generating matrix}, or \emph{generator}, for the sequence is a matrix
whose rows form a generating set for the module of relations; it is said to be
\begin{itemize}
  \item \emph{minimal} if the matrix is reduced \cite{Wolovich74,Kailath80};
  \item \emph{ordered weak Popov} if the matrix is in weak Popov form
    \cite{MulSto03} with pivots on the diagonal;
  \item \emph{canonical} if the matrix is in Popov form \cite{Popov72,Kailath80}.
\end{itemize}
In this context, an important quantity related to the sequence is the
determinantal degree $\deg(\det(\relbas))$, invariant for all generators
$\relbas \in \relbasSpace$. Hereafter, we denote it by $\degDet$.

\begin{lemma}
  Consider $\seq = (\seqelt{k})_{k\in\NN} \subset \seqeltSpace$ and
  its generating series $\seqpm = \sum_{k\ge 0} \seqelt{k} / \var^{k+1} \in
  \field\Poxi^{\rdim \times \cdim}$.
  \begin{itemize}
    \item A vector $\rel \in \relSpace$ is a relation for $\seq$ if and only if
      the entries of $\num = \rel \seqpm$ are in $\polRing$. Then, $\deg(\num)
      < \deg(\rel)$.
    \item A matrix $\relbas \in \relbasSpace$ is a generator for $\seq$ if and
      only if $\deg(\det(\relbas)) = \degDet$ and the entries of $\nummat =
      \relbas \seqpm$ are in $\polRing$. Then, $\rdeg{\nummat} <
      \rdeg{\relbas}$ componentwise.
  \end{itemize}
\end{lemma}
\begin{proof}
  Let $\rel = \sum_{0 \le k \le \degBd} p_k \var^k$. For $\delta \ge 0$, the
  coefficient of $\num$ of degree $-\delta-1<0$ is $\sum_{0\le k \le \degBd}
  p_k \seqelt{k+\delta}$. Hence the equivalence, by definition of a relation.
  The degree comparison is clear since $\seqpm$ has only terms of (strictly)
  negative degree.

  Concerning the second item, thanks to the first item applied to each row of
  $\relbas$, it is enough to assume that all the rows of $\relbas$ are
  relations for $\seq$, and to prove that $\relbas$ is a generator if and only
  if $\deg(\det(\relbas)) = \degDet$. Our assumption implies that $\relbas$ is
  a left multiple $\relbas = \mat{U} \mat{G}$ for some matrix $\mat{U}$ and
  some generator $\mat{G}$. Obviously if $\relbas$ is a generator then
  $\deg(\det(\relbas)) = \degDet$. Conversely, if $\deg(\det(\relbas)) =
  \degDet$ then $\mat{U}$ is unimodular.
\end{proof}

%alternative definition from \cite{Thome02}.
%
%\begin{definition}[\cite{Thome02}]
%  \label{dfn:relation_thome}
%  Let $\seq = (\seqelt{k})_{k\in\NN} \subset \matSpace[\rdim][\cdim]$ be a
%  sequence of $\rdim\times\cdim$ matrices over $\field$. We define the
%  generating series $\seqpm = \sum_{k\ge 0} \seqelt{k} \var^k \in
%  \psMatSpace[\rdim][\cdim]$. Then, a vector $\rel \in \relSpace$ is said to be
%  a \emph{(linear recurrence) relation for $\seq$} if the product $\rel\seqpm$
%  has polynomial entries, that is, $\rel \seqpm \in \relSpace$.
%\end{definition}
%
%Assume there is a nontrivial relation $\rel = \sum_k p_k \var^k$ for $\seq$, we
%have
%\begin{equation}
%  \label{eqn:relation_thome}
%  \sum_{k=0}^{d} p_{k} \seqelt{\delta - k} = 0 \quad \text{ for all } d \ge
%  \deg(\rel) \text{ and } \delta \ge \max(d,\deg(\seqpm \rel)+1).
%\end{equation}
%The alternative definition focuses on this type of relation.
%\begin{lemma}
%  \label{lem:link_defs}
%  For a given sequence $\seq \subset \matSpace[\rdim][\cdim]$, a nonzero vector
%  $\rel \in \relSpace$ is a relation for \cref{dfn:relation_thome} if and only
%  if there exists $d \ge \deg(\rel)$ such that the reverse $\var^{d}
%  \rel(\var^{-1})$ is a relation for \cref{dfn:relation_ky}.
%\end{lemma}
%\begin{proof}
%  First, we assume that $\var^{d} \rel(\var^{-1}) = \sum_{k=0}^{d} p_{d-k}
%  \var^k$ is a relation for \cref{dfn:relation_ky}, for some integer $d \ge
%  \deg(\rel)$. This means that, for all $\delta \ge 0$, we have $0 =
%  \sum_{k=0}^{d} \seqelt{\delta + k} p_{d-k} = \sum_{k=0}^{d} \seqelt{\delta+d
%  - k} p_{k}$. This implies that $\seqpm\rel$ has polynomial entries (and
%  $\deg(\seqpm\rel) \le d$).
%
%  Now, we assume that $\rel$ is a relation for \cref{dfn:relation_thome}.
%  Taking $d = \max(\deg(\rel),\deg(\seqpm \rel)+1)$ in
%  \cref{eqn:relation_thome}, we obtain $\sum_{k=0}^{d} \seqelt{\delta - k}
%  p_{k} = 0$ for all $\delta \ge d$. This implies $\sum_{k=0}^{d}
%  \seqelt{\delta-d + k} p_{d-k} = 0$ for all $\delta\ge d$, or equivalently,
%  $\sum_{k=0}^{d} \seqelt{\delta+k} p_{d-k} = 0$ for all $\delta\ge 0$.
%  Therefore the reverse $\var^{d} \rel(\var^{-1})$ is a relation for
%  \cref{dfn:relation_ky}.
%\end{proof}

\section{Computing a minimal matrix generators}
\label{sec:computing_matrix_generators}

We consider the following problem.

\begin{problembox}[htbp]
  \problemInfo
  {}
  {Minimal generator}
  {pbm:mingen}

  \dataInfos{Input}{%itemize
    \item sequence $\seq = (\seqelt{k})_k \subset \matSpace[\rdim][\cdim]$,
    \item degree bound $\degBd \in \NN$.
  }

  \dataInfos{Assumptions}{%itemize
    \item the sequence $\seq$ is linearly recurrent,
    \item the canonical generating matrix of $\seq$ has degree at most $\degBd
      \in \NN$.
  }

  \dataInfo{Output}{
    a minimal generating matrix for $\seq$.
  }
\end{problembox}


\bibliographystyle{plain}
\bibliography{biblio.bib}

\end{document}
