%%%preamble.
\documentclass[12pt]{article}

\usepackage[english]{babel}
\usepackage{enumerate}
\usepackage{fullpage}
\usepackage[colorlinks=true,linkcolor=cyan]{hyperref}
\usepackage{color}
\newcommand{\todo}[1]{\textcolor{red}{#1}}
\newcommand{\fixme}[1]{\textcolor{blue}{#1}}

% for enumerate / itemize: define reasonable margins
\usepackage[shortlabels]{enumitem}
\setlist{topsep=0.25\baselineskip,partopsep=0pt,itemsep=1pt,parsep=0pt}

% math and theorem names
\usepackage{amsmath,amsfonts,amssymb,amsthm,thmtools}
\declaretheorem[style=plain,parent=section]{definition}
\declaretheorem[sibling=definition]{theorem}
\declaretheorem[sibling=definition]{corollary}
\declaretheorem[sibling=definition]{proposition}
\declaretheorem[sibling=definition]{lemma}
\declaretheorem[style=remark,sibling=definition,qed={\qedsymbol}]{remark}
\declaretheorem[style=remark,sibling=definition,qed={\qedsymbol}]{example}
\declaretheoremstyle[headpunct={},notebraces={\textbf{--}~}{}]{algorithm}
\declaretheorem[style=algorithm]{problem}
\declaretheorem[style=algorithm]{algorithm}

% for our algorithms & problems
\usepackage{mdframed}

\usepackage[capitalize]{cleveref}
\crefname{problem}{Problem}{Problems}
\Crefname{problem}{Problem}{Problems}

\title{Matrix Berlekamp-Massey}

\author{}
\date{\today}

%%%notation
%misc
\newcommand{\storeArg}{} % aux, not to be used in document!!
\newcounter{notationCounter}
%spaces
\newcommand{\NN}{\mathbb{Z}_{\ge 0}} % nonnegative integers
\newcommand{\var}{X} % variable for univariate polynomials
\newcommand{\field}{\mathbb{K}} % base field
\newcommand{\polRing}{\field[\var]} % polynomial ring
\newcommand{\Pox}{[\mkern-3mu[ \var ]\mkern-3.2mu]}
\newcommand{\Poxi}{[\mkern-3mu[ \var^{-1} ]\mkern-3.2mu]}
\newcommand{\psRing}{\field\Pox}
\newcommand{\matSpace}[1][\rdim]{\renewcommand\storeArg{#1}\matSpaceAux} % polynomial matrix space, 2 opt args
\newcommand{\matSpaceAux}[1][\storeArg]{\field^{\storeArg \times #1}} % not to be used in document
\newcommand{\polMatSpace}[1][\rdim]{\renewcommand\storeArg{#1}\polMatSpaceAux} % polynomial matrix space, 2 opt args
\newcommand{\polMatSpaceAux}[1][\storeArg]{\polRing^{\storeArg \times #1}} % not to be used in document
\newcommand{\psMatSpace}[1][\rdim]{\renewcommand\storeArg{#1}\psMatSpaceAux} % polynomial matrix space, 2 opt args
\newcommand{\psMatSpaceAux}[1][\storeArg]{\psRing^{\storeArg \times #1}} % not to be used in document
\newcommand{\mat}[1]{\mathbf{\MakeUppercase{#1}}} % for a matrix
\newcommand{\row}[1]{\mathbf{\MakeLowercase{#1}}} % for a matrix
\newcommand{\col}[1]{\mathbf{\MakeLowercase{#1}}} % for a matrix
\newcommand{\matCoeff}[1]{\MakeLowercase{#1}} % for a coefficient in a matrix
\newcommand{\rdim}{m} % row dimension
\newcommand{\cdim}{n} % column dimension
\newcommand{\diag}[1]{\mathrm{Diag}(#1)}  % diagonal matrix with diagonal entries #1
\newcommand{\seqelt}[1]{S_{#1}} % element of sequence of matrices
\newcommand{\seqeltSpace}{\matSpace[\rdim][\cdim]} % element of sequence of matrices
\newcommand{\seq}{\mathcal{S}} % sequence of matrices
\newcommand{\seqpm}{\mat{S}} % power series matrix from a sequence
\newcommand{\rel}{\col{p}} % linear relation
\newcommand{\relbas}{\mat{P}} % linear relation
\newcommand{\relSpace}{\polMatSpace[1][\rdim]} % space for linear relations
\newcommand{\relbasSpace}{\polMatSpace[\rdim][\rdim]} % space for linear relations
\newcommand{\num}{\row{q}} % numerator for linear recurrence relation
\newcommand{\nummat}{\mat{Q}} % numerator for linear recurrence relation basis
\newcommand{\rem}{\row{r}} % remnant for linear recurrence relation
\newcommand{\remmat}{\mat{R}} % remnant for linear recurrence relation basis
\newcommand{\remSpace}{\polMatSpace[1][\cdim]} % space for linear relations
\newcommand{\degBd}{d} % bound on degree of minimal generator
\newcommand{\degBdr}{d_{r}} % bound on degree of a right minimal generator
\newcommand{\degBdl}{d_{\ell}} % bound on degree of a left minimal generator
\newcommand{\degDet}[1][\seq]{\operatorname{\Delta}(#1)}
\newcommand{\rdeg}[2][]{\mathrm{rdeg}_{{#1}}(#2)} % shifted row degree
\newcommand{\cdeg}[2][]{\mathrm{cdeg}_{{#1}}(#2)} % shifted column degree
\newcommand{\sys}{\mat{F}} % input matrix series to approximant basis
\newcommand{\appMod}[2]{\mathcal{A}(#1,#2)} % module of approximants for #2 at order #1

%% ------------------------------------------------
%% --------- For problems and algorithms ----------
%% ------------------------------------------------
\newcommand{\argfig}[1]{\begin{figure}[#1]} % to be able to feed an optional argument to the inside figure

\newenvironment{algobox}[1][htbp]{
  \newcommand{\algoInfo}[3]{
    \begin{algorithm}[{name=[\algoname{##2}:~##1]\algoname{##2}}]
    \label{##3}
    ~ \hfill
    {\small\emph{(##1)}}
    \smallskip

  }
  \newcommand{\dataInfos}[2]{
    \algoword{##1:}
      \begin{itemize}[topsep=0pt]
          ##2
      \end{itemize}
    \smallskip
  }
  \newcommand{\dataInfo}[2]{
    \algoword{##1:} ##2 
    %\smallskip
  }
  \newcommand{\algoSteps}[1]{
    %\smallskip
    \addtolength{\leftmargini}{-0.35cm}
    \begin{enumerate}[{\bf 1.}]
        ##1
    \end{enumerate}
    \smallskip
  }

  \expandafter\argfig\expandafter{#1}
    \centering
    \begin{minipage}{0.99\textwidth}
    \begin{mdframed}
      \smallskip
    }
    {
    \end{algorithm}
    \end{mdframed}
    \end{minipage}
  \end{figure}
}

\newenvironment{problembox}[1][htbp]{
  \newcommand{\problemInfo}[3]{
    \begin{problem}[{name=[\emph{##2}\ifx&##1&\else##1\fi]\emph{##2}}]
    \label{##3}
    ~\smallskip

  }
  \newcommand{\dataInfos}[2]{
    \emph{##1:}
    \begin{itemize}[topsep=0pt]
      ##2
    \end{itemize}
    \smallskip
  }
  \newcommand{\dataInfo}[2]{
    \emph{##1:}  ##2
  }

  \expandafter\argfig\expandafter{#1}
    \centering
    \begin{minipage}{0.75\textwidth}
    \begin{mdframed}
    }
    {
    \end{problem}
    \end{mdframed}
    \end{minipage}
  \end{figure}
}


\begin{document}
  \maketitle

\section{Linearly recurrent matrix sequences}
\label{sec:relations}

When considering sequences of matrices over a field $\field$, \emph{linearly
recurrent sequences} and \emph{minimal generators} are defined similarly as for
scalar sequences.

\begin{definition}[{\cite[Sec.\,3]{KalVil01}}]
  %% also \cite[Def.\,4.2]{Turner02}.. any earlier ref?
  \label{dfn:recurrence_relation}
  Let $\seq = (\seqelt{k})_{k\in\NN} \subset \matSpace[\rdim][\cdim]$ be a
  matrix sequence and let $\rel = \sum_{0\le k\le \degBd} p_k \var^k \in
  \relSpace$ be a polynomial vector.  Then,
  \begin{itemize}
    \item $\rel$ is said to be a \emph{(vector) relation for $\seq$} if
      $\sum_{k=0}^{\degBd} p_{k} \seqelt{\delta + k} = 0$ for all $\delta \ge
      0$;
    \item $\seq$ is said to be \emph{linearly recurrent} if there exists a
      nontrivial relation for $\seq$.
  \end{itemize}
\end{definition}

Hereafter, the word generator \emph{never} refers to these relations.  This
word will be reserved to generating sets for the set of all relations.

We will now see that, as in the scalar case, one can define the minimal
generator of a sequence as the denominator in some irreducible fraction
description of the generating series of the sequence. The set of relations for
$\seq$ is a $\polRing$-submodule of $\relSpace$, which \todo{has rank $\rdim$
if $\seq$ is linearly recurrent}. This is showed in \cite[Fact\,1]{KalYuh13} or
in \cite{Turner02} for sequences having a \emph{scalar} recurrence relation.
Let us now link this with the property of being linearly recurrent as in the
above definition.

\begin{lemma}
  The sequence $\seq$ is linearly recurrent if and only if there exists a
  polynomial $P(\var) = \sum_{0\le k\le \degBd} p_k \var^k \in \polRing$ such
  that $\sum_{k=0}^{\degBd} p_{k} \seqelt{\delta + k} = 0$ for all $\delta \ge
  0$.
\end{lemma}
\begin{proof}
  \todo{I cannot find this result in the literature..!} \fixme{actually, is it
  even true?} (but we don't care since in our case the sequence obviously
  admits a scalar relation)
\end{proof}

Then, the fact that the relation module has rank $\rdim$ is straightforward: since
the sequence is linearly recurrent, it has a scalar recurrence polynomial
$P(\var)$, hence for each $i$ the vector $[0 \; \cdots \; 0 \; P(\var) \; 0 \;
\cdots \; 0]$ with $P(\var)$ at position $i$ is a relation for $\seq$.

A \emph{matrix generator} for the sequence is a matrix whose rows form a
generating set for the module of relations; it is said to be
\begin{itemize}
  \item \emph{minimal} if the matrix is row reduced \cite{Wolovich74,Kailath80};
  \item \emph{ordered weak Popov} if the matrix is in weak Popov form
    \cite{MulSto03} with pivots on the diagonal;
  \item \emph{canonical} if the matrix is in Popov form \cite{Popov72,Kailath80}.
\end{itemize}
In this context, an important quantity related to the sequence is the
determinantal degree $\deg(\det(\relbas))$, invariant for all generators
$\relbas \in \relbasSpace$. Hereafter, we denote it by $\degDet$.

\begin{lemma}
  \label{lem:linearly_recurrent}
  Consider a matrix sequence $\seq = (\seqelt{k})_{k\in\NN} \subset
  \seqeltSpace$ and its generating series $\seqpm = \sum_{k\ge 0} \seqelt{k} /
  \var^{k+1} \in \field\Poxi^{\rdim \times \cdim}$.  Then, a vector $\rel \in
  \relSpace$ is a relation for $\seq$ if and only if the entries of $\num =
  \rel \seqpm$ are in $\polRing$; furthermore, in this case, $\deg(\num) <
  \deg(\rel)$.
  %A matrix $\relbas \in \relbasSpace$ is a generator for $\seq$ if and only if
  %$\deg(\det(\relbas)) = \degDet$ and the entries of $\nummat = \relbas
  %\seqpm$ are in $\polRing$. Then, $\rdeg{\nummat} < \rdeg{\relbas}$
  %componentwise.
\end{lemma}
\begin{proof}
  Let $\rel = \sum_{0 \le k \le \degBd} p_k \var^k$. For $\delta \ge 0$, the
  coefficient of $\num$ of degree $-\delta-1<0$ is $\sum_{0\le k \le \degBd}
  p_k \seqelt{k+\delta}$. Hence the equivalence, by definition of a relation.
  The degree comparison is clear since $\seqpm$ has only terms of (strictly)
  negative degree.
  %
  %  Concerning the second item, thanks to the first item applied to each row
  %  of $\relbas$, it is enough to assume that all the rows of $\relbas$ are
  %  relations for $\seq$, and to prove that $\relbas$ is a generator if and
  %  only if $\deg(\det(\relbas)) = \degDet$. Our assumption implies that
  %  $\relbas$ is a left multiple $\relbas = \mat{U} \mat{G}$ for some matrix
  %  $\mat{U}$ and some generator $\mat{G}$. Obviously if $\relbas$ is a
  %  generator then $\deg(\det(\relbas)) = \degDet$; conversely, if
  %  $\deg(\det(\relbas)) = \degDet$ then $\mat{U}$ is unimodular.
\end{proof}

\begin{corollary}
  A matrix sequence $\seq = (\seqelt{k})_{k\in\NN} \subset \seqeltSpace$ is
  linearly recurrent if and only if its generating series $\seqpm = \sum_{k\ge
  0} \seqelt{k} / \var^{k+1} \in \field\Poxi^{\rdim \times \cdim}$ can be
  written as a matrix fraction $\seqpm = \relbas^{-1} \nummat$ where $\relbas
  \in \relbasSpace$ is nonsingular and $\nummat \in
  \polMatSpace[\rdim][\cdim]$. In this case,
  \begin{itemize}
    \item we have $\rdeg{\nummat} < \rdeg{\relbas}$ and $\deg(\det(\relbas))
      \ge \degDet$,
    \item $\relbas$ is a matrix generator of $\seq$ if and only if the fraction
      is irreducible (i.e. $\mat{U} \relbas + \mat{V} \nummat = \mat{I}$ for
      some $\mat{U}$, $\mat{V}$),
    \item $\relbas$ is a matrix generator of $\seq$ if and only if
      $\deg(\det(\relbas)) = \degDet$.
  \end{itemize}
\end{corollary}

\bigskip

For more details:
  \begin{itemize}
    \item \cite[Sec.\,1]{Villard97} when the sequence is of the form $\seq =
      (\mat{U} \mat{A}^k \mat{V})_k$. Note that in this case the generating
      series can be written $\seqpm = \mat{U} (\var \mat{I} - \mat{A})^{-1}
      \mat{V}$. Link with so-called realizations from control theory
      \cite{Kailath80}\ldots
    \item \cite[Chap.\,4]{Turner02} has things related to Hankel matrices (but
      it is extremely detailed, including many properties which are actually
      about polynomial matrices and completely independent of the ``linear
      recurrence'' context)
  \end{itemize}

%alternative definition from \cite{Thome02}.
%
%\begin{definition}[\cite{Thome02}]
%  \label{dfn:relation_thome}
%  Let $\seq = (\seqelt{k})_{k\in\NN} \subset \matSpace[\rdim][\cdim]$ be a
%  sequence of $\rdim\times\cdim$ matrices over $\field$. We define the
%  generating series $\seqpm = \sum_{k\ge 0} \seqelt{k} \var^k \in
%  \psMatSpace[\rdim][\cdim]$. Then, a vector $\rel \in \relSpace$ is said to be
%  a \emph{(linear recurrence) relation for $\seq$} if the product $\rel\seqpm$
%  has polynomial entries, that is, $\rel \seqpm \in \relSpace$.
%\end{definition}
%
%Assume there is a nontrivial relation $\rel = \sum_k p_k \var^k$ for $\seq$, we
%have
%\begin{equation}
%  \label{eqn:relation_thome}
%  \sum_{k=0}^{d} p_{k} \seqelt{\delta - k} = 0 \quad \text{ for all } d \ge
%  \deg(\rel) \text{ and } \delta \ge \max(d,\deg(\seqpm \rel)+1).
%\end{equation}
%The alternative definition focuses on this type of relation.
%\begin{lemma}
%  \label{lem:link_defs}
%  For a given sequence $\seq \subset \matSpace[\rdim][\cdim]$, a nonzero vector
%  $\rel \in \relSpace$ is a relation for \cref{dfn:relation_thome} if and only
%  if there exists $d \ge \deg(\rel)$ such that the reverse $\var^{d}
%  \rel(\var^{-1})$ is a relation for \cref{dfn:relation_ky}.
%\end{lemma}
%\begin{proof}
%  First, we assume that $\var^{d} \rel(\var^{-1}) = \sum_{k=0}^{d} p_{d-k}
%  \var^k$ is a relation for \cref{dfn:relation_ky}, for some integer $d \ge
%  \deg(\rel)$. This means that, for all $\delta \ge 0$, we have $0 =
%  \sum_{k=0}^{d} \seqelt{\delta + k} p_{d-k} = \sum_{k=0}^{d} \seqelt{\delta+d
%  - k} p_{k}$. This implies that $\seqpm\rel$ has polynomial entries (and
%  $\deg(\seqpm\rel) \le d$).
%
%  Now, we assume that $\rel$ is a relation for \cref{dfn:relation_thome}.
%  Taking $d = \max(\deg(\rel),\deg(\seqpm \rel)+1)$ in
%  \cref{eqn:relation_thome}, we obtain $\sum_{k=0}^{d} \seqelt{\delta - k}
%  p_{k} = 0$ for all $\delta \ge d$. This implies $\sum_{k=0}^{d}
%  \seqelt{\delta-d + k} p_{d-k} = 0$ for all $\delta\ge d$, or equivalently,
%  $\sum_{k=0}^{d} \seqelt{\delta+k} p_{d-k} = 0$ for all $\delta\ge 0$.
%  Therefore the reverse $\var^{d} \rel(\var^{-1})$ is a relation for
%  \cref{dfn:relation_ky}.
%\end{proof}

\section{Computing minimal matrix generators}
\label{sec:computing_matrix_generators}

Now, we focus on the following algorithmic problem: we are given a linearly
recurrent sequence and we want to find a matrix generator. If we want our
algorithm to run efficiently (or simply, in finite time), we cannot access
infinitely many terms of the sequence. We therefore ask for an additional
input, which one often has when considering a sequence coming from some
application: a bound on the degree of any minimal matrix generator.  Note that
all minimal generators have the same degree: that of the canonical generator.
If available, a bound on the determinantal degree $\degDet$ is sufficient; yet
better bounds can be available and will yield better efficiency.  We now focus
on \cref{pbm:mingen}.

\begin{problembox}[htbp]
  \problemInfo
  {}
  {Minimal matrix generator}
  {pbm:mingen}

  \dataInfos{Input}{%itemize
    \item sequence $\seq = (\seqelt{k})_k \subset \seqeltSpace$,
    \item degree bounds $(\degBdl,\degBdr) \in \NN^2$.
  }

  \dataInfos{Assumptions}{%itemize
    \item the sequence $\seq$ is linearly recurrent,
    \item the left (resp.~right) canonical matrix generator of $\seq$ has
      degree at most $\degBdl$ (resp.~$\degBdr$).
  }

  \dataInfo{Output}{
    a minimal matrix generator for $\seq$.
  }
\end{problembox}

We now show how the additional information of $\degBd$ allows us to find a
matrix generator by considering only a small chunk of the sequence, rather than
all its terms.

The fast computation of matrix generators is usually handled via algorithms for
computing minimal approximant bases \cite{Villard97,Turner02,GioLeb14}. The
next result gives the main idea behind this approach. This result is similar to
\cite[Thm.\,4.7,\,4.8,\,4.9,\,4.10]{Turner02}, but in some sense the reversal
is on the input sequence rather than on the output matrix generator (and also
this section 4.2 of \cite{Turner02} provides many more details related to the
mechanisms and output properties in the approximant basis algorithm, which we
do not consider here).

We recall from \cite{BarBul92,BecLab94} that, given a matrix $\sys \in
\polMatSpace[\rdim][\cdim]$ and an integer $d \in \NN$, the set of
\emph{approximants for $\sys$ at order $d$} is defined as
\[
  \appMod{\sys}{d} = \{ \rel \in \relSpace \mid \rel \sys = 0 \bmod \var^d \}.
\]

Then, the following lemma shows that relations for $\seq$ can be retrieved as
subvectors of approximants at order about $\degBdl+\degBdr$ for a matrix
involving the first $\degBdl+\degBdr$ entries of the sequence $\seq$. Note that
these bounds $\degBdl,\degBdr$ are the same as Turner's $\gamma_1,\gamma_2$
\cite[Def.\,4.6~and\,4.7]{Turner02}.


\begin{theorem}
  \label{thm:mingen_via_appbas}
  Let $\seq = (\seqelt{k})_k \subset \seqeltSpace$ be a linearly recurrent
  sequence. For $\degBd>0$, define
  \begin{equation}
    \label{eqn:series_to_approximate}
    \sys =
    \begin{bmatrix}
      \sum_{0\le k < \degBd} \seqelt{k} \var^{\degBd-k-1} \\ - \mat{I}_{\cdim}
    \end{bmatrix} \in \polMatSpace[(\rdim+\cdim)][\cdim].
  \end{equation}
  Then, for any relation $\rel \in \relSpace$ for $\seq$, there exists $\rem
  \in \remSpace$ such that $\deg(\rem) < \deg(\rel)$ and $[\rel \;\; \rem] \in
  \appMod{\sys}{\degBd}$.

  Now, consider $(\degBdl,\degBdr) \in \NN^2$ such that the left (resp.~right)
  canonical matrix generator of $\seq$ has degree at most $\degBdl$
  (resp.~$\degBdr$), and define $\sys$ for $\degBd = \degBdl+\degBdr+1$.  For
  any vectors $\rel \in \relSpace$ and $\rem \in \remSpace$, if $[\rel \;\;
  \rem] \in\appMod{\sys}{\degBdl+\degBdr+1}$ and $\deg([\rel \;\;
  \rem])\le\degBdl$, then $\rel$ is a relation for $\seq$ and
  $\deg(\rem)<\deg(\rel)$.
  
  As a corollary, if $\mat{B} \in \polMatSpace[(\rdim+\cdim)][(\rdim+\cdim)]$
  is a basis of $\appMod{\sys}{\degBdl+\degBdr+1}$, then
  \begin{itemize}
    \item if $\mat{B}$ is in Popov (resp. ordered weak Popov) form then its
      $\rdim\times\rdim$ leading principal submatrix is the canonical (resp.
      ordered weak Popov) matrix generator for $\seq$;
    \item if $\mat{B}$ is row reduced then it has exactly $\rdim$ rows of
      degree $\le\degBdl$, and the corresponding submatrix $[\relbas \;\;
      \remmat]$ of $\mat{B}$ is such that $\relbas\in\relSpace$ is a minimal
      matrix generator for $\seq$.
      %the form $[\rel \;\; \rem]$ with $\rel \in \relSpace$, $\rem \in
      %\remSpace$, and $\deg(\rem) < \deg(\rel) \le \degBdl$
  \end{itemize}
\end{theorem}
\begin{proof}
  From \cref{lem:linearly_recurrent}, if $\rel$ is a relation for $\seq$ then
  $\num = \rel \seqpm$ has polynomial entries, where $\seqpm = \sum_{k\ge 0}
  \seqelt{k} \var^{-k-1}$. Then, the vector $\rem = - \rel (\sum_{k \ge \degBd}
  \seqelt{k} \var^{\degBd-k-1})$ has polynomial entries, has degree less than
  $\deg(\rel)$, and is such that $[\rel \;\; \rem] \sys = \num \var^{\degBd}$,
  hence $[\rel \;\; \rem] \in \appMod{\sys}{\degBd}$.


  
  Then, the three items are straightforward consequences.
\end{proof}

\begin{corollary}
  Assuming $\rdim = \Theta(\cdim)$, any of these matrix generators (minimal,
  Popov, \ldots) can be computed in $O(\rdim^\omega \mathsf{M}(\degBd)
  \log(\degBd))$ operations in $\field$, where $d =\max(\degBdl,\degBdr)$.
\end{corollary}

We would prefer to say that we compute the canonical form, rather than a
minimal one. In theory, exactly the same asymptotic cost bound (but not yet in
the literature, so this needs some short explanation; except if we do not care
about logarithmic factors then this is in the literature).

With our implementation, asking for the canonical form should induce a slowdown
factor of at most $2$.

\bibliographystyle{plain}
\bibliography{biblio.bib}

\end{document}
