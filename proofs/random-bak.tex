\documentclass[12pt]{article}

\usepackage{amsfonts,amsmath,amsthm}
\usepackage{mathrsfs}
\usepackage{fullpage}

\newtheorem{definition}{Definition}
\newtheorem{theorem}[definition]{Theorem}
\newtheorem{corollary}[definition]{Corollary}
\newtheorem{proposition}[definition]{Proposition}
\newtheorem{lemma}[definition]{Lemma}
\newtheorem{remark}[definition]{Remark}



\def\K{\mathbb{K}}
\def\Deg{D}
\def\scrY{\mathscr{Y}}

\begin{document}	

Let $A$ be in $\K^{\Deg\times \Deg}$. Let $s_1,\dots,s_r$ be the
invariant factors of $t I - A$, ordered in such a way that $s_2$
divides $s_1$, $s_3$ divides $s_2$, etc, and let $d_i=\deg(s_i)$ for
all $i$; for $i > r$, we let $s_i=1$, with $d_i=0$. 

We fix an integer $M$ and we define $\nu=d_1 +\cdots +d_M \le \Deg$
and $\delta =\lceil \nu/M\rceil \le \lceil \Deg/M\rceil$.
We choose matrices $X,Y \in \mathbb{K}^{D\times M}$, and we let
$F_X^{A,Y}$ be the minimal generator in Popov form of the sequence
$(X^\perp\, A^i\, Y)_{i \ge 0}$. We also denote by $\sigma_1,\dots,\sigma_t$ the
invariant factors of $F_X^{A,Y}$, for some $t \le M$. As above, for $i
> t$, we let $\sigma_i =1$.

\begin{theorem}
  For a generic choice of $X$ and $Y$, we have:
  \begin{itemize}
  \item $F_X^{A,Y}$ has degree $\delta$;
  \item $s_i = \sigma_i$ for $1 \le i \le M$.
  \end{itemize}
\end{theorem}
\begin{proof}
Let $F^{A,Y}$ be the minimal generating polynomial of the sequence
$(A^i Y)_{i \ge 0}$.  We denote by $\langle Y \rangle$ the vector
space generated by the columns of $Y,AY,A^2Y,\dots$. We also write
$N_Y=\dim(\langle Y \rangle)$.

First, we prove that for any $Y$ in $\K^{N \times M}$, for a generic
$X$ in $\K^{N\times M}$, $F_X^{A,Y}=F^{A,Y}$.  Indeed,
by~\cite[Lemma~4.2]{Villard97a}, there exists matrices $P_Y$ in
$\K^{N\times N_Y}$ and $A_Y \in \K^{N_Y \times N_Y}$, with $P_Y$ of
full rank $N_Y$, and where $A_Y$ is a matrix of the restriction of $A$
to $\langle Y \rangle$, such that $F_X^{A,Y}=F^{A,Y}$ if and only if
the dimension of the span of $[Z ~ B_Y Z ~B_Y^2 Z ~ \cdots]$ is equal to
$N_Y$, with $B_Y=A_Y^\perp$ and $Z=P_Y^\perp X \in \K^{N_Y \times M}$.

We prove that this is the case for a generic $X$. By construction, one
can find a basis of $\langle Y \rangle$ in which the matrix of $A_Y$
is block-companion, with $M' \le M$ blocks (take the $A_Y$-span of the
first column of $Y$, then of the second column, working modulo the
previous vector space, etc.) Thus, $B_Y$ is similar to a
block-companion matrix with $M'$ blocks as well; since $Z$ has $M$
columns, $S$ has full dimension $N_Y$ for a generic $Z$ (and for a
generic $X$, since $P_Y$ has rank $N_Y$). Thus, for generic choices of
$X$ and $Y$, $F_X^{A,Y}=F^{A,Y}$.

Let us next introduce a matrix $\scrY$ of indeterminates of size $N
\times M$, and let $F^{A,\scrY}$ be the minimal generating polynomial
of the ``generic'' sequence $(A^i \scrY)_{i \ge 0}$. The notation
$\langle \scrY \rangle$ and $N_\scrY$ are defined as above.  In
particular, by~\cite[Proposition 6.1]{Villard97a}, the minimal
generating polynomial $F^{A,\scrY}$ has degree $\delta$ and
determinantal degree $\nu$.

Now, for a generic $Y$ in $\K^{N\times M}$, $N_Y=N_\scrY$. Indeed,
$\langle \scrY\rangle$ is the span of $[\scrY ~ A \scrY ~ \cdots ~
  A^{N-1} \scrY]$, whereas $\langle Y\rangle$ is the span of $[Y ~ A Y
  ~ \cdots ~A^{N-1} Y]$. Take a maximal non-zero minor $\mu$ of
$K_\scrY$; as soon as $\mu(Y)\ne 0$, we have equality of the
dimensions. On the other hand, by~\cite[Lemma~4.3]{Villard97a}, for
any $Y$ (including $\scrY$), the degree of $F^{A,Y}$ is equal to the
first index $d$ such that $\dim({\rm span}([Y ~ A Y ~ \cdots ~A^{d-1}
  Y]))=N_Y$. As a result, for generic $Y$, $F^{A,Y}$ and $F^{A,\scrY}$
have the same degree, that is, $\delta$.  The first item is proved.

We conclude by proving that for generic $X,Y$, the invariant factors
$\sigma_1,\dots,\sigma_M$ of $F_X^{A,Y}$ are $s_1,\dots,s_M$.
By~\cite[Theorem~2.12]{KaVi04}, for any $X$ and $Y$ in $\K^{N\times
  M}$, for $i=1,\dots,M$, the $i^{th}$ invariant factor $\sigma_i$ of
$F_X^{A,Y}$ divides $s_i$, so that $\deg(\det(F_X^{A,Y}))\le\nu$, with
equality if and only if $\sigma_i=s_i$ for all $i \le M$.

For $Y$ as above and any integers $e,d$, we let ${\rm Hk}_{e,d}(Y)$ be
the block Hankel matrix
$$ {\rm Hk}_{e,d}(Y) =
\begin{bmatrix}
I \\  A \\  A^2 \\ \vdots  \\  A^{e-1}
\end{bmatrix}
\begin{bmatrix}
Y & AY & A^2Y &\cdots&  A^{d-1}Y
\end{bmatrix}
$$ By~\cite[Eq.~(2.6)]{KaVi04}, ${\rm rank}({\rm Hk}_{e,d}(Y)) =
\deg(\det(F^{A,Y}))$ for $d \ge \deg(F^{A,Y})$ and $e \ge N$.  We take
$e=N$, so that ${\rm rank}({\rm Hk}_{N,d}(Y)) = \deg(\det(F^{A,Y}))$
for $d \ge \deg(F^{A,Y})$. On the other hand, the sequence ${\rm
  rank}({\rm Hk}_{N,d}(Y))$ is constant for $d \ge N$; as a result,
${\rm rank}({\rm Hk}_{N,N}(Y)) = \deg(\det(F^{A,Y}))$. For the same
reason, we also have ${\rm rank}({\rm Hk}_{N,N}(\scrY)) =
\deg(\det(F^{A,\scrY}))$, so that for a generic $Y$, $F^{A,Y}$ and
$F^{A,\scrY}$ have the same determinantal degree, that is, $\nu$.  As
a result, for generic $X$ and $Y$, we also have
$\deg(\det(F_X^{A,Y}))=\nu$, and the conclusion follows.
\end{proof}


\bibliographystyle{plain}
\bibliography{biblio}


\end{document}
