\documentclass[12pt]{article}
\usepackage{fullpage,url,amsmath,amsthm,amssymb,epsfig,color,xspace,mathrsfs}
\pagestyle{empty}
\newcommand{\todo}[1]{\textcolor{red}{#1}}

\usepackage{fullpage}

\thispagestyle{empty}

\begin{document}
\noindent\textbf{Lemma 1:} Let $U,V \in \mathbb{K}^{D\times M}$
be matrices with random entries and $s = (U^{tr}T_1^iV)_{i \ge 0}$. If 
$Z = \sum_{i = 0}^{\infty} s^{(i)}/ t^{i+1}$, then each entry of $Z$ is
in the form $n_*/P$, where $P$ is the minimum scalar generator for
$s$.\\

\noindent\textit{Proof:} Rewrite $U,V$ as 
$U = [u_1,u_2,\cdots,u_M], V= [v_1,v_2,\cdots,v_M]$, then 
$$s^{(i)} = 
\begin{bmatrix}
u_1^{tr}T_1^iv_1 & u_1^{tr}T_1^iv_2 & \cdots   & u_1^{tr}T_1^iv_M \\
u_2^{tr}T_1^iv_1 & \cdots           & \cdots   & u_2^{tr}T_1^iv_M \\
\vdots           & \ddots           & \ddots   & \vdots \\
u_M^{tr}T_1^iv_1 & \cdots           & \cdots   & u_M^{tr}T_1^iv_M
\end{bmatrix}$$

Thus, 	
$$ Z = 
\begin{bmatrix}
\sum u_1^{tr}T_1^iv_1/t^{i+1} & \cdots  & \cdots & \sum u_1^{tr}T_1^iv_M/t^{i+1} \\
\vdots                        & \ddots  & \ddots & \vdots \\
\sum u_M^{tr}T_1^iv_1/t^{i+1} & \cdots  & \cdots & \sum u_M^{tr}T_1^iv_M/t^{i+1}
\end{bmatrix}$$
So each entry separately is what would be computed in the scalar case. Therefore,
we can rewrite each entry as $n_*/P$ for some numerator $n_*$.\\

\noindent\textbf{Lemma 2:} Let $S$ be the minimum generating polynomial matrix for $s$
and $D = ASB$ be the Smith normal form of $S$. Furthermore, let $s_1, \cdots s_D$ be
invariant factors of $S$ such that $s_1 | s_2 | \cdots | s_D$. Then, there exists a vector $\tilde{u}$
such that $\tilde{u} S = [0, \cdots, 0, s_D]$\\

\noindent\textit{Proof:} Let $[b_1,\cdots,b_D]$ be the last row of $B$ and 
$w = [\frac{s_Db_1}{s1},\frac{s_Db_2}{s2},\cdots,\frac{s_Db_{D-1}}{s_{D-1}},b_D]$ (since $s_i | s_D$), then
\begin{align*}
	(w A) A^{-1} D &=  [\frac{s_Db_1}{s1},\frac{s_Db_2}{s2},\cdots,\frac{s_Db_{D-1}}{s_{D-1}},b_D]
	\begin{bmatrix}
	s_1 &        & \\
	    & \ddots & \\
	    &        & s_D
	\end{bmatrix}\\
	    &= [s_Db_1, s_Db_2, \cdots, s_Db_D]\\
	    &= [0,\cdots,0,s_D] B
\end{align*}
Therefore, if we choose $\tilde{u} = w A$, we get
$ \tilde{u} S = (w A) A^{-1} D B^{-1} = [0,\cdots,0,s_D]$ as needed.\\

\noindent\textbf{Lemma 2 (bis):}
Let $S$ be the minimum generating polynomial matrix for $s$.  Then, there
exists a vector $\tilde{u}$ such that $\tilde{u} S = [0, \ldots, 0, s_D]$,
where $s_D$ is the largest invariant factor of $S$.

\medskip\noindent\textit{Proof:} Since $S$ is nonsingular, its first $M-1$
columns are free. Then, a minimal left kernel basis for this $(M-1) \times M$
submatrix of $S$ is a row vector $\tilde{v} \in \mathbb{K}^{1 \times M}$ such
that $\tilde{v} S = [0,\ldots,0,\lambda]$, where the polynomial $\lambda$ is
the GCD of the entries in the last column of $S$. By definition, $s_D$ is
divisible by $\lambda$, so that $\tilde{u} = \frac{s_D}{\lambda}\tilde{v}$ has
the wanted property.

\medskip\noindent\textit{Remark:}
One may alternatively:
\begin{itemize}
  \item compute the Smith form of $S$, along with the unimodular
    transformations, and use the formulas in the proof of Lemma 2 (first
    version);
  \item compute $s_D$ via the inversion algorithm of Zhou-Labahn-Storjohann
    (Sec. 5.1, 2015), and then solve the system to find $\tilde{u}$.
\end{itemize}
The first solution is significantly more expensive if the full transformations
are needed. Here we might be happy with only the transformations with entries
reduced modulo the corresponding invariant factors, in which case both
solutions have the same theoretic speed. Yet, note that computing the Smith
form efficiently is randomized (Storjohann, 2003) and uses high-order lifting
(not in LinBox), while the second solution is deterministic and uses tools
close to approximant basis computation (already partially implemented in
LinBox). In the second solution, the system solving can be done via minimal
kernel basis \todo{(and also via lifting since the matrix $S$ is reduced, but
no issue with large degrees?)}.

\bigskip
\noindent\textbf{Lemma 3:} If $\sum_{i = 0}^{\infty} s^{(i)} t^i = S^{-1}N$, then $deg(N)$ is
less than or equal to $deg(S)$. TODO\\

\noindent\textbf{Theorem 1:} If 
$S^{-1}N = \sum_{i=0}^{\infty} s^{(i)}/t^{i+1}$, the first entry
of the last row of $\tilde{u} N$ is  the numerator of the generating function for 
$(u_{M}^{tr} T_1^i v_{1})_{i \ge 0}$.\\

\noindent\textit{Proof:} Let $S^{-1}N = \sum_{i=0}^{\infty} s^{(i)}/t^{i+1}$, 
then by lemma 1 
$$ N = S \sum_{i=0}^{\infty} s^{(i)}/t^{i+1} = S 
\begin{bmatrix}
n_{1,1} / P & \cdots & n_{1,M} / P \\
\vdots      & \ddots & \vdots \\
n_{M,1} / P & \cdots & n_{M,M} / P
\end{bmatrix}
$$
From theorem 1 of (randomXY-proof), we know that the $i^{th}$ invariant factor of
$XI - A$ is equal to the $i^{th}$ invariant factor of $S$ for generic choice of
$U,V$. Thus, $s_D = P$ and by lemma 2
\begin{align*}
\tilde{u} N &= \tilde{u} S
				\begin{bmatrix}
				n_{1,1} / P & \cdots & n_{1,M} / P \\
				\vdots            & \ddots & \vdots \\
				n_{M,1} / P & \cdots & n_{M,M} / P
				\end{bmatrix} \\
            &= [0,\cdots,0,P]
               \begin{bmatrix}
               n_{1,1} / P & \cdots & n_{1,M} / P \\
               \vdots            & \ddots & \vdots \\
               n_{M,1} / P & \cdots & n_{M,M} / P
               \end{bmatrix}\\
            &= [n_{M,1} , \cdots , n_{M,M}]
\end{align*}

\end{document}



