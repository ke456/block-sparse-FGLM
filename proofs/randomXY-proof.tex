\documentclass[12pt]{article}
\usepackage{amsfonts}
\pagestyle{empty}

\usepackage{fullpage}

\thispagestyle{empty}

\begin{document}	
\noindent\textbf{Relevant Theorems and misc:}\\
\noindent\textit{Proposition 6.1:} The minimal generating polynomial $D_{\bar{Y}}$ for the generic
sequence $\{A^i\bar{Y}\}$ has determinantal degree $\bar{\nu}$.\\

\noindent\textit{Theorem 8.1 (Villard):} Let $A$ be a $N \times N$ matrix over $K$ with minimal polynomial $\pi_A(\lambda)$,
and let $Y$ with $n \ge \phi$ columns chosen at random. If $K = GF(q)$ then $Prob\{dim \langle Y \rangle = \bar{\nu} \} \ge
\Phi(\pi_A,\phi)$.\\

\noindent\textit{Proposition 8.2} Let $A$ be a $N \times N$ matrix over $K$ with minimal polynomial $\pi_A(\lambda)$,
and let $X$ and $Y$ be chosen at random with $m$ rows and $n$ columns. If $m \ge min\{\phi,n\}$, then $D_Y(\lambda) = D_W(\lambda)$
with probability no less than $\Phi(\pi_A,min\{\phi,n\})$.\\

\noindent\textit{2.6 of KaVi04:} $rank(Hk_{e,d}) = deg(det(F_X^{A,Y}))$ for $d \ge deg(F_X^{A,Y})$ and $e \ge n$\\

\noindent\textit{2.7 of KaVi04:} $\nu = max\{ rank(Hk_{e,d}(A,X,Y))\}$ over all possible $e,d,X,Y$. Moreover, $\nu$ is equal to the sum of the
degrees of the first $M$ invariant factors of $\lambda I -A$ (where $M$ is the size of $X,Y$)\\

\noindent\textit{2.12 of KaVi04:} Let $s_i,\dots, s_\phi$ be all the invariant factors of $\lambda I - A$. This $i^{th}$ invariant factor of
$F_X^{A,Y}$ divides $s_i$. Furthermore, there exist matrices $W,Z$ st $\forall i$, $1 \le i \le min(M,\phi)$, the $i^{th}$ invariant factor of
$F_W^{A,Z}$ is equal to $s_i$ (all other remaining ones are equal to 1).\\

\noindent\textbf{Proposition 1:} If we choose random matrices $X,Y$, then $deg(det(F_X^{A,Y})) = \bar{\nu}$ 
with high probability.\\

\noindent\textbf{Proof:} Directly follows from the theorems 8.1 and 8.2 and proposition 6.1\\

\noindent\textbf{Proposition 2:} For random choice of blocking matrices $W,Z$, the $i^{th}$ invariant factor of $F_W^{A,Z}$ is equal to the $i^{th}$ 
invariant factor of $A$ for $1 \le i \le min(M,\phi)$ (with all other remaining factors equal to 1) with high probability.\\

\noindent\textbf{Proof:} 
We choose $\mathbb{X},\mathbb{Y}$ as the specialization of $\bar{X},\bar{Y}$ given in (Villard, corollary 6.4). From equation
(2.17) of theorem 2.12, we have that
$$ deg(det(F_{\mathbb{X}}^{A,\mathbb{Y}})) = \bar{\nu} = max_{X,Y}(deg(det(F_X^{A,Y}))) = \nu $$
Thus, by proposition 1, for any random matrices $W,Z$, with high probability,
$$ deg(det(F_W^{A,Z})) = \nu$$
Now, assume $deg(det(F_W^{A,Z})) = \nu$ and 
let $\bar{s_i}$ be the $i^{th}$ invariant factor $F_W^{A,Z}$. Then, by the first assertion of theorem 2.12, $\bar{s_i}$ divides $s_i$. Since
$\nu$ is the sum of the degrees of the invariant factors of $\lambda I - A$, this can only happen if $s_i = \bar{s_i}$ by the same reasoning as the end of theorem 2.12.

\end{document}
