\documentclass[12pt]{article}
\usepackage{amsmath,amsthm,amssymb,epsfig,color,xspace,mathrsfs}
\pagestyle{empty}


\thispagestyle{empty}

\begin{document}	
\noindent\textbf{Relevant Theorems and misc:}\\
\noindent\textit{Proposition 6.1:} The minimal generating polynomial $D_{\bar{Y}}$ for the generic
sequence $\{A^i\bar{Y}\}$ has determinantal degree $\bar{\nu}$.\\

\noindent\textit{Theorem 8.1 (Villard):} Let $A$ be a $N \times N$ matrix over $K$ with minimal polynomial $\pi_A(\lambda)$,
and let $Y$ with $n \ge \phi$ columns chosen at random. If $K = GF(q)$ then $Prob\{dim \langle Y \rangle = \bar{\nu} \} \ge
\Phi(\pi_A,\phi)$.\\

\noindent\textit{Proposition 8.2} Let $A$ be a $N \times N$ matrix over $K$ with minimal polynomial $\pi_A(\lambda)$,
and let $X$ and $Y$ be chosen at random with $m$ rows and $n$ columns. If $m \ge min\{\phi,n\}$, then $D_Y(\lambda) = D_W(\lambda)$
with probability no less than $\Phi(\pi_A,min\{\phi,n\})$.\\

\noindent\textit{2.6 of KaVi04:} $rank(Hk_{e,d}) = deg(det(F_X^{A,Y}))$ for $d \ge deg(F_X^{A,Y})$ and $e \ge n$\\

\noindent\textit{2.7 of KaVi04:} $\nu = max\{ rank(Hk_{e,d}(A,X,Y))\}$ over all possible $e,d,X,Y$. Moreover, $\nu$ is equal to the sum of the
degrees of the first $M$ invariant factors of $\lambda I -A$ (where $M$ is the size of $X,Y$)\\

\noindent\textit{2.12 of KaVi04:} Let $s_i,\dots, s_\phi$ be all the invariant factors of $\lambda I - A$. This $i^{th}$ invariant factor of
$F_X^{A,Y}$ divides $s_i$. Furthermore, there exist matrices $W,Z$ st $\forall i$, $1 \le i \le min(M,\phi)$, the $i^{th}$ invariant factor of
$F_W^{A,Z}$ is equal to $s_i$ (all other remaining ones are equal to 1).\\

\noindent We will now prove that if we choose two matrices $X,Y \in \mathbb{K}^{D\times M}$ generically 
(generic in the Schwartz/Zippel sense) and $s_i$ and $\bar{s_i}$ are $i^{th}$ invariant factor of 
$\lambda I - A$ and $F_X^{A,Y}$ respectively, then for $1 \le i \le M$,
$s_i = \bar{s_i}$.

\noindent First, by the definition of invariants factors, for any choice of $X,Y$
\begin{align*}
	dim(span(X,A^{tr}X,(A^{tr})^2X, (A^{tr})^3X,\cdots)) &\le \sum_{i=1}^{M} deg(s_i)\\
	dim(span(Y,AY,A^2Y,A^3Y,\dots)) &\le \sum_{i=1}^{M} deg(s_i)
\end{align*}
(Note: equality is possible for both inequalities). Then define two block Hankel matrices (with as many
rows and columns as it takes to maximize the rank)
$$ H_Y =
\begin{bmatrix}
I \\ A \\ A^2 \\ A^3 \\ \vdots
\end{bmatrix}
\begin{bmatrix}
Y & AY &A^2Y & \cdots
\end{bmatrix}
$$
$$ H_{X,Y} =
\begin{bmatrix}
X^{tr} \\ X^{tr}A \\ X^{tr} A^2 \\ X^{tr} A^3 \\ \vdots
\end{bmatrix}
\begin{bmatrix}
Y & AY & A^2Y & A^3Y& \cdots
\end{bmatrix}
$$
Now,
\begin{align*}
rank(H_Y) &= dim(span(Y,AY,A^2Y,A^3Y,\dots))\\ \intertext{and}
rank(H_{X,Y}) &= dim(span(X^{tr}Y,X^{tr}AY,X^{tr}A^2Y,\cdots)) \le rank(H_Y)
\end{align*} 
Let $W,Z$ be generic choices for $X,Y$ respectively.
A generic choice of $Y$ maximizes $dim(span(Y,AY,A^2Y,A^3Y,\dots))$; thus, we get
$$rank(H_Z) = dim(span(Z,AZ,A^2Z,A^3Z,\dots)) = \sum_{i = 1}^{M} deg(s_i)$$
A generic choice of $X$ makes $F_X^{A,Y} = F^{A,Y}$ and by (2.6), 
$rank(H_{W,Z}) = deg(det(F_W^{A,Z})) = deg(det(F^{A,Z})) = rank(H_Z)$. Therefore,
$$deg(det(F_W^{A,Z})) = \sum_{i=1}^{M} deg(s_i)$$
Since $\bar{s_i}$ divides $s_i$, in order to have $deg(det(F_W^{A,Z})) = \sum_{i=1}^{M} deg(s_i)$,
it must be the case that $\bar{s_i} = s_i$ for $1 \le i \le M$ as needed.

\end{document}



























