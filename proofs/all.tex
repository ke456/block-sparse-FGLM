\documentclass[12pt]{article}

\usepackage{bbm,fullpage}
\usepackage{amsmath}
\usepackage{alltt, amssymb, amsthm}
\usepackage{mathrsfs}
\usepackage{bm}


\def\C {\ensuremath{\mathbb{C}}}
\def\Q {\ensuremath{\mathbb{Q}}}
\def\N {\ensuremath{\mathbb{N}}}
\def\R {\ensuremath{\mathbb{R}}}
\def\Z {\ensuremath{\mathbb{Z}}}
\def\F {\ensuremath{\mathbb{F}}}
\def\H {\ensuremath{\mathbb{H}}}
\def\K {\ensuremath{\mathbb{K}}}
\def\Kbar {{\ensuremath{\overline{\mathbb{K}}}}}
\def\A {\ensuremath{\mathbb{A}}}
\def\D {\ensuremath{D}}
\def\m {\ensuremath{\mathfrak{m}}}
\def\todo#1{(\textbf{todo:} #1)}

\def\scrM {\ensuremath{\mathscr{M}}}
\def\calL {\ensuremath{\mathcal{L}}}
\def\scrP {\ensuremath{\mathscr{P}}}
\def\scrS {\ensuremath{\mathscr{S}}}
\def\ann {\ensuremath{\mathrm{ann}}}
\def\rk {\ensuremath{\mathrm{rk}}}

\DeclareBoldMathCommand{\bell}{\ell}
\DeclareBoldMathCommand{\bu}{u}
\DeclareBoldMathCommand{\bv}{v}
\DeclareBoldMathCommand{\bX}{X}
\DeclareBoldMathCommand{\bx}{x}
\DeclareBoldMathCommand{\balpha}{\alpha}
\DeclareBoldMathCommand{\bbeta}{\beta}

\newtheorem*{Example}{Example}
\newtheorem{Coro}{Corollary}
\newtheorem{Def}{Definition}
\newtheorem{Theo}{Theorem}
\newtheorem{Prop}{Proposition}
\newtheorem{Lemma}{Lemma}

\title{Some generating series formulas}
\date{}

\begin{document}

\maketitle

In what follows, $\K$ is a field.  Let $I$ be an ideal in
$\K[X_1,\dots,X_n]$ and $Q=\K[X_1,\dots,X_n]/I$ be the associated
residue class ring. Suppose that $V=V(I)$ has dimension zero, and
write it $V=\{\balpha_1,\dots,\balpha_d\},$ with all $\balpha_i$'s in
$\Kbar^n$, and $\balpha_i=(\alpha_{i,1},\dots,\alpha_{i,n})$ for all
$i$.  We also let $\D$ be the dimension of $Q$, so that $d \le \D$,
and {\em we assume that ${\rm char}(\K)$ is greater than $D$}. In this
section, we recall and generalize results from the appendix
of~\cite{BoSaSc03}.

For $i$ in $\{1,\dots,d\}$, let $Q_i$ be the local algebra at
$\balpha_i$, that is $Q_i=\Kbar[X_1,\dots,X_n]/I_i$, with $I_i$ is the
$\m_{\balpha_i}$-primary component of $I$. By the Chinese Remainder
Theorem, $Q\otimes_\K \Kbar=\Kbar[X_1,\dots,X_n]/I$ is isomorphic to
the direct product $Q_1\times \cdots \times Q_d$.  We let $N_i$ be the
{\em nil-index} of $Q_i$, that is, the maximal integer $N$ such that
$\m_{\alpha_i}^N$ is not contained in $I_i$; for instance, $N_i=0$ if
and only if $Q_i$ is a field, if and only if $\alpha_i$ is a
non-singular root of $I$. We also let
$\D_i=\dim_\Kbar(Q_i) \ge N_i$, so that $\D=\D_1 + \cdots + \D_d$.


%% \begin{Lemma}
%%   For a generic choice of $t_0,\dots,t_n$, the minimal polynomial of
%%   $t= t_0+ t_1 X_1 + \cdots + t_n X_n$ has degree $(N_1+1) + \cdots + (N_d+1)$.
%% \end{Lemma}
%% \begin{proof}
%%   For $i$ in $\{1,\dots,d\}$, minimal polynomial of $t$ in $Q_i$ is of
%%   the form $(T-t(\alpha_i))^{m_i}$, and we have to prove that
%%   $m_i=N_i$ for a generic choice of $t_0,\dots,t_n$.

%%   Since $t-t(\alpha_i)=t_1 X_1 + \cdots + t_n X_n$ is in
%%   $\m_{\alpha_i}$, its $(N_i+1)$-th power vanishes and thus $m_i \le
%%   N_i$. Conversely, consider the $N_i$th power of $t-t(\alpha_i)$:
%%   write $t-t(\alpha_i)$ as $t_1(X_1-\alpha_1) + \cdots +
%%   t_n(X_n-\alpha_n)$, and let all monomials of degree $N_i$ be
%%   $\mu_1,\dots,\mu_S$. Then,  $(t-t(\alpha_i))^{N_i}$
%%   can be written as 
%% $$(t-t(\alpha_i))^{N_i}=\sum_{1 \le j \le S} c_j\,
%%   \mu_j(t_1,\dots,t_n)\, \mu_j(X_1-\alpha_1,\dots,X_n-\alpha_n),$$ for
%%   some constants $c_j$ that are non-zero due to our assumption
%%   on the characteristic of $\K$.

%%   We fix a monomial basis of $Q_i$, and we let $\mu'_j$ be coefficient
%%   vector of the normal form of
%%   $\mu_j(X_1-\alpha_1,\dots,X_n-\alpha_n)$ on this basis, for
%%   $j=1,\dots,S$. Hence, the coefficients of the normal form of
%%   $(t-t(\alpha_i))^{N_i}$ are the linear combinations $\sum_{1
%%     \le j \le S} c_j \mu_j(t_1,\dots,t_n) \mu'_j$. By assumption, at
%%   least one of the $\mu'_j$ is non-zero. Since none of the $c_j$'s
%%   vanishes, the above linear combination is non-zero for a generic
%%   choice of $t_1,\dots,t_n$.
%% \end{proof}

\paragraph{Basic facts on linearly recurrent sequences.} Consider a sequence $(\ell_i)_{i \ge 0} \in \K^\N$
and the associated generating series $S=\sum_{i \ge 0} \ell_i T^i \in
\K[[T]]$. The sequence $(\ell_i)$ is {\rm linearly recurrent} if and
only if its generating series is {\em rational}m that is, if there
exist polynomials $N,D$ in $\K[T]$ such that $S=N/D$; these
polynomials are unique if we assume $\gcd(N,D)=1$ and $D(0)=1$. When
this is the case, given a degree bound $\delta$ on such $N$ and $D$,
we can recover them by means of a rational reconstruction
algorithm~\cite{GaGe13}. In the same vein, we say that a degree $m$
polynomial $P\in\K[T]$ {\em cancels} a sequence $\ell$ if $p_0 \ell_i
+ \cdots + p_m \ell_{i+d}=0$ for all $i \ge 0$, where $p_0,\dots,p_m$
are the coefficients of $P$; this is equivalent to ${\rm rev}(P) S$
being a polynomial, with $S=\sum_{i \ge 0} \ell_i T^i$ and ${\rm
  rev}(P)=T^m P(1/T)$.

%% Given such $N$ and $D$, we define $\tilde D
%% = T^{\max(\deg(N)+1,\deg(D))}D(1/T)$.  By construction, $\tilde D$
%% lies in $\K[T]$, and one can check that it is indeed the (monic) minimal
%% polynomial of sequence $(\ell_i)_{i \ge 0}$.

Let $\ell^{(1)},\dots,\ell^{(s)}$ be sequences as above, and suppose
that $P_1,\dots,P_s$ are polynomials that respectively cancel these
sequences, all $P_i$'s being pairwise coprime. Then, bla bla bla.


\paragraph{Structure of the dual.}
Fix $i$ in $1,\dots,d$.  There exists a basis of the dual ${\rm
  hom}_\Kbar(Q_i,\Kbar)$ consisting of linear forms
$(\lambda_{i,j})_{1\le j \le \D_i}$ of the form
$$\lambda_{i,j}: f \mapsto (\Lambda_{i,j}(f))(\balpha_i),$$
where $\Lambda_{i,j}$ is the operator
$$f \mapsto \Lambda_{i,j}(f) = \sum_{\mu=(\mu_1,\dots,\mu_n) \in
  S_{i,j}} c_{i,j,\mu} \frac{ \partial^{\mu_1 + \cdots + \mu_n} f}
{\partial X_1^{\mu_1} \cdots \partial X_n^{\mu_n}},$$ for some finite
subset $S_{i,j}$ of $\N^n$ and constants $c_{i,j,\mu}$ in
$\Kbar$. 
%% Let $w_{i,j}$ be the maximum of all $|\mu|$ for $\mu$ in
%% $S_{i,j}$, with $|\mu| = \mu_1 +\cdots + \mu_n $.
%% By~\cite[Lemma~3.3]{Mourrain97}, we have that $\max_j w_{i,j} =N_i$
%% for all $i$.
For instance, when $\balpha_i$ is non-singular, there is (up to a
scalar multiple) only one function $\lambda_{i,j}$, say
$\lambda_{i,1}$, and it takes the form $\lambda_{i,1}(f) =
f(\balpha_i)$. 

More generally, we will always take $\lambda_{i,1}$ of the form
$\lambda_{i,1}(f) = f(\balpha_i)$, and for $j>1$, and
$\mu=(0,\dots,0)$, we set $c_{i,j,\mu}=0$ (so all terms in
$\Lambda_{i,j}$ have order $1$ or more).  Thus, introducing 
parameters $\ell_{i,1}$ and $\bell_i=(\ell_{i,m})_{m
  =2,\dots,D_i}$, we deduce the existence of homogeneous linear forms
$P_{i,\mu}$ such that any $\lambda$ in ${\rm hom}_\Kbar(Q_i,\Kbar)$
can be written as 
$$\lambda:f \mapsto (\Lambda(f))(\balpha_i),\quad\text{with}\quad
\Lambda(f) = \sum_{\mu=(\mu_1,\dots,\mu_n) \in S_i} c_{\mu}
 \frac{ \partial^{\mu_1 + \cdots + \mu_n} f}
{\partial X_1^{\mu_1} \cdots \partial X_n^{\mu_n}},
$$ where $S_i$ is the union of $S_{i,1},\dots,S_{i,D_i}$, and where
the coefficients $c_\mu$ can be descibed in parametric manner as
$c_{(0,\dots,0)}=\ell_{i,1}$ and $c_{\mu} = P_{i,\mu}(\bell_i)$ for all
$\mu$ in $S_i$, $\mu \ne (0,\dots,0)$.

If $\lambda$ is non-zero, we can then define its {\em order} $\omega$
and {\em symbol} $\pi$. The former is the maximum of all
$|\mu|=\mu_1+\cdots+\mu_n$ for $\mu=(\mu_1,\dots,\mu_n)$ in $S_i$ 
such that $c_\mu$ is non-zero;
by~\cite[Lemma~3.3]{Mourrain97} we have $\omega \le N_i-1$. Then, we let
$$\pi =\sum_{\mu \in S_i,\ |\mu|=\omega} c_{\mu} X_1^{\mu_1} \cdots
X_n^{\mu_n}$$ be the {\em symbol} of $\lambda$; by construction,
this is a non-zero polynomial.  

Finally, we say a word about global objects.  Fix a linear form $\lambda:
Q \to \K$. By the Chinese Remainder Theorem, there exist unique
$\lambda^{(1)},\dots,\lambda^{(d)}$, with $\lambda^{(i)}$ in ${\rm
  hom}_\Kbar(Q_i,\Kbar)$ for all $i$, such that the extension
$\lambda_\Kbar: Q\otimes_\K \Kbar \to \Kbar$ decomposes as $\lambda_\Kbar =
\lambda^{(1)} + \cdots + \lambda^{(d)}$. We call {\em support} of $\lambda$ the
subset $\mathfrak{S}$ of $\{1,\dots,d\}$ such that $\lambda^{(i)}$ is
non-zero exactly for $i$ in $\mathfrak{S}$.  As a consequence, for all
$f$ in $Q$, we have
\begin{align}\label{eq:fui}
\lambda(f) &= \lambda^{(1)}(f) + \cdots + \lambda^{(d)}(f)\nonumber\\
&=  \sum_{i \in \mathfrak{S}} \lambda^{(i)}(f).
\end{align}

\paragraph{Using a generic separating element.} We now show
how to compute a parametrization of $V(I)$ through the introduction
of a generic combination of variables $t=t_1 X_1 + \cdots +t_n X_n$.

\begin{Lemma}\label{lemma:formula}
  Let $\ell$ be in ${\rm hom}_\K(Q,\K)$, suppose that $\ell$ is supported on
  some subset 
  $\mathfrak{S}$ of $\{1,\dots,d\}$ and let $\{\pi_i \mid i \in \mathfrak{S}\}$
  and  $\{w_i \mid i \in \mathfrak{S}\}$
  be as above. 

  Let $t=t_1 X_1 + \cdots +t_n X_n$, for some $t_1,\dots,t_n$ in $\K$
  and let $v$ be in $\K[X_1,\dots,X_n]$. Then, we have the equality
  \begin{align}\label{eq:sumgenseries}
  \sum_{\ell \ge 0} \ell(v t^i)T^\ell =
\sum_{i \in \mathfrak{S}} \frac{
  v(\balpha_i)\, w_i!\, \pi_{i}(t_1,\dots,t_n)
  T^{w_i} + (1-t(\balpha_i)    T)A_i}
  {(1-t(\balpha_i) T)^{w_{i}+1}},    
  \end{align}
 for some polynomials $A_1,\dots,A_d$ in
  $\Kbar[T]$, with $A_i$ of degree less than $w_i$ for all $i$
  in $\mathfrak{S}$.
\end{Lemma}
\begin{proof}
  Take $v$ and $t$ as above. Consider first
  an operator of the form $f \mapsto \frac{ \partial^{\mu_1+\cdots+\mu_n}  f}
  {\partial X_1^{\mu_1} \cdots \partial X_n^{\mu_n}}$. Then, we have
  the following generating series identities, with coefficients in 
  $\K(X_1,\dots,X_n)$:
\begin{align*}
\sum_{\ell \ge 0} 
\frac{ \partial^{|\mu|} ( v t^\ell )} {\partial X_1^{\mu_1} \cdots
  \partial X_n^{\mu_n}}
 T^\ell 
&=  \sum_{\ell \ge 0} 
\frac{ \partial^{|\mu|} (v t^\ell T^\ell)} {\partial X_1^{\mu_1} \cdots
  \partial X_n^{\mu_n}}\\
&=  
\frac{ \partial^{|\mu|} } {\partial X_1^{\mu_1} \cdots
  \partial X_n^{\mu_n}}
 \left (\sum_{\ell \ge 0} v t^\ell T^\ell\right ) \\
&= \frac{ \partial^{|\mu|} } {\partial X_1^{\mu_1} \cdots
  \partial X_n^{\mu_n}}
 \left (\frac v{1-tT} \right ) \\
&= v\, |\mu|!\, \prod_{1 \le k \le n} 
\left (\frac{ \partial t} {\partial X_k} \right)^{\mu_k}
\frac {T^{|\mu|}}{(1-t T)^{|\mu|+1}} + \frac{P_{|\mu|}(\bX,T)}{(1-t T)^{|\mu|}} + \cdots + \frac{P_{1}(\bX,T)}{(1-t T)}\\
&= v\, |\mu|!\, \prod_{1 \le k \le n} 
t_k^{\mu_k}
\frac {T^{|\mu|}}{(1-t T)^{|\mu|+1}} + \frac{P(\bX,T)}{(1-t T)^{|\mu|}},
\end{align*}
for some polynomials $P_1,\dots,P_{|\mu|},P$ in $\K[\bX,T]$ that
depend on the choices of $\mu$, $v$ and $t$, with $\deg(P_i,T) < i$
for all $i$ and thus $\deg(P,T) < |\mu|$.

Take now a linear combination of such operators, such as 
$f \mapsto \sum_{\mu \in R} c_\mu \frac{ \partial^{\mu_1 +\cdots + \mu_n}  f } {\partial X_1^{\mu_1} \cdots
  \partial X_n^{\mu_n}}$. The corresponding generating series
becomes
\begin{align*}
\sum_{\ell \ge 0} 
\sum_{\mu \in R} c_\mu \frac{ \partial^{|\mu|} ( v t^\ell )} {\partial X_1^{\mu_1} \cdots
  \partial X_n^{\mu_n}}
 T^\ell 
&=
v\,\sum_{\mu \in R} c_\mu
 |\mu|!\, \prod_{1 \le k \le n} 
t_k^{\mu_k}
\frac {T^{|\mu|}}{(1-t T)^{|\mu|+1}} +\sum_{\mu \in R} \frac{P_\mu(\bX,T)}{(1-t T)^{|\mu|}},
\end{align*}
where each $P_\mu$ has degree less than $|\mu|$ in $T$.
Let $w$ be the maximum of all $|\mu|$ for $\mu$ in $R$. We can rewrite 
the above as
\begin{align*}
v\, w! 
\sum_{\mu \in R, |\mu|=w} c_\mu
\, \prod_{1 \le k \le n} 
t_k^{\mu_k}
\frac {T^{w}}{(1-t T)^{w+1}}
 + \frac{A(\bX,T)}{(1-t T)^{w}},
\end{align*}
for some polynomial $A$ of degre less than $w$ in $T$. If we let 
$\pi =\sum_{\mu \in R,\ |\mu|=w} c_{\mu} X_1^{\mu_1} \cdots
  X_n^{\mu_n}$, this becomes
\begin{align*}
\sum_{\ell \ge 0} 
\sum_{\mu \in R} c_\mu \frac{ \partial^{|\mu|} ( v t^\ell )} { X_1^{\mu_1} \cdots
 X_n^{\mu_n}}
 T^\ell 
&=
v\, w! \,  \pi(t_1,\dots,t_n)
\frac {T^{w}}{(1-t T)^{w+1}}
 + \frac{A(\bX,T)}{(1-t T)^{w}}.
\end{align*}
Applying this formula to the sum in~\eqref{eq:fui}, we obtain the
claim in the lemma.
\end{proof}

\noindent Let us suppose that
\begin{itemize}
\item for all $i$ in $\mathfrak{S}$,  $ \pi_i(t_1,\dots,t_n)$ is nonzero;
\item $t$ is a separating element for $ \{\balpha_i \mid i \in \mathfrak{S}\}$.
\end{itemize}
Take $v=1$ in the previous lemma, and
let us rewrite the sum in~\eqref{eq:sumgenseries} as $A/B$, with
\begin{align*}
  A&=\sum_{i \in \mathfrak{S}} \Big(\big[
 w_i!\, \pi_{i}(t_1,\dots,t_n)
  T^{w_i} + (1-t(\balpha_i)    T)A_i\big]
\prod_{j \in \mathfrak{S}-\{i\}}(1-t(\balpha_j) T)^{w_{j}+1}\Big)
\\
B&=\prod_{i \in \mathfrak{S}}(1-t(\balpha_i) T)^{w_{i}+1}.
\end{align*}
We claim that $A$ and $B$ are coprime. Indeed, any root of $B$ is of
the form $1/t(\balpha_i)$ for $i$ in $\mathfrak{S}$; we claim that all
values $A(1/t(\balpha_i))$ are nonzero. Indeed, for $i' \ne i$, the
term $\prod_{j \in \mathfrak{S}-\{i'\}}(1-t(\balpha_j) T)^{w_{j}+1}$
vanishes at $T=1/t(\balpha_i)$, whereas our two assumptions
respectively imply that $ w_i!\, \pi_{i}(t_1,\dots,t_n)
T^{w_i} + (1-t(\balpha_i) T)A_i$ and $\prod_{j \in
  \mathfrak{S}-\{i\}}(1-t(\balpha_j) T)^{w_{j}+1}$
are nonzero at  $T=1/t(\balpha_i)$, so that $A(1/t(\balpha_i))$ is nonzero.

As a result, given $2D$ terms of the sequence $\ell(t^i)$, we can
reconstruct $A$ and $B$ as above. We claim that we can actually
recover the polynomial $\tilde B = \prod_{i \in
  \mathfrak{S}}(T-t(\balpha_i) )^{w_{i}+1}$.  Remark that $\tilde B$
may not agree with the reversed polynomial ${\rm rev}(B)=T^{\deg(B)}
B(1/T) = \prod_{i \in \mathfrak{S}, t(\balpha_i)\ne 0}(T-t(\balpha_i)
)^{w_{i}+1}$; knowing $B$ (and thus ${\rm rev}(B)$), we need to
determine whether there exists $i_0$ in $\mathfrak{S}$ such that
$t(\balpha_{i_0})=0$, and if so, the corresponding exponent $w_{i_0}$.
\begin{itemize}
\item Suppose first that all values $t(\balpha_i)$ are nonzero.  Then,
  $B$ has degree $\sum_{i \in \mathfrak{S}} (w_i+1)$, so that we have
  the inequality $\deg(B) > \deg(A)$.
\item Suppose now that $t(\balpha_{i_0})=0$, for some $i_0$ in
  $\mathfrak{S}$. Then, $B$ has degree $\sum_{i \in
  \mathfrak{S}-\{i_0\}} (w_i+1)$; let us verify that we have $\deg(A)
  \ge \deg(B)$ in this case.  For $i$ in $ \mathfrak{S}-\{i_0\}$, the
  term $\big[ w_i!\, \pi_{i}(t_1,\dots,t_n) T^{w_i} +
    (1-t(\balpha_i) T)A_i\big] \prod_{j \in
    \mathfrak{S}-\{i\}}(1-t(\balpha_j) T)^{w_{j}+1}$ has degree less
  than $\deg(B)$, whereas that same term for $i=i_0$ has degree
  $\sum_{i \in \mathfrak{S}} (w_i+1)-1 \ge \deg(B)$.

  Thus, $A$ itself has degree greater than (or equal to) $\deg(B)$,
  and the difference $\deg(A)-\deg(B)$ is precisely $w_{i_0}$.
\end{itemize}
In particular, in either case, we can determine the integer $\delta=\sum_{i \in \mathfrak{S}} (w_i+1)-1$,
which is an upper bound on the degree of $A$.
This allows us to define $\tilde A = T^{\delta}A(1/T)$, that is,
$$\tilde A = 
\sum_{i \in \mathfrak{S}} \Big(\big[
 w_i!\, \pi_{i}(t_1,\dots,t_n) + (T-t(\balpha_i)  )\tilde A_i\big]
\prod_{j \in \mathfrak{S}-\{i\}}(T-t(\balpha_j) )^{w_{j}+1}\Big),$$
with $\tilde A_i = T^{w_i-1} A_i(1/T) \in \K[T]$. In particular, 
the value $\tilde A(t(\balpha_i))$ is 
$$\tilde A(t(\balpha_i))= w_i!\, \pi_{i}(t_1,\dots,t_n)\prod_{j \in
  \mathfrak{S}-\{i\}}(t(\balpha_i)-t(\balpha_j) )^{w_{j}+1},$$ which is
nonzero. In other words, $\tilde A$ is a unit modulo $C= \prod_{i \in
  \mathfrak{S}}(T-t(\balpha_i) )$.

Let us finally consider the formula in Lemma~\ref{lemma:formula} 
for an arbitrary $v$, still under our two assumptions above. 
The sum in~\eqref{eq:sumgenseries} can now be rewritten as $A_v/B$, with
\begin{align*}
  A_v&=\sum_{i \in \mathfrak{S}} \Big(\big[
 v(\balpha_i) w_i!\, \pi_{i}(t_1,\dots,t_n)
  T^{w_i} + (1-t(\balpha_i)    T)A_i\big]
\prod_{j \in \mathfrak{S}-\{i\}}(1-t(\balpha_j) T)^{w_{j}+1}\Big).
\end{align*}
We do not claim that $A_v$ and $B$ are necessarily coprime, but
since $B$ is known, we can recover $A_v$ using only $D$ 
terms in the sequence $\ell(v t^i)$. As above, we can deduce
$\tilde A_v
 = T^{\delta}A_v(1/T)$, that is,
$$\tilde A_v = 
\sum_{i \in \mathfrak{S}} \Big(\big[
 v(\balpha_i) w_i!\, \pi_{i}(t_1,\dots,t_n) + (T-t(\balpha_i)  )\tilde A_i\big]
\prod_{j \in \mathfrak{S}-\{i\}}(T-t(\balpha_j) )^{w_{j}+1}\Big).$$
Now, the value of $\tilde A_v$ at $t(\balpha_i)$ is 
\begin{align*}
\tilde A_v(t(\balpha_i))&=v(\balpha_i) w_i!\, \pi_{i}(t_1,\dots,t_n)\prod_{j \in
  \mathfrak{S}-\{i\}}(t(\balpha_i)-t(\balpha_j) )^{w_{j}+1}\\
&= v(\balpha_i) \tilde A(t(\balpha_i)).
\end{align*}
Thus, the polynomial 
$$P_v = \frac{\tilde A_v}{\tilde A} \bmod C$$
is well-defined, and $P_v(t(\balpha_i))=v(\balpha_i)$ holds for all $i$ in $\mathfrak{S}$.



\paragraph{Using $X_1$.}
In this paragraph, we discuss how to avoid (as much as possible) using
a generic linear form $t=t_1 X_1 + \cdots + t_n X_n$, and how to use
(say) $X_1$ instead; this is motivated by the fact that the
multiplication matrix by $X_1$ is sparser than that of $t$ (since the
matrix of $t$ is a combination of those of $X_1,\dots,X_n$), sometimes
by a substantial amount. Of course, there is no guarantee that $X_1$
is a separating element for $V$. As a result, we will compute a
decomposition of $V$ into two components $V'$ and $V''$; $X_1$
will be a separating element for $V'$, whereas we will use a generic
linear form to describe $V''$.

We characterize the set $V'$ mentioned above as follows: for $i$ in
$\{1,\dots,d\}$, $\balpha_i$ is in $V'$ if and only if:
\begin{itemize}
\item for $i'$ in $\{1,\dots,d\}$, with $i'\ne i$, $\alpha_{i',1} \ne \alpha_{i,1}$;
\item $Q_i$ is a reduced algebra (equivalently, $I_i$ is radical).
\end{itemize}
We denote by $\mathfrak{S}'\subset \{1,\dots,d\}$ the set of corresponding indices
$i$, and we let $\mathfrak{S}''=\{1,\dots,d\}-\mathfrak{S}'$. We can 
then define $V''$ as $V''=\cup_{i \in \mathfrak{S}''} V(Q_i)$, so that 
$V$ is indeed the disjoint union of $V'$ and $V''$; remark that $X_1$ is a
separating element for $V'$.

The projection $\varphi: V \to \Kbar$ defined by
$\varphi(\balpha_i)=\alpha_{i,1}$ plays a special role in this
construction.  We thus let $r_1,\dots,r_c$ be the pairwise distinct
values taken by $X_1$ on $V$, for some $c \le d$.  For
$j=1,\dots,c$, we write $T_j$ for the set of all indices $i$ in
$\{1,\dots,d\}$ such that $\varphi(\balpha_{i})=r_j$; the sets
$T_1,\dots,T_c$ form a partition of $\{1,\dots,d\}$. When $T_j$ has
cardinality~$1$, we denote it as $T_j=\{\sigma_j\}$, for some index
$\sigma_j$ in $\{1,\dots,d\}$, and the preimage of $r_j$ by $\varphi$
will thus be $\balpha_{\sigma_j}$.


For $i=1,\dots,d$, let us write $\nu_i$ for the degree of the minimal
polynomial of $X_1$ in $\Kbar[X_1,\dots,X_n]/Q_i$.  In particular,
$\nu_i=1$ if $Q_i$ is radical, but the converse may not hold.  For $j$
in $\{1,\dots,c\}$, we define $m_j$ as the maximum of all $\nu_i$, for
$i$ in~$T_j$.  
\begin{Lemma}
  Let $j$ be in $\{1,\dots,c\}$ such that $m_j=1$, let $\ell$ be a
  linear form over $\prod_{i \in T_j} Q_i$ and let $t=t_2 X_2
  + \cdots + t_n X_n$. Define constants $a,b,c$ in $\Kbar$ by
  $$a=\ell(1),\quad b=\ell(t),\quad c=\ell(t^2).$$
  %% \begin{align*}
  %%   a&= \mathcal{N}(\ell(X_1^i), T-r_j) \\
  %%   b&= \mathcal{N}(\ell(t X_1^i), T-r_j) \\
  %%   c&= \mathcal{N}(\ell(t^2 X_1^i), T-r_j).
  %% \end{align*}
  Then, for a generic choice of $\ell$ and $t$, $j$ is in $\mathfrak{s}$
  if and only if $ac=b^2$.
\end{Lemma}
\begin{proof}
  The assumption that $m_j=1$ means that for all $i$ in $T_j$,
  $\nu_i=1$, that is, $X_1-r_j$ belongs to $Q_i$. Now, the linear 
  form $\ell$ can be uniquely written as a sum $\ell=\sum_{i \in T_j}
  \ell_i$, where each $\ell_i$ is in ${\rm hom}_\Kbar(Q_j,\Kbar)$.
  The fact that $\nu_i=1$ then implies that $\ell_i$ takes the form 
$$\ell_i: f \mapsto (\Lambda_{i}(f))(\balpha_i),$$
where $\Lambda_{i}$ is a differential operator that does not 
involve $\partial/\partial X_1$. Thus, we can write it as
$$\Lambda_i: f \mapsto \ell_{i,0} f + \sum_{2 \le r \le n}
P_{i,r}(\bell_i) \frac{\partial}{\partial X_j} f + \sum_{2 \le r \le s
  \le n} P_{r,s,k}(\bell_i) \frac{\partial^2}{\partial X_j\partial
  X_k} f + \tilde\Lambda_i(f),$$ where all terms in $\tilde \Lambda_i$
have order at least $3$, $\ell_{i,0}$ and $\bell_i=(\ell_{i,m})_{m
  =1,\dots,D_i-1}$ are indeterminates and
$(P_{i,r})_{2 \le r \le n}$ and $(P_{i,r,s})_{2 \le r \le s \le n}$
are linear forms in $\bell_i$.

We obtain
\begin{align*}
\Lambda_i(1)   &= \ell_{i,0} \\
\Lambda_i(t)   &= \ell_{i,0} t +\sum_{2 \le r \le n}P_{i,r}t_r \\
\Lambda_i(t^2) &= \ell_{i,0} t^2  +2 t \sum_{2 \le r \le n}P_{i,r}t_r   
                                                + 2\sum_{2 \le r \le s \le n} P_{i,r,s}t_rt_s,
\end{align*}
which gives
\begin{align*}
a  &= \sum_{i\in T_j}\ell_{i,0} \\
b  &= \sum_{i\in T_j}\ell_{i,0} t(\balpha_i) +\sum_{i \in T_j, 2 \le r \le n}P_{i,r}t_r \\
c &= \sum_{i\in T_j}\ell_{i,0} t(\balpha_i)^2  +2  \sum_{i \in T_j, 2 \le r \le n}P_{i,r}t_r t(\balpha_i)   
                                                +2 \sum_{i \in T_j, 2 \le r \le s \le n} P_{i,r,s}t_rt_s.
\end{align*}
Suppose first that $j$ is in $\mathfrak{s}$. Then, $T_j=\{\sigma_j\}$, so we 
have only one term $\Lambda_{\sigma_j}$ to consider, and $Q_{\sigma_j}$ 
is reduced, so that all coefficients $P_{\sigma_j,r}$ and
$P_{\sigma_j,r,s}$ vanish. Thus, we are left in
this case with
$$
a = \ell_{\sigma_j,0}, \quad
b = \ell_{\sigma_j,0} t(\balpha_{\sigma_j}), \quad
c = \ell_{\sigma_j,0} t(\balpha_{\sigma_j})^2,
$$ so that we have $ac=b^2$, for {\em any} choice of $\ell$ and
$t$. Now, we suppose that $j$ is not in $\mathfrak{s}$, and we prove
that for a generic choice of $\ell$ and $t$, $ac-b^2$ is non-zero.
The quantity $ac-b^2$ is a polynomial in the coefficients $\ell_{i,0}$
and $(\bell_i)_{i\in T_j}$, and $(t_i)_{i \in \{2,\dots,n\}}$,
and we have to show that it is not identically zero. We discuss
two cases; in both of them, we prove that a suitable specialization
of $ac-b^2$ is non-zero.

Suppose first that for at least one index $\sigma$ in $T_j$,
$Q_\sigma$ is not reduced. In this case, there exists as least one 
index $\rho$ such that
$P_{\sigma,\rho}(\bell_\sigma)$ is non-zero. Let us set all $\ell_{\sigma',0}$
and $\bell_{\sigma'}$ to $0$, for $\sigma'$ in $T_j-\{\sigma\}$, as well
as $\ell_{\sigma,0}$, and all $t_r$ for $r\ne \rho$. Then,
$ac-b^2$ becomes $-(P_{\sigma,\rho}(\bell_\sigma)t_\rho)^2$, which is 
non-zero, so that $ac-b^2$ itself must be non-zero.

Else, since $j$ is not in $\mathfrak{s}$, $T_j$ must have cardinality
at least $2$; besides, we are assuming that all $Q_\sigma$ are
reduced, for $\sigma$ in $T_j$.  Suppose that $\sigma$ and $\sigma'$
are two indices in $T_j$; we set all indices $\ell_{\sigma'',0}$ to
zero, for $\sigma''$ in $T_j-\{\sigma,\sigma'\}$. Since
all $P_{\sigma,r},P_{\sigma',r},P_{\sigma,r,s}$ and $P_{\sigma',r,s}$
vanish, we are left with
$$
a=\ell_{\sigma,0}+\ell_{\sigma',0},\quad
b=\ell_{\sigma,0}t(\balpha_{\sigma})+\ell_{\sigma',0}t(\balpha_{\sigma'}),\quad
c=\ell_{\sigma,0}t(\balpha_{\sigma})^2+\ell_{\sigma',0}t(\balpha_{\sigma'})^2.
$$
Then, $ac-b^2$ is equal to $2\ell_{\sigma,0}\ell_{\sigma',0}(t(\balpha_{\sigma})-t(\balpha_{\sigma'}))^2$,
which is non-zero, since $\balpha_\sigma \ne \balpha_{\sigma'}$.
\end{proof}



%% Take $v$ in $\K[X_2,\dots,X_n]$ and consider an operator of the form
%% $f \mapsto \frac{ \partial^{\mu_2+\cdots+\mu_n} f} {\partial
%%   X_2^{\mu_2} \cdots \partial X_n^{\mu_n}}$. Then, we have the
%% following generating series identities, with coefficients in
%% $\K(X_1,\dots,X_n)$:
%%   \begin{align*}
%%     \sum_{i \ge 0} 
%%     \frac{ \partial^{\mu_2+\cdots+\mu_n} ( v X_1^i )} {\partial X_2^{\mu_2} \cdots \partial X_n^{\mu_n}} T^i 
%%     &= \frac{ \partial^{\mu_2+\cdots+\mu_n} } {\partial X_1^{\mu_1} \cdots \partial X_n^{\mu_n}}
%%     \left (\frac v{1-X_1T} \right ) \\
%%     &=  \frac{\partial^{\mu_2+\cdots+\mu_n} v} {\partial X_2^{\mu_2} \cdots \partial X_n^{\mu_n}}
%%     \frac {1}{1-X_1 T}.
%%   \end{align*}







Let $M \in \K[T]$ be the minimal polynomial of $X_1$ in $Q$; then, we have
$$M=\prod_{j=1}^c (T-r_j)^{m_j}.$$
\begin{Lemma}
  For a generic linear form $\ell$ in ${\rm hom}_\K(Q,\K)$, the minimal
  polynomial of the sequence $(\ell(X_1^i))_{i \ge 0}$ is the minimal
  polynomial $M$ of $X_1$ in $Q$.
\end{Lemma}

Up to reindexing, we can assume that $c' \in \{0,\dots,c\}$ is such
that for $j$ in $\{1,\dots,c\}$, $m_j=1$ if and only if $j$ is in
$\{1,\dots,c'\}$.

We let $c'' \in \{0,\dots,c'\}$
be such that for $j$ in $\{1,\dots,c'\}$, $T_j$ has cardinality $1$ if
and only if $j$ is in $\{1,\dots,c''\}$. Thus, for $j$ in
$\{1,\dots,c''\}$, there is a unique point $\balpha_{\sigma_j}$ in $V$ such
that $\varphi(\balpha_{\sigma_j})=r_j$ and $X_1=r_j$ holds in 
the corresponding local algebra $Q_{\sigma_j}$.

The following lemma is similar in spirit to Lemma~\ref{lemma:formula};
the proof is however slightly different, taking into account the
specific form of the sequence we consider.
%% \begin{Lemma} Let $\ell$ be  in ${\rm hom}_\K(Q,\K)$
%%   and  $v$ in $\K[X_2,\dots,X_n]$.
%%   Then, there exists constants $(\eta_{j})_{j\in\{1,\dots,c''\}}$
%%   and
%%   $(\eta_{j,i})_{j\in\{c''+1,\dots,c'\},i\in T_j}$, that depend 
%%   only $\ell$, 
%%   and polynomials  $N_{c'+1},\dots,N_c$, that depend on
%%   $\ell$ and $v$, such that we  have
%%   $$\sum_{i \ge 0} \ell(v X_1^i)T^i = \sum_{j=1}^{c''} \frac{ \eta_{j}v(\balpha_i)}{1-r_j T}
%% +\sum_{j=1}^{c'} \frac{\sum_{i \in T_j} \eta_{j,i}v(\balpha_i)}{1-r_j T}
%%   +\sum_{j=c'+1}^{c} \frac{N_j(T)}{(1-r_j T)^{m_j}}.$$
%%   In addition,  
%%   $N_j$ has degree less than $m_j$ and is coprime with $1-r_j T$
%%   for all $j$ i $\{c'+1,\dots,c\}$.
%% \end{Lemma}

, and let $\tilde M$ 
be its reverse, that is, $\tilde M=\prod_{j=1}^c (1-r_jT)^{m_j}$.
Knowing the latter polynomial allows us to 
determine $A=\prod_{j=1}^{c'} (1-r_jT)$, as well as the 
numerator






 In particular, the minimum
polynomial of $X_1$ modulo $\cap_{i \in T_j} Q_i$ is $(T-r_j)^{m_j}$,
and the minimal polynomial of $X_1$ modulo $I=\cap_{i=1,\dots,d} Q_i$
is $\prod_{j=1,\dots,c} (T-c_j)^{m_j}$.

 For $v$ in $\K[X_2,\dots,X_n]$, we have
$$\sum_{i \ge 0} u(v X_1^i)T^i = \sum_{j=1}^{c'} \frac{\sum_{i \in T_j} v(\alpha_i)}{1-r_j T}
+\sum_{j=c'+1}^{c} \frac{N_j(T)}{(1-r_j T)^{m_j}},$$










This lemma shows how to compute a description of $V(I)$. We make
the following assumptions:
\begin{itemize}
\item $u$ is chosen generically, so that the polynomials $\pi_i$,
  $i=1,\dots,d$, are all non-zero;
\item $t=t_1 X_1 + \cdots + t_n X_n$ is a {\em separating element} for
  $V(I)$, in the sense that the values $t(\alpha_i)$ are pairwise
  distinct, for $i=1,\dots,d$;
\item $\pi_i(t_1,\dots,t_n)$ is non-zero for all $i=1,\dots,d$.
\end{itemize}
As a result, we can write
$$\sum_{\ell \ge 0} u(X_i t^i)T^\ell = \frac{ }{ }$$


%%  \sum_{1 \le i \le d} \left (
%%   v(\alpha_i)\, w_i!\, \pi_{i}(t_1,\dots,t_n) \frac{T^{w_i}}
%%   {(1-t(\alpha_i) T)^{w_{i}+1}} + \frac{A_i(T)}{(1-t(\alpha_i)
%%     T)^{w_{i}}} \right ),$$ for some polynomials $A_1,\dots,A_d$ in
%%   $\Kbar[T]$, with $A_i$ of degree less than $w_i$ for all $i$.
%% \end{Lemma}





%% We choose a random linear form $\ell: Q \to \K$, and we start by
%% computing the sequences
%% $$\ell_i=\ell(X_n^i) \quad (0 \le i \le 2\D)$$
%% and 
%% $$\ell_{i,j}=\ell(X_n^i X_j) \quad (0 \le i < \D).$$ By
%% Berlekamp-Massey, we can compute the minimal polynomial $T_n$ of the
%% sequence $\ell_i$, which is also the minimal polynomial of $X_n$
%% \todo{prove}, and its squarefree part $T_n$. Using the formulas above,
%% we can deduce from the $\ell_{i,j}$ polynomials $T_1,\dots,T_{n-1}$
%% such that
%% $$X_1-T_1(X_n),\dots,X_{n-1}-T_{n-1}(X_n), T_n(X_n)$$
%% is the lexicographic basis of $\sqrt{I}$ for the 
%% order $X_1 > \cdots > X_n$. \todo{finish}


\bibliographystyle{plain}
\bibliography{all}

\end{document}




