\documentclass[12pt,english]{article}

\usepackage{fullpage,url,amsmath,amsthm,amssymb,epsfig,color,xspace,mathrsfs}
\usepackage{hyperref}
\usepackage{nopageno}

% let's take some less standard font
\usepackage{libertine}
\usepackage{libertinust1math}
\usepackage[T1]{fontenc}

\title{Amazing title}
\author{\textsc{Proposal for a Special Session} \\
  for the 24th Conference on Applications of Computer Algebra}
\date{}

\begin{document}

\maketitle

\textbf{Organizers:}
% Vincent: I wrote by alphabetical order... feel free to modify as you like.
\begin{itemize}
  \item Vincent Neiger (XLIM, University of Limoges, France) \\
    {\small \verb+vincent.neiger@unilim.fr+}
  \item Hamid Rahkooy (University of Waterloo, Ontario, Canada) \\
    {\small \verb+hamid.rahkooy@uwaterloo.ca+}
  \item \'Eric Schost (University of Waterloo, Ontario, Canada) \\
    {\small \texttt{eric.schost@uwaterloo.ca}}
\end{itemize}

\textbf{Abstract:}

TODO

\bigskip

The points are: \\
  - fast algorithms \\
  - implementations \\
  - related to polynomial system solving / zero-dimensional ideals \\
  - sparse FGLM and (multi-dimensional) linearly recurrent sequences \\
  - the focus is not on the initial Grobner basis computation with F4/F5, but rather with what we do once we have this basis (and the multiplication matrices) \\
  - deflation?

\end{document}
