\documentclass[12pt,english]{article}

\usepackage{fullpage,url,amsmath,amsthm,amssymb,epsfig,color,xspace,mathrsfs}
\usepackage{hyperref}
\usepackage{nopageno}

% let's take some less standard font
\usepackage{libertine}
\usepackage{libertinust1math}
\usepackage[T1]{fontenc}

\title{Algorithms for zero-dimensional ideals}
\author{\textsc{Proposal for a Special Session} \\
  for the 24th Conference on Applications of Computer Algebra}
\date{}

\begin{document}

\maketitle

\textbf{Organizers:}
% Vincent: I wrote by alphabetical order... feel free to modify as you like.
\begin{itemize}
  \item Vincent Neiger (XLIM, University of Limoges, France) \\
    {\small \verb+vincent.neiger@unilim.fr+}
  \item Hamid Rahkooy (University of Waterloo, Ontario, Canada) \\
    {\small \verb+hamid.rahkooy@uwaterloo.ca+}
  \item \'Eric Schost (University of Waterloo, Ontario, Canada) \\
    {\small \texttt{eric.schost@uwaterloo.ca}}
\end{itemize}

\textbf{Abstract:}

In the last decades, a lot of progress has been made on the study of efficient
algorithms related to zero-dimensional ideals, including for solving polynomial
systems, i.e. determining the finite set of roots common to a given collection
of multivariate polynomials.  During this process, it has turned out that these
algorithms heavily rely on some routines from linear algebra.  This session
will focus on the design and the implementation of algorithms specifically
tailored for the particular linear algebra problems encountered in this kind of
computations.  Applications of these techniques will also be considered, such
as algebraic cryptanalysis and decoding algorithms for algebraic geometry
codes.

Polynomial system solving often involves computing a first Gr\"obner basis,
typically with the F5 algorithm, and then working on finding a representation
of the sought roots, using for example the FGLM algorithm \cite{FaGiLaMo93}.
In the first step, one has to deal with matrices of large dimension which are
sparse and exhibit a noticeable structure.  The second step corresponds to
finding the nullspace of a matrix with a ``multi-Krylov'' structure: the matrix
is formed by some vector and its images by successive powers of the so-called
multiplication matrices.

It has been observed that these multiplication matrices are usually sparse, a
feature that one wants to exploit to obtain faster algorithms.  One solution,
followed in \cite{Steel15}, is to rely on a strategy inspired from the block
Wiedemann algorithm \cite{Coppersmith94,Villard97a} to compute the generator
for a linearly recurrent matrix sequence.  Another solution, presented in
\cite{FaMo17}, is to rely on the computation of generators for a
multi-dimensional linearly recurrent sequence.

This revived interest into the latter problem, with the goal of designing
algorithms which outperform the Berlekamp-Massey-Sakata algorithm, known for
its applications to the decoding of algebraic geometry codes.  Some approaches
have already been described \cite{BerBoyFau17,Mourrain17}, involving
computations with matrices that have a multi-layered block-Hankel structure.
This turns out to have links with the decomposition of symmetric tensors
\cite{BrCoMots10}.

This session aims at gathering the main actors behind the recent advances, and
naturally all researchers interested in this topic and its future developments.

%Invitations: Gilles Villard, Elias Tsigaridas, Jean-Charles Faugere, Jeremy
  %Berthomieu, Daniel Augot, Bernard Mourrain, Allan Steel. About
  %deflation: Hauenstein, Szanto, etc.

% YES :

% NO : Chenqi Mou,

\bibliographystyle{plain}
\bibliography{biblio.bib}

\end{document}
