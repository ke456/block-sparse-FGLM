\documentclass[12pt,english]{article}

\usepackage{fullpage,url,amsmath,amsthm,amssymb,epsfig,color,xspace,mathrsfs}
\usepackage{hyperref}
\usepackage{nopageno}

% let's take some less standard font
\usepackage{libertine}
\usepackage{libertinust1math}
\usepackage[T1]{fontenc}

\title{Amazing title}
\author{\textsc{Proposal for a Special Session} \\
  for the 24th Conference on Applications of Computer Algebra}
\date{}

\begin{document}

\maketitle

\textbf{Organizers:}
% Vincent: I wrote by alphabetical order... feel free to modify as you like.
\begin{itemize}
  \item Vincent Neiger (XLIM, University of Limoges, France) \\
    {\small \verb+vincent.neiger@unilim.fr+}
  \item Hamid Rahkooy (University of Waterloo, Ontario, Canada) \\
    {\small \verb+hamid.rahkooy@uwaterloo.ca+}
  \item \'Eric Schost (University of Waterloo, Ontario, Canada) \\
    {\small \texttt{eric.schost@uwaterloo.ca}}
\end{itemize}

\textbf{Abstract:}

In the last decades, a lot of progress has been made on the design and the
implementation of efficient algorithms related to zero-dimensional ideals,
including for solving polynomial systems, i.e. determining the finite set of
roots common to a given collection of multivariate polynomials.  During this
process, it has turned out that these algorithms heavily rely on routines from
linear algebra.  This session focuses on the study of algorithms specifically
tailored for the particular linear algebra problems encountered in this kind of
computations.

More precisely, polynomial system solving often involves computing a first
Gr\"obner basis, typically with the F5 algorithm,
% TODO citation to Faugere's paper about F5
and then using for example
the FGLM algorithm
% TODO citation to the FGLM paper
to obtain a representation of the sought roots.  In the
first step, one has to deal with matrices of large dimension which are sparse
and exhibit a noticeable structure.  The second step corresponds to finding the
nullspace of matrix with a structure which may be called multi-Krylov: the matrix
is formed by a given vector and its images by the successive powers of the
so-called multiplication matrices.

It was also observed recently
% TODO citation to Faugere Mou
that, by exploiting the sparsity of these multiplication matrices, one can
reduce the problem to the computation of generators for a multi-dimensional
linearly recurrent sequence. 
% TODO should probably say a word about Wiedemann as in our work and Steel's
% work, this would also be the opportunity to cite Gilles Villard's work
This revived interest into this problem, with the goal of designing algorithms
which outperform the Berlekamp-Massey-Sakata algorithm, well-known for its
applications to the decoding of algebraic geometry codes.  Some approaches have
already been described,
% TODO cite work of Berthomieu et al, and work of Mourrain at ISSAC 17
involving computations with matrices that have a multi-layered block-Hankel
structure.  This was noticed to have direct links with the decomposition of
symmetric tensors.
% TODO cite work of Tsigaridas - Mourrain et al

% TODO Vincent->Hamid: don't hesitate to insert something about deflation... I am not able to.

This session aims at gathering the main actors behind these recent
developments, and naturally all researchers interested in future developments
on this topic.

%Invitations: Gilles Villard, Elias Tsigaridas, Jean-Charles Faugere, Jeremy
  %Berthomieu, Daniel Augot, Bernard Mourrain, Allan Steel, Chenqi Mou. About
  %deflation: Hauenstein, Szanto, etc.
\end{document}
