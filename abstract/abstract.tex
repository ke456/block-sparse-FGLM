\documentclass[12pt]{article}

\usepackage{fullpage,url,amsmath,amsthm,amssymb,epsfig,color,xspace,sigsam,mathrsfs}
\usepackage[
pdftitle={abstract}]
{hyperref}

\title{Sparse FGLM using the block Wiedemann algorithm}
\author{Seung Gyu Hyun, Vincent Neiger, Hamid Rahkooy, \' Eric Schost}

\titlehead{Block Sparse FGLM Algorithm}
\articlehead{ISSAC 2017 poster abstract}
\authorhead{Seung Gyu Hyun, Vincent Neiger, Hamid Rahkooy, \' Eric Schost}



\begin{document}

\maketitle

\paragraph{Overview}
Computing the Gr\"obner basis of an ideal with respect to a term
ordering is an essential step in solving systems of polynomials (in
what follows, we restrict our attention to systems with finitely many
solutions.)  Certain term orderings, such as the degree reverse
lexicographical ordering (degrevlex), make the computation of the
Gr\"obner basis faster, while other orderings, such as the
lexicographical ordering (lex), make it easier to find the coordinates
of the solutions. Thus, one typically first compute a Gr\"obner basis
using a degrevlex ordering, then converts it to either a lex Gr\"obner
basis, or a related representation, such as Rouillier's Rational
Univariate Representation~\cite{Rouillier99}.

Consider a zero-dimensional ideal $I \subset \mathbb{K}[x_1, \dots,
  x_n]$, given by means of a monomial basis $\mathscr{B}$ of
$Q=\mathbb{K}[x_1, \dots, x_n]/I$, together with the multiplication
matrices $T_1,\dots, T_n \in \mathbb{K}^{D \times D}$ of respectively
$x_1,\dots,x_n$ in $Q$ in $\mathscr{B}$, with
$D=\dim_\mathbb{K}(Q)$. In all that follows, we assume that $x_n$ {\em
  separates} the points of $V(I)$. However, we do not assume that $I$
is radical, or that it is in {\em shape position} (that is, that its
lex basis for the order $x_1 > \cdots > x_n$ has the form
$(x_1-G_1(x_n),\dots,x_{n-1}-G_{n-1}(x_n),G_n(x_n))$); note that our
assumption can be ensured by a generic change of coordinates, while
the shape position one may not hold for any choice of
coordinates. Under our assumption, the {\em radical} of $I$ is in
shape position, with Gr\"obner basis
$R=(x_1-R_1(x_n),\dots,x_{n-1}-R_{n-1}(x_n),R_n(x_n))$.

The FGLM algorithm~\cite{FaGiLaMo93} computes the Gr\"obner basis of
$I$ for the lex order $x_1 > \cdots > x_n$ with a runtime cubic in the dimension
$D$. Although recent work has reduced the runtime exponent from $3$ to
the exponent of matrix multiplication
$\omega$~\cite{FaGaHuRe13,Neiger16}, linear algebra techniques based
on duality often turn out to be more efficient. The basic idea behind
these techniques is to compute values of one (or possibly several)
linear forms $\ell: Q \to \mathbb{K}$ at well-chosen elements of $Q$,
and to deduce our output by solving some well-structured linear system
(typically, Hankel).

Several forms of these ideas exist. The Rational Univariate
Representation algorithm takes for $\ell$ the trace ${\rm tr}: Q \to
\mathbb{K}$ and computes values of the form ${\rm tr}(x_n^i)$ and
${\rm tr}(x_j x_n^i)$; from there, the output of the algorithm is
equivalent to the data of $R$, together with the multiplicities of all
points. Values such as ${\rm tr}(x_n^i)$ can be computed as $v T_n^i
e_n$, where $v$ is the row vector of the traces of the monomial basis
$\mathscr{B}$, and $e_n$ is the column vector of coordinates of $x_n$ on
$\mathscr{B}$. Computing the values of the trace on $\mathscr{B}$ is
however costly, so several algorithms use random linear forms
instead. In~\cite{BoSaSc03}, the authors show how we can recover the
Gr\"obner basis of $R$ of $\sqrt{I}$ (together with the nil-indices of
all points) from the values $\ell(x_n^i)$ and $\ell(x_j x_n^i)$, for a
random $\ell$, whereas Faug\`ere and Mou show in~\cite{FaMo17} how to
compute the basis $G$ of $I$ itself using similar sets of values, when
$I$ is in shape position (in general, the algorithm falls back on
Berlekamp-Massey-Sakata techniques). Quite importantly, they also
demonstrate that exploiting the sparsity of matrices $T_1,\dots,T_n$
is critical for the efficiency of these algorithms.

The computation of sequences such as $\ell(x_n^i)$ is difficult to
parallelize.  In this work, in the continutation of~\cite{BoSaSc03},
we use techniques inspired by Coppersmith's block Wiedemann
algorithm~\cite{Coppersmith93} to compute the Gr\"obner basis $R$ of
$\sqrt{I}$.

\noindent{\bf Computing $R_{n}$.} We choose an integer $m$, random
matrices $u \in \mathbb{K}^{m \times D}$, $v \in \mathbb{K}^{D \times
  m}$, and we consider the matrix sequence $s = (uT_n^iv)_{i \ge
  0}$. For generic matrices $u, v$, Theorem 2.12 in~\cite{KaVi04}
shows that if $S \in \mathbb{K}[x]^{m\times m}$ is a minimal
generating polynomial of $(s_i)_{i \ge 0}$, its determinant is a
multiple of the minimal polynomial of $x_n$ in $Q$, and divides its
characteristic polynomial. This polynomial must then factor as
$\prod_{\alpha=(\alpha_1,\dots,\alpha_n) \in V(I)}
(x-\alpha_n)^{e_\alpha}$, for certain positive integers $e$; in
particular, its squarefree part (written in the variable $x_n$) is the last polynomial $R_n$ in $R$.

\medskip\noindent{\bf Using $u$-resultant techniques.} We propose  to recover the 
whole Gr\"obner basis $R$ of $\sqrt{I}$ by computing a truncation of
a polynomial akin to the $u$-resultant; we show here this idea for the computation
of $R_{n-1}$.

Let $\lambda$ be a new variable and consider the matrix $T_\lambda=T_n
+ \lambda T_{n-1}$; this is the multiplication matrix by $x_n +
\lambda x_{n-1}$ in $Q \otimes_{\mathbb{K}}
\mathbb{K}(\lambda)$. Applying the above algorithm to the sequence
$s_\lambda=(uT_\lambda^iv)_{i \ge 0}$ gives us the polynomial
$M_\lambda=\prod_{\alpha=(\alpha_1,\dots,\alpha_n) \in V(I)}
(x-(\alpha_n+\lambda \alpha_{n-1}))$; it is well-known that the
polynomial $R_{n-1}$ can be deduced from the coefficients of
$\lambda^0$ and $\lambda^1$ in $M_\lambda$ in quasi-linear time.

Hence, it is enough to compute $M_\lambda \bmod \lambda^2$. For this,
we compute the sequence $s_\lambda \bmod \lambda^2=(s_i + i s^*_i \lambda)_{i \ge 0}$,
with $s^*_i= uT_n^{i-1} T_{n-1}v$, and we apply e.g.\ the matrix
Berlekamp Massey algorithm to this sequence, over the coefficient ring
$\mathbb{K}[\lambda]/\lambda^2$; in generic coordinates, we expect that
the calculation carries over as if we were over a field.

\medskip\noindent{\bf The case of radical ideals.} It is of course desirable to 
avoid computing minimal matrix polynomials over non-reduced rings, if
possible. For radical ideals, we propose an alternative solution that
avoids this issue (extending this idea to arbitrary $I$ is work in progress).

Suppose that we have computed the minimal generating polynomial $S$,
together with its determinant (in the radical case, it coincides with
$R_n(x_n)$); we can then compute the numerator matrix $N$ of the
generating series $G=\sum_{i \ge 0} s_i/x^{i+1}$ by a single product
of polynomial matrices $N=S G$. We repeat this operation with the
generating series $\sum_{i \ge 0} s^*_i/x^{i+1}$, for $s^*_i$ as in
the previous paragraph, to obtain a numerator matrix $N^*$. The Smith
form of $S$, $D=A S B$, only has $R_n(x)$ as a non-trivial
invariant factor. Then, let $n$ and $n^*$ be the entries of indices
$(1,m)$ in respectively $AN$ and $AN^*$; we can verify that
$R_{n-1}=n^*/n \bmod R_n$.

As an example, we will run the algorithm on $I = \langle f_1, f_2, f_3 \rangle \subset \mathbb{F}(9001)[x_1,x_2,x_1]$,
\begin{align*}
f_1 &= 3536x_3^2 +6536x_3x_2 + 3900x_2^2 + 3722x_3x_1 + 580x_2x_1 + 7635x_1^2 + 4203x_3 + 1386x_2 + 2491x_1 + 250\\
f_2 &= 3987x_3^2 + 3953x_3x_2 +6122x_2^2 +5115x_3x_1 +7660x_2x_1 +8669x_1^2 + 4098x_3 +7705x_2 + 2449x_1 + 1134\\
f_3 &= 8589x_3^2 + 1291x_3x_2 +5321x_2^2 + 765x_3x_1 +6052x_2x_1 +8178x_1^2 + 2764x_3 + 957x_2 +7079x_1 + 517.
\end{align*}
We choose $m = 2$ and two random matrices
\begin{align*}
u &= \begin{bmatrix}
291&  22& 337& 924& 414& 666& 574& 707\\
57& 801& 513& 135& 447& 107& 942& 320
\end{bmatrix}\\
v &= \begin{bmatrix}
553&  81&  56& 261& 109& 890& 477&  53\\
725& 642& 905& 612& 952& 158& 235& 783
\end{bmatrix} ^ {t}
\end{align*}
The monomial basis of $Q$ is $\mathscr{B}=[x_1^3, x_1^2, x_2x_1, x_3x_1, x_1, x_2, x_3, 1]$, with $D = 8$. 
Computing the minimal generating polynomial of $s_i=u T_3^i v$, we get
$$S= \begin{bmatrix}
8226 + 5622x + 7693x^2 + 5033x^3 + x^3&     2919 + 8706x + 3736x^2 + 4829x^3\\
7928 + 3675x + 7471x^2 + 3510x^3  &        3113 + 5615x + 2702x^2 + 3353x^3 + x^4
\end{bmatrix},$$
with determinant
$$R_3 =x^8 + 8386x^7 + 8262x^6 + 7301x^5 + 318x^4 + 4870x^3 + 715x^2 + 8568x + 8433.$$
We compute the numerator matrix $N$ of the generating series $\sum_{i \ge 0} s_i/x^{i+1}$
and multiply it by the change of basis of $A$ obtained from the Smith form computation of 
$S$; the $(1,2)$-entry in the result is
$$n = 5527x^7 + 2064x^6 + 4391x^5 + 1308x^4 + 6797x^3 + 1328x^2 + 6317x + 8700.$$
Computing the numerator matrix $N^*$ of the generating series $\sum_{i \ge 0} s'_i/x^{i+1}$,
with $s'_i = uT_3^{i-1} T_2 v$, and doing as above, we obtain
$$n^* = 8649x^7 + 3840x^6 + 4938x^5 + 1734x^4 + 1525x^3 + 2362x^2 + 1780x + 2397.$$
The polynomial $R_2(x_3)$ is then given by $n^*/n \bmod R_3$, written in the variable
$x_3$, and
is equal to 
$$R_2 = 2287x_3^7 + 8269x_3^6 + 8475x_3^5 + 6889x_3^4 + 4785x_3^3 + 3960x_3^2 + 3902x_3 + 4153.$$
Doing the same with the sequence $uT_3^{i-1} T_1 v$ gives us a polynomial $R_1(x_3)$,
which finally leads us to the Gr\"obner basis $(x_1-R_1(x_3), x_2-R_2(x_3),R_3(x_3))$
of $I$.



%% It is possible to verify the correctness of this algorithm algebraically, given that the Smith normal form of
%% S is as described. We know experimentally that this is the case, but we do not know how to prove this relationship
%% between the Smith normal form and the least common left multiple matrix $S$. Further work must also be done to
%% make the algorithm work for ideals that are not radical.

\bibliographystyle{plain}
\bibliography{abstract}


\end{document}

