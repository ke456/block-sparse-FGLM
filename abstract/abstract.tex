\documentclass[12pt]{article}

\usepackage{fullpage,url,amsmath,amsthm,amssymb,epsfig,color,xspace}
\usepackage[
pdftitle={abstract}]
{hyperref}

\begin{document}
Computing the Groebner basis of an ideal with respect to a term ordering 
is an essential step in solving systems of polynomials.
Certain term orderings, such as the degree reverse lexicographical ordering (degrevlex),
make the computation of the Groebner basis faster, while other orderings, 
such as the lexicographical ordering (lex), make it easier to interpret when
used for polynomial system solving. Thus, the Groebner basis is often computed
first using degrevlex ordering then convert to lex ordering.

One such algorithm is the sparse FGLM. Given a Groebner basis $G$ of a zero 
dimensional ideal $I \subset \mathbb{K}[x_1, \dots, x_n]$ 
in shape position and the multiplication matrices $T_1,\dots, T_n \in \mathbb{K}^{D \times D}$,
where $D$ is the number of monomials in the canonical basis of $G$, it computes 
linear sequences of the form $s_1^{(i)} = (uT_1^iv_1)_{1\le i \le 2D}$ and
$s_j^{(i)} = (uT_1^iv_j)_{1\le i \le D}$, $2 \le j \le n$, for some vectors $u,v_j \in 
\mathbb{K}^{D\times 1}$. One can view this as an application of the Wiedemann algorithm 
for sparse linear algebra The computation of these sequences is difficult to parallelize 
since previous terms are required to compute the next. 

When the ideal is radical, we are developing an algorithm inspired by the block Wiedemann algorithm of Coppersmith
that uses small matrices instead of single vectors. The block Wiedemann algorithm has been 
widely studied and have shown to be successful in problems such as integer factorization.
More precisely, given the same input as above, let $u,v \in \mathbb{K}^{D \times M}$, $M < D$, be 
matrices with random entries; compute the matrix sequences $s_1^{(i)} := (uT_1^iv)_{1\le i \le \frac{2D}{M}}$. 
Then, we find $C_0, \dots, C_\frac{D}{M}$ such that for any $t \ge 0$,
$$ C_0 \cdot s_1^{(t)} + C_1 \cdot s_1^{(t+1)} + \cdots + C_{\frac{D}{M}} \cdot s_1^{(t+\frac{D}{M})}  = 0$$
Then assign
$$ S(x_1) := C_0  x_1^{\frac{D}{M}} + C_1  x_1^{\frac{D}{M} - 1} + \cdots C_{\frac{D}{M}} x_1^0$$
$$ \bar{P} := \det(S) , P := \bar{P}^{(rev)}$$
$$ N_1 := S * (\sum_{i\le D/M} s_1^{(i)} x_1^i) \mod x_1^{\frac{D}{M}}$$
where $\bar{P}^{(rev)}$ is the reverse of $\bar{P}$. According to both Coppersmith and Villard,
$P$ is the minimal polynomial of $T_1$. Next, compute the Smith normal form 
of $S$ such that $D = A S B$ is a diagonal matrix of the form
$$\begin{bmatrix}
	1 &  &        & \\
	  & 1&        &  \\ 
	  &  & \ddots & \\
	  &  &        & \bar{P}
\end{bmatrix}$$
and assign
$$ n_1 := (A*N_1)[0,M-1]^{(rev)}$$

Now, for $j \in [2,n]$, compute 
$$s_j^{(i)} = (uT_1^i T_j v)_{1\le i \le D}$$
$$N_j := S*(\sum_{i\le D/M} s_j^{(i)} x_1^i) \mod x_1^{\frac{D}{M}}$$
$$n_j := (A*N_j)[0,1]^{(rev)}$$
$$F_j := \frac{n_j}{n_1} \mod P$$
Finally, return $G_2 = [P, x_2 - F_2, x_3 - F_3, \dots, x_n - F_n]$, which is the Groebner basis
$I$ with respect to lex ordering.

As an example, we will run the algorithm on $I = \langle f_1, f_2, f_3 \rangle \subset GF(9001)[x_1,x_2,x_3]$,
\begin{align*}
f_1 &= 3536x_1^2 +6536x_1x_2 + 3900x_2^2 + 3722x_1x_3 + 580x_2x_3 + 7635x_3^2 + 4203x_1 + 1386x_2 + 2491x_3 + 250\\
f_2 &= 3987x_1^2 + 3953x_1x_2 +6122x_2^2 +5115x_1x_3 +7660x_2x_3 +8669x_3^2 + 4098x_1 +7705x_2 + 2449x_3 + 1134\\
f_3 &= 8589x_1^2 + 1291x_1x_2 +5321x_2^2 + 765x_1x_3 +6052x_2x_3 +8178x_3^2 + 2764x_1 + 957x_2 +7079x_3 + 517
\end{align*}
We choose $M = 2$ and two random matrices in $GF(9001)^{D\times M}$
\begin{align*}
u &= \begin{bmatrix}
291&  22& 337& 924& 414& 666& 574& 707\\
57& 801& 513& 135& 447& 107& 942& 320
\end{bmatrix}\\
v &= \begin{bmatrix}
553&  81&  56& 261& 109& 890& 477&  53\\
725& 642& 905& 612& 952& 158& 235& 783
\end{bmatrix} ^ {t}\\
\end{align*}
The canonical basis is $[x_3^3, x_3^2, x_2x_3, x_1x_3, x_3, x_2, x_1, 1]$ and $D = 8$. Doing the above computations we get
$$S= \begin{bmatrix}
8226x_1^4 + 5622x_1^3 + 7693x_1^2 + 5033x_1 + 1&     2919x_1^4 + 8706x_1^3 + 3736x_1^2 + 4829x_1\\
7928x_1^4 + 3675x_1^3 + 7471x_1^2 + 3510x_1& 3113x_1^4 + 5615x_1^3 + 2702x_1^2 + 3353x_1 + 1
\end{bmatrix}$$
$$P =x_1^8 + 8386x_1^7 + 8262x_1^6 + 7301x_1^5 + 318x_1^4 + 4870x_1^3 + 715x_1^2 + 8568x_1 + 8433$$
$$n_1 = 5527x_1^7 + 2064x_1^6 + 4391x_1^5 + 1308x_1^4 + 6797x_1^3 + 1328x_1^2 + 6317x_1 + 8700$$
It is easy to verify that $P$ is indeed the minimal polynomial of $T_1$.
For $x_2$ and $x_3$, we get
$$n_2 = 8649x_1^7 + 3840x_1^6 + 4938x_1^5 + 1734x_1^4 + 1525x_1^3 + 2362x_1^2 + 1780x_1 + 2397$$
$$F_2 = 2287x_1^7 + 8269x_1^6 + 8475x_1^5 + 6889x_1^4 + 4785x_1^3 + 3960x_1^2 + 3902x_1 + 4153$$
$$n_3 = 6771x_1^7 + 7066x_1^6 + 5042x_1^5 + 1134x_1^4 + 4434x_1^3 + 5715x_1^2 + 8773x_1 + 781$$
$$F_3 = 3999x_1^7 + 8566x_1^6 + 2338x_1^5 + 1227x_1^4 + 4962x_1^3 + 7623x_1^2 + 4661x_1 + 4463$$
Finally, we have that $G2 = [P, x_2-F_2, x_3-F_3]$ is the Groebner basis of I with lex ordering.

\end{document}

