\documentclass[final,1p,times,authoryear]{elsarticle}

\usepackage{bm} % for bold \bm
\usepackage{mathrsfs} % for \mathscr
\usepackage{algorithm, pseudocode}

% for enumerate / itemize: define reasonable margins
\usepackage[shortlabels]{enumitem}
\setlist{topsep=0.25\baselineskip,partopsep=0pt,itemsep=1pt,parsep=0pt}

% math and theorem names
\usepackage{amsmath,amsfonts,amssymb,amsthm,thmtools}
\declaretheoremstyle[headfont=\normalfont\bfseries,bodyfont=\normalfont]{myremark}
\declaretheorem[style=plain,parent=section]{definition}
\declaretheorem[sibling=definition]{theorem}
\declaretheorem[sibling=definition]{corollary}
\declaretheorem[sibling=definition]{proposition}
\declaretheorem[sibling=definition]{lemma}
\declaretheorem[style=myremark,sibling=definition,qed={\qedsymbol}]{remark}
\declaretheorem[style=remark,sibling=definition,qed={\qedsymbol}]{example}

% references / hyperlinks
\usepackage[linkcolor=black,colorlinks=true,citecolor=blue,urlcolor=blue]{hyperref} 
\usepackage[capitalise]{cleveref}
\crefname{subsec}{Subsection}{Subsections}
\Crefname{subsec}{Subsection}{Subsections}
\crefname{problem}{Problem}{Problems}
\Crefname{problem}{Problem}{Problems}

%% ------------------------------------------------
%% --------- FOR MATRIX BERLEKAMP-MASSEY ----------
%% ------------------------------------------------
%%% 
%%%notation
%misc
\newcommand{\storeArg}{} % aux, not to be used in document!!
\newcounter{notationCounter}
%spaces
\newcommand{\NN}{\mathbb{N}} % nonnegative integers
\newcommand{\var}{T} % variable for univariate polynomials
\newcommand{\field}{\mathbb{K}} % base field
\newcommand{\polRing}{\field[\var]} % polynomial ring
\newcommand{\Pox}{[\mkern-3mu[ \var ]\mkern-3.2mu]}
\newcommand{\Poxi}{[\mkern-3mu[ \var^{-1} ]\mkern-3.2mu]}
\newcommand{\psRing}{\field\Pox}
\newcommand{\matSpace}[1][\rdim]{\renewcommand\storeArg{#1}\matSpaceAux} % polynomial matrix space, 2 opt args
\newcommand{\matSpaceAux}[1][\storeArg]{\field^{\storeArg \times #1}} % not to be used in document
\newcommand{\polMatSpace}[1][\rdim]{\renewcommand\storeArg{#1}\polMatSpaceAux} % polynomial matrix space, 2 opt args
\newcommand{\polMatSpaceAux}[1][\storeArg]{\polRing^{\storeArg \times #1}} % not to be used in document
\newcommand{\psMatSpace}[1][\rdim]{\renewcommand\storeArg{#1}\psMatSpaceAux} % polynomial matrix space, 2 opt args
\newcommand{\psMatSpaceAux}[1][\storeArg]{\psRing^{\storeArg \times #1}} % not to be used in document
\newcommand{\mat}[1]{\bm{\MakeUppercase{#1}}} % for a matrix
\newcommand{\row}[1]{\bm{\MakeLowercase{#1}}} % for a matrix
\newcommand{\col}[1]{\bm{\MakeLowercase{#1}}} % for a matrix
\newcommand{\matCoeff}[1]{\MakeLowercase{#1}} % for a coefficient in a matrix
\newcommand{\rdim}{m} % row dimension
\newcommand{\cdim}{{m'}} % column dimension
\newcommand{\diag}[1]{\mathrm{Diag}(#1)}  % diagonal matrix with diagonal entries #1
\newcommand{\seqelt}[1]{\bm{F}_{#1}} % element of sequence of matrices
\newcommand{\seqeltSpace}{\matSpace[\rdim][\cdim]} % element of sequence of matrices
\newcommand{\seq}{\mat{\mathcal{F}}} % sequence of matrices
\newcommand{\seqLelt}[1]{\bm{L}_{#1}} % element of sequence of matrices
\newcommand{\seqL}{\mat{\mathcal{L}}} % sequence of matrices
\newcommand{\seqpm}{\mat{Z}} % power series matrix from a sequence
\newcommand{\rel}{\col{p}} % linear relation
\newcommand{\relbas}{\mat{P}} % linear relation
\newcommand{\relSpace}{\polMatSpace[1][\rdim]} % space for linear relations
\newcommand{\relbasSpace}{\polMatSpace[\rdim][\rdim]} % space for linear relations
\newcommand{\num}{\row{q}} % numerator for linear recurrence relation
\newcommand{\nummat}{\mat{Q}} % numerator for linear recurrence relation basis
\newcommand{\rem}{\row{r}} % remnant for linear recurrence relation
\newcommand{\remmat}{\mat{R}} % remnant for linear recurrence relation basis
\newcommand{\remSpace}{\polMatSpace[1][\cdim]} % space for linear relations
\newcommand{\degBd}{d} % bound on degree of minimal generator
\newcommand{\degBdr}{d_{r}} % bound on degree of a right minimal generator
\newcommand{\degBdl}{d_{\ell}} % bound on degree of a left minimal generator
\newcommand{\degDet}[1][\seq]{\operatorname{\Delta}(#1)}
\newcommand{\rdeg}[2][]{\mathrm{rdeg}_{{#1}}(#2)} % shifted row degree
\newcommand{\cdeg}[2][]{\mathrm{cdeg}_{{#1}}(#2)} % shifted column degree
\newcommand{\sys}{\mat{F}} % input matrix series to approximant basis
\newcommand{\appMod}[2]{\mathcal{A}(#1,#2)} % module of approximants for #2 at order #1
\newcommand{\basis}{\mathscr{B}}
\newcommand{\trace}{\operatorname{trace}}
\newcommand{\softO}[1]{O{\tilde{~}}(#1)} % module of approximants for #2 at order #1
\newcommand{\genseries}{Z}
\newcommand{\minpoly}{P}
\newcommand{\mainalgoname}{\mathsf{ BlockParametrization}}
\newcommand{\lf}{X}
\newcommand{\mf}{Y}
\newcommand{\residueI}{\mathscr{Q}}
\newcommand{\sqfree}{Q}
\newcommand{\trsp}[1]{#1^{\mathsf{T}}} % transpose of a matrix
\newcommand{\itrsp}[1]{#1^{\mathsf{-T}}} % inverse transpose of a matrix


\newcommand{\density}{\rho}

\def\M {\ensuremath{\mathsf{M}}}
\def\PP {\ensuremath{\mathsf{P}}}
\def\Deg{D}
\def\dg{\kappa}

\def\C {\ensuremath{\mathbb{C}}}
\def\Q {\ensuremath{\mathbb{Q}}}
\def\N {\ensuremath{\mathbb{N}}}
\def\R {\ensuremath{\mathbb{R}}}
\def\Z {\ensuremath{\mathbb{Z}}}
\def\F {\ensuremath{\mathbb{F}}}
\def\H {\ensuremath{\mathbb{H}}}
\def\K{\mathbb{K}}
\def\K {\ensuremath{\mathbb{K}}}
\def\Kbar {{\ensuremath{\overline{\mathbb{K}}}}}
\def\A {\ensuremath{\mathbb{A}}}
\def\D {\ensuremath{D}}
\def\m {\ensuremath{\mathfrak{m}}}

\def\scrY{\mathscr{Y}}
\def\scrM {\ensuremath{\mathscr{M}}}
\def\calL {\ensuremath{\mathcal{L}}}
\def\scrP {\ensuremath{\mathscr{P}}}
\def\scrS {\ensuremath{\mathscr{S}}}
\def\scrU {\ensuremath{\mathscr{U}}}
\def\scrV {\ensuremath{\mathscr{V}}}
\def\ann {\ensuremath{\mathrm{ann}}}
\def\rk {\ensuremath{\mathrm{rk}}}

\DeclareBoldMathCommand{\bell}{\ell}
\DeclareBoldMathCommand{\be}{e}
\DeclareBoldMathCommand{\bu}{u}
\DeclareBoldMathCommand{\bv}{v}
\DeclareBoldMathCommand{\bX}{X}
\DeclareBoldMathCommand{\bx}{x}
\DeclareBoldMathCommand{\balpha}{\alpha}
\DeclareBoldMathCommand{\bbeta}{\beta}
\DeclareBoldMathCommand{\mA}{A}
\DeclareBoldMathCommand{\mB}{B}
\DeclareBoldMathCommand{\mD}{D}
\DeclareBoldMathCommand{\mF}{F}
\DeclareBoldMathCommand{\mG}{G}
\DeclareBoldMathCommand{\mI}{I}
\DeclareBoldMathCommand{\mM}{M}
\DeclareBoldMathCommand{\mNs}{N^*}
\DeclareBoldMathCommand{\mN}{N}
\DeclareBoldMathCommand{\mS}{S}
\DeclareBoldMathCommand{\mT}{T}
\DeclareBoldMathCommand{\mU}{U}
\DeclareBoldMathCommand{\mV}{V}
\DeclareBoldMathCommand{\mW}{W}
\DeclareBoldMathCommand{\mX}{X}
\DeclareBoldMathCommand{\mY}{Y}
\DeclareBoldMathCommand{\mZ}{Z}
\DeclareBoldMathCommand{\bell}{\ell}
\DeclareBoldMathCommand{\be}{e}
\DeclareBoldMathCommand{\bu}{u}
\DeclareBoldMathCommand{\bv}{v}
\DeclareBoldMathCommand{\bX}{X}
\DeclareBoldMathCommand{\bx}{x}
\DeclareBoldMathCommand{\balpha}{\alpha}
\DeclareBoldMathCommand{\bbeta}{\beta}

\newcommand{\mUt}{\trsp{\mU}}


\begin{document}

\begin{frontmatter}

  \title{Block-Krylov techniques in the context of \\ sparse-FGLM algorithms}

  \author{Seung Gyu Hyun}
  \address{Cheriton School of Computer Science, University of Waterloo}

  \author{Vincent Neiger}
  \address{Univ.~Limoges, CNRS, XLIM, UMR 7252, F-87000 Limoges, France}

  \author{Hamid Rahkooy}
  \address{Cheriton School of Computer Science, University of Waterloo}

  \author{\'Eric Schost}
  \address{Cheriton School of Computer Science, University of Waterloo}

  \begin{abstract}
    Consider a zero-dimensional ideal $I$ in $\K[X_1,\dots,X_n]$.  Inspired by
    Faug\`ere and Mou's Sparse FGLM algorithm,
    we use Krylov sequences based on multiplication matrices of $I$ in order to
    compute a description of its zero set by means of univariate polynomials.

    Steel recently showed how to use Coppersmith's block-Wiedemann algorithm in
    this context; he describes an algorithm that can be easily parallelized, but
    only computes parts of the output in this manner. Using generating series
    expressions going back to work of Bostan, Salvy, and Schost, we show how to
    compute the entire output for a small overhead, without making any assumption
    on the ideal $I$ other than it having dimension zero. We then propose a refinement of this idea that partially
    avoids the introduction of a generic linear form.  We comment on experimental
    results obtained by an implementation based on the C++ libraries LinBox, Eigen and
    NTL.
  \end{abstract}

  \begin{keyword}
    Polynomial systems; Block-Krylov algorithms; Sparse FGLM.
  \end{keyword}

\end{frontmatter}


\section{Introduction}

Computing the Gr\"obner basis of an ideal with respect to a given term
ordering is an essential step in solving systems of polynomials.
Certain term orderings, such as the degree reverse lexicographic
ordering (\textit{degrevlex}), tend to make the computation of the
Gr\"obner basis faster. This has been observed empirically since the
1980's and is now supported by theoretical results, at least for some
``nice'' families of inputs, such as complete intersections or certain
determinantal systems~\citep{Faugere02,FaSaSp13,BaFaSa15}.  On the
other hand, other orderings, such as the lexicographic ordering
(\textit{lex}), make it easier to find the coordinates of the
solutions, or to perform arithmetic operations in the corresponding
residue class ring.  For instance, for a zero-dimensional radical
ideal $I$ in generic coordinates in $\K[X_1,\dots,X_n]$, for some
field $\K$, the Gr\"obner basis of $I$ for the lexicographic ordering
with $X_1 > \cdots > X_n$ has the form
\begin{equation}\label{eq:shapelemma}
  \{ X_1 - R_1(X_n),\dots,X_{n-1}-R_{n-1}(X_n),R_n(X_n)\},
\end{equation}
with all $R_i$'s, for $i =1,\dots,n-1$, of degree less than
$\deg(R_n)$ (and $R_n$ squarefree); this is known as the {\em
shape lemma}~\citep{GiMo89}. The points in the variety $V(I) \subset
\Kbar{}^n$ are then
$$\{ ( R_1(\tau), \dots, R_{n-1}(\tau), \tau ) \mid \tau \in \Kbar
\;\,\text{is a root of}\;\, R_n\}.$$ As a result, the standard approach to
solve a zero-dimensional system by means of Gr\"obner basis
algorithms is to first compute a Gr\"obner basis for a degree ordering
and then convert it to a more exploitable output, such as a lexicographic
basis. As pointed out in~\citep{FaMo17}, the latter step, while of
polynomial complexity, can now be a bottleneck in practice. This
paper will thus focus on this step; in order to describe our 
contributions, we first discuss previous work on the question.

Let $I$ be a zero-dimensional ideal in $\K[X_1,\dots,X_n]$.  As
input, we assume that we know a monomial basis $\basis$ of
$\residueI=\K[X_1,\dots,X_n]/I$, together with the multiplication
matrices $\mM_1,\dots,\mM_n$ of respectively $X_1,\dots,X_n$ in this
basis. We denote by $D$ the degree of $I$, which is the vector space
dimension of $\residueI$. We should stress that starting from a
degree Gr\"obner basis of $I$, computing the multiplication matrices
efficiently is not a straightforward task. \citep{FaGiLaMo93}
showed how to do it in time $O(nD^3)$; more
recently, algorithms have been given with cost bound
$\softO{nD^\omega}$~\citep{FaGaHuRe13,FaGaHuRe14,Neiger16}, at least
for some favorable families of inputs. Here, the notation
$O\tilde{~}$ hides polylogarithmic factors and $\omega$ is a feasible
exponent for matrix multiplication over a ring (commutative, with
$1$). While improving these results is an interesting question in
itself, we will not address it in this paper.

Given such an input, including the multiplication matrices, the FGLM
algorithm~\citep{FaGiLaMo93} computes the lexicographic Gr\"obner basis
of $I$ in $O(nD^3)$ operations in $\K$.  While the algorithm has an
obvious relation to linear algebra, lowering the runtime to
$O\tilde{~}(nD^\omega)$ was only recently
achieved~\citep{FaGaHuRe13,FaGaHuRe14,Neiger16}.

Polynomials as in~\cref{eq:shapelemma} form a very useful data
structure, but there is no guarantee that the lexicographic Gr\"obner
basis of $I$ has such a shape (even in generic coordinates). When it does, we will say that $I$ is
in {\em shape position}; some sufficient conditions for being in shape
position are detailed in~\citep{BeMoMaTr94}.  As an alternative, one
may then use the Rational Univariate Representation algorithm of
\citet{Rouillier99} (see also~\citep{AlBeRoWo94,BeWo96} for related
considerations). The output is a description of the zero-set $V(I)$ by
means of univariate rational functions
\begin{equation}\label{eq:RUR}
  \left\{  F(T)=0, \quad X_1 = \frac{G_1(T)}{G(T)}, \dots,X_n = \frac{G_n(T)}{G(T)} \right\},
\end{equation}
where the multiplicity of a root $\tau$ of $F$ coincides with that of
$I$ at the corresponding point
$(G_1(\tau)/G(\tau),\dots,G_n(\tau)/G(\tau)) \in V(I)$. The fact that
we use rational functions makes it possible to control
precisely the bit-size of their coefficients, if working over $\K=\Q$.

The algorithms of~\citet{AlBeRoWo94, BeWo96, Rouillier99} rely on
\emph{duality}, which will be at the core of our algorithms as well.
Indeed, these algorithms compute sequences of values of the form
$v_s=(\trace(\lf^s))_{s \ge 0}$ and 
$v_{s,i}=(\trace(\lf^s X_i))_{s \ge 0}$, where $\trace: \residueI \to \K$ is the trace 
form and $\lf=t_1 X_1 + \cdots + t_n X_n$ is some $\K$-linear combination of the variables.
From these values, one may then recover the output in~\cref{eq:RUR} by means
of structured linear algebra calculations.

A drawback of this approach is that we need to know the trace of all
elements of the basis $\basis$; while feasible in polynomial time,
this is by no means straightforward. \citet{BoSaSc03}
introduced randomization to alleviate this issue. They show
that computing values such as $\ell(\lf^s)$ and $\ell(\lf^s X_i)$, where
$\lf$ is as above and 
$\ell$ is a random $\K$-linear form $\residueI \to \K$, allows one to deduce
a description of $V(I)$ of the form
\begin{equation}\label{eq:BoSaSc03}
  \{  \sqfree(T)=0, \quad X_1 = V_1(T), \dots,X_n = V_n(T) \},
\end{equation}
where $\sqfree$ is a monic squarefree polynomial in $\K[T]$ and $V_i$ is in
$\K[T]$ of degree less than $\deg(\sqfree)$ for all $i$. The tuple
$((\sqfree,V_1,\dots,V_n),\lf)$ computed by  such algorithms
will be called a {\em zero-dimensional parametrization} of $V(I)$. In
particular, it generally differs from the description
in~\cref{eq:RUR}, since the latter keeps track of the multiplicities
of the solutions (the algorithm in~\citep{BoSaSc03} actually computes
the {\em nil-indices} of the solutions). Remark that starting from
such an output, we can reconstruct the local structure of $I$ at 
its roots, using algorithms from~\citep{MaMoMo96},~\citep{Mourrain97} or~\citep{NeRaSc17}.

The most costly part of the algorithm of~\citep{BoSaSc03} is the
computation of the values $\ell(\lf^s)$ and $\ell(\lf^s X_i)$; the
rest essentially boils down to applying the Berlekamp-Massey algorithm
and univariate polynomial arithmetic. \citet{FaMo17} pointed out that the
multiplication matrices
$\mM_1,\dots,\mM_n$ can be expected to be sparse; for generic inputs,
they gave precise estimates on the sparsity of these matrices,
assuming the validity of a conjecture by
\citet{MorenoSocias91}.  On this basis, they designed
several forms of {\em sparse FGLM} algorithms. For instance, if $I$ is
in shape position, the algorithms in \citep{FaMo17} recover its
lexicographic basis, which is as in~\eqref{eq:shapelemma}, by also
considering values of a linear form $\ell:\residueI \to \K$. For less
favorable inputs, these algorithms fall back either on the
algorithm of \citet{Sakata90} or on plain FGLM.

The ideas at play in these algorithms are essentially based on Krylov subspace
methods, using projections as well as Berlekamp-Massey techniques, along the
lines of the algorithm of \citet{Wiedemann86} to solve sparse linear systems.
These techniques have also been widely used in integer
factorization or discrete logarithm calculations, going back to~\citep{LaOd90}.
It has become customary to design block versions of such algorithms to
parallelize their bottleneck, as pioneered by \citet{Coppersmith94}
in the context of integer factorization: it is then natural to adapt this
strategy to our situation. This was already discussed by
\citet{Steel15}, where he showed how to compute the analogue of the
polynomial $\sqfree$ in \cref{eq:BoSaSc03} using such techniques. In that
reference, one is only interested in the solutions in the base field $\K$ ($\K$
being a finite field in that context): the algorithm computes the roots of
$\sqfree$ in $\K$ and substitutes them in the input system, before computing a
Gr\"obner basis in $n-1$ variables for each of them.

Our first contribution is to give a block version of the algorithm
in~\citep{BoSaSc03} that extends the approach introduced
in~\citep{Steel15} to compute all polynomials in~\cref{eq:BoSaSc03} for
essentially the same cost as the computation of $\sqfree$. More
precisely, the bottleneck of the algorithm of~\citep{Steel15} is the
computation of a block-Krylov sequence; we show that once this
sequence has been computed, not only $\sqfree$ but also all other
polynomials in the zero-dimensional parametrization can be efficiently
obtained. Compared with the algorithms of~\citep{FaMo17}, a notable
difference is that our algorithm deals with any ideal zero-dimensional
$I$ (for instance, we do not require shape position), but the
base field must have sufficiently large characteristic (and our output is
somewhat weaker than a Gr\"obner basis, since multiplicities are not
computed). While we focus on the case where the multiplication
matrices are sparse, we also give a cost analysis for the case of
dense multiplication matrices.

Our second contribution is a refinement of our first algorithm, where
we try to avoid computations with a generic linear form $\lf =t_1 X_1 +
\cdots + t_n X_n$ to the extent possible (this is motivated by the
fact that the multiplication matrix of $\lf$ is often denser than those
of the variables $X_i$). The algorithm first computes a zero-dimensional
parametrization of a subset of $V(I)$ for which we can take $\lf$
equal to (say) $X_1$, and falls back on the previous approach for the
residual part; if the former set has large cardinality, this is
expected to provide a speed-up over the general algorithm.

For experimental purposes, our algorithms have been implemented in C++ using
the libraries LinBox~\citep{LinBox}, Eigen~\citep{Eigen} and NTL~\citep{NTL}.

The paper is organized as follows. The next section mainly reviews
known results on scalar and matrix recurrent sequences, and introduces
a simple useful algorithm to compute a so-called {\em scalar
numerator} for such sequences. \cref{sec:seq0} describes sequences
that arise in the context of FGLM-like algorithms; we prove slightly
refined versions of results from~\citep{BoSaSc03} that will be 
used throughout the paper. The main algorithm is given in \cref{sec:main},
and the refinement mentioned above is in \cref{sec:original}.


\paragraph{Complexity model.}
We measure the cost of our algorithms by counting basic operations in
$\K$ at unit cost. Most algorithms are randomized; they involve the
choice of a vector $\gamma \in \K^S$ of field elements, for an integer
$S$ that depends on the size of our input, and success is guaranteed
if the vector $\gamma$ avoids a hypersurface of the parameter space
$\K^S$.

Suppose that the input ideal $I$ is generated by polynomials
$F_1,\dots,F_t$.  Given a zero-dimensional parametrization
$((\sqfree,V_1,\dots,V_n),\lf)$ computed by our algorithms, one can
always evaluate $F_1,\dots,F_t$ at $X_1 =V_1,\dots,X_n=V_n$, doing all
computations modulo $\sqfree$. This allows us to verify whether the
output describes a subset of $V(F_1,\dots,F_t)$, but not whether we
have found all solutions.  If $\deg(Q)$ coincides with the dimension
of $\residueI=\K[X_1,\dots,X_n]/I$, we can infer that we have all
solutions (and that $I$ is radical), so our output is correct.

In what follows, we assume $\omega>2$, in order to simplify a few cost
estimates. We use a time function $d \mapsto \M(d)$ for the cost of
univariate polynomial multiplication over $\K$, for which we assume
the super-linearity properties of \citep[Section~8.4]{GaGe13}.  Then,
two $m\times m$ matrices over $\K[T]$ whose degree is less than $d$
can be multiplied using $O(m^\omega \M(d))$ operations in $\K$.

\paragraph{Acknowledgments.} We wish to thank Chenqi Mou and Dave Saunders for
useful discussions. This research was partially supported by NSERC (Schost's
NSERC Discovery Grant) and by the CNRS-INS2I Institute through its program for
young researchers (Neiger's project ARCADIE).

%%%%%%%%%%%%%%%%%%%%%%%%%%%%%%%%%%%%%%%%%%%%%%%%%%%%%%%%%%%%
%%%%%%%%%%%%%%%%%%%%%%%%%%%%%%%%%%%%%%%%%%%%%%%%%%%%%%%%%%%%
%%%%%%%%%%%%%%%%%%%%%%%%%%%%%%%%%%%%%%%%%%%%%%%%%%%%%%%%%%%%

\section{Linearly recurrent sequences}

This section starts with a review of known facts on linearly recurrent
sequences: we first discuss the scalar case, and then we show how the ideas
carry over to matrix sequences. 

The main results we will need from the first three subsections are
\cref{coro:cost_approx} and \cref{randXY}, which give cost estimates
for the computation of a {\em minimal matrix generator} of a linearly
recurrent matrix sequence, as well as a degree bound for such
generators, when the matrices we consider are obtained from a Krylov
sequence. These results are for the most part not new
(see~\citep{Villard97,Villard97a,KalVil01,Turner02,KaVi04}), but the cost
analysis we give uses results not available when those references were
written.  The fourth and last subsection presents a useful
result for our main algorithm that allows us to compute a
``numerator'' for a scalar sequence from a similar object obtained for
a matrix sequence.

%%%%%%%%%%%%%%%%%%%%%%%%%%%%%%%%%%%%%%%%%%%%%%%%%%%%%%%%%%%%

\subsection{Scalar sequences} \label{section:linseq}

Let $\K$ be a field and consider a sequence $\mathcal{L}=(\ell_s)_{s
\ge 0} \in \K^\N$. We say that a degree $d$ polynomial $\minpoly =
p_0 + \cdots + p_d T^d \in\K[T]$ {\em cancels} the sequence
$\mathcal{L}$ if $p_0 \ell_s + \cdots + p_d \ell_{s+d}=0$ for all $s
\ge 0$. The sequence $\mathcal{L}$ is {\em linearly recurrent} if
there exists a nonzero polynomial that cancels it.  The {\em minimal
polynomial} of a linearly recurrent sequence
$\mathcal{L}=(\ell_s)_{s \ge 0}$ is the monic polynomial of lowest
degree that cancels it; the {\em order} of $\mathcal{L}$ is the degree
of this polynomial~$\minpoly$.

One can rephrase these properties in terms of the generating series $S=\sum_{s
\ge 0} \ell_s T^s \in \K[[T]]$ associated to $\mathcal{L}$.  Then, $\minpoly$
cancels $\mathcal{L}$ if and only if ${\rm rev}(\minpoly) S$ is a polynomial,
where ${\rm rev}(\minpoly)=T^d \minpoly(1/T)$; in this case, ${\rm
rev}(\minpoly) S$ must have degree less than $d$.  The sequence $\mathcal{L}$
is linearly recurrent if and only if there exist polynomials $A,B$ in $\K[T]$
such that $S=A/B$; these polynomials are unique if we assume $\gcd(A,B)=1$ and
$B(0)=1$.  Given these $A$ and $B$, the minimal polynomial of $\mathcal{L}$ is
$\minpoly = T^{\max(\deg(A)+1,\deg(B))}B(1/T)$. 

It is often easier to work with a closed form that has the actual
minimal polynomial as its denominator, rather than its reverse; this
is done by working with generating series in the variable $1/T$.  More precisely, let
$\genseries = \sum_{s\ge0} \ell_s / T^{s+1}$ and $\minpoly$
be any polynomial; then, $\minpoly$ cancels the sequence $\mathcal{L}$
if and only if $Q=\minpoly \genseries $ is a polynomial, in which case $Q$
must have degree less than $\deg(\minpoly)$.  Using generating series in
$1/T$, the minimal polynomial of $\mathcal{L}$ is thus the
monic polynomial $\minpoly$ of lowest degree for which there exists $Q \in
\K[T]$ such that $\genseries=Q/\minpoly$.

In particular, given a scalar sequence and a
polynomial that cancels it, there is a well-defined notion of
associated numerator. This is formalized in the next definition:
\begin{definition}
  \label{def:omega}
  Let $\mathcal{L}=(\ell_s)_{s \ge 0}\in \K^\N$ be a sequence and $P$ be a
  polynomial that cancels $\mathcal{L}$. Then, the {\em numerator} of $\mathcal{L}$
  with respect to $P$ is denoted by $\Omega(\mathcal{L},P)$ and defined as 
  \[
    \Omega(\mathcal{L},P) = P \genseries, \quad\text{where}\quad
    \genseries=\sum_{s \ge 0} \frac {\ell_s}{T^{s+1}}.
  \]
  In particular, $\Omega(\mathcal{L},P)$ is a polynomial of
  degree less than $\deg(P)$.
\end{definition}

Here are two remarks concerning this definition:
\begin{itemize}
  \item We may write this definition using expressions in
    variable $T$ instead of $1/T$ as well: given
    $\mathcal{L}=(\ell_s)_{s\ge0}$ and $P$ as above, we recover
    $\Omega(\mathcal{L},P)$ by considering $S=\sum_{s\ge 0} \ell_s
    T^s$ and computing $A = {\rm rev}(P) S$, with ${\rm rev}(P)=T^d
    P(1/T)$ and $d=\deg(P)$; then
    $$\Omega(\mathcal{L} ,P) = T^{d - 1} A (1/T).$$ 
  \item We only need to know the first $d=\deg(P)$ coefficients
    $\ell_0,\dots,\ell_{d-1}$ to compute $\Omega(\mathcal{L}, P)$. Explicitly, we
    have
    \[
      \Omega(\mathcal{L},P) = \left( P \sum_{s=0}^{d-1} \ell_{d - 1- s} T^s \right) {\rm~div~} T^d,
    \]
    where div denotes the quotient in the Euclidean division.
\end{itemize}

\medskip

Now, we give an example of this construction. Consider the Fibonacci sequence
$\mathcal{L} = (1,1,2,3,5,8,\dots)$, which is linearly recurrent with minimal
polynomial $P=T^2-T-1$, and define $S= \sum_{s\ge 0} \ell_{s} T^s$ and
${\rm rev}(P) = 1-T-T^2$. Then, we can write $S$ in closed form as
$$S = \frac{A}{{\rm rev}(P)} = \frac{1}{1-T-T^2},$$
so that $\Omega(\mathcal{L},P)=T^{2-1} \cdot 1=T$. Equivalently, defining
$\genseries = \sum_{s\ge0} \ell_{s}/T^{s+1}$, we recover 
\[
  \Omega(\mathcal{L},P) = (T^2-T-1)\left (\frac 1T +\frac 1{T^2} + \frac
  2{T^3} + \frac 3{T^4} + \cdots \right ) =T.
\]
The numerators thus defined are related to the Lagrange interpolation
polynomials; this will explain why they play an important role in our main
algorithm. Explicitly, suppose that $\mathcal{L} = (a_1^s + \cdots + a_n^s)_{s
\ge 0}$ for some pairwise distinct elements $a_1,\dots,a_n \in \K$. Then, its
minimal polynomial is $\minpoly=\prod_{i=1}^n (T-a_i)$, and the numerator
$\Omega(\mathcal{L}, P)$ is  $ \sum_{i=1}^n \prod_{i'\ne i} (T-a_{i'})$.

Let us go back to the general case of a sequence $\mathcal{L}
=(\ell_s)_{s\ge0}\in\K^\N$.  In terms of complexity, assuming that
$\mathcal{L}$ is known to have order at most $d$ and that we are given its
first $2d$ terms, we can recover its minimal polynomial $P$ by means of the
Berlekamp-Massey algorithm, or of Euclid's algorithm applied to rational
function reconstruction; this can be done in $O(\M(d)\log(d))$ operations in
$\K$ \citep{BrGuYu80}. Given $P$ and $\ell_0,\dots,\ell_{\deg(P)-1}$, we can
deduce $\Omega(\mathcal{L},P)$ for the cost $\M(\deg(P))$ of one polynomial
multiplication in degree $\deg(P)$.


%%%%%%%%%%%%%%%%%%%%%%%%%%%%%%%%%%%%%%%%%%%%%%%%%%%%%%%%%%%%

\subsection{Linearly recurrent matrix sequences}\label{section:matrix_seq}

Next, we discuss the analogue of these ideas for matrix sequences; our main
goal is to give a cost estimate for the computation of a suitable {\em matrix
generator} for a matrix sequence, obtained by means of recent algorithms
for approximant bases. A similar discussion, but relying on the approximant
basis algorithm in \citep{BecLab94}, can be found in~\citep[Chapter~4]{Turner02}.
Note also that the latter reference uses the generating series in $\var$ while
here we use that in $1/\var$, thus obtaining the sought generator directly as a
submatrix of the computed approximant basis, without requiring polynomial
reversals.

We first define the notion of linear recurrence for matrix sequences over a
field $\field$ as in~\citep[Section~3]{KalVil01}
or~\citep[Definition~4.2]{Turner02}, hereby extending the above notions for
scalar sequences.
\begin{definition}
  \label{dfn:recurrence_relation}
  Let $\seq = (\seqelt{s})_{s\ge 0} \subset \seqeltSpace$ be a matrix
  sequence.  Then,
  \begin{itemize}
    \item a polynomial $P = p_0 + \cdots + p_\degBd T^\degBd \in \polRing$ is
      a \emph{scalar relation for $\seq$} if the identity $p_0 \seqelt{s} +
      \cdots + p_{\degBd} \seqelt{s+\degBd} = \mat{0}$ holds for all $s \ge 0$;
    \item a polynomial vector 
      $\rel = \row{p}_0 + \cdots + \row{p}_\degBd T^\degBd \in
      \relSpace$ is  a \emph{(left, vector) relation for $\seq$} if
      $  \row{p}_0 \seqelt{s} + \cdots + \row{p}_{\degBd} \seqelt{s+\degBd} = \row{0}$ holds for all
      $s \ge 0$;
    \item $\seq$ is  \emph{linearly recurrent} if it admits  a
      nonzero scalar relation.
  \end{itemize}
\end{definition}
For designing efficient algorithms, it will be useful to rely on
operations on polynomials, that is, truncated power series; hence the
following characterization of vector relations.

\begin{lemma}
  \label{lem:linearly_recurrent}
  Consider a matrix sequence $\seq = (\seqelt{s})_{s\ge 0} \subset
  \seqeltSpace$ and its generating series $\seqpm = \sum_{s\ge 0} \seqelt{s} /
  \var^{s+1} \in \field\Poxi^{\rdim \times \cdim}$.  Then, $\rel \in \relSpace$
  is a vector relation for $\seq$ if and only if the entries of $\num = \rel
  \seqpm$ are in $\polRing$; furthermore, in this case, $\deg(\num) <
  \deg(\rel)$.
\end{lemma}
\begin{proof}
  Write $\rel = \sum_{0 \le k \le \degBd} \row{p}_k \var^k$. For $s \ge 0$,
  the coefficient of $\num$ of degree $-s-1<0$ is $\sum_{0\le k \le
  \degBd} \row{p}_k \seqelt{s+k}$. Hence the equivalence, by definition of
  a relation.  The degree comparison is clear since $\seqpm$ has only
  terms of negative degree.
\end{proof}

Concerning the algebraic structure of the set of vector relations, we have the
following basic result, which can be found for example in
\citep{Villard97,KalVil01,Turner02}.

\begin{lemma}
  \label{lem:module_rank}
  The sequence $\seq$ is linearly recurrent if and only if the set of left
  vector relations for $\seq$ is a $\polRing$-submodule of $\relSpace$ of rank
  $\rdim$.
\end{lemma}
%%%% kEEP for reference, thanks.
%% \begin{proof}
%% 	The set of vector relations for $\seq$ is a $\polRing$-submodule of
%% 	$\relSpace$, and hence is free of rank at most $\rdim$
%% 	\citep[Chapter~12]{DumFoo04}.
%%
%% 	If $\seq$ is linearly recurrent, let $p \in \polRing$ be a nontrivial scalar
%% 	relation for $\seq$. Then each vector $[0 \; \cdots \; 0 \; p \; 0 \; \cdots
%% 	\; 0]$ with $p$ at index $1 \le i \le \rdim$ is a vector relation for $\seq$,
%% 	hence $\seq$ has rank $\rdim$.  Conversely, if $\seq$ has rank $\rdim$, then
%% 	it has a basis with $\rdim$ vectors, which form a matrix in $\relbasSpace$;
%% 	the determinant of this matrix is a nontrivial scalar relation for $\seq$.
%% \end{proof}

\noindent
(Note however that in general a matrix sequence may admit nontrivial vector
relations and have no scalar relation, and therefore not be linearly recurrent
with the present definition; in this case the module of vector relations has
rank less than $\rdim$.)

\begin{definition}
  \label{dfn:matrix_generator}
  Let $\seq \subset \seqeltSpace$ be linearly recurrent.  A \emph{(left) matrix
  generator} $\mat{P}$ for $\seq$ is a matrix in $\relbasSpace$ whose rows form a basis
  of the module of left vector relations for $\seq$. This basis is said to be
  \begin{itemize}
    \item \emph{minimal} if $\mat{P}$ is in reduced form \citep{Wolovich74,Kailath80};
    \item \emph{canonical} if $\mat{P}$ is in Popov form \citep{Popov72,Kailath80}.
  \end{itemize}
\end{definition}
\noindent Note that the canonical generator is also a minimal generator.
Besides, all matrix generators $\relbas \in \relbasSpace$ have the same
determinantal degree $\deg(\det(\relbas))$, which we denote by $\degDet$.  

We now point out that matrix generators are denominators in some irreducible
fraction description of the generating series of the sequence; this is a direct
consequence of \cref{lem:linearly_recurrent,lem:module_rank} and of basic
properties of polynomial matrices. As suggested in the previous subsection,
working with generating series in $1/T$ turns out to be the most convenient
choice here.  

In what follows, for an $r \times s$ matrix $\mat{P}$ with entries in
$\K[T]$, we denote by $\rdeg{\mat{P}}$ its row degree, that is, the
size-$r$ vector of the degrees of its rows; similarly, we denote by
$\cdeg{\mat{P}}$ its column degree, which is a size-$s$ vector.

\begin{corollary}
  Let $\mat{P}$ be a nonsingular matrix in $\relbasSpace$. The rows
  of $\mat{P}$ are relations for a matrix sequence $\seq =
  (\seqelt{s})_{s\ge 0} \subset \seqeltSpace$ if and only if the
  generating series $\seqpm = \sum_{s\ge 0} \seqelt{s} / \var^{s+1}
  \in \field\Poxi^{\rdim \times \cdim}$ can be written as a matrix
  fraction $\seqpm = \relbas^{-1} \nummat$, with $\nummat \in
  \polMatSpace[\rdim][\cdim]$. In this case, we have $\rdeg{\nummat} <
  \rdeg{\relbas}$ termwise and $\deg(\det(\relbas)) \ge \degDet$; furthermore,
  $\relbas$ is a matrix generator for $\seq$ if and only if
  $\deg(\det(\relbas)) = \degDet$.
  %%  or, equivalently, the fraction $\relbas^{-1} \nummat$ is
  %% irreducible (that is, $\relbas \mat{U} + \nummat \mat{V} = \mat{I}$
  %% for some polynomial matrices $\mat{U}$ and $\mat{V}$).
\end{corollary}

Given a nonsingular matrix of relations $\relbas$ for  $\seq$, we will thus
write $\mat{\Omega}(\seq, \relbas)= \relbas \seqpm  \in
\polMatSpace[\rdim][\cdim]$, generalizing \cref{def:omega}. 

By the previous corollary, this is a polynomial matrix, whose $i$th
row has degree less than the $i$th row of $\mat{P}$ for $1\le
i\le\rdim$.  As in the scalar case, if $\mat{P}$ has degree $d$, we
only need to know $\seqelt{0},\dots,\seqelt{d-1}$ to recover
$\mat{\Omega}(\seq, \relbas)$ as
\begin{align}\label{eq:computeOmega}
  \mat{\Omega}(\seq, \relbas) =  \left ( \relbas \cdot \sum_{s=0}^{d-1} \seqelt{d-1-s} T^s \right ) \;\mathrm{div}\; T^d,
\end{align}
where ``$\mathrm{div}\; T^d$'' means we keep the quotient of each entry by $T^d$.
The cost of computing it then depends on the cost of
rectangular matrix multiplication in sizes $(\rdim,\rdim)$ and
$(\rdim,\cdim)$. For simplicity, we only give the two most useful
cases: if $\cdim=\rdim$, this takes $O(\rdim^\omega \M(d))$ operations
in $\K$; if $\cdim =1$, the cost becomes $O(\rdim^2 \M(d))$.

%%% Remark to keep in comment for now:
%%% (not so important here, and not clear what to cite to say it concisely)
%% in fact if $\cdim=1$ and d terms we will have average row degree ~d/m
%% --> the cost would be better, O(m^w M(d/m))
%% more generally, average row degree ~nd/m -->  O(m^w M(nd/m))

In all the previous discussion, we remark that we may also consider
vector relations operating on the right: in particular,
\cref{lem:linearly_recurrent} shows that if the sequence is linearly
recurrent then these right relations form a submodule of
$\polMatSpace[\cdim][1]$ of rank $\cdim$. Thus, a linearly recurrent
sequence also admits a canonical right generator.

Now, we focus on our algorithmic problem: given a linearly recurrent sequence,
find a minimal matrix generator.  We assume the availability of bounds
$(\degBdl,\degBdr)$ on the degrees of the canonical left and right generators,
which allow us to control the number of terms of the sequence we will access
during the algorithm.  Since taking the Popov form of a reduced matrix does not
change the degree, any minimal left matrix generator $\relbas$ has the same
degree $\deg(\relbas)$ as the canonical left generator: thus, $\degBdl$ is also
a bound on the degree of any minimal left generator. The same remark holds for
$\degBdr$ and minimal right generators.  The bounds $(\degBdl,\degBdr)$
correspond to $(\gamma_1,\gamma_2)$ in \citep[Definitions~4.6~and~4.7]{Turner02}
and $(\delta_l,\delta_r)$ in \citep[Section~4.2]{Villard97a}.  The next result
is similar to \citep[Theorem~4.5]{Turner02}.

\begin{lemma}
  \label{lem:finitely_many_terms}
  Let $\seq = (\seqelt{s})_{s \ge 0} \subset \seqeltSpace$ be linearly
  recurrent and let $\degBdr \in \NN$ be such that the canonical
  right matrix generator of $\seq$ has degree at most $\degBdr$.  Then,
  a vector $\rel =\row{p}_0 + \cdots +\row{p}_{\degBd}\var^\degBd \in \relSpace$ is a left
  relation for $\seq$ if and only if $\row{p}_0 \seqelt{s} + \cdots +
  \row{p}_{\degBd} \seqelt{s + \degBd} = \row{0}$ holds for $s \in
  \{0,\ldots,\degBdr-1\}$.
\end{lemma}
\begin{proof}
  Since the canonical right generator $\relbas
  \in \polMatSpace[\cdim]$ is in column Popov form, we have $\relbas =
  \mat{L}\,\diag{\var^{t_1},\ldots,\var^{t_{\cdim}}} - \mat{Q}$, where
  $\cdeg{\mat{Q}} < \cdeg{\relbas} = (t_1,\ldots,t_{\cdim})$
  termwise and $\mat{L} \in \matSpace[\cdim]$ is unit upper
  triangular. Define the matrix $\mat{U} =
  \diag{\var^{\degBdr-t_1},\ldots,\var^{\degBdr-t_{\cdim}}}
  \mat{L}^{-1}$, which is in $\polMatSpace[\cdim]$ since $\degBdr \ge
  \deg(\relbas) = \max_j t_j$. Then, the columns of the right multiple
  $\relbas \mat{U} = \var^{\degBdr} \mat{I}_\cdim - \mat{Q} \mat{U}$
  are right relations for $\seq$, and we have $\deg(\mat{Q} \mat{U}) <
  \degBdr$. As a consequence, writing $\mat{Q} \mat{U} = \sum_{0 \le k
  < \degBdr} \mat{Q}_k \var^k$, we have $\seqelt{s+\degBdr} =
  \sum_{0 \le k < \degBdr} \seqelt{s+k} \mat{Q}_k$ for all $s \ge 0$.

  Assuming that $\row{p}_0 \seqelt{s} + \cdots + \row{p}_{\degBd} \seqelt{s +
  \degBd} = \row{0}$ holds for all $s \in \{0,\ldots,\degBdr-1\}$, we
  prove by induction that this holds for all $s\in\NN$. Let $s \ge
  \degBdr-1$ and assume that this identity holds for all integers up
  to $s$. Then, the identity concluding the previous paragraph implies
  that
  \begin{align*}
    \sum_{0 \le k \le \degBd} \row{p}_{k} \seqelt{s+1 + k} & =
    \sum_{0 \le k \le \degBd} \row{p}_{k} \left(\sum_{0\le j<\degBdr} \seqelt{s+1+k-\degBdr+j} \mat{Q}_j\right) \\
                                                           & = \sum_{0\le j<\degBdr} 
                                                           \underbrace{\left(\sum_{0 \le k \le \degBd} \row{p}_{k} \seqelt{s+1-\degBdr+j+k}\right)}_{=\, 0 \text{ since } s+1-\degBdr+j \le s} \mat{Q}_j = \row{0},
  \end{align*}
  and the proof is complete.
\end{proof}

The fast computation of matrix generators is usually handled via minimal
approximant bases algorithms \citep{Villard97,Turner02,GioLeb14}; the next
result gives the main idea behind this approach. This result is similar to
\citep[Theorem~4.6]{Turner02} (see also Theorems~4.7 to~4.10 in that reference),
using however the reversal of the input sequence rather than that of the output
matrix generator.

We recall from \citep{BarBul92,BecLab94} that, given a matrix $\sys \in
\polMatSpace[\rdim][\cdim]$ and an integer $d \in \NN$, the set of
\emph{approximants for $\sys$ at order $d$} is defined as
\[
  \appMod{\sys}{d} = \{ \rel \in \relSpace \mid \rel \sys = \row{0} \bmod \var^d \}.
\]
Then, the next theorem shows that relations for $\seq$ can be retrieved as
subvectors of approximants at order about $\degBdl+\degBdr$ for a matrix
involving the first $\degBdl+\degBdr$ entries of $\seq$. 

\begin{theorem}
  \label{thm:mingen_via_appbas}
  Let $\seq = (\seqelt{s})_{s \ge 0} \subset \seqeltSpace$ be
  linearly recurrent and let $(\degBdl,\degBdr) \in \NN^2$ be
  such that the canonical left (resp.~right) matrix generator of
  $\seq$ has degree $\le\degBdl$ (resp.~$\le \degBdr$).  For
  $\degBd>0$, define
  \[
    \sys =
    \begin{bmatrix}
      \sum_{0\le s < \degBd} \seqelt{s} \var^{\degBd-s-1} \\ - \mat{I}_{\cdim}
    \end{bmatrix} \in \polMatSpace[(\rdim+\cdim)][\cdim].
  \]
  Suppose that  $\degBd \ge \degBdr+1$ and let $\mat{B} \in \polMatSpace[(\rdim+\cdim)][(\rdim+\cdim)]$
  be a basis of $\appMod{\sys}{\degBdl+\degBdr+1}$. Then,
  \begin{itemize}
    \item if $\mat{B}$ is in Popov form then its $\rdim\times\rdim$ leading
      principal submatrix is the canonical matrix generator for $\seq$;
    \item if $\mat{B}$ is row reduced then it has exactly $\rdim$ rows of
      degree $\le\degBdl$, and they form a submatrix $[\relbas \;\; \remmat] \in
      \polMatSpace[\rdim][(\rdim+\cdim)]$ of $\mat{B}$ such that $\relbas$ is a
      minimal matrix generator for~$\seq$.
  \end{itemize}
\end{theorem}
\begin{proof}
  We first observe that for any relation $\rel \in \relSpace$ for $\seq$, there exists $\rem \in
  \remSpace$ such that $\deg(\rem) < \deg(\rel)$ and $[\rel \;\; \rem]
  \in \appMod{\sys}{\degBd}$. Indeed, from
  \cref{lem:linearly_recurrent}, if $\rel$ is a relation for $\seq$
  then $\num = \rel \seqpm$ has polynomial entries, where $\seqpm =
  \sum_{s\ge 0} \seqelt{s} \var^{-s-1}$. Then the vector $\rem = -
  \rel (\sum_{s \ge \degBd} \seqelt{s} \var^{\degBd-s-1})$ has
  polynomial entries, has degree less than $\deg(\rel)$, and is such
  that $[\rel \;\; \rem] \sys = \num \var^{\degBd}$.

  Conversely, we show that for any vectors $\rel \in \relSpace$ and $\rem
  \in \remSpace$, if $[\rel \;\; \rem] \in\appMod{\sys}{\degBd}$ and
  $\deg([\rel \;\; \rem])\le\degBd-\degBdr-1$, then $\rel$ is a
  relation for $\seq$. Indeed, if $[\rel \;\; \rem]
  \in\appMod{\sys}{\degBd}$ we have $\rel (\sum_{0\le s< \degBd}
  \seqelt{s} \var^{\degBd-s-1}) = \rem \bmod \var^\degBd$. Since
  $\degBd\ge\degBdr+1$ and $\deg([\rel \;\;
  \rem])\le\degBd-\degBdr-1$, this implies that the coefficients of
  degree $\degBd-\degBdr$ to $\degBd-1$ of $\rel(\sum_{0\le s <
  \degBd} \seqelt{s} \var^{\degBd-s-1})$ are zero. Then,
  \cref{lem:finitely_many_terms} shows that $\rel$ is a relation for
  $\seq$.

  The two items in the theorem follow.
\end{proof}

Using fast approximant basis algorithms, we obtain the main results
from this section. They are stated in slightly more generality than
needed, since in our main algorithm, we will always use $\cdim =
\rdim$.  However, we believe that the more general form stated here
may find further applications. The proof is a direct consequence of
the previous theorem, using the algorithms of
respectively~\citep{GiJeVi03},~\citep{ZhoLab12}, and~\citep{JeaNeiVil18}.
\begin{corollary}\label{coro:cost_approx}
  Let $\seq \subset \seqeltSpace$ be a linearly recurrent sequence and
  let $\degBd = \degBdl+\degBdr+1$, where $(\degBdl,\degBdr) \in
  \NN^2$ are such that the canonical left (resp.~right) matrix
  generator of $\seq$ has degree $\le\degBdl$ (resp.~$\le \degBdr$).
  Then, given $\seqelt{0},\dots,\seqelt{d-1}$, one can compute
  \begin{itemize}
    \item a minimal left matrix generator
      for $\seq$ using $O(\cdim^\omega \M(\degBd)
      \log(\degBd))$ operations in $\field$ if $\cdim \in \Omega(\rdim)$;
    \item a minimal left matrix generator for $\seq$ using 
      $O(\rdim^\omega \M(\cdim\degBd/\rdim) \log(\degBd))$
      operations in $\field$ if $\cdim \in O(\rdim)$. This bound
      assumes $\M(kd) \in O(k^{\omega-1} \M(d))$; this holds in
      particular if $d \mapsto \M(d)$ is quasi-linear.
    \item the canonical left matrix generator for $\seq$
      using $O((\rdim+\cdim)^{\omega} \M(\frac{\cdim\degBd}{\rdim+\cdim})
      \log(\frac{\cdim\degBd}{\rdim+\cdim})^2)$ operations in $\field$, assuming
      \(\rdim+\cdim \le \cdim\degBd\).
  \end{itemize}
\end{corollary}
In particular, when $\cdim = \rdim$, we can find a minimal
left matrix generator of $\seq$ in $O(\rdim^\omega \M(\degBd)
\log(\degBd))$ operations in $\field$.

%%%%%%%%%%%%%%%%%%%%%%%%%%%%%%%%%%%%%%%%%%%%%%%%%%%%%%%%%%%%

\subsection{Application to the block Wiedemann algorithm}\label{ssec:appliW}

We next apply the results seen above to a particular class of
matrix sequences, namely the Krylov sequences used in Coppersmith's block
Wiedemann algorithm~\citep{Coppersmith94}. In what follows,
the integer $\cdim$ of the previous subsection is now taken equal to $\rdim$.


Let $\mM$ be in $\mathbb{K}^{D \times D}$ and $\mU,\mV \in
\mathbb{K}^{D \times m}$ be two blocking matrices for some $m\le D$. We can
then define
the Krylov sequence $\seq_{\mU,\mV}=(\seqelt{s,\mU,\mV})_{s \ge 0} \subset
\matSpace[m]$ by
$$\seqelt{s,\mU,\mV} = \mUt \mM^s \mV, \quad s \ge 0.$$ This
sequence is linearly recurrent, since the minimal polynomial of $\mM$
is a scalar relation for it. The following theorem states some useful
properties of any minimal left generator of $\seq_{\mU,\mV}$, with in
particular a bound on its degree, for generic choices of $\mU$ and
$\mV$; we also state properties of the invariant factors of such a
generator.  These results are not new, as all statements can be
found in~\citep{Villard97a} and~\citep{KaVi04} (see also~\citep{Kaltofen95}
for an analysis of the block Wiedemann algorithm). We chose to give a
close-to-self-contained presentation, that relies on these
two references for the key properties.

We let $s_1, \dots, s_r$ be the nontrivial invariant factors of $T\mI_D
- \mM$, ordered such that $s_i$ divides $s_{i-1}$ for \(2 \le i \le r\), and
let $d_i = $ deg$(s_i)$ for all $i$; for $i > r$, we let $s_i$ = 1,
with $d_i = 0$.  We define $\nu = d_1 + \cdots + d_m \le D$ and
$\delta = \lceil \nu / m \rceil \le \lceil D / m \rceil$.

\begin{theorem}
  \label{randXY}
  For a generic choice of $\mU$ and $\mV$ in $\K^{D \times m}$, the
  following holds.  Let $\mat{P}_{\mU,\mV}$ be a minimal left
  generator for $\seq_{\mU,\mV}$ and denote by $\sigma_1, \dots,
  \sigma_k$ the invariant factors of $\mat{P}_{\mU,\mV}$, for some $k
  \le m$, ordered as above, and write $\sigma_{k+1}=\cdots=\sigma_m=1$.
  Then,
  \begin{itemize}
    \item $\mat{P}_{\mU,\mV}$ is a minimal left generator for the
      matrix sequence $\seqL_{\mU} = (\mUt \mM^s)_{s \ge 0}$;
    \item $\mat{P}_{\mU,\mV}$ has degree $\delta$;
    \item $s_i = \sigma_i$ for $1 \le i \le m$.
  \end{itemize}
\end{theorem}
\begin{proof}
  Without loss of generality, we assume that $\mat{P}_{\mU,\mV}$ is the
  canonical left generator for $\seq_{\mU,\mV}$. The sequence $\seqL_{\mU}$ is
  linearly generated as well, and we let $\mat{P}_{\mU}$ be the canonical left
  generator for it.  We denote by $\langle \mU \rangle$ the vector space
  generated by the rows of $\mUt$, $\mUt \mM$, $\mUt \mM^2$, $\dots$, and we write
  $D_\mU=\dim(\langle \mU \rangle)$.

  First, we prove that for any $\mU$ in $\K^{D \times m}$, for a
  generic $\mV$ in $\K^{D\times m}$,
  $\mat{P}_{\mU}=\mat{P}_{\mU,\mV}$.  Indeed,
  by~\citep[Lemma~4.2]{Villard97a} (which considers right-generators),
  there exist matrices $\mat{J}_\mU$ in $\K^{D\times D_\mU}$ and
  $\mN_\mU \in \K^{D_\mU \times D_\mU}$, with $\mat{J}_\mU$ of full
  rank $D_\mU$, and where $\mN_\mU$ is a matrix of the restriction of
  $\trsp{\mM}$ to $\langle \mU \rangle$, such that
  $\mat{P}_{\mU,\mV}=\mat{P}_{\mU}$ if and only if the dimension of
  the vector space generated by the rows of the span of
  $\trsp{\mV}\mat{J}_\mU, \trsp{\mV} \mat{J}_\mU \mN_\mU, \trsp{\mV}
  \mat{J}_\mU \mN_\mU^2,\dots$ is equal to $D_\mU$.
  %% , with $\mB_\mV=\trsp{\mM}_\mV$
  %% and $\mZ=\mat{P}_\trsp{\mV} \mU \in \K^{D_\mV \times m}$.

  By construction, one can find a basis of $\langle \mU \rangle$ in
  which the matrix of $\mN_\mU$ is block-companion, with $\mu \le m$
  blocks (take the $\mN_\mU$-span of the first column of $\mU$, then of
  the second column, working modulo the previous vector space, etc.). 
  Thus, $\trsp{\mN}_\mU$ is similar to a block-companion matrix with $\mu$
  blocks as well; since $\mat{J}_\mUt \mV$ has $m$ columns, the span of
  the columns of
  $\mat{J}_\mUt \mV,  \trsp{\mN}_\mU\mat{J}_\mUt \mV,\dots$
  has full dimension $D_\mU$ for a
  generic $\mat{J}_\mUt \mV$, and thus for a generic $\mV$, since $\mat{J}_\mU$ has rank
  $D_\mU$. As a result, as claimed, for generic choices of $\mU$ and $\mV$,
  $\mat{P}_{\mU,\mV}=\mat{P}_{\mU}$.

  Let us next introduce a matrix $\scrU$ of indeterminates of size $D
  \times m$, and let $\mat{P}_{\scrU}$ be the canonical matrix generator
  of the ``generic'' sequence $(\trsp{\scrU} \mM^s)_{s \ge
  0}$. The notation $\langle \scrU \rangle$ and $D_\scrU$ are
  defined as above.  In particular,
  by~\citep[Proposition~6.1]{Villard97a}, the canonical matrix generator
  $\mat{P}_{\scrU}$ has degree $\delta$ and determinantal
  degree $\nu$.

  %%  Indeed,
  %%   $\langle \scrV\rangle$ is the span of $K_\scrV=[\scrV ~ \mM \scrV ~
  %%     \cdots ~ \mM^{N-1} \scrV]$, whereas $\langle \mV \rangle$ is the
  %%   span of $[\mV ~ \mM \mV ~ \cdots ~\mM^{N-1} \mV]$.
  %% Take a
  %%  maximal nonzero minor $\mu$ of $K_\scrV$; as soon as $\mu(Y)\ne 0$,
  %%  we have equality of the dimensions.
  Now, for a generic $\mU$ in $\K^{D\times m}$, $D_\mU=D_\scrU$. On
  the other hand, by~\citep[Lemma~4.3]{Villard97a}, for any $\mU$
  (including $\scrU$), the degree of $\mat{P}_{\mU}$ is equal to the
  first index $d$ such that the dimension of the span of the rows of
  $\mUt,\mUt \mM,\dots, \mUt \mM^{d-1}$ is $D_\mU$. As
  a result, for generic $\mU$, $\mat{P}_{\mU}$ and $\mat{P}_{\scrU}$
  have the same degree, that is, $\delta$.  Taking the first item into
  account, we see that the second item is proved as well.

  We conclude by proving that for generic $\mU,\mV$, the invariant
  factors $\sigma_1,\dots,\sigma_m$ of $\mat{P}_{\mU,\mV}$ are
  $s_1,\dots,s_m$.  By~\citep[Theorem~2.12]{KaVi04} (which considers 
  right-generators as well), for any $\mU$ and
  $\mV$ in $\K^{D\times m}$, for $i=1,\dots,m$, the $i$-th invariant
  factor $\sigma_i$ of $\mat{P}_{\mU,\mV}$ divides $s_i$, so that
  $\deg(\det(\mat{P}_{\mU,\mV}))\le\nu$, with equality if and only
  if $\sigma_i=s_i$ for all $i \le m$.

  For $\mU$ as above and any integers $e,d$, we let ${\rm
  Hk}_{e,d}(\mU)$ be the block Hankel matrix
  $$ {\rm Hk}_{d,e}(\mU) =
  \begin{bmatrix}
    \mUt \\     \mUt \mM \\     \mUt \mM^2 \\ \vdots  \\      \mUt\mM^{d-1}
  \end{bmatrix}
  \begin{bmatrix}
    \mI_D & \mM & \mM^2 &\cdots& \mM^{e-1}
  \end{bmatrix}.
  $$ By~\citep[Eq.~(2.6)]{KaVi04}, ${\rm rank}({\rm Hk}_{d,e}(\mU)) =
  \deg(\det(\mat{P}_{\mU}))$ for $d \ge \deg(\mat{P}_{\mU})$
  and $e \ge D$.  We take $e=D$, so that ${\rm rank}({\rm
  Hk}_{d,D}(\mU)) = \deg(\det(\mat{P}_{\mU}))$ for $d \ge
  \deg(\mat{P}_{\mU})$. On the other hand, the sequence ${\rm
  rank}({\rm Hk}_{d,D}(\mU))$ is constant for $d \ge D$; as a
  result, ${\rm rank}({\rm Hk}_{D,D}(\mU)) =
  \deg(\det(\mat{P}_{\mU}))$. For the same reason, we also have ${\rm
  rank}({\rm Hk}_{D,D}(\scrU)) = \deg(\det(\mat{P}_{\scrU}))$, so that
  for a generic $\mU$, $\mat{P}_{\mU}$ and $\mat{P}_{\scrU}$
  have the same determinantal degree, that is, $\nu$.  As a result,
  for generic $\mU$ and $\mV$, we also have
  $\deg(\det(\mat{P}_{\mU,\mV}))=\deg(\det(\mat{P}_{\mU}))=\nu$, and the conclusion follows.
\end{proof} 

In particular, combining \cref{coro:cost_approx} and \cref{randXY}, we
deduce that for generic $\mU,\mV$, given the first $2 \lceil D/m
\rceil+1$ terms of the sequence $\seq_{\mU,\mV}$, we can recover a
minimal matrix generator  $\mat{P}_{\mU,\mV}$ of it using $O(m^{\omega-1} \M(D) \log(D))
\subset \softO{m^{\omega-1} D}$ operations in $\K$.

Besides, the theorem shows that for generic $\mU$ and $\mV$, the
largest invariant factor $\sigma_1$ of $\mat{P}_{\mU,\mV}$ is the
minimal polynomial $\minpoly=s_1$ of $\mM$.  Given $\mat{P}_{\mU,\mV}$,
$\minpoly$ can thus be computed by solving a linear system
$\mat{P}_{\mU,\mV} \, \col{x} = \col{y}$, where $\col{y}$ is a vector of
$m$ randomly chosen elements in $\K$: for a generic choice of
$\col{y}$, the least common multiple of the denominators of the
entries of $\col{x}$ is $\minpoly$.  Thus, both $\minpoly$ and $\col{x}$
can be computed using
high-order lifting \citep[Algorithm~5]{Stor03} on input
$\mat{P}_{\mU,\mV}$ and $\col{y}$; by
\citep[Corollary~16]{Stor03}, this costs
\begin{align}
  & O\Bigg(m^{\omega} \M(D/m) \log(m) + \sum_{0 \le i \le \log_2(m)} 4^i (2^{-i}m)^\omega \M(D/m) \nonumber \\
  & \quad\qquad + m^2 \M(D/m)\log(D/m)\log(m) + m \M(D) \log(D) \Bigg) \nonumber \\
  & \subset O(m^\omega \M(D/m) \log(m) + m \M(D)\log(D)\log(m))  \nonumber \\
  & \subset O(m^{\omega-1} \M(D) \log(D)\log(m))  \label{eqn:hol_cost}\\
  & \subset \softO{m^{\omega-1}D} \nonumber
\end{align}
operations in $\K$.
The latter algorithm is randomized, since it chooses a random point in $\K$ to
compute (parts of) the expansion of the inverse of
$\mat{P}_{\mU,\mV}$.

%% Note: since $\mat{P}_{\mU,\mV}$ is reduced, we could use the expansion of the reversal

An alternative solution would be to compute the Smith form of
$\mat{P}_{\mU,\mV}$ using an algorithm such as that in
\citep[Section~17]{Stor03}, yet the cost would be slightly higher on the
level of logarithmic factors.

%%%%%%%%%%%%%%%%%%%%%%%%%%%%%%%%%%%%%%%%%%%%%%%%%%%%%%%%%%%%

\subsection{Computing a scalar numerator}\label{ssec:scalar_numer}

Let us keep the notation of the previous subsection.  The main
advantage of using the block Wiedemann algorithm is that it allows one
to distribute the bulk of the computation in a straightforward manner:
on a platform with $m$ processors (or cores, \dots), one would
typically compute the $D \times m$ matrices $\mat{L}_s=\mUt\mM^s$
for $s=0,\dots,2\lceil D/m \rceil$ by having each processor compute a
sequence $\row{u}_i \mM^s$, where $\row{u}_i$ is the $i$th row of
$\mUt$. From these values, we then deduce the matrices
$\seqelt{s,\mU,\mV}=\mUt \mM^s \mV = \mat{L}_s\mV$ for
$s=0,\dots,2\lceil D/m \rceil$. Note that in our description, we
assume for simplicity that memory is not an issue, so that we can
store all needed elements from e.g. sequence $\mat{L}_s$; we discuss
this in more detail in \cref{ssec:mainalgo}.

Our main algorithm will also resort to scalar numerators of
the form $\Omega((\row{u}_i \mM^s \col{w})_{s \ge 0},
\minpoly)$, where $\row{w}$ is a given vector in $\K^{D \times 1}$ and
$\minpoly$ is the minimal polynomial of $\mM$. Since $\minpoly$ may
have degree $D$, and then $\Omega((\row{u}_i \mM^s \col{w})_{s \ge 0},
\minpoly)$ itself may have degree $D-1$,
the definition of $\Omega$ suggests that we
may need to compute up to $D$ terms of the sequence $\row{u}_i \mM^s
\col{w}$, which we would of course like to avoid. We now present an
alternative solution which involves solving a univariate polynomial linear system
and computing a matrix numerator, but only uses the sequence elements
$\mat{L}_s= \mUt \mM^s$ for $s=0,\dots,\lceil D/m \rceil-1$
which have already been computed.

Fix $i$ in $1,\dots,m$ and let $\row{a}_i$ be the row vector defined
by $$\row{a}_i =[0~\cdots~0~\minpoly~0~\cdots~0]  (\mat{P}_{\mU,\mV})^{-1} ,$$
where the minimal polynomial $\minpoly$ appears at the $i$th entry  of the
left-hand row vector. 
\begin{lemma}\label{utilde}
  For generic $\mU,\mV$, the row vector $\row{a}_i$ has polynomial
  entries of degree at most~$\deg(P) \le D$.
\end{lemma}
\begin{proof}
  For generic $\mU,\mV$, we saw that $\minpoly$ is the largest invariant factor
  of $ \mat{P}_{\mU,\mV}$; thus, the product $\minpoly\
  (\mat{P}_{\mU,\mV})^{-1}$ has polynomial entries. Since $\row{a}_i$ the $i$th
  row of this matrix, $\row{a}_i$ has polynomial entries.  Now, since
  $\mat{P}_{\mU,\mV}$ is reduced, the predictable degree property
  \citep[Theorem~6.3-13]{Kailath80} holds; it implies that each entry of
  $\row{a}_i$ has degree at most the maximum of the degrees of the entries of
  $\row{a}_i\mat{P}_{\mU,\mV}$. This maximum is $\deg(\minpoly)$.
\end{proof}
To compute $\row{a}_i$, we use again Storjohann's high-order lifting;
according to \cref{eqn:hol_cost}, the cost is $ O(m^{\omega-1} \M(D)
\log(D) \log(m)) \subset \softO{m^{\omega-1}D}$ operations in $\K$.
Once $\row{a}_i$ is known, the following lemma shows that we can
recover the scalar numerator $\Omega((\row{u}_i \mM^s \col{w})_{s \ge
0}, \minpoly)$ as a dot product.
\begin{lemma}\label{lemma:omegaOmega}
  For a generic choice of $\mU$ and $\mV$, and for any $\col{w}$ in
  $\K^{D \times 1}$, $ \mat{P}_{\mU,\mV}$ is a nonsingular matrix of
  relations for the sequence $\mat{\mathcal{E}} =
  (\col{e}_s)_{s\ge0}$, with $\col{e}_s=\mUt \mM^s\, \col{w}$
  for all $s$, and we have
  \begin{align}\label{eq:OmegaOmega}
    [~\Omega((\row{u}_i \mM^s \col{w})_{s \ge 0}, \minpoly)~] = \row{a}_i\cdot \mat{\Omega}(\mat{\mathcal{E}}, \mat{P}_{\mU,\mV})
  \end{align}
  with $\row{a}_i$ in $\K[\var]^{1 \times m}$ and 
  $\mat{\Omega}(\mat{\mathcal{E}} , \mat{P}_{\mU,\mV}) \in \K[\var]^{m \times 1}$.
\end{lemma}
\begin{proof}
  The first item in \cref{randXY} shows that for a generic choice of
  $\mU$ and $\mV$, $\mat{P}_{\mU,\mV}$ cancels the sequence $(\mUt
  \mM^s)_{s \ge 0}$, and thus the sequence $\mat{\mathcal{E}}$ as well;
  this proves the first point. Then, the equality
  in \cref{eq:OmegaOmega} directly follows from the definitions:
  \begin{align*}
    [~ \Omega((\row{u}_i \mM^s \col{w})_{s \ge 0}, \minpoly)~]  &= [~\minpoly~]\ \sum_{s \ge 0} \frac{\row{u}_i \mM^s \col{w}}{T^{s+1}}\\
                                                                &=  [~0~\cdots~0~\minpoly~0~\cdots~0~]\  \sum_{s \ge 0} \frac{\mUt \mM^s \col{w}}{T^{s+1}}\\
                                                                &=  [~0~\cdots~0~\minpoly~0~\cdots~0~]\ (\mat{P}_{\mU,\mV})^{-1} \mat{P}_{\mU,\mV} \sum_{s \ge 0} \frac{\mUt \mM^s \col{w}}{T^{s+1}}\\
                                                                &=  \row{a}_i\cdot \mat{\Omega}(\mat{\mathcal{E}}, \mat{P}_{\mU,\mV}).
                                                                \qedhere
  \end{align*}
\end{proof}

The algorithm to compute the scalar numerator $\Omega((\row{u}_i \mM^s
\col{w})_{s \ge 0}, \minpoly)$ follows; in the algorithm, we
assume that we know $\mat{P}_{\mU,\mV}$, $P$, $\row{a}_i$, and the
matrices $\mat{L}_s=\mUt \mM^s \in \K^{m \times D}$ for
$s=0,\dots,\lceil D/m \rceil-1$.

\begin{algorithm}[H]
  \caption{{\sf ScalarNumerator}($\mat{P}_{\mU,\mV}, \minpoly, \row{w}, i, \row{a}_i,(\mat{L}_s)_{0 \le s < \lceil D/m\rceil}$)}
  {\bf Input:} \vspace{-0.5em}
  \begin{itemize}
    \item a minimal generator $\mat{P}_{\mU,\mV}$ of $(\mUt \mM^s \mV)_{s \ge 0}$
    \item the minimal polynomial $P$ of $\mM$
    \item $\row{w}$ in $\K^{D \times 1}$
    \item $i$ in $\{1,\dots,m\}$
    \item $\row{a}_i =  [0~\cdots~0~\minpoly~0~\cdots~0]  (\mat{P}_{\mU,\mV})^{-1} \in \K[T]^{1\times m}$
      where $\minpoly$ appears at the $i$th entry
    \item $\mat{L}_s = \mUt \mM^s \in \K^{m\times D}$, for $s=0,\dots,\lceil D/m\rceil-1$
  \end{itemize}
  {\bf Output:}  \vspace{-0.5em}
  \begin{itemize}
    \item         the scalar numerator $\Omega((\row{u}_i \mM^s \col{w})_{s \ge 0}, \minpoly) \in \K[T]$
  \end{itemize}
  \begin{enumerate}
    \item compute $\col{E}_s = \mat{L}_s \col{w} \in \K^{m \times 1}$ for $s=0,\dots,\lceil D/m\rceil-1$
    \item use these values to compute the matrix numerator $ \mat{\Omega}(\mat{\mathcal{E}}, \mat{P}_{\mU,\mV}) \in \K[T]^{m \times 1}$ 
      by \cref{eq:computeOmega}
    \item {\bf return} the entry of the $1 \times 1$ matrix $\row{a}_i\cdot  \mat{\Omega}(\mat{\mathcal{E}}, \mat{P}_{\mU,\mV})$ 
  \end{enumerate}
  \label{algo:scalar_numerator}
\end{algorithm}

Computing the first $\lceil D/m \rceil$ values of the sequence
$\mat{\mathcal{E}}=(\col{e}_s)_{s \ge 0}$ is done by using the
equality $\col{E}_s = \mat{L}_s \col{w}$ and takes $O(D^2)$ field
operations. Then, applying the cost estimate given after
\cref{eq:computeOmega}, we see that we have enough terms to compute
$\mat{\Omega}(\mat{\mathcal{E}}, \mat{P}_{\mU,\mV}) \in \K[T]^{m
\times 1}$ and that it takes $O(m^2 \M(D/m)) \subset O(m \M(D))$ operations in
$\K$. Then, the dot product with $\row{a}_i$ takes $O(m \M(D))$
operations, since both vectors have size $m$ and entries of degree at
most $D$. Thus, the runtime is 
\(O(D^2 + m \M(D)) \subset \softO{D^2}\)
operations in $\K$.


%%%%%%%%%%%%%%%%%%%%%%%%%%%%%%%%%%%%%%%%%%%%%%%%%%%%%%%%%%%%
%%%%%%%%%%%%%%%%%%%%%%%%%%%%%%%%%%%%%%%%%%%%%%%%%%%%%%%%%%%%
%%%%%%%%%%%%%%%%%%%%%%%%%%%%%%%%%%%%%%%%%%%%%%%%%%%%%%%%%%%%

\section{Sequences associated to a zero-dimensional ideal}\label{sec:seq0}

We now focus on our main question: computing a zero-dimensional
parametrization of an algebraic set of the form $V=V(I)$, for some
zero-dimensional ideal $I$ in $\K[X_1,\dots,X_n]$. 

We write $V=\{\balpha_1,\dots,\balpha_\dg\},$ with $\balpha_i=(\alpha_{i,1},\dots,\alpha_{i,n}) \in \Kbar{}^n$ for
all $i$.  We also let $\D$ be the dimension of
$\residueI=\K[X_1,\dots,X_n]/I$, so that $\dg \le \D$, and {\em we
assume that ${\rm char}(\K)$ is greater than $D$}.

In this section, we recall and expand on results from the appendix
of~\citep{BoSaSc03}, with the objective of computing a zero-dimensional
parametrization of $V$. These results were themselves inspired by 
those in~\citep{Rouillier99}, the latter being devoted to computations
with the trace linear form $\mathrm{tr}: \residueI\to\K$.

At this stage, we do not discuss data structures or complexity (this
is the subject of the next section); the algorithm in this
section simply describes what polynomials should be computed in order
to obtain a zero-dimensional parametrization.

%%%%%%%%%%%%%%%%%%%%%%%%%%%%%%%%%%%%%%%%%%%%%%%%%%%%%%%%%%%%

\subsection{The structure of the dual}\label{ssec:dual}

For $i$ in $\{1,\dots,\dg\}$, let $\residueI_i$ be the local algebra at
$\balpha_i$, that is $\residueI_i=\Kbar[X_1,\dots,X_n]/I_i$, with $I_i$ the
$\m_{\balpha_i}$-primary component of $I$. By the Chinese Remainder
Theorem, $\residueI\otimes_\K \Kbar=\Kbar[X_1,\dots,X_n]/I$ is isomorphic to
the direct product $\residueI_1\times \cdots \times \residueI_\dg$.  We let $N_i$ be the
{\em nil-index} of $\residueI_i$, that is, the maximal integer $N$ such that
$\m_{\alpha_i}^N$ is not contained in $I_i$; for instance, $N_i=0$ if
and only if $\residueI_i$ is a field, if and only if $\balpha_i$ is a
nonsingular root of $I$. We also let $\D_i=\dim_\Kbar(\residueI_i)$, so that
we have $D_i \ge N_i$ and $\D=\D_1 + \cdots + \D_\dg$.

The sequences we consider below are of the form $(\ell(\lf^s))_{s \ge
0}$, for $\ell$ a $\K$-linear form $\residueI \to \K$ and $\lf$ in $\residueI$
(we will often write $\ell \in {\rm hom}_\K(\residueI,\K)$). For
such sequences, the following standard result will be useful
(see e.g.~\citep[Propositions~1 \& 2]{BoSaSc03} for a proof).
\begin{lemma}\label{lemma:minpoly}
  Let $\lf$ be in $\residueI$ and let $P \in \K[T]$ be its minimal
  polynomial. For a generic choice of $\ell$ in ${\rm hom}_\K(\residueI,\K)$,
  $P$ is the minimal polynomial of the sequence $(\ell(\lf^s))_{s \ge
  0}$.
\end{lemma}

The following results are classical; they go back
to~\citep{Macaulay16}, and have been used in computational algebra
since the 1990's~\citep{MaMoMo96,Mourrain97}. Fix $i$ in $1,\dots,\dg$.
There exists a basis of the dual ${\rm hom}_\Kbar(\residueI_i,\Kbar)$
consisting of linear forms $(\lambda_{i,j})_{1\le j \le \D_i}$ of the
form
$$\lambda_{i,j}: f \mapsto (\Lambda_{i,j}(f))(\balpha_i),$$
where $\Lambda_{i,j}$ is the operator
$$f \mapsto \Lambda_{i,j}(f) = \sum_{\mu=(\mu_1,\dots,\mu_n) \in
S_{i,j}} c_{i,j,\mu} \frac{ \partial^{\mu_1 + \cdots + \mu_n} f}
{\partial X_1^{\mu_1} \cdots \partial X_n^{\mu_n}},$$ for some finite
subset $S_{i,j}$ of $\N^n$ and nonzero constants $c_{i,j,\mu}$ in
$\Kbar$. 
%% Let $w_{i,j}$ be the maximum of all $|\mu|$ for $\mu$ in
%% $S_{i,j}$, with $|\mu| = \mu_1 +\cdots + \mu_n $.
%% By~\citep[Lemma~3.3]{Mourrain97}, we have that $\max_j w_{i,j} =N_i$
%% for all $i$.
For instance, when $\balpha_i$ is nonsingular, we have $D_i=1$, so
there is only one function $\lambda_{i,j}$, namely $\lambda_{i,1}$; we
write it $\lambda_{i,1}(f) = f(\balpha_i)$.

More generally, we can always take $\lambda_{i,1}$ of the form
$\lambda_{i,1}(f) = f(\balpha_i)$; for $j>1$, we can then also assume
that $S_{i,j}$ does not contain $\mu=(0,\dots,0)$ (that is, all terms
in $\Lambda_{i,j}$ have order $1$ or more). Thus, introducing new
variables $(U_{i,j})_{j =1,\dots,D_i}$, we deduce the existence of
nonzero homogeneous linear forms $P_{i,\mu}$ in
$(U_{i,j})_{j=1,\dots,D_i}$ such that for any $\ell$ in ${\rm
hom}_\Kbar(\residueI_i,\Kbar)$, there exists $\bu_i=(u_{i,j}) \in
\Kbar{}^{D_i}$ such that we have
\begin{align}\label{ell_param}
  \ell: f \mapsto \ell(f)
&= \sum_{j=1}^{D_i} u_{i,j} \lambda_{i,j}(f)\nonumber\\
&= \sum_{j=1}^{D_i} u_{i,j} \big(\Lambda_{i,j}(f)\big)(\balpha_i)\nonumber\\
&= \sum_{j=1}^{D_i} u_{i,j}
\sum_{\mu=(\mu_1,\dots,\mu_n) \in
S_{i,j}} c_{i,j,\mu} \frac{ \partial^{\mu_1 + \cdots + \mu_n} f}
{\partial X_1^{\mu_1} \cdots \partial X_n^{\mu_n}}(\balpha_i)\nonumber\\
&= \sum_{\mu=(\mu_1,\dots,\mu_n) \in S_i} P_{i,\mu}(\bu_i)
\frac{ \partial^{\mu_1 + \cdots + \mu_n} f}
{\partial X_1^{\mu_1} \cdots \partial X_n^{\mu_n}}(\balpha_i),
\end{align}
where $S_i$ is  the union of $S_{i,1},\dots,S_{i,D_i}$,
with in particular $P_{i,(0,\dots,0)}=U_{i,1}$ and where $P_{i,\mu}$
depends only on $(U_{i,j})_{j =2,\dots,D_i}$ for all $\mu$ in $S_i$,
$\mu \ne (0,\dots,0)$. Explicitly, we can write $P_{i,\mu}=\sum_{j\in
\{1,\dots,D_i\} \;\mid\; \mu \in S_{i,j}} c_{i,j,\mu}
U_{i,j}$. 

Fix $\ell$ nonzero in ${\rm hom}_\Kbar(\residueI_i,\Kbar)$, written
as in \cref{ell_param}. We can then
define its {\em order} $w$ and {\em symbol} $\pi$. The former is
the maximum of all $|\mu|=\mu_1+\cdots+\mu_n$ for
$\mu=(\mu_1,\dots,\mu_n)$ in $S_i$ such that $P_{i,\mu}(\bu_i)$ is
nonzero; by~\citep[Lemma~3.3]{Mourrain97} we have $w \le
N_i-1$. Then, we let
$$\pi =\sum_{\mu \in S_i,\ |\mu|=w}P_{i,\mu}(\bu_i) X_1^{\mu_1} \cdots
X_n^{\mu_n}$$ be the {\em symbol} of $\ell$; by construction,
this is a nonzero polynomial. In the following paragraphs, we will
need the next lemma.

\begin{lemma}\label{lemma:symbol0}
  Fix $i$ in $\{1,\dots,\dg\}$. For a generic choice of $\ell$ in
  ${\rm hom}_\Kbar(\residueI_i,\Kbar)$ and of $t_1,\dots,t_n$ in $\Kbar{}^n$,
  $\pi_i(t_1,\dots,t_n)$ is nonzero.
\end{lemma}
\begin{proof}
  Let $\Omega$ be the maximum of all $|\mu|=\mu_1+\cdots+\mu_n$ for
  $\mu=(\mu_1,\dots,\mu_n)$ in $S_i$, and define 
  $$\Pi =\sum_{\mu \in S_i,\ |\mu|=\Omega}P_{i,\mu} X_1^{\mu_1}
  \cdots X_n^{\mu_n} \in
  \Kbar[U_{i,1},\dots,U_{i,D_i},X_1,\dots,X_n];$$ this is by
  construction a nonzero polynomial.  As a result, for a generic choice of
  $\bu_i=(u_{i,1},\dots,u_{i,D_i})$, which defines a linear form
  $\ell$ in ${\rm hom}_\Kbar(\residueI_i,\Kbar)$ as in \cref{ell_param},
  and of $t_1,\dots,t_n$ in $\Kbar{}^n$, the value
  $\Pi(u_{i,1},\dots,u_{i,D_i},t_1,\dots,t_n)$ is nonzero. Thus, the symbol of
  such a linear form $\ell$ is $\pi
  =\sum_{\mu \in S_i,\ |\mu|=\Omega}P_{i,\mu}(\bu_i) X_1^{\mu_1}
  \cdots X_n^{\mu_n},$ and $\pi(t_1,\dots,t_n)$ is then nonzero.
\end{proof}

Finally, we say a word about global objects.  Fix a linear form $\ell:
\residueI \to \K$. By the Chinese Remainder Theorem, there exist unique
$\ell_1,\dots,\ell_\dg$, with $\ell_i$ in ${\rm hom}_\Kbar(\residueI_i,\Kbar)$
for all $i$, such that the extension $\ell_\Kbar: \residueI\otimes_\K \Kbar
\to \Kbar$ decomposes as $\ell_\Kbar = \ell_1 + \cdots + \ell_\dg$.
Note that formally, we should write 
$\ell_\Kbar = \ell_1 \circ \phi_1 + \cdots + \ell_\dg \circ \phi_\dg$,
where for all $i$, $\phi_i$ is the canonical projection $\residueI \to \residueI_i$;
we will however omit these projection operators for simplicity.

We call {\em support} of $\ell$ the subset $\mathfrak{S}$ of
$\{1,\dots,\dg\}$ such that $\ell_i$ is nonzero exactly for $i$ in
$\mathfrak{S}$.  As a consequence, for all $f$ in $\residueI$, we have
\begin{equation}\label{eq:fui}
  \ell(f) = \ell_1(f) + \cdots + \ell_\dg(f)
  =  \sum_{i \in \mathfrak{S}} \ell_i(f).
\end{equation}
For $i$ in $\mathfrak{S}$, we denote by $w_i$ and $\pi_i$ respectively
the order and the symbol of $\ell_i$. For such a subset $\mathfrak{S}$
of $\{1,\dots,\dg\}$, we also write $\residueI_\mathfrak{S}=\prod_{i
  \in \mathfrak{S}} \residueI_i$ and $V_\mathfrak{S}= \{\balpha_i \mid
i \in \mathfrak{S}\}$.

%%%%%%%%%%%%%%%%%%%%%%%%%%%%%%%%%%%%%%%%%%%%%%%%%%%%%%%%%%%%

\subsection{A fundamental formula}  \label{ssec:genseries}

Let $\lf$ be in $\residueI$ and $\ell$ in ${\rm hom}_\K(\residueI,\K)$.  The
sequences $(\ell(\lf^s))_{s\ge 0}$, and more generally the sequences $(\ell(v
\lf^s))_{s\ge 0}$ for $v$ in $\residueI$, are the core ingredients of our
algorithm.  This is justified by the following lemma, which gives a
description of generating series of the form $\sum_{s \ge 0} \ell(v
\lf^s)/T^{s+1}$. A slightly less precise version of it is in~\citep{BoSaSc03};
the more explicit expression given here will be needed in the last section of
this paper.

\begin{lemma}\label{lemma:formula}
  Let $\ell$ be in ${\rm hom}_\K(\residueI,\K)$, with support $\mathfrak{S}$,
  and let $\{\pi_i \mid i \in \mathfrak{S}\}$ and $\{w_i \mid i \in
  \mathfrak{S}\}$ be the symbols and orders of $\{\ell_i \mid i \in \mathfrak{S}\}$,
  for $\{\ell_i \mid i \in \mathfrak{S}\}$ as in \cref{ssec:dual}.

  Let $\lf=t_1 X_1 + \cdots +t_n X_n$, for some $t_1,\dots,t_n$ in $\K$
  and let $v$ be in $\K[X_1,\dots,X_n]$. Then, we have the equality
  \begin{align}\label{eq:sumgenseries}
    \sum_{s \ge 0} \frac{\ell(v \lf^s)}{T^{s+1}} = \sum_{i \in \mathfrak{S}}
    \frac{ v(\balpha_i)\, w_i!\, \pi_{i}(t_1,\dots,t_n) +
    (T-\lf(\balpha_i))A_{v,i}} {(T-\lf(\balpha_i))^{w_{i}+1}},
  \end{align}
  for some polynomials $\{A_{v,i} \in \Kbar[T] \mid i \in \mathfrak{S}\}$ which
  depend on the choice of $v$ and are such that $A_{v,i}$ has degree less than
  $w_i$ for all $i$ in $\mathfrak{S}$.
\end{lemma}
\begin{proof}
  Take $v$ and $\lf$ as above. Consider first an operator of the form $f
  \mapsto \frac{ \partial^{|\mu|} f} {\partial X_1^{\mu_1} \cdots
  \partial X_n^{\mu_n}}$, where we write
  $|\mu|=\mu_1+\cdots+\mu_n$. Then, we have the following generating
  series identities, with coefficients in $\K(X_1,\dots,X_n)$:
  \begin{align*}
    \sum_{s \ge 0} 
    \frac{ \partial^{|\mu|} ( v \lf^s )} {\partial X_1^{\mu_1} \cdots
    \partial X_n^{\mu_n}}
    &\frac{1}{T^{s+1}} 
    =  \sum_{s \ge 0} 
    \frac{ \partial^{|\mu|} (v \lf^s/T^{s+1})} {\partial X_1^{\mu_1} \cdots
    \partial X_n^{\mu_n}}\\
    &=  
    \frac{ \partial^{|\mu|} } {\partial X_1^{\mu_1} \cdots
    \partial X_n^{\mu_n}}
    \left (\sum_{s \ge 0} \frac{v \lf^s}{T^{s+1}}\right ) \\
    &= \frac{ \partial^{|\mu|} } {\partial X_1^{\mu_1} \cdots
    \partial X_n^{\mu_n}}
    \left (\frac v{T-\lf} \right ) \\
    &= \left (v\, |\mu|!\,    \frac {1}{(T-\lf)^{|\mu|+1}} \prod_{1 \le k \le n} 
      \left (\frac{ \partial \lf} {\partial X_k} \right)^{\mu_k}
    \right ) + \frac{P_{|\mu|}}{(T-\lf)^{|\mu|}} + \cdots + \frac{P_{1}}{(T-\lf)}\\
    &=\left ( v\, |\mu|!\,    \frac {1}{(T-\lf)^{|\mu|+1}} \prod_{1 \le k \le n} 
      t_k^{\mu_k}
    \right ) + \frac{H}{(T-\lf)^{|\mu|}},
  \end{align*}
  for some polynomials $P_1,\dots,P_{|\mu|},H$ in $\K[X_1,\dots,X_n,T]$ that
  depend on the choices of $\mu$, $v$ and $X$, with $\deg_T(P_i) < i$
  for all $i$ and thus $\deg_T(H) < |\mu|$.

  Take now a $\Kbar$-linear combination of such operators, such as $f \mapsto
  \sum_{\mu \in R} c_\mu \frac{ \partial^{|\mu|} f } {\partial
  X_1^{\mu_1} \cdots \partial X_n^{\mu_n}}$ for some finite subset $R$ of
  $\N^n$. The corresponding generating series becomes
  \begin{align*}
    \sum_{s \ge 0} \sum_{\mu \in R} c_\mu \frac{ \partial^{|\mu|} ( v
    \lf^s )} {\partial X_1^{\mu_1} \cdots \partial X_n^{\mu_n}}
    \frac{1}{T^{s+1}} &= v\,\sum_{\mu \in R} \left( c_\mu |\mu|!\,  \frac {1}{(T-\lf )^{|\mu|+1}} \prod_{1
      \le k \le n} t_k^{\mu_k} \right )+\sum_{\mu
    \in R} \frac{H_\mu}{(T-\lf)^{|\mu|}},
      \end{align*}
      where each $H_\mu \in \Kbar[X_1,\dots,X_n,T]$ has degree in $T$ less than
      $|\mu|$.  Let $w$ be the maximum of all $|\mu|$ for $\mu$ in
      $R$. We can rewrite the above as
      \begin{align*}
        v\, w! 
        \sum_{\mu \in R, |\mu|=w}\left ( c_\mu
          \,    \frac {1}{(T-\lf )^{w+1}} \prod_{1 \le k \le n} 
        t_k^{\mu_k}\right )
        + \frac{A}{(T-\lf )^{w}},
      \end{align*}
      for some polynomial $A \in \Kbar[X_1,\dots,X_n,T]$ of degree less than $w$ in $T$. 
      Then, if we let 
      $\pi =\sum_{\mu \in R,\ |\mu|=w} c_{\mu} X_1^{\mu_1} \cdots
      X_n^{\mu_n}$, this becomes
      \begin{align*}
        \sum_{s \ge 0} 
        \sum_{\mu \in R} c_\mu \frac{ \partial^{|\mu|} ( v \lf^s )} { X_1^{\mu_1} \cdots
        X_n^{\mu_n}}
        \frac{1}{T^{s+1}} 
    &=
    v\, w! \,  \pi(t_1,\dots,t_n)
    \frac {1}{(T-\lf )^{w+1}}
    + \frac{A}{(T-\lf )^{w}}.
      \end{align*}
      Applying this formula to the sum $\ell=\sum_{i \in
      \mathfrak{S}}\ell_i$ from \cref{eq:fui}, and taking into account
      the expression in \cref{ell_param} for each $\ell_i$, we obtain the
      claim in the lemma.
    \end{proof}

    \noindent 
    The most useful consequence of this lemma is the following
    interpolation formula involving the operator $\Omega$ of the previous
    section, which generalizes a comment we made in \cref{section:linseq}.

    Fix a subset $\mathfrak{S}$ of $\{1,\dots,\dg\}$, and let $\lf$
    be as in the previous lemma. The mapping
    $\lf:V_\mathfrak{S} \to \Kbar$ defined by $\balpha_i \mapsto
    \lf(\balpha_{i})$ plays a special role in the formula in the previous
    lemma; this leads us to the following definitions.
    \begin{itemize}
      \item We consider $\ell$ and $\lf$ as in \cref{lemma:formula}, such
        that $\ell$ has support $\mathfrak{S}$.
      \item $\mathfrak{T}$ is the subset of $\mathfrak{S}$ consisting of
        all indices $i$ such that 
        \begin{itemize}
          \item $\lf(\balpha_{i'}) \ne \lf(\balpha_i)$ for $i' \ne i$ in $\mathfrak{S}$;
          \item $\pi_i(t_1,\dots,t_n)$ is nonzero.
        \end{itemize}
      \item $\{r_1,\dots,r_c\}$ are the pairwise distinct values taken by $\lf$ on
        $V_\mathfrak{S}$, for some $c \le |\mathfrak{S}|$.
      \item $\mathfrak{t}$ is the set of all indices $j$ in
        $\{1,\dots,c\}$ such that
        \begin{itemize}
          \item the fiber $\lf^{-1}(r_j) \subset V_{\mathfrak{S}}$ contains a single
            point, written $\balpha_{\sigma_j}$;
          \item   $\pi_{\sigma_j}(t_1,\dots,t_n)$ is nonzero.
        \end{itemize}
        Remark that $j \mapsto \sigma_j$ induces a one-to-one correspondence
        between  $\mathfrak{t}$ and  $\mathfrak{T}$, and that $\lf(\balpha_{\sigma_j})=r_j$ 
        for all $j$ in $\mathfrak{t}$.
    \end{itemize}

    \begin{lemma} \label{lemma:anyv}
      Let $\ell$, $\lf$ and all other notation be as above. Let further
      $\minpoly$ be the minimal polynomial of $\lf$ in
      $\residueI_\mathfrak{S}$. Suppose that $\minpoly$ is also the minimal
      polynomial of the sequence $(\ell(\lf^s))_{s \ge 0}$. Then,
      for $v$ in $\K[X_1,\dots,X_n]$, $\minpoly$
      cancels the sequence $(\ell(v \lf^s))_{s\ge0}$ and
      there
      exist nonzero constants $\{c_j \mid j \in \mathfrak{t}\}$ such that
      for $v$ in $\K[X_1,\dots,X_n]$,
      $$\Omega((\ell(v \lf^s))_{s\ge0},\minpoly) = c_{j} v(\balpha_{\sigma_j}) \quad \text{for all $j$ in $\mathfrak{t}$.}$$
    \end{lemma}
    \begin{proof}
      For $j=1,\dots,c$, we write $T_j$ for the set of all indices $i$ in
      $\mathfrak{S}$ such that $\lf(\balpha_{i})=r_j$; the sets
      $T_1,\dots,T_c$ form a partition of $\mathfrak{S}$. When $T_j$ has
      cardinality~$1$, we thus have $T_j=\{\sigma_j\}$.

      Take an arbitrary $v$ in $\K[X_1,\dots,X_n]$ and let us
      collect terms in \cref{eq:sumgenseries} as
      \begin{align*}
        \sum_{s \ge 0} \frac{\ell(v \lf^s)}{T^{s+1}} =&
        \sum_{j \in \{1,\dots,c\}}
        \sum_{i \in T_j} \frac{
          v(\balpha_{i})   w_{i}!\, \pi_{i}(t_1,\dots,t_n)
        + (T-r_{j} )A_{v,i}}
        {(T-r_{j} )^{w_{i}+1}}\\
        =&
        \sum_{j \in \mathfrak{t}}
        \frac{
          v(\balpha_{\sigma_{j}})  w_{\sigma_j}!\, \pi_{\sigma_j}(t_1,\dots,t_n)
        + (T-r_{j}  )A_{v,\sigma_j} }
        {(T-r_{j} )^{w_{\sigma_j}+1}}\\
         &+
         \sum_{j \in \{1,\dots,c\}-\mathfrak{t}}
         \frac{   \sum_{i \in T_j} \Big( \big[
               v(\balpha_{i})   w_{i}!\, \pi_{i}(t_1,\dots,t_n)
             + (T-r_{j}  )A_{v,i} \big ]
         (T-r_{j} )^{y_j-(w_i+1)}\Big)}
         {(T-r_{j} )^{y_j}},
      \end{align*}
      where $y_j$ is the maximum of all $w_i$ for $i$ in $T_j$.  Remark
      that for $v=1$, our condition that $\pi_i(t_1,\dots,t_n)$ is
      nonzero for $i$ in $\mathfrak{T}$, together with our assumption on
      the characteristic of $\K$, imply that in the second line, all terms
      in the first sum are nonzero and in reduced form.

      After simplifying terms in the second sum, we can rewrite the
      expression above as
      \begin{align*}
        \sum_{s \ge 0} \frac{\ell(v \lf^s)}{T^{s+1}} =&
        \sum_{j \in \mathfrak{t}} \frac{
          v(\balpha_{\sigma_{j}})   w_{\sigma_{j}}!\, \pi_{{\sigma_{j}}}(t_1,\dots,t_n)
        + (T-r_{j} )A_{v,\sigma_{j}}}
        {(T-r_{j} )^{w_{\sigma_{j}}+1}}  
        +\sum_{j \in  \{1,\dots,c\}-\mathfrak{t}}
        \frac{D_{v,j}}
        {(T-r_{j})^{z_{v,j}}},
      \end{align*}
      for some positive integers $\{z_{v,j} \mid j\in
        \{1,\dots,c\}-\mathfrak{t}\}$ and polynomials $\{D_{v,j} \mid j\in
      \{1,\dots,c\}-\mathfrak{t}\}$ such that for all $j$ in $
      \{1,\dots,c\}-\mathfrak{t}$, we have $\deg(D_{v,j}) < z_{v,j}$ and
      $\gcd(D_{v,j}, T-r_{j} )=1$. Some of the polynomials $D_{v,j}$ may
      vanish, so we let $\mathfrak{u}_v \subset
      \{1,\dots,c\}-\mathfrak{t}$ be the set of all $j$ for which this is
      not the case.  We then arrive at our final form for this sum, namely
      \begin{align}\label{eq:sumgenseries_collect1}
        \sum_{s \ge 0}  \frac{\ell(v \lf^s)}{T^{s+1}} =&
        \sum_{j \in \mathfrak{t}} \frac{
          v(\balpha_{\sigma_{j}})   w_{\sigma_{j}}!\, \pi_{{\sigma_{j}}}(t_1,\dots,t_n)
        + (T-r_{j}  )A_{v,\sigma_{j}}}
        {(T-r_{j} )^{w_{\sigma_{j}}+1}}  
        +\sum_{j \in  \mathfrak{u}_v}
        \frac{D_{v,j}}
        {(T-r_{j} )^{z_{v,j}}},
      \end{align}
      where all terms in the second sum are nonzero and in reduced form
      (and similarly for the first sum, for $v=1$).
      This implies that the minimal
      polynomial of the sequence $(\ell(v\lf^s))_{s \ge 0}$ is 
      $$\minpoly_v=\prod_{j \in \mathfrak{t}} (T-r_{j})^{\zeta_j} \prod_{j
      \in \mathfrak{u}_v} (T-r_{j})^{z_{v,j}},$$
      for some integers $\{\zeta_j \le w_{\sigma_{j}}+1 \mid j \in \mathfrak{t}\}$; for $v=1$, 
      we actually have $\zeta_j = w_{\sigma_{j}}+1$ for all such~$j$.

      Now, for $v=1$, we assume that the minimal polynomial of the sequence
      $(\ell(\lf^s))_{s \ge 0}$ is the minimal polynomial $\minpoly$ of $t$ in
      $\residueI_\mathfrak{S}$.  Writing $\mathfrak{u}=\mathfrak{u}_1$ and
      $z_j=z_{1,j}$ for all $j$ in $\mathfrak{u}$, we can thus write it as
      $$\minpoly=\prod_{j \in \mathfrak{t}} (T-r_{j})^{w_{\sigma_{j}}+1}
      \prod_{j \in \mathfrak{u}} (T-r_{j})^{z_{j}}.$$ Since it is the
      minimal polynomial of $\lf$ in $\residueI_\mathfrak{S}$, it also cancels the
      sequence $(\ell(v \lf^s))_{s \ge 0}$ for any $v$, so that for all
      $v$, $\minpoly_v$ divides $\minpoly$ (which proves the first point
      in the lemma), and in particular $\mathfrak{u}_v$ is contained in
      $\mathfrak{u}$.
      We can then rewrite the
      sum in \cref{eq:sumgenseries_collect1} as $$ \sum_{s \ge 0} \frac{\ell(v \lf^s)}{T^{s+1}} =\frac{\Omega((\ell(v \lf^s))_{s \ge 0} ,\minpoly)}{\minpoly},$$ with
      \begin{align*}
    & \Omega((\ell(v \lf^s))_{s \ge 0} ,\minpoly)= \\
    &\;\; \sum_{j \in \mathfrak{t}} \Big(\big[
        v(\balpha_{\sigma_{j}})  w_{\sigma_j}!\, \pi_{\sigma_j}(t_1,\dots,t_n)
      + (T-r_j) A_{v,\sigma_j}\big]
      \prod_{\iota \in \mathfrak{t}-\{j\}}(T-r_{\iota} )^{w_{\sigma_{\iota}}+1}
    \Big)\prod_{j \in \mathfrak{u}}(T-r_j )^{z_j}
    \\
    &\;\; +
    \Big(\prod_{j \in \mathfrak{t}}(T-r_{j} )^{w_{\sigma_j}+1}\Big)
    \sum_{j \in \mathfrak{u}_v} \Big (
      (T-r_j)^{z_{j}-z_{v,j}} D_{v,j}
    \prod_{\iota \in \mathfrak{u}-\{j\}}(T-r_{\iota} )^{z_{\iota}}\Big).
      \end{align*}
      In particular, 
      for $k$ in $\mathfrak{t}$,
      the value $\Omega((\ell(v \lf^s))_{s \ge 0} ,\minpoly)(r_k)$ is 
      \begin{align*}
        \Omega((\ell(v \lf^s))_{s \ge 0} ,\minpoly)(r_k)&= v(\balpha_{\sigma_{k}}) w_{\sigma_k}!\, \pi_{\sigma_k}(t_1,\dots,t_n)
        \prod_{\iota \in \mathfrak{t}-\{k\}}(r_\iota-r_{k} )^{w_{\sigma_{\iota}}+1}
        \prod_{j \in \mathfrak{u}}(r_j-r_{k} )^{z_{k}}\\
                                                        &= v(\balpha_{\sigma_{k}}) c_{k},
      \end{align*}
      with 
      $$c_{k}=
      w_{\sigma_k}!\, \pi_{\sigma_k}(t_1,\dots,t_n)
      \prod_{\iota \in \mathfrak{t}-\{k\}}(r_\iota-r_{k} )^{w_{\sigma_{\iota}}+1}
      \prod_{j \in \mathfrak{u}}(r_j-r_{k} )^{z_{k}}$$
      for $k$ in $\mathfrak{t}$. This is a nonzero constant, independent 
      of $v$, which completes the proof.
    \end{proof}

    %%%%%%%%%%%%%%%%%%%%%%%%%%%%%%%%%%%%%%%%%%%%%%%%%%%%%%%%%%%%

    \subsection{Computing a zero-dimensional parametrization}  \label{ssec:abstractlago}

    As a first application, the following algorithm shows how to compute a
    zero-dimensional parametrization of $V_{\mathfrak{S}}$. Our main usage
    of it will be with $\mathfrak{S}=\{1,\dots,\dg\}$, in which case
    $V_{\mathfrak{S}}=V$, but in the last section of the paper, we will
    also work with strict subsets.

    \begin{algorithm}[H]
      \caption{$\mathsf{Parametrization}(\ell,\lf)$}  ~\\
      {\bf Input:} \vspace{-0.5em}
      \begin{itemize}\setlength\itemsep{0em}
        \item  a linear form $\ell$ over $\residueI_\mathfrak{S}$
        \item $\lf=t_1 X_1 + \cdots + t_n X_n$
      \end{itemize}
      {\bf Output:}  \vspace{-0.5em}
      \begin{itemize}
        \item              polynomials $((\sqfree,V_1,\dots,V_n),\lf)$, with $\sqfree,V_1,\dots,V_n$ in $\K[T]$
      \end{itemize}
      \begin{enumerate}\setlength\itemsep{0em}
        \item let $\minpoly$ be the minimal polynomial of the sequence $(\ell(\lf^s))_{s \ge 0}$
        \item let $\sqfree$ be the squarefree part of $\minpoly$
        \item let $C_1 = \Omega((\ell(\lf^s))_{s\ge0} ,\minpoly)$
        \item \textbf{for} $i=1,\dots,n$ \textbf{do} \\
          \phantom{for} let $C_{X_i} = \Omega((\ell(X_i \lf^s))_{s\ge0}, \minpoly)$ 
        \item \textbf{return} $((\sqfree, C_{X_1}/ C_1 \bmod \sqfree, \dots, C_{X_n}/ C_{1} \bmod \sqfree),\lf)$
      \end{enumerate}
      \label{algo:para2}
    \end{algorithm}

    \begin{lemma}\label{lemma:para2}
      Suppose that $\ell$ is a generic element of ${\rm
      hom}_{\Kbar}(\residueI_\mathfrak{S},\Kbar)$ and that $\lf$ is a generic
      linear form. Then the output $((\sqfree,V_1,\dots,V_n),\lf)$ of
      $\mathsf{Parametrization}(\ell,\lf)$ is a zero-dimensional
      parametrization of $V_{\mathfrak{S}}$.
    \end{lemma}
    \begin{proof}
      A generic choice of $\lf$ separates the points of
      $V_{\mathfrak{S}}$, and we saw in \cref{lemma:symbol0} that for a
      generic choice of $\ell$ and $\lf$, $\pi_i(t_1,\dots,t_n)$ vanishes
      for no index $i$ in $\mathfrak{S}$.  As a result, with the notation
      introduced prior to \cref{lemma:anyv}, we have
      $\mathfrak{T}=\mathfrak{S}$ and $\mathfrak{t}$ consists of
      $|\mathfrak{S}|$ pairwise distinct values.

      Besides, we recall that for a generic $\ell$ in ${\rm
      hom}_{\Kbar}(\residueI_\mathfrak{S},\Kbar)$, the minimal polynomials of
      $(\ell(\lf^s))_{s \ge 0}$ and of $\lf$ are the same
      (\cref{lemma:minpoly}).  Thus, the polynomial $\minpoly$ computed
      at Step~1 is indeed the minimal polynomial of $\lf$ (with roots
      $\{r_j \mid j \in \mathfrak{t}\}$), and we can apply \cref{lemma:anyv};
      then, for any root $r_j$ of $\minpoly$, and $i=1,\dots,n$, we
      have
      $$\frac{ C_{X_i}(r_j)}{ C_1(r_j)} = \frac{\Omega((\ell(X_i
      \lf^s))_{s\ge0}, \minpoly)(r_j)}{\Omega((\ell(\lf^s))_{s\ge0} ,\minpoly)(r_j)}=
      \frac{c_j \alpha_{\sigma_j,i}}{c_j} = \alpha_{\sigma_j,i},$$ so that
      $ C_{X_i}/ C_1 \bmod P$ is the $i$th polynomial in the
      zero-dimensional parametrization of $V$ corresponding to $\lf$.
    \end{proof}

    We demonstrate how this algorithm works through a small example, in 
    which we already know the coordinates of the solutions. Let 
    $$I = \langle (X_1-1)(X_2-2),(X_1-3)(X_2-4)\rangle \subset
    \F_{101}[X_1,X_2].$$ Then, $V(I) = \{\balpha_1,\balpha_2\}$,
    with $\balpha_1= (1,4)$ and $\balpha_2=(3,2)$; we take
    $\lf=X_1$, which separates the points of $V(I)$.  We choose the linear form
    \[
      \ell: \F_{101}[X_1,X_2]/I \to \F_{101},\;
      f \mapsto \ell(f) = 17 f(\balpha_1) + 33 f(\balpha_2);
    \]
    then, $\mathfrak{S}=\{1,2\}$, and the symbols $\pi_1$ and $\pi_2$ 
    are respectively the constants 17 and 33. We have
    \begin{align*}
      \ell(X_1^s) &= 17 \cdot 1^s + 33 \cdot 3^s\\
      \ell(X_2X_1^s) &= 17 \cdot 4 \cdot 1^s + 33 \cdot 2 \cdot 3^s.
    \end{align*} 
    We associate a generating series to each sequence:
    \begin{align*}
      Z_1 = \sum_{s \ge 0} \frac{\ell(X^s_1)}{T^{s+1}}
&= \frac{17}{T-1} + \frac{33}{T-3}
= \frac{17(T-3)+33(T-1)}{(T-1)(T-3)} \\
Z_{X_2} = \sum_{s\ge0} \frac{\ell(X_2X_1^s)}{T^{s+1} }
&= \frac{17\cdot 4}{T-1} + \frac{33 \cdot 2}{T-3}
= \frac{17\cdot 4 (T-3) + 33\cdot 2(T-1)}{(T-1)(T-3)}.
    \end{align*}
    These generating series have for common denominator $\minpoly = (T-1)(T-3)$,
    whose roots are the coordinates of $X_1$ in $V(I)$;
    their numerators are respectively
    $$C_{1} = \Omega((\ell(X^s_1))_{s\ge 0},\minpoly) = 17 (T-3) + 33(T-1)$$
    and
    $$C_{X_2} = \Omega((\ell(X_2X^s_1))_{s\ge 0},\minpoly) = 17\cdot 4 (T-3) + 33\cdot 2(T-1).$$
    Now, let
    \begin{align*}
      V_2 
&=\frac{C_{X_2}}{C_1} \mod \minpoly\\
&=\frac{17\cdot 4 (T-3) + 33\cdot 2(T-1)}{17(T-3)+33(T-1)} \mod \minpoly\\
&=100 T +5.
    \end{align*}
    Then, $V_2(1) = 4$ and $V_2(3) = 2$, as expected.

    %%%%%%%%%%%%%%%%%%%%%%%%%%%%%%%%%%%%%%%%%%%%%%%%%%%%%%%%%%%%
    %%%%%%%%%%%%%%%%%%%%%%%%%%%%%%%%%%%%%%%%%%%%%%%%%%%%%%%%%%%%
    %%%%%%%%%%%%%%%%%%%%%%%%%%%%%%%%%%%%%%%%%%%%%%%%%%%%%%%%%%%%

    \section{The main algorithm}\label{sec:main}

    In this section, we extend the algorithm of~\citep{BoSaSc03} to compute
    a zero-dimensional parametrization of $V(I)$, for some
    zero-dimensional ideal $I$ of $\K[X_1,\dots,X_n]$, by using blocking
    methods. As input, we assume that we know a monomial basis
    $\basis=(b_1,\dots,b_D)$ of $\residueI=\K[X_1,\dots,X_n]/I$, together
    with the multiplication matrices $\mM_1,\dots,\mM_n$ of respectively
    $X_1,\dots,X_n$ in this basis; for definiteness, we suppose that the
    first basis element in $\basis$ is~$b_1=1$. As before, we let $D$ denote the
    dimension of $\residueI$.

    The first subsection presents the main algorithm. Its main feature is
    that after we compute the Krylov sequence used to find a minimal
    matrix generator, we recover all entries of the output for a minor
    cost, without computing another Krylov sequence. We make no assumption
    on $I$ (radicality, shape position, \dots), except of course that it
    has dimension zero; however, we assume (as in the previous subsection)
    that the characteristic of $\K$ is greater than $D$. 

    Then, in the second subsection we present a simple example, and in the third we
    show experimental results of an implementation based on the C++ libraries
    LinBox, Eigen and NTL.

    %%%%%%%%%%%%%%%%%%%%%%%%%%%%%%%%%%%%%%%%%%%%%%%%%%%%%%%%%%%%

    \subsection{Description, correctness and cost analysis}\label{ssec:mainalgo}

    We mentioned that the method of \citet{Steel15} already uses the block
    Wiedemann algorithm to compute the minimal polynomial $\minpoly$ of
    $\lf=t_1 X_1 + \cdots + t_n X_n$; \textcolor{red}{given sufficiently many terms of the
      sequence $(\mUt \mM^s \mV)$, this is done by means of polynomial
      lattice reduction, without computing the matrix generator
    $\mat{P}_{\mU,\mV}$}. Knowing the roots of $\minpoly$ in $\K$, that
    algorithm uses an ``evaluation'' method for the rest (several
    Gr\"obner basis computations, all with one variable less).

    Our algorithm computes the whole zero-dimensional parametrization of
    $V(I)$ for essentially the same cost as the computation of the minimal
    polynomial. In what follows, $\col{\varepsilon}_1$ denotes the
    size-$D$ column vector whose only nonzero entry is a $1$ in the first
    row: $\col{\varepsilon}_1 =\trsp{[1~0~\cdots~0]}$.

    \begin{algorithm}[H]
      \caption{$\mainalgoname(\mM_1,\dots,\mM_n,\mU,\mV,\lf)$}
      {\bf Input:} \vspace{-0.5em}
      \begin{itemize}
        \item $\mM_1,\dots,\mM_n$ multiplication matrices defined as above
        \item  $\mU,\mV \in \mathbb{K}^{D \times m}$, for some block dimension  $m \in \{1,\dots,D\}$
        \item $\lf =t_1 X_1 + \cdots + t_n X_n$
      \end{itemize}
      {\bf Output:}  \vspace{-0.5em}
      \begin{itemize}
        \item         polynomials $((\sqfree,V_1,\dots,V_n),\lf)$, with $\sqfree,V_1,\dots,V_n$ in $\K[T]$
      \end{itemize}
      \begin{enumerate}
        \item\label{mainstep1}   let $\mM = t_1 \mM_1 + \cdots + t_n \mM_n$
        \item\label{mainstep3} { compute $\mat{L}_s = \mUt\mM^s$ for $s=0,\dots,2d-1$, with $d = \lceil D/m \rceil$}
        \item\label{mainstep4} { compute $\seqelt{s,\mU,\mV}= \mat{L}_s\mV$ for $s=0,\dots, 2d-1$}
        \item\label{mainstep5} { compute a minimal matrix generator $\mat{P}_{\mU,\mV}$ of $(\seqelt{s,\mU,\mV})_{0 \le s < 2d}$}
        \item\label{mainstep6} { let $\minpoly$ be the largest invariant factor of $\mat{P}_{\mU,\mV}$}
        \item\label{mainstep7} { let $\sqfree$ be  the squarefree part  of $\minpoly$}
        \item\label{mainstep8} { let $\row{a}_1 = [P~0 ~\cdots~ 0] (\mat{P}_{\mU,\mV})^{-1}$}
        \item\label{mainstep9}  let $C_1 = \mathsf{{\sf ScalarNumerator}}(\mat{P}_{\mU,\mV}, \minpoly, \col{\varepsilon}_1, 1, \row{a}_1, 
          (\mat{L}_s)_{0 \le s < d})$
        \item\label{mainstep10} \textbf{for} $i=1,\dots,n$ \textbf{do} \\
          \phantom{for}  let $C_{X_i} = \mathsf{{\sf ScalarNumerator}}(\mat{P}_{\mU,\mV}, \minpoly, \mM_i\col{\varepsilon}_1, 1, \row{a}_1, (\mat{L}_s)_{0 \le s < d})$
        \item\label{mainstep11}     \textbf{return} $((\sqfree, C_{X_1}/ C_1 \bmod \sqfree, \dots, C_{X_n}/ C_{1} \bmod \sqfree),\lf)$
      \end{enumerate}  \label{algo:block-sparse-fglm}
    \end{algorithm}

    We first prove correctness of the algorithm, for generic choices of
    $t_1,\dots,t_n$, $\mU$ and $\mV$. The first step computes the
    multiplication matrix $\mM=t_1 \mM_1 + \cdots + t_n \mM_n$ of $\lf=t_1
    X_1 + \cdots + t_n X_n$.
    Then, we compute the first $2d$ terms of the sequence $\seq_{\mU,\mV}=
    (\mUt\mM^s\mV)_{s\ge 0}$. For generic choices of these two
    matrices, as discussed in \cref{ssec:appliW}, \cref{randXY} shows that
    the matrix polynomial $\mat{P}_{\mU,\mV}$ is indeed a minimal left
    generator of the sequence $\seq_{\mU,\mV}$, that $\minpoly$ is the minimal 
    polynomial of $\lf$ and $\sqfree$ its squarefree part.

    We find the rest of the polynomials in the output by following
    \cref{algo:para2}. In particular, the scalar numerators
    needed in this algorithm are computed using \cref{algo:scalar_numerator}
    ($\mathsf{ScalarNumerator}$); indeed, applying \cref{lemma:omegaOmega},
    we see that calling this algorithm
    at Steps~\ref{mainstep9} and~\ref{mainstep10}
    computes $$C_1 = \Omega((\row{u}_i \mM^s \col{\varepsilon}_1)_{s \ge
    0}, \minpoly) \quad\text{and}\quad C_{X_i} = \Omega((\row{u}_1 \mM^s
    \mM_i \col{\varepsilon}_1)_{s \ge 0}, \minpoly),\ \ i=1,\dots,n.$$ 
    %% The
    %% key result supporting this is \cref{lemma:omegaOmega} which shows that
    %% for a generic choice of $\mU$ and $\mV$, we have
    %% $$[~C_1~] = \row{a}_1\cdot \mat{\Omega}((\mUt \mM^s \col{\varepsilon}_1)_{s \ge 0}, \mat{P}_{\mU,\mV})$$
    %% and
    %% $$[~C_{X_i}~] = \row{a}_1\cdot \mat{\Omega}((\mUt \mM^s \mM_i \col{\varepsilon}_1)_{s \ge 0}, \mat{P}_{\mU,\mV}),\ \ i=1,\dots,n;
    %% $$
    %% Algorithm $\mathsf{\sf{ScalarNumerator}}$ precisely computes these quantities.

    Let $\ell:\residueI \to \K$ be the linear form $f = \sum_{i=1}^D f_i b_i \mapsto 
    \sum_{1 \le i \le D} f_i u_{i,1}$, where $u_{i,1}$ is the entry at position
    $(i,1)$ in $\mU$. The two polynomials above can be rewritten
    as 
    $$C_1 = \Omega( (\ell(X^s))_{s \ge 0}, \minpoly) \quad\text{and}\quad
    C_{X_i} = \Omega( (\ell(X_i X^s))_{s \ge 0}, \minpoly),$$ so they
    coincide with the polynomials computed in \cref{algo:para2}.  Then,
    \cref{lemma:para2} shows that for generic $\mU$ and $\lf$, the output
    of $\mainalgoname$
    is indeed a zero-dimensional parametrization of $V(I)$.  

    \begin{remark}
      As already pointed out in \cref{ssec:scalar_numer}, the algorithm is
      written assuming that memory usage is not a limiting factor (this
      makes it slightly easier to write the pseudo-code). As described
      here, the algorithm stores $\Theta(D^2)$ field elements in the
      sequence $\mat{L}_s$ computed at Step~\ref{mainstep3}, since they
      are re-used at Steps~\ref{mainstep9} and~\ref{mainstep10}.  We may
      instead discard each matrix $\mat{L}_s$ after it is used, by
      computing on the fly the column vectors needed for Steps~\ref{mainstep9}
      and~\ref{mainstep10}.

      If the multiplication matrices are dense, little is to be gained
      this way (since in the worst case they use themselves $n D^2$ field elements),
      but savings can be substantial if these  matrices are sparse.
    \end{remark}

    For the cost analysis, we will mainly focus on a sparse model: we let $\density
    \in [0,1]$ denote the density of $\mM$ and the $\mM_i$'s, that is,
    all these matrices have at most $\density D^2$ nonzero entries.  As a result, a
    matrix-vector product by $\mM$ can be done in $O(\density D^2)$ operations in
    $\K$ (we briefly discuss variants of the algorithm using dense linear algebra
    at the end of this subsection). In particular, the cost incurred at
    Step~\ref{mainstep1} to compute $\mM$ is $O(\density n D^2)$.

    In this context, the main purpose of Coppersmith's blocking strategy
    is to allow for easy parallelization. Computing the matrices
    $\mat{L}_s=\mUt\mM^s$, for $s=0,\dots,2d-1$, is the bottleneck
    of the algorithm, but this can be parallelized. This is done by
    working row-wise, computing independently the sequences
    $(\row{\ell}_{i,s})_{0 \le s < 2d}$ of the $i$th rows of
    $(\mat{L}_s)_{0 \le s < 2d}$ as $\row{\ell}_{i,0}=\row{u}_i$ and
    $\row{\ell}_{i,s+1} = \row{\ell}_{i,s}
    \mM$ for all $i,s$, where $\row{u}_i$ is the $i$th row of $\mUt$.
    For a fixed $i \in \{1,\dots,m\}$, computing $(\mat{\ell}_{i,s})_{0
    \le i < 2d}$ costs $O(d \density D^2) = O(\density D^3/m )$ field operations. If
    we are able to compute $m$ vector-matrix products in parallel at once,
    the {\em span} of Step~\ref{mainstep3} is thus $O(\density D^3/m)$, whereas
    the total work is $O(\density D^3)$.

    At Step~\ref{mainstep4}, we can then compute $\seqelt{s,\mU,\mV}=\mUt\mM^s\mV$, for $s=0,\dots,2d-1$ by the
    product
    $$
    \begin{bmatrix}
      \mUt\\
      \mUt \mM\\
      \mUt \mM^2\\
      \vdots\\
      \mUt \mM^{2d-1}\\
    \end{bmatrix} \mV
    = 
    \begin{bmatrix}
      \mUt \mV\\
      \mUt \mM \mV\\
      \mUt \mM^2 \mV\\
      \vdots \\
      \mUt \mM^{2d} \mV\\
    \end{bmatrix}
    $$
    of size $O(D) \times D$ by $D \times m$; since $m \le D$, this  costs $O(m^{\omega-2}D^2)$
    base field operations.

    Recall from \cref{section:matrix_seq} that we can compute a minimal
    matrix generator $\mat{P}_{\mU,\mV}$ in time $O(m^{\omega}
    \M(D/m) \log(D/m))$, which is in $O(m^{\omega-1} \M(D) \log(D))$.  In
    \cref{ssec:appliW,ssec:scalar_numer}, we saw that the
    largest invariant factor $P$ and the vector $\row{a}_1$ can be
    computed in time $O(m^{\omega-1} \M(D) \log(D) \log(m))$.  Computing $\sqfree$
    takes time $O(\M(D) \log(D))$.

    In \cref{ssec:scalar_numer}, we saw that each call to {\sf ScalarNumerator}
    takes $O(D^2 + m\M(D))$ operations in $\K$, for a total of $O(nD^2 + n
    m\M(D))$; the final modular calculations modulo $\sqfree$ at Step \ref{mainstep11} take time $O(n\M(D) + \M(D) \log(D))$.
    Altogether, assuming perfect parallelization 
    at Step~\ref{mainstep3}, the total span is
    $$O\left (\density \frac{D^3}m + m^{\omega-1} \M(D) \log(D) \log(m) + nD^2 + nm\M(D)\right ),$$
    and the total work is
    $$O\left (\density D^3 + m^{\omega-1} \M(D) \log(D) \log(m) + nD^2 +
    nm\M(D)\right ).$$ Although one may work on parallelizing other steps than
    Step~\ref{mainstep3}, we note that this step is simultaneously the most costly in
    theory and in practice, and the easiest to parallelize. 

    \medskip

    We conclude this section by a discussion of ``dense'' versions of the
    algorithm (to be used when the density $\density$ is close to $1$). 
    If we use a dense model for our matrices, our algorithms
    should rely on dense matrix multiplication. We will see two possible
    approaches, which respectively take $m=1$ and $m=D$; we will not
    discuss how they parallelize, merely pointing out that one may simply
    parallelize dense matrix multiplications throughout the algorithms.

    Let us first discuss the modifications in the algorithm to apply if we choose
    $m=1$. In this case, blocking has no effect. We compute the
    row-vectors $\mat{L}_s$, for $s=0,\dots,2D-1$, using the
    square-and-multiply technique used in the algorithm of \citet{Keller85},
    for $O(D^\omega \log(D))$ operations in
    $\K$. For generic choices of $\mU$ and $\mV$, the canonical matrix generator
    $\mat{P}_{\mU,\mV}$ is equal to the minimum
    polynomial $\minpoly$ of $\mM$, and can be computed efficiently by the
    Berlekamp-Massey algorithm (or a fast version of it); besides,
    $\row{a}_1 = P (\mat{P}_{\mU,\mV})^{-1} = 1$. Computing the scalar
    numerators is simply a power series multiplication in degree at most
    $D$. Altogether, the runtime is $O(D^{\omega} \log(D) + nD^2)$, where
    the second term gives the cost of computing $\mM$.

    When $m = D$, $\mU,\mV \in \mathbb{K}^{D \times D}$ are square
    matrices and $d = D/m = 1$. The canonical matrix generator of
    $\seq_{\mU,\mV} = (\mUt\mV, \mUt\mM\mV,\dots)$ is
    $\mat{P}_{\mU,\mV} = T \mat{I}_D - \mUt\mM\itrsp{\mU}$ and its
    largest invariant factor $\minpoly$ and $\row{a}_1$ can be computed in
    $O(D^\omega \log(D))$ operations in $\K$ using high-order lifting.

    The numerator $\mat{\Omega}(\mUt \mM \col{\varepsilon}_1,
    \mat{P}_{\mU,\mV})$ is then seen to be $\mUt \col{\varepsilon}_1$, that is, the
    first column of $\mUt$; we recover $C_1$ from it through a dot
    product with $\row{a}_1$. Similarly, the numerator
    $\mat{\Omega}(\mUt \mM \mM_i \col{\varepsilon}_1,
    \mat{P}_{\mU,mV})$ is $\mUt \mM_i \col{\varepsilon}_1$, and gives
    us $C_{X_i}$. Altogether, the runtime is again $O(D^{\omega} \log(D) +
    nD^2)$, where the second term now gives the cost of computing $\mM$,
    as well as $C_1$ and all $C_{X_i}$.

    \begin{remark}
      Our algorithm only computes the first invariant factor of
      $\mat{P}_{\mU,\mV}$, that is, of $T \mat{I}_D-\mM$. A natural
      question is whether computing further invariant factors can be of
      any use in the algorithm (or possibly can help us determine part of
      the structure of the algebras $\residueI_i$).
    \end{remark}

    %%%%%%%%%%%%%%%%%%%%%%%%%%%%%%%%%%%%%%%%%%%%%%%%%%%%%%%%%%%%

    \subsection{Example}

    We give an example of our algorithm with a non-radical system as input. Let
    $$
    I = 
    \sbox0{$\begin{array}{c}
        X_1^3 + 88X_1^2 + 56X_1 + 21,\\
        X_1^2X_2 + 91X_1^2 + 92X_1X_2 + 90X_1 + 20X_2 + 2,\\
        X_1X_2^2 + 81X_1X_2 + 100X_1 + 96X_2^2 + 100X_2 + 5,\\
        X_1^2X_2 + 81X_1^2 + 93X_1X_2 + 59X_1 + 16X_2 + 84,\\
        X_1X_2^2 + 71X_1X_2 + 99X_1 + 97X_2^2 + 19X_2 + 8,\\
        X_2^3 + 61X_2^2 + 96X_2 + 20
    \end{array}$}
    \mathopen{\resizebox{1.2\width}{\ht0}{$\Bigg\langle$}}
      \usebox{0}
    \mathclose{\resizebox{1.2\width}{\ht0}{$\Bigg\rangle$}}
    \subset \F_{101}[X_1,X_2];
    $$ the corresponding residue class ring $\residueI=\F_{101}[X_1,X_2]/I$
    has dimension $D=4$.  Although it is not obvious from the generators,
    the ideal $I$ is simply $\mathfrak{m}_1^2 \mathfrak{m}_2$, where
    $\mathfrak{m}_1 =\langle X_1-4,X_2-10\rangle$ and $\mathfrak{m}_2
    =\langle X_1-5,X_2-20\rangle$ (this immediately implies that $D=3+1=4$).

    We choose $\lf = 2X_1 + 53 X_2$, so that the multiplication matrices of $X_1$, $X_2$ and $\lf$
    \textcolor{red}{in the basis $\basis=(1,X_2,X_2,X_2^2)$} of $\residueI$ are respectively
    \begin{align*}
      \mM_1 = \begin{bmatrix}
        7   & 91 & 100& 0\\
        41  & 2  & 20 & 0\\
        100 & 10 & 8  & 1\\
        1   & 71 & 86 & 0
      \end{bmatrix},\,
      \mM_2 = \begin{bmatrix}
        40&  1 & 91 & 0\\
        5 &  0 &  2 & 1\\
        0 &  0 & 10 & 0\\
        81&  0 & 71 & 0
      \end{bmatrix},\,\text{and }
      \mM = \begin{bmatrix}
        13 & 33&  74&  0\\
        44 &  4&  45&  53\\
        99 & 20&  41&  2\\
        53 & 41&  97&  0
      \end{bmatrix}.
    \end{align*}

    We choose $m = 2$ and take $\mU,\mV \in \F_{101}^{D\times m}$ with 
    entries
    \[ \mU = \begin{bmatrix}
        84& 38\\
        29& 58\\
        80& 43\\
        7& 82
      \end{bmatrix},\quad
      \mV = \begin{bmatrix}
        6&  97\\
        83&  58\\
        0&  95\\
        59&  89
      \end{bmatrix}.
    \]
    We compute the first $2d=2\lceil D/m\rceil =4$ terms in the matrix
    sequence $\seq_{\mU,\mV} = (\mUt\mM^s\mV)_{s\ge0}$ and its
    minimum matrix generator $\mat{P}_{\mU,\mV}$. This is done
    by first computing
    \begin{align}\label{exampleproducts}
      \mUt =
      \begin{bmatrix}
        84&  29&  80&   7\\
        38&  58&  43&  82
      \end{bmatrix},
& 
\quad \mUt \mM 
=
\begin{bmatrix}
  54&  28&  67&  81\\
  34&  52&  90&  29
\end{bmatrix},
\\[2mm]
\mUt \mM^2 
=
\begin{bmatrix}
  33&  91&   3&  2\\
  47&  77&  47&  7
\end{bmatrix},
 &\quad \mUt \mM^3 
 =
 \begin{bmatrix}
   89&  80&  87&  82\\
   34&  56&  55&  34
 \end{bmatrix}, \nonumber
\end{align}

from which we get, by right-multiplication by $\mV$,
\[
  \seq_{\mU,\mV} =
  \begin{bmatrix}
    92& 75\\  
    83& 51
  \end{bmatrix},
  \begin{bmatrix}
    54& 34\\  
    70& 73
  \end{bmatrix},
  \begin{bmatrix}
    92& 54\\  
    16& 74
  \end{bmatrix},
  \begin{bmatrix}
    94& 51\\
    91& 51
  \end{bmatrix},\dots
\]
and then we obtain the canonical matrix generator
\[
  \mat{P}_{\mU,\mV} =
  \begin{bmatrix}
    T^2 + 60T + 62 &       88T + 25\\
    100T + 33 & T^2 + 84T + 78
  \end{bmatrix}.
\]

The largest invariant factor of $\mat{P}_{\mU,\mV}$ is 
$P = T^3 + 76T^2 + 100T + 7$,
with squarefree part $\sqfree=T^2+8T+61$.
Next, we compute the row vector $\row{a}_1 = [T+16,13]$ 
by solving $\row{a}_1 = [P~~0] (\mat{P}_{\mU,\mV})^{-1}$,
and the matrix numerator
\[
  \mat{\Omega}( (\mUt \mM^s\col{\varepsilon}_1)_{s \ge 0}, \mat{P}_{\mU,\mV}) 
  =\left [\begin{matrix} 84T + 55 \\ 38T + 11 \end{matrix}\right ]
\]
which is made from the entries of nonnegative degree in the product
\[\mat{P}_{\mU,\mV} 
  \left (
    \begin{bmatrix}
      84\\  
      38
    \end{bmatrix}\cdot \frac 1T+
    \begin{bmatrix}
      54\\  
      34
    \end{bmatrix}\cdot \frac 1{T^2}+
  \cdots \right),
\]
where the columns are the first columns of the
matrices in \cref{exampleproducts} (as per \cref{eq:computeOmega}, we only need $d=2$
terms in the right-hand side).  From this, we find the scalar
numerator 
\[
  C_1 = \Omega((\row{u}_1 \mM^s\col{\varepsilon}_1)_{s \ge
  0}, \minpoly) = 84T^2 + 75T + 13
\]
by means of the dot product $[~C_1~] = \row{a}_1\cdot
\mat{\Omega}( (\mUt \mM^s\col{\varepsilon}_1)_{s \ge 0},
\mat{P}_{\mU,\mV})$.

We then find $C_{X_1} = 88T^2 + 47T  + 16$, by proceeding similarly: we compute
the matrix numerator
\[
  \mat{\Omega}( (\row{u}_1 \mM^s \mM_1\col{\varepsilon}_1)_{s \ge 0},
  \mat{P}_{\mU,\mV}) = \begin{bmatrix} 88T + 19 \\ 57T+40
  \end{bmatrix}
\]
and we take the dot product with $\row{a}_1$.
Thus, we obtain the polynomial $V_1 = C_{X_1}/C_1 \bmod \sqfree =
15T+14$. 
We compute $V_2= 49T+9$ in the same way,
and our output is 
$$((T^2+8T+61, 15T+14, 49T+9), 2X_1 + 53 X_2).$$ As a sanity check, we
recall that $V( I)$ has two points in $\F_{101}^2$, namely
$(4,10)$ and $(5,20)$; accordingly, $Q$ has two roots in $\F_{101}$,
$33$ and $60$, and we have $(V_1(33) = 4, V_2(33) = 10)$ and $(V_1(60) = 5,
V_2(60) = 20)$, as expected.

%%%%%%%%%%%%%%%%%%%%%%%%%%%%%%%%%%%%%%%%%%%%%%%%%%%%%%%%%%%%

\subsection{Experimental results}\label{section:ex}

In \cref{tbl:timings_mainalgo}, we give the timings in seconds for
different values of $m$ for \cref{algo:block-sparse-fglm}.  Our
implementation is based on Shoup's NTL for univariate polynomials, LinBox
for dense polynomial matrices and matrix generator
computations, and Eigen for sparse matrix-vector
products. It is dedicated to small prime fields; thus,
as is done in~\citep{fflas-ffpack} for dense matrices, we use machine
floats to do the bulk of the calculations. Explicitly, we rely on
Eigen's {\tt SparseMatrix<double, RowMajor>} class to store our
multiplication matrices and do the matrix-vector products, reducing
the results modulo $p$ afterwards.

All timings are measured on an Intel Xeon CPU E5-2667 with 128 GB RAM
and 8 cores (16 available through hyperthreading). For each value of
$m$ in $\{1,3,6\}$, we create and run $m$ threads in parallel.

In all cases, we start from multiplication matrices computed from a
{degrevlex} Gr\"obner basis in Magma~\citep{BoCaPl97}. The timings
reported here do not include this precomputation; instead, we refer
the reader to~\citep{FaMo17} for extensive experiments comparing the
runtime of two similar algorithmic stages (degree basis computation and
conversion to a lex ordering).  Rather than optimizing our
implementation, our main focus here was to demonstrate the effects of
parallelization, and how the Krylov sequence computation dominates the
runtime in these examples. In particular, we believe that
improvements are possible for the polynomial matrix computations
(computing minimal matrix generator, its largest invariant factor,
\dots), but this is by no means a bottleneck here.

All our inputs as well as our source code are available at
\url{https://git.uwaterloo.ca/sghyun/Block-sparse-FGLM}.
Some systems
are well-known (Katsura or Eco from~\citep{Morgan88}), while we also
consider several families of randomly generated inputs (some are
inspired from~\citep{FaMo17}). Systems rand1-$i$ have $3$ variables and
randomly generated equations (of degree depending on $i$), and
rand2-$i$ are similar, with $4$ variables. These systems are
generically radical and in shape position (for the projection on the
first coordinate axis; the last column uses that
convention). Systems mixed1-$i$ and mixed2-$i$ have similar numbers of
variables as the previous ones, but some of their solutions are
multiple, so the ideals are not radical (and not in shape
position). Systems mixed3-$i$ have only one multiple root (the others
are simple), in increasing numbers of variables; they are not in shape
position. Systems W1-$\kappa$-$n$-$p$ are determinantal equations that
describe the computation of critical points for the projection $
(\alpha_1,\dots,\alpha_n) \mapsto \alpha_1$ on $V(f_1,\dots,f_p)$,
where $f_1,\dots,f_p$ are $p$ equations of degree $\kappa$ in $n$
variables.


We used OpenMP {\tt parallel
for} pragmas to parallelize the computation of the above sequence,
as described in \cref{ssec:mainalgo}.
In the columns $m=1,m=3,m=6$, the numbers in parentheses indicate the
part of the total time spent in computing the Krylov sequence
$(\mUt \mM^s)_{0 \le s< 2d}$, with $d=\lceil D/m\rceil$. This is
always by far the dominant factor.  In other words, the immense
majority of the time is spent doing sparse matrix / vector products
for matrices with machine float entries. 

Increasing $m$ has two effects. On the plus side, plainly, it
decreases the length of the sequence above. On the other side, while
the algorithm performs better by a factor often close to 3 for $m=3$,
the gain is not as significant for $m=6$, so that parallelization is
less effective. It is an interesting question to clarify this issue.  Besides, increasing $m$ also increases the time to
compute the output polynomials, through minimal generator computation,
high-order lifting, \dots. However, from the observed speedup and the
ratios in the table, we can conclude that this effect is minor.


\begin{table}[H]
  \centering
  \def\arraystretch{1.2}
  \setlength\tabcolsep{6pt}
  \caption{Timings (in seconds) for polynomials over $\F_{65537}$}
  \label{tbl:timings_mainalgo}
  \begin{tabular}{c|c|c|c|c|c|c|c}
    \textbf{name}& $\bm{n}$ & $\bm{D}$ & \textbf{density} $\boldsymbol\rho$ & $\bm{m = 1}$ & $\bm{m = 3}$ & $\bm{m = 6}$ 
                 & \textbf{radical/shape} \\
                 \hline
    rand1-26&3 &17576&0.06& 692(0.98) & 307(0.969) & 168(0.926) & yes/yes \\
    rand1-28&3 &21952&0.05&1261(0.983) & 471(0.971) & 331(0.944)& yes/yes  \\
    rand1-30&3 &27000&0.05&2191(0.986) & 786(0.974) & 512(0.946)& yes/yes  \\
    rand2-10&4 &10000&0.14&301(0.981) & 109(0.964) & 79(0.934)& yes/yes  \\
    rand2-11&4 &14641&0.13&851(0.987) & 303(0.975) & 239(0.961) & yes/yes \\
    rand2-12&4&20736&0.12&2180(0.99) & 784(0.982) & 648(0.972)& yes/yes  \\
    mixed1-22&3 &10864&0.07&207(0.973) & 75(0.947) & 58(0.909) &no/no \\
    mixed1-23&3 &12383&0.07&294(0.976) & 107(0.95) & 92(0.925) &no/no  \\
    mixed1-24&3 &14040&0.07&413(0.979) & 150(0.958) & 125(0.934) &no/no  \\
    mixed2-10&4 &10256&0.16&362(0.984) & 130(0.969) & 113(0.954)  &no/no \\
    mixed2-11&4 &14897&0.14&989(0.988) & 384(0.98) & 278(0.965)  &no/no \\
    mixed2-12&4 &20992&0.13&2480(0.991) & 892(0.984) & 807(0.977)  &no/no \\
    mixed3-12&12 &4109&0.5&75(0.963) & 27(0.941) & 21(0.929)  &no/no \\
    mixed3-13&13&8206&0.48&554(0.982) & 198(0.973) & 171(0.968)   &no/no\\
    eco12 &12 & 1024 & 0.55 & 1(0.801) & 1(0.641) & 1(0.63) & yes/yes  \\
    sot1 &5 & 8694 & 0.01 & 21(0.745) & 9(0.62) & 9(0.552) & yes/no \\
    W1-6-5-2 & 5 & 18000 & 0.2 & 2362(0.992) & 859(0.986) & 696(0.979)& yes/yes  \\
    W1-4-6-2 & 6 & 6480 & 0.32 & 184(0.981) & 66(0.965) & 54(0.951)& yes/yes  \\
    katsura10 &11 & 1024 & 0.63 & 1(0.836) & 1(0.679) & 1(0.672)  & yes/yes
  \end{tabular}
\end{table}

We refer the reader to experiments with systems of comparable (or
higher) degrees presented in~\citep{FaMo17} (the algorithm in that
reference does not use a blocking strategy). Recall that the output
in~\citep{FaMo17} is somewhat stronger than here (the authors compute a
Gr\"obner basis of the input ideal, so multiplicities are kept). On
the other hand, for ideals not in shape position, Table~2 of that
reference reports only the calculation of the last polynomial in the
Gr\"obner basis, whereas our algorithm makes no distinction between
ideals in shape position or not.

% We can determine the effectiveness of parallel computations by
% comparing combined CPU times (that is, total work).  For example, in
% test rand1-1 we get $4124, 4350, 6008$ seconds for $m=1,4,8$
% respectively; we observed similar patterns for other tests. If the
% parallelization was perfect, the combined CPU time should be roughly
% equal for all values of $m$, since computing $(\mUt
% \mM^s)_{0 \le s< 2d}$ took more than $99\%$ of the total time for all three
% values of $m$ in this test. 

%%%%%%%%%%%%%%%%%%%%%%%%%%%%%%%%%%%%%%%%%%%%%%%%%%%%%%%%%%%%
%%%%%%%%%%%%%%%%%%%%%%%%%%%%%%%%%%%%%%%%%%%%%%%%%%%%%%%%%%%%
%%%%%%%%%%%%%%%%%%%%%%%%%%%%%%%%%%%%%%%%%%%%%%%%%%%%%%%%%%%%

\section{Using the original coordinates}\label{sec:original}

In this last section, we propose a refinement of the algorithm given
previously; the main new feature is that we avoid using a generic
linear form $\lf = t_1 X_1 + \cdots + t_n X_n$ as much as possible.
Indeed, such a linear combination is likely to result in a
multiplication matrix $\mM = t_1 \mM_1 + \cdots + t_n \mM_n$
significantly more dense than $\mM_1,\dots,\mM_n$. In all this
section, we will work under the assumption that $\mM_1$ is the sparsest
matrix among $\mM_1,\dots,\mM_n$, and try to rely on computations
involving $\mM_1$ as much as possible.

All notation, such as $I$, $\residueI=\K[X_1,\dots,X_n]/I$, its basis
$\basis=(b_1,\dots,b_D)$, the local algebras $\residueI_i$, \dots, are as in the  previous two sections.
In particular, we write $V=V(I)=\{\balpha_1,\dots,\balpha_\dg\},$ with
all $\balpha_i$'s in $\Kbar{}^n$, and
$\balpha_i=(\alpha_{i,1},\dots,\alpha_{i,n})$ for all $i$.

%%%%%%%%%%%%%%%%%%%%%%%%%%%%%%%%%%%%%%%%%%%%%%%%%%%%%%%%%%%%

\subsection{Overview}

The algorithm will decompose $V$ into two parts: for the first part,
we will be able to use $X_1$ as a linear form in our zero-dimensional
parametrization; the remaining points will be dealt with using a
random linear form $\lf=t_1 X_1 + \cdots + t_n X_n$ as
above. Throughout, we rely on the following operations: evaluations of
linear forms on successive powers of a given element in $\residueI$
(such as $1,X_1,X_2^2,\dots$ or $1,\lf,\lf^2,\dots$) and (polynomial)
linear algebra, as we did before, as well as some operations on
univariate polynomials related to the so-called {\em power
projection}~\citep{Shoup94,Shoup99}.

The main algorithm is as follows. For the moment, we only describe its
main structure; the subroutines are described in the next subsections.

\begin{algorithm}[H]
  \caption{$\mathsf{BlockParametrizationWithSplitting}(\mM_1,\dots,\mM_n,\mU,\mV,\lf,\mf$)}
  {\bf Input:} \vspace{-0.5em}
  \begin{itemize}
    \item $\mM_1,\dots,\mM_n$ defined as above
    \item  $\mU,\mV \in \mathbb{K}^{D \times m}$, for some block dimension  $m \in \{1,\dots,D\}$
    \item $\lf =t_1 X_1 + \cdots + t_n X_n$
    \item $\mf =y_2 X_2 + \cdots + y_n X_n$
  \end{itemize}
  {\bf Output:}  \vspace{-0.5em}
  \begin{itemize}
    \item polynomials $((\sqfree,V_1,\dots,V_n),\lf)$, with $\sqfree,V_1,\dots,V_n$ in $\K[T]$
  \end{itemize}
  \begin{enumerate}
    \item let $(F,G_1,\dots,G_n,X_1)=\mathsf{BlockParametrizationX}_1(\mM_1,\dots,\mM_n,\mU,\mV,\mf)$
    \item let $(\mat{\Delta}_s),(\mat{\Delta}'_s),(\mat{\Delta}''_s)$ be correction matrices associated
      to the previous calculation
    \item let $(R,W_1,\dots,W_n,\lf)=\mainalgoname{\sf Residual}(\mM_1,\dots,\mM_n,\mU,\mV,$ \\
      \phantom{bla} \hfill $(\mat{\Delta}_s),(\mat{\Delta}'_s),(\mat{\Delta}''_s),\lf)$
    \item let $((\sqfree,V_1,\dots,V_n),\lf)=\mathsf{Union}(((F,G_1,\dots,G_n),X_1), ((R,W_1,\dots,W_n),\lf))$
    \item \textbf{return} $((\sqfree,V_1,\dots,V_n),\lf)$
  \end{enumerate}
\end{algorithm}

The call to $\mathsf{BlockParametrizationX}_1$ computes a zero-dimensional
parametrization of a subset $V'$ of $V$, such that $X_1$ separates the
points of $V'$ (that is, takes pairwise distinct values on the points
of $V'$); this is done by using sequences of the form $(\mUt
\mM_1^s \mV)_{s \ge 0}$. The next stage computes three 
sequences of correction matrices, using objects that were computed 
in the call to  $\mathsf{BlockParametrizationX}_1$.

We then apply a modified version of Algorithm $\mainalgoname$, which we call
$\mainalgoname{\sf Residual}$. It computes a zero-dimensional
parametrization of $V''=V-V'$ using sequences such as $(\mUt
\mM^s \mV)_{s \ge 0} - \mat{\Delta}_s$, where $\mM= t_1 \mM_1 + \cdots
+ t_n \mM_n$. Subtracting the correction terms $\mat{\Delta}_s$ has
the effect of removing from $V$ the points in $V'$; in those ``lucky''
cases where $V'$ is a large subset of $V$, $V''=V-V'$ may only contain
a few points, and few values for the latter sequences will be needed.

The last step involves changing coordinates in $((F,G_1,\dots,G_n),X_1)$ to use
$\lf$ as a linear form instead, and performing the union of the two components
$V'$ and $V''$. These operations can be done in time $O(n D^{(\omega+1)/2})$
\citep[Lemmas~2 \&~3]{PoSc13b}; we will not discuss them further.

\begin{remark}
  Another possibility is to return the parametrizations
  $(F,G_1,\dots,G_n),X_1)$ and $((R,W_1,\dots,W_n),\lf)$ without
  performing the union; this has the obvious advantage of partially
  decomposing the output.
\end{remark}

%%%%%%%%%%%%%%%%%%%%%%%%%%%%%%%%%%%%%%%%%%%%%%%%%%%%%%%%%%%%

\subsection{Describing the subset \texorpdfstring{$V'$}{V'} of \texorpdfstring{$V$}{V}}

In this paragraph, we give the details of Algorithm
$\mathsf{BlockParametrizationX}_1$. This is done by specializing the
discussion of \cref{ssec:genseries} to the case $\lf=X_1$:
in the notation of that section, we take
$\mathfrak{S}=\{1,\dots,\dg\}$, that is, $V_{\mathfrak{S}}=V$, and we
let $r_1,\dots,r_c$ be the pairwise distinct values taken by $X_1$ on
$V$, for some $c \le \dg$.  For $j=1,\dots,c$, we write $T_j$ for the
set of all indices $i$ in $\{1,\dots,\dg\}$ such that
$\alpha_{i,1}=r_j$; the sets $T_1,\dots,T_c$ form a partition of
$\{1,\dots,\dg\}$. When $T_j$ has cardinality~$1$, we denote it as
$T_j=\{\sigma_j\}$, for some index $\sigma_j$ in $\{1,\dots,\dg\}$, so
that $\alpha_{\sigma_j,1}=r_j$.

For $i=1,\dots,\dg$, let us write $\nu_i$ for the degree of the minimal
polynomial of $X_1$ in $\residueI_i$; thus, this polynomial is
$(T-\alpha_{i,1})^{\nu_i}$. For $j$ in $\{1,\dots,c\}$, we define
$\mu_j$ as the maximum of all $\nu_i$, for $i$ in~$T_j$. As a result, the minimal
polynomial of $X_1$ in $\prod_{j \in T_j} \residueI_j$ is 
$(T-r_j)^{\mu_j}$, and the minimal polynomial of $X_1$ in $\residueI$ is
$M=\prod_{j \in \{1,\dots,c\}} (T-r_j)^{\mu_j}$.

Recall that for any linear form $\ell: \residueI \to \K$, the
extension $\ell: \residueI\otimes_\K \Kbar \to \Kbar$ can be written
uniquely as $\ell=\sum_{i\in \{1,\dots,\dg\}} \ell_i$, with
$\ell_i:\residueI_i \to \Kbar$; collecting terms, $\ell$ may also be
written as $\ell=\sum_{j \in \{1,\dots,c\}} \lambda_j$, with
$\lambda_j=\sum_{i \in T_j} \ell_i$.  Given such an $\ell$, we first
explain how to compute values of the form $\lambda_j(1)$. We will do
this for some values of $j$ only, namely those $j$ for which
$\mu_j=1$. This result is close in spirit to \cref{lemma:anyv}, but
does not assume that the projection $X_1: V \to \Kbar$ is one-to-one
(that lemma makes such an assumption, but works for a more general
linear mapping $\lf:V\to \Kbar$).

\begin{lemma}\label{lemma:valuelambda}
  Let $\ell$ be in ${\rm hom}_\K(\residueI,\K)$ and let $M$ be the minimal
  polynomial of $X_1$ in $\residueI$. Then, the polynomial
  $\Omega((\ell(X_1^s))_{s\ge0},M)$ is well-defined and satisfies
  $$\Omega((\ell(X_1^s))_{s\ge0},M)(r_j) = \lambda_j(1) M'(r_j) \quad \text{for all $j$ in $\{1,\dots,c\}$ such that $\mu_j=1$.}$$
\end{lemma}
\begin{proof}
  Let $\mathfrak{e}$ be the set of all indices $j$ in $\{1,\dots,c\}$
  such that $\mu_j=1$, and let $\mathfrak{f}=\{1,\dots,c\}-\mathfrak{e}$;
  this definition allows us to split the generating series
  of sequence $(\ell(X_1^s))_{s\ge 0}$ as
  \begin{align*}
    \sum_{s \ge 0} \frac{\ell(X_1^s)}{T^{s+1}  }
  &= \sum_{j \in \{1,\dots,c\}}\sum_{i\in T_j} 
  \sum_{s \ge 0} \frac{\ell_i(X_1^s)}{T^{s+1}}  \\
  &=\sum_{j \in \mathfrak{e}}\sum_{i\in T_j}\sum_{s \ge 0}  \frac{\ell_i(X_1^s)}{T^{s+1}} +
  \sum_{j \in \mathfrak{f}}\sum_{i\in T_j}\sum_{s \ge 0}  \frac{\ell_i(X_1^s)}{T^{s+1}}.
  \end{align*}
  Using \cref{lemma:formula} with $\lf=X_1$ and $v=1$, any sum $\sum_{s \ge 0} \ell_i(X_1^s)/T^{s+1}$ 
  in the second summand
  can be rewritten as 
  $$\frac{E_i}{(T-r_j)^{v_i}},$$
  for some integer $v_i$, and for some polynomial $E_i \in \Kbar[T]$ of degree less than
  $v_j$. Next, take $j$ in $\mathfrak{e}$. Since $\mu_j=1$, $\nu_i=1$ for all $i$ in $T_j$,
  so that
  each such $\ell_i$ takes the form 
  $$\ell_i: f \mapsto (\Lambda_{i}(f))(\balpha_i),$$ where $\Lambda_{i}$
  is a differential operator that does not involve $\partial/\partial
  X_1$. Since all terms of positive order in $\Lambda_i$ involve one of
  $\partial/\partial X_2,\dots,\partial/\partial X_n$, they cancel
  $X_1^s$ for $s\ge 0$. Thus, $\ell_i(X_1^s)$ can be rewritten 
  as $\ell_{i,1} \alpha_{i,1}^s$, for some constant $\ell_{i,1}$,
  and the generating series of these terms is 
  $$\frac {\ell_{i,1}}{T-\alpha_{i,1}}=\frac {\ell_{i,1}}{T-r_j}.$$
  Remarking  that we can write $\ell_{i,1}=\ell_i(1)$,
  altogether, the sum in question can be written
  \begin{align*}
    \sum_{s \ge 0} \frac{\ell(X_1^s)}{T^{s+1}  }
  &=\sum_{j \in \mathfrak{e}} 
  \frac{ \sum_{i\in T_j}  \ell_{i}(1) }{T-r_j }
  + \sum_{j \in \mathfrak{f}} \frac{D_j}{(T-r_j )^{x_j}}\\
  &= \sum_{j \in \mathfrak{e}} 
  \frac{ \lambda_j(1) }{T-r_j }
  + \sum_{j \in \mathfrak{f}} \frac{D_j}{(T-r_j )^{x_j}}
  \end{align*}
  for some integers $\{x_j \mid j \in \mathfrak{f}\}$ and
  polynomials $\{D_j \mid j \in \mathfrak{f}\}$ such that
  $\deg(D_j) < x_j$ holds, and with $D_j$ and $T-r_j $
  coprime. In particular, the minimal polynomial of  $(\ell(X_1^s))_{s\ge 0}$ is $N=\prod_{j\in
  \mathfrak{e}}(T-r_j) \prod_{j \in  \mathfrak{f}}(T-r_j)^{x_j}$.

  %% On the other hand, the minimal polynomial $M$ of $X_1$ can be
  %% rewritten as $M=\prod_{j\in \mathfrak{e}}(T-r_j) \prod_{j \in
  %% \mathfrak{f}}(T-r_j)^{\mu_j}$.  

  The minimal polynomial of the sequence $(\ell(X_1^s))_{s \ge
  0}$ divides $M$, so that $x_j \le \mu_j$ holds for all $j$
  in $\mathfrak{f}$.  As a result,
  $\Omega((\ell(X_1^s))_{s\ge0},M)$ is well-defined and is given by
  \begin{align*}
    \Omega((\ell(X_1^s))_{s\ge0},M)=&
    \sum_{j \in \mathfrak{e}}
    \Big(
    \lambda_j(1) \prod_{\iota \in \mathfrak{e}-\{j\}}(T-r_\iota)\Big)
    \Big(\prod_{j \in \mathfrak{f}}(T-r_j)^{\mu_j} \Big)\\
                                    &+
                                    \sum_{j\in \mathfrak{f}}
                                    \Big(  (T-r_j)^{\mu_j-x_j} D_j
                                    \prod_{\iota \in \mathfrak{f}-\{j\}}(T-r_j)^{\mu_\iota}\Big)
                                    \Big(\prod_{j\in \mathfrak{e}} (T-r_j) \Big).
  \end{align*}
  This implies that $$\Omega((\ell(X_1^s))_{s\ge0},M)(r_k) =\lambda_k(1) 
  \prod_{\iota \in \mathfrak{e}-\{k\}}(r_k-r_\iota)
  \prod_{j \in \mathfrak{f}}(r_k-r_j)^{\mu_j} = \lambda_k(1) M'(r_k)$$ 
  holds for all $k$ in $\mathfrak{e}$.
\end{proof}

We now show how this result allows us to use sequences of the form
$(\ell(X_1^s))_{s \ge 0}$ to compute a zero-dimensional
parametrization of a subset $V'$ of $V$. Precisely, we characterize
the set $V'$ as follows: for $i$ in $\{1,\dots,\dg\}$, $\balpha_i$ is
in $V'$ if and only if:
\begin{itemize}
  \item for $i'$ in $\{1,\dots,\dg\}$, with $i'\ne i$, $\alpha_{i',1} \ne
    \alpha_{i,1}$;
  \item $\residueI_i$ is a reduced algebra, or equivalently, $I_i$ is radical (see \cref{ssec:dual} 
    for the notation used here).
\end{itemize}
We denote by $\mathfrak{A}\subset \{1,\dots,\dg\}$ the set of
corresponding indices $i$, and we let
$\mathfrak{B}=\{1,\dots,\dg\}-\mathfrak{A}$, so that we have
$V'=V_{\mathfrak{A}}$ and $V''=V_{\mathfrak{B}}$.  Remark that $X_1$
separates the points of $V'$.

Correspondingly, we define $\mathfrak{a}$ as the set of all indices
$j$ in $\{1,\dots,c\}$ such that $\sigma_j$ is in $\mathfrak{A}$. In
other words, $j$ is in $\mathfrak{a}$ if and only if $T_j$ has
cardinality $1$, so that $T_j=\{\sigma_j\}$, and
$\residueI_{\sigma_j}$ is reduced.  The algorithm in this paragraph
will compute a zero-dimensional parametrization of $V_{\mathfrak{A}}$;
we will use the following lemma to perform the decomposition.

\begin{lemma}\label{lemma:acb2}
  Let $j$ be in $\{1,\dots,c\}$ such that $\mu_j=1$, let $\lambda$ be
  a linear form over $\prod_{i \in T_j} \residueI_i$ and let $\mf=y_2
  X_2 + \cdots + y_n X_n$, for some $y_2,\dots,y_n$ in $\K$. Define constants
  $a,b,c$ in $\Kbar$ by
  $$a=\lambda(1),\quad b=\lambda(\mf),\quad c=\lambda(\mf^2).$$
  Then, $j$ is in $\mathfrak{a}$
  if and only if, for a generic choice of $\lambda$ and $\mf$, $ac=b^2$.
\end{lemma}
\begin{proof}
  The assumption that $\mu_j=1$ means that for all $i$ in $T_j$,
  $\nu_i=1$. The linear 
  form $\lambda$ can be uniquely written as a sum $\lambda=\sum_{i \in T_j}
  \ell_i$, where each $\ell_i$ is in ${\rm hom}_\Kbar(\residueI_i,\Kbar)$.
  The fact that all $\nu_i$ are equal to $1$ then implies that each $\ell_i$ takes the form 
  $$\ell_i: f \mapsto (\Lambda_{i}(f))(\balpha_i),$$
  where $\Lambda_{i}$ is a differential operator that does not 
  involve $\partial/\partial X_1$. Thus, as in \cref{ell_param}, we can write a general
  $\Lambda_i$ of this form as
  $$\Lambda_i: f \mapsto u_{i,1} f + \sum_{2 \le r \le n}
  P_{i,r}(u_{i,2},\dots,u_{i,D_i}) \frac{\partial}{\partial X_j} f +
  \sum_{2 \le r \le s \le n} P_{i,r,s}(u_{i,2},\dots,u_{i,D_i})
  \frac{\partial^2}{\partial X_j\partial X_k} f +
  \tilde\Lambda_i(f),$$ where all terms in $\tilde \Lambda_i$ have
  order at least $3$, $\bu_i=(u_{i,1},\dots,u_{i,D_i})$ are parameters and
  $(P_{i,r})_{2 \le r \le n}$ and $(P_{i,r,s})_{2 \le r \le s \le n}$
  are linear forms in $u_{i,2},\dots,u_{i,D_i}$.
  We obtain
  \begin{align*}
    \Lambda_i(1)   &= u_{i,1} \\
    \Lambda_i(\mf)   &= u_{i,1} \mf +\sum_{2 \le r \le n}P_{i,r}(u_{i,2},\dots,u_{i,D_i})y_r \\
    \Lambda_i(\mf^2) &= u_{i,1} \mf^2  +2 \mf \sum_{2 \le r \le n}P_{i,r}(u_{i,2},\dots,u_{i,D_i})y_r + 
    2\sum_{2 \le r \le s \le n} P_{i,r,s}(u_{i,2},\dots,u_{i,D_i})y_ry_s,
  \end{align*}
  which gives
  \begin{align*}
    a  &= \sum_{i\in T_j}u_{i,1} \\
    b  &= \sum_{i\in T_j}u_{i,1} \mf(\balpha_i) +\sum_{i \in T_j, 2 \le r \le n}P_{i,r}(u_{i,2},\dots,u_{i,D_i})y_r \\
    c &= \sum_{i\in T_j}u_{i,1} \mf(\balpha_i)^2  +2  \sum_{i \in T_j, 2 \le r \le n}\mf(\balpha_i) P_{i,r}(u_{i,2},\dots,u_{i,D_i})y_r    
   \\ & \phantom{{ }= \sum_{i\in T_j}u_{i,1} \mf(\balpha_i)^2}
   +2 \sum_{i \in T_j, 2 \le r \le s \le n} P_{i,r,s}(u_{i,2},\dots,u_{i,D_i})y_ry_s.
  \end{align*}
  Suppose first that $j$ is in $\mathfrak{a}$. Then, $T_j=\{\sigma_j\}$, so we 
  have only one term $\Lambda_{\sigma_j}$ to consider, and $\residueI_{\sigma_j}$ 
  is reduced, so that all coefficients $P_{\sigma_j,r}$ and
  $P_{\sigma_j,r,s}$ vanish. Thus, we are left in
  this case with
  $$
  a = u_{\sigma_j,1}, \quad
  b = u_{\sigma_j,1} \mf(\balpha_{\sigma_j}), \quad
  c = u_{\sigma_j,1} \mf(\balpha_{\sigma_j})^2,
  $$ so that we have $ac=b^2$, for {\em any} choice of $\lambda$ and
  $\mf$. Now, we suppose that $j$ is not in $\mathfrak{a}$, and we prove
  that for a generic choice of $\lambda$ and $\mf$, $ac-b^2$ is nonzero.
  The quantity $ac-b^2$ is a polynomial in the parameters
  $(\bu_i)_{i\in T_j}$, and $(y_i)_{i \in \{2,\dots,n\}}$, and we have
  to show that it is not identically zero. We discuss two cases; in both
  of them, we prove that a suitable specialization of $ac-b^2$ is
  nonzero.

  Suppose first that for at least one index $\sigma$ in $T_j$,
  $\residueI_\sigma$ is not reduced. In this case, there exists as least one
  index $\rho$ in $\{2,\dots,n\}$ such that
  $P_{\sigma,\rho}(u_{\sigma,2},\dots,u_{\sigma,D_\sigma})$ is not
  identically zero. Let us set all $\bu_{\sigma'}$
  to $0$, for $\sigma'$ in $T_j-\{\sigma\}$, as well as $u_{\sigma,1}$,
  and all $y_r$ for $r\ne \rho$. Then, under this specialization,
  $ac-b^2$ becomes
  $-(P_{\sigma,\rho}(u_{\sigma,2},\dots,u_{\sigma,D_\sigma})y_\rho)^2$,
  which is nonzero, so that $ac-b^2$ itself must be nonzero.

  Else, since $j$ is not in $\mathfrak{a}$, we can assume that $T_j$
  has cardinality at least $2$, with $\residueI_\sigma$ reduced for all $\sigma$
  in $T_j$ (so that $P_{\sigma,r}$ and $P_{\sigma,r,s}$ vanish for 
  all such $\sigma$ and all $r,s$). Suppose that $\sigma$ and $\sigma'$ are two indices in
  $T_j$; we set all indices $u_{\sigma'',1}$ to zero, for $\sigma''$
  in $T_j-\{\sigma,\sigma'\}$. We are left with
  $$
  a=u_{\sigma,1}+u_{\sigma',1},\quad
  b=u_{\sigma,1}\mf(\balpha_{\sigma})+u_{\sigma',1}\mf(\balpha_{\sigma'}),\quad
  c=u_{\sigma,1}\mf(\balpha_{\sigma})^2+u_{\sigma',1}\mf(\balpha_{\sigma'})^2.
  $$
  Then, $ac-b^2$ is equal to $2u_{\sigma,1}u_{\sigma',1}(\mf(\balpha_{\sigma})-\mf(\balpha_{\sigma'}))^2$,
  which is nonzero, since $\balpha_\sigma \ne \balpha_{\sigma'}$.
\end{proof}

The previous lemmas allow us to write Algorithm
$\mathsf{BlockParametrizationX}_1$. After computing $M$, we determine its
factor $F=\prod_{j \in \{1,\dots,c\}, \mu_j=1} (T-r_j)$, which is
obtained by taking the squarefree part of $M$ and dividing it by
its gcd with $\gcd(M,M')$. We split this polynomial further using the previous
lemma in order to find $\prod_{j \in \mathfrak{a}} (T-r_j)$, and we
conclude using the same kind of calculations as in Algorithm
$\mainalgoname$.

\begin{algorithm}[H]
  \caption{$\mainalgoname{\sf X}_1(\mM_1,\dots,\mM_n,\mU,\mV,\mf$)}
  {\bf Input:} \vspace{-0.5em}
  \begin{itemize}
    \item multiplication matrices $\mM_1,\dots,\mM_n$ in $\K^{D \times D}$
    \item  $\mU,\mV \in \mathbb{K}^{D \times m}$, for some block dimension  $m \in \{1,\dots,D\}$
    \item $\mf =y_2 X_2 + \cdots + y_n X_n$
  \end{itemize}
  {\bf Output:}  \vspace{-0.5em}
  \begin{itemize}
    \item polynomials $((F,G_1,\dots,G_n),X_1)$, with $F,G_1,\dots,G_n$ in $\K[T]$
  \end{itemize}
  \begin{enumerate}
    \item\label{X1step3} { compute $\mat{L}_s = \mUt\mM_1^s$ for $s=0,\dots,2d-1$, with $d = \lceil D/m \rceil$}
    \item\label{X1step4} { compute $\seqelt{s,\mU,\mV}= \mat{L}_s\mV$ for $s=0,\dots, 2d-1$}
    \item\label{X1step5} { compute a minimal matrix generator $\mat{P}_{\mU,\mV}$ of $(\seqelt{s,\mU,\mV})_{0 \le s < 2d}$}
    \item\label{X1step6} { let $M$ be the largest invariant factor of $\mat{P}_{\mU,\mV}$}
    \item\label{X1step7} { let $F$ be  the squarefree part  of $M$}
    \item\label{X1step7b} let $F = F /\gcd(F, \gcd (M, M'))$
    \item\label{X1step8} { let $\row{a}_1 = [M~0 ~\cdots~ 0] (\mat{P}_{\mU,\mV})^{-1}$}
    \item let $\mN = y_2 \mM_2 + \cdots+y_n \mM_n$
    \item \textbf{for} $i=0,1,2$ \textbf{do} \\
      \phantom{for}  let $A_i = \mathsf{{\sf ScalarNumerator}}(\mat{P}_{\mU,\mV}, M, 
      \mN^i \col{\varepsilon}_1, 1, \row{a}_1, (\mat{L}_s)_{0 \le s<d})$
    \item\label{X1step9} let $F = \gcd(F, A_0A_2-A_1^2)$
    \item\label{X1step10} \textbf{for} $i=2,\dots,n$ \textbf{do} \\
      \phantom{for}  let $A_{X_i} = \mathsf{{\sf ScalarNumerator}}(\mat{P}_{\mU,\mV}, M, \mM_i\col{\varepsilon}_1, 1, \row{a}_1, (\mat{L}_s)_{0 \le s <d})$
    \item\label{X1step11}   \textbf{return} $((F,T,A_{X_2}/ A_0 \bmod F, \dots,A_{X_n}/A_{0} \bmod F),X_1)$
  \end{enumerate}  \label{algo:X1block-sparse-fglm}
\end{algorithm}


\begin{lemma}
  For generic choices of $\mU$, $\mV$ and $\mf$, the output
  $((F,G_1,\dots,G_n),X_1)$ of $\mathsf{BlockParametrizationX}_1$ is a
  zero-dimensional parametrization of $V_{\mathfrak{A}}$.
\end{lemma}
\begin{proof}
  As in the case of $\mainalgoname$, for generic choices of $\mU$ and
  $\mV$, the degree bound $\deg(\mat{P}_{\mU,\mV}) \le d$ holds and
  $M$ is the minimal polynomial of $X_1$ in $\residueI$; hence, after
  Step~\ref{X1step7b}, we have $F=\prod_{j \in \{1,\dots,c\}, \mu_j=1}
  (T-r_j)$. 

  The calculation of $A_0,A_1,A_2$ and $A_{X_2},\dots,A_{X_n}$ is
  justified as in $\mainalgoname$, by means of
  \cref{lemma:omegaOmega}. For $i=1,\dots,D$, let $u_{i,1}$ is the entry at position
  $(i,1)$ in $\mU$, and define the linear 
  form $\ell: \residueI \to \K$ by 
  $$\ell(f) = \sum_{i=1}^D f_i u_{i,1}, \quad\text{for}\quad f =
  \sum_{i=1}^D f_i b_i.$$ Then, the above lemma proves that $A_i =
  \Omega((\ell(\mf^i X_1^s))_{s\ge0},M)$ for $i=0,1,2$ and $A_{X_i} =
  \Omega((\ell(X_i X_1^s))_{s\ge0},M)$ for $i=2,\dots,n$.

  Take then $j$ in $\{1,\dots,c\}$ such that $\mu_j=1$, that is, a
  root of $F$ as computed at Step~\ref{X1step7b}. By
  \cref{lemma:valuelambda}, for $i=0,1,2$, we have $ A_i(r_j) = M'(r_j)
  (\mf^i \cdot \lambda_j)(1)$, where $\lambda_j =\sum_{i \in T_j}
  \ell_i$, where the $\ell_i$'s are the components of $\ell$,
  and where $\mf^i \cdot \lambda_j$ is the linear form $f \mapsto \lambda_j(\mf^i f)$.

  As a result, the value of $ A_0  A_2 -  A_1^2$ at
  $r_j$ is (up to the nonzero factor $M'(r_j)^2$) equal to the
  quantity $ac-b^2$ defined in \cref{lemma:acb2}, so for a
  generic choice of $\ell$ (that is, of $\mU$) and $\mf$, it vanishes if and only if $j$ is
  in $\mathfrak{a}$. Thus, after Step~\ref{X1step9}, 
  $F$ is equal to $\prod_{j \in \mathfrak{a}} (T-r_j)$.

  The last step is to compute the zero-dimensional parametrization of
  $V_{\mathfrak{A}}$. This is done using again
  \cref{lemma:valuelambda}. Indeed, for $j$ in $\mathfrak{a}$, 
  $T_j$ is simply equal to $\{\sigma_j\}$, so that we have, for $i=2,\dots,n$,
  $$ A_0(r_j)=M'(r_j) \lambda_j(1) \quad\text{and}\quad 
  A_{X_i}(r_j) = M'(r_j) (X_i \cdot \lambda_j)(1) = M'(r_j) \lambda_j(X_i),$$
  where as above, $X_i \cdot \lambda_j$ is the linear form $f \mapsto \lambda_j(X_i f)$.
  Now, since $j$
  is in $\mathfrak{a}$, $\residueI_{\sigma_j}$ is reduced, so that there
  exists a constant $\lambda_{j,1}$ such that for all $f$ in
  $\Kbar[X_1,\dots,X_n]$, $\lambda_j(f)$ takes the form $\lambda_{j,1}
  f(\balpha_{\sigma_j})$. This shows that, as claimed,
  $$\frac{ A_{X_j}(r_j)}{ A_0 (r_j)} = 
  %% \frac {\Omega((\ell(X_j X_1^s)_{s\ge0},M)(r_j)}{\Omega((\ell(X_1^s))_{s\ge0},M)(r_j)}=
  \frac
  {M'(r_j) \lambda_{j,1} \alpha_{\sigma_j,i}}{M'(r_j) \lambda_{j,1}} = \alpha_{\sigma_j,i},$$
  since $M'(r_j)$ is nonzero.
  For $i=1$, since we use $X_1$ as a separating variable for $V_{\mathfrak{A}}$, 
  we simply insert the polynomial $T$ into our list.
\end{proof}

The cost analysis is the same as that of Algorithm $\mainalgoname$, the
crucial difference being that the density $\rho_1$ of $\mM_1$ plays the 
role of the density $\rho$ of $\mM$ used in that algorithm.

%%%%%%%%%%%%%%%%%%%%%%%%%%%%%%%%%%%%%%%%%%%%%%%%%%%%%%%%%%%%

\subsection{Computing correction matrices}

Next, we describe an operation of decomposition for linear forms
$\residueI \to \K$; this is essentially akin to the Chinese Remainder
Theorem. We then use it to compute the sequences of correction
matrices $(\mat{\Delta}_s),(\mat{\Delta}'_s),(\mat{\Delta}''_s)$
defined in Algorithm $\mathsf{BlockParametrizationWithSplitting}$.

As a preamble, we introduce the notation $\PP(r,t)$ for the cost of a
{\em power projection} operation, as defined
in~\citep{Shoup94,Shoup99}: given a polynomial $F$ in $\K[T]$ of degree
$r$, a linear form $\ell: \K[T]/F \to \K$, and $H$ in $\K[T]/F$, the
goal is to compute $(\ell(H^s))_{0 \le s < t}$, for some upper bound
$t$. We denote this operation by ${\sf PowerProjection}(F,H,\ell,t)$; this is
essentially the analogue for univariate polynomials of the Krylov
computations that we heavily rely on in this paper. Here, $\ell$ 
is represented by the vector $(\ell(1),\ell(T),\dots,\ell(T^{r-1}))$.

\citet[Theorem~4]{Shoup94} showed that this can be done in
$\PP(r,t)=O(r^{(\omega-1)/2} t)$ operations in $\K$ for $t \le r$. For
$t \ge r$, we first solve the problem up to index $r$ in time
$O(r^{(\omega+1)/2})$; then we use the fact that the sequence 
$(\ell(H^s))_{s \ge 0}$
is linearly recurrent to compute all further values in time
$O(t\M(r)/r)$, as for instance in~\citep[Proposition~1]{BoFlSaSc06}.
Thus, for $t \ge r$, we take $\PP(r,t)=O(r^{(\omega+1)/2} +
t\M(r)/r)$.

\medskip

Let $\mathfrak{A}$ and $\mathfrak{B}$ be defined as in the previous 
subsection, and let $D_{\mathfrak{A}}$ be the number of points in
$V_{\mathfrak{A}}$. Since $\residueI_i$ is a reduced algebra for all
$i$ in $\mathfrak{A}$, $D_\mathfrak{A}$ is also the dimension of
$\residueI_\mathfrak{A} = \prod_{i \in \mathfrak{A}}
\residueI_i$ and $\residueI_\mathfrak{B}=\prod_{i \in \mathfrak{B}}
\residueI_i$ has dimension $D_{\mathfrak{B}}=D-D_{\mathfrak{A}}$.

Consider a linear form $\ell: \residueI \to \K$; we still denote
$\ell$ for its extension $\residueI \otimes_\K \Kbar \to \Kbar$.  It
can then be decomposed as $\ell= \ell_{\mathfrak{A}} +
\ell_{\mathfrak{B}}$, with $\ell_\mathfrak{A}: \residueI_\mathfrak{A}
\to \Kbar$ and $\ell_\mathfrak{B}: \residueI_\mathfrak{B} \to \Kbar$.
Remark that the support of $\ell_\mathfrak{B}$ is contained in
$\mathfrak{B}$, and actually equal to $\mathfrak{B}$ for a generic
$\ell$.

Suppose that we are given the minimal polynomial $M$ of $X_1$ in
$\residueI$, the numerator $C=\Omega( (\ell(X_1^s))_{s \ge 0}, M)$, as
well as the zero-dimensional parametrization $((F,G_1,\dots,G_n),X_1)$
of $V_\mathfrak{A}$ computed in the previous paragraph.  Given
$\lf=t_1 X_1 + \cdots+ t_n X_n$, and an upper bound $\tau$, we show
how to compute the values $\ell_\mathfrak{A}(\lf^s)$, for
$s=0,\dots,\tau-1$.

Let $E=M/F$; the division is exact and  $E$ and $F$ are coprime, by construction.
The equality $\ell= \ell_{\mathfrak{A}} + \ell_{\mathfrak{B}}$ implies
an equality between generating series
\begin{align*}
  \sum_{s \ge 0} \frac{\ell(X_1^s)}{T^{s+1}} & = \sum_{s \ge 0}\frac{\ell_\mathfrak{A}(X_1^s)}{T^{s+1}}  
  +\sum_{s \ge 0} \frac{\ell_\mathfrak{B}(X_1^s)}{T^{s+1}}\\
                                             & = \frac{A}{F} + \frac{B}{E},
\end{align*}
for some polynomials $A,B$ in $\K[T]$, with $\deg(A) < \deg(F)$ and
$\deg(B) < \deg(E)$. With  $C=\Omega( (\ell(X_1^s))_{s \ge 0}, M)$, we deduce the equality
$$\frac{C}{M}=\frac{A}{F} + \frac{B}{E},$$ from which we find $A = C/E
\bmod F$. Knowing $A$ and $F$ allows us to compute the values
$\ell_\mathfrak{A}(X_1^s)$, for $s=0,\dots,D_\mathfrak{A}-1$, by
Laurent series expansion.  Since $\residueI_\mathfrak{A}$ is reduced,
we have $\lf = t_1 G_1 + \cdots t_n G_n$ in $\residueI_\mathfrak{A}$,
where $G_1,\dots,G_n$ are polynomials in the zero-dimensional
parametrization of $V_\mathfrak{A}$. As a result, we can finally
compute $\ell_\mathfrak{A}(\lf^s)$, for $s=0,\dots,\tau-1$ by
applying our algorithm for univariate power projection to $G=t_1 G_1 + \cdots t_n G_n$.

The following pseudo-code summarizes this discussion; the cost of the
algorithm is $O(\M(D_\mathfrak{A})\log(D_\mathfrak{A}) +
\PP(D_\mathfrak{A},\tau) +nD_\mathfrak{A})$ operations in $\K$, where the
first term accounts for the cost of the first three steps,
$\PP(D_\mathfrak{A},\tau)$ is the cost of power projection
and 
the term $O(nD_\mathfrak{A})$ describes the cost of computing $G$ as defined above.

\begin{algorithm}[H]
  \caption{${\sf Decompose}(M, C, ((F,G_1,\dots,G_n),X_1), \lf,\tau)$} {\bf
  Input:} \vspace{-0.5em}
  \begin{itemize}
    \item minimal polynomial $M$ of $X_1$ in $\residueI$
    \item numerator $C=\Omega( (\ell(X_1^s))_{s \ge 0}, M)$
    \item zero-dimensional parametrization $((F,G_1,\dots,G_n),X_1)$ of $V_\mathfrak{A}$
    \item $\lf =t_1 X_1 + \cdots + t_n X_n$
    \item a bound $\tau$
  \end{itemize}
  {\bf Output:}  \vspace{-0.5em}
  \begin{itemize}
    \item $\ell_\mathfrak{A}(\lf^s)$, for $s=0,\dots,\tau-1$
  \end{itemize}
  \begin{enumerate}
    \item let $E=M/F$
    \item let $A=C/E \bmod F$
    \item compute the first $D_\mathfrak{A}$ terms $(v_0,\dots,v_{D_\mathfrak{A}-1})$ of the Laurent series $A/F$
    \item {\bf return} ${\sf PowerProjection}(F, t_1 G_1 + \cdots + t_n G_n, (v_0,\dots,v_{D_\mathfrak{A}-1}),\tau)$
  \end{enumerate}
\end{algorithm}

We will use this algorithm to compute the correction matrices to be
used in Algorithm $\mainalgoname{\sf Residual}$. We assume that we
have stored various quantities computed in Algorithm
$\mainalgoname{\sf X}_1$: the sequence of matrices
$(\seqelt{s,\mU,\mV})_{0 \le s < 2d}$, the matrix generator
$\mat{P}_{\mU,\mV}$, the minimal polynomial $M$ of $X_1$, and of
course the zero-dimensional parametrization $((F,G_1,\dots,G_n),X_1)$.


Let $\mU$ and $\mV$ be the blocking matrices used in
$\mainalgoname{\sf X}_1$. For $i=1,\dots,m$, we let
$\ell_i: \residueI \to \K$ be the linear form whose values on the
basis $\basis=(b_1,\dots,b_D)$ are given by the $i$-th column of
$\mU$. In other words, $\ell_i(f) = \sum_{j=1}^D f_j u_{j,i}$, for
$f=\sum_{j=1}^D f_j b_j$. Similarly, for $j=1,\dots,m$, we let
$\gamma_j$ be the element of $\residueI$ whose coefficient vector on
the basis $\basis$ is the $j$-th column of $\mV$.

Hereafter, we also write $d_\mathfrak{B}=\lceil D_\mathfrak{B}/m \rceil$,
in analogy with the definition of $d$ used so far.
\begin{itemize}
  \item To each $(i,j)$ in $\{1,\dots,m\}\times \{1,\dots,m\}$ is
    associated a linear form $\ell_{i,j}: \residueI\to \K$ defined by
    $\ell_{i,j}(f) =\ell_i(\gamma_j f)$ for all $f$ in $\residueI$.
    Then, the entry $(i,j)$ of the matrix sequence $(\mUt \mM_1^s
    \mV)_{s \ge0}$ is the scalar sequence $(\ell_{i,j}(X_1^s))_{s \ge
    0}$.

    \smallskip

    For all such $(i,j)$, since we know the minimal polynomial $M$ of
    $X_1$, we can compute the scalar numerator $C_{i,j} \in \K[T]$
    associated to $\ell_{i,j}$ and $M$. This is done by applying
    Algorithm $\mathsf{{\sf ScalarNumerator}}$ of
    \cref{ssec:scalar_numer}, using the row vector $\row{a}_i$ defined
    in that section, together with the sequence of matrices
    $\seqelt{s,\mU,\mV}$ and the matrix generator $\mat{P}_{\mU,\mV}$
    computed in Algorithm $\mainalgoname{\sf X}_1$.

    \smallskip

    Once $C_{i,j}$ is known, we can call ${\sf Decompose}$, which allows us
    to compute $\ell_{i,j,\mathfrak{A}}(\lf^s)$, for
    $s=0,\dots,2d_\mathfrak{B}-1$.  We can then construct the sequence
    $(\mat{\Delta}_s)_{0 \le s < 2d_\mathfrak{B}}$ of matrices in
    $\K^{m\times m}$ by setting the $(i,j)$-th entry of $\mat{\Delta}_s$
    to be $\ell_{i,j,\mathfrak{A}}(\lf^s)$.

    \smallskip

  \item To each $(i,k)$ in $\{1,\dots,m\}\times \{1,\dots,n\}$ is
    associated a linear form $\ell'_{i,k}: \residueI\to \K$ defined by
    $\ell'_{i,k}(f) =\ell_i(X_k f)$ for all $f$ in $\residueI$.  Then,
    the $i$th entry of the sequence of column vectors $(\mUt
    \mM_1^s \mM_k \col{\varepsilon}_1)_{s \ge0}$ is the scalar sequence
    $(\ell'_{i,k}(X_1^s))_{s \ge 0}$, where $\col{\varepsilon}_1$ is the
    column vector $\trsp{[1~0\cdots~0]}$ we already used several times.

    \smallskip

    Proceeding as before, we construct the sequence of $m \times n$ matrices
    $(\mat{\Delta}'_s)_{0 \le s < d_\mathfrak{B}}$ by setting the
    $(i,k)$-th entry of $\mat{\Delta}'_s$ to be
    $\ell'_{i,k,\mathfrak{A}}(\lf^s)$. Note that we will only need $d_\mathfrak{B}$
    entries in this sequence.

    \smallskip

  \item Finally, we apply this process to the linear forms $\ell_i$
    themselves; they are such that the $i$th entry of the sequence of
    column vectors $(\mUt \mM_1^s \col{\varepsilon}_1)_{s \ge0}$ is
    the scalar sequence $(\ell_{i}(X_1^s))_{s \ge 0}$. Using again
    $\mathsf{{\sf ScalarNumerator}}$ and {\sf Decompose}, we construct
    the sequence of column vectors $(\mat{\Delta}''_s)_{0 \le s <
    d_\mathfrak{B}}$ by setting the $i$-th entry of $\mat{\Delta}''_s$
    to $\ell_{i,\mathfrak{A}}(\lf^s)$.
\end{itemize}

In terms of cost, we need to compute all vectors $\row{a}_i$, for
$i=1,\dots,m$; using the analysis of \cref{eqn:hol_cost}, this takes
$O(m^{\omega} \M(D) \log(D) \log(m))$ operations in $\K$.  Then, the total time
spent in {\sf ScalarNumerator} is $m(m+n+1)$ times the cost reported in
\cref{ssec:scalar_numer}, which was $O(D^2 + m\M(D))$; similarly, the total
cost incurred by ${\sf Decompose}$ is $m(m+n+1)$ times the cost of a single
call, which was reported above.

%%%%%%%%%%%%%%%%%%%%%%%%%%%%%%%%%%%%%%%%%%%%%%%%%%%%%%%%%%%%

\subsection{Describing the residual set}

We finally describe Algorithm $\mainalgoname{\sf Residual}$.  Let
$\mathfrak{A}$ and $\mathfrak{B}$ be as in the previous section.  This
part of the main algorithm computes a zero-dimensional parametrization
of the residual set $V_\mathfrak{B}=V-V_\mathfrak{A}$. For this, we
are going to call a modified version Algorithm
$\mathsf{BlockParametrization}$, where we update the values of our
matrix sequences before computing the minimal matrix generator, using
the correction matrices defined just above.

The resulting algorithm is as follows. A superficial difference with
$\mathsf{BlockParametrization}$ is that names of the main variables
have been changed (so as not to create any confusion with those used
in $\mainalgoname{\sf X}_1$). More importantly, using the correction
matrices makes it possible for us to compute fewer terms in the
sequences, namely only $2\lceil D_\mathfrak{B}/m\rceil$ and
$\lceil D_\mathfrak{B}/m\rceil$, respectively. Hence, if $D_\mathfrak{B} \ll D$
(that is, $V_\mathfrak{B}$ contains few points, with small
multiplicities), this last stage of the algorithm will be fast.

The algorithm uses a subroutine called {\sf ScalarNumeratorCorrected}
at Steps~8 and~9.  It is similar to Algorithm {\sf ScalarNumerator} of
\cref{ssec:scalar_numer}, with a minor difference: instead of computing the
vectors $\mat{D}_s \col{\varepsilon}_1$, resp.\ $\mat{D}_s\mM_i
\col{\varepsilon}_1$, at the first step of  {\sf ScalarNumerator}, it computes $\mat{D}_s
\col{\varepsilon}_1-\mat{\Delta}''_s$, resp.\ $\mat{D}_s\mM_i
\col{\varepsilon}_1-\mat{\Delta}'_{s,i}$, where $\mat{\Delta}'_{s,i}$
is the $i$th column of $\mat{\Delta}'_{s}$.

\begin{algorithm}[H]
  \caption{$\mainalgoname{\sf Residual}(\mM_1,\dots,\mM_n,\mU,\mV,(\mat{\Delta}_s),(\mat{\Delta}'_s),(\mat{\Delta}''_s),\lf)$}
  {\bf Input:} \vspace{-0.5em}
  \begin{itemize}
    \item $\mM_1,\dots,\mM_n$ defined as above
    \item  $\mU,\mV \in \mathbb{K}^{D \times m}$, for some block dimension  $m \in \{1,\dots,D\}$
    \item sequences of correction matrices $(\mat{\Delta}_s),(\mat{\Delta}'_s),(\mat{\Delta}''_s)$
    \item $\lf =t_1 X_1 + \cdots + t_n X_n$
  \end{itemize}
  {\bf Output:}  \vspace{-0.5em}
  \begin{itemize}
    \item  polynomials $((R,W_1,\dots,W_n),\lf)$, with $R,W_1,\dots,W_n$ in $\K[T]$
  \end{itemize}
  \begin{enumerate}
    \item\label{residualstep1}   let $\mM = t_1 \mM_1 + \cdots + t_n \mM_n$
    \item\label{residualstep3} { compute $\mat{D}_s = \mUt\mM^s$ for $s=0,\dots,2d_\mathfrak{B}-1$, with $d_\mathfrak{B} = \lceil D_\mathfrak{B}/m \rceil$}
    \item\label{residualstep4} { compute $\mat{E}_{s,\mU,\mV}= \mat{D}_s\mV-\mat{\Delta}_s$ for $s=0,\dots, 2d_\mathfrak{B}-1$}
    \item\label{residualstep5} { compute a minimal matrix generator $\mat{S}_{\mU,\mV}$ of $(\mat{E}_{s,\mU,\mV})_{0 \le s < 2d_\mathfrak{B}}$}
    \item\label{residualstep6} { let $S$ be the largest invariant factor of $\mat{S}_{\mU,\mV}$}
    \item\label{residualstep7} { let $R$ be  the squarefree part  of $S$}
    \item\label{residualstep8} { let $\row{a}_1 = [S~0 ~\cdots~ 0] (\mat{S}_{\mU,\mV})^{-1}$}
    \item\label{residualstep9}  let $C_1 = \mathsf{{\sf ScalarNumeratorCorrected}}(\mat{S}_{\mU,\mV}, S, \col{\varepsilon}_1, 1, \row{a}_1,  (\mat{D}_s)_{0 \le s < d_\mathfrak{B}}, (\mat{\Delta}''_s)_{0 \le s < d_\mathfrak{B}})$
    \item\label{residualstep10} \textbf{for} $i=1,\dots,n$ \textbf{do} \\
      \phantom{for}let $C_{X_i} = \mathsf{{\sf ScalarNumeratorCorrected}}(\mat{S}_{\mU,\mV},S, \mM_i\col{\varepsilon}_1, 1, \row{a}_1, (\mat{D}_s)_{0 \le s < d_\mathfrak{B}}, (\mat{\Delta}'_s)_{0 \le s < d_\mathfrak{B}})$
    \item\label{residualstep11}     \textbf{return} $((R, C_{X_1}/ C_1 \bmod R, \dots, C_{X_n}/ C_{1} \bmod R),\lf)$
  \end{enumerate}  \label{algo:block-sparse-fglm-residual}
\end{algorithm}

Let us prove correctness. Since $\residueI=\residueI_\mathfrak{A}
\times \residueI_\mathfrak{B}$, we may assume without loss of
generality that our multiplication matrices are block diagonal, with
two blocks corresponding respectively to bases of $\residueI_\mathfrak{A}$
and $\residueI_\mathfrak{B}$; if not, apply a change of basis to 
reduce to this situation, updating $\mU$ and $\mV$ accordingly. 

We denote by $\mM_\mathfrak{A}$ and $\mM_\mathfrak{B}$ the 
two blocks on the diagonal of matrix $\mM$.
The projection matrices can also be divided into blocks, namely as
$$\mU = \left [\begin{matrix}\mU_\mathfrak{A} \\\mU_\mathfrak{B}
\end{matrix}\right ] \quad\text{and}\quad
\mV = \left [\begin{matrix}\mV_\mathfrak{A} \\\mV_\mathfrak{B}
\end{matrix}\right ],$$
and we have $\mUt \mM^s \mV = \mUt_\mathfrak{A}
\mM_\mathfrak{A}^s \mV_\mathfrak{A} + \mUt_\mathfrak{B}
\mM_\mathfrak{B}^s \mV_\mathfrak{B}$ for $s \ge 0$. The first summand
is none other than the matrix $\mat{\Delta}_s$, so that
$\mat{E}_{s,\mU,\mV}$ is equal to $\mUt_\mathfrak{B}
\mM_\mathfrak{B}^s \mV_\mathfrak{B}$. These are thus the kind of
Krylov matrices we would obtain if we were working with a basis of
$\residueI_\mathfrak{B}$, and shows that we have enough
terms to compute a minimal matrix generator
$\mat{S}_{\mU,\mV}$, at least for generic $\mU$ and $\mV$. 
Similarly, $S$ is generically the minimal polynomial of $X_1$ in
$Q_\mathfrak{B}$, and $R$ its squarefree part.

The same considerations justify the computation of $C_1$ and
$C_{X_1},\dots,C_{X_n}$. Indeed, subtracting the correction matrices
implies that 
$$C_1 = \Omega( (\ell_1(\lf^s)-\ell_{1,\mathfrak{A}}(\lf^s))_{s \ge 0}, S) =  \Omega( (\ell_{1,\mathfrak{B}}(\lf^s))_{s \ge 0}, S)$$
and
$$C_{X_i} = \Omega( (\ell_1(X_i \lf^s)-\ell_{1,\mathfrak{A}}(X_i \lf^s))_{s \ge 0}, S) =  \Omega( (\ell_{1,\mathfrak{B}}(X_i\lf^s))_{s \ge 0}, S),$$
for $i=1,\dots,n$. As shown in \cref{ssec:abstractlago}, these are precisely 
the polynomials we need to compute a zero-dimensional parametrization
of $V_\mathfrak{B}$.

The cost analysis is similar to that of Algorithm $\mainalgoname$,
with the important exception that the sequence length $d=\lceil
D/m\rceil$ can then be replaced by the (hopefully much smaller)
$d_\mathfrak{B}=\lceil D_\mathfrak{B}/m\rceil$.


%%%%%%%%%%%%%%%%%%%%%%%%%%%%%%%%%%%%%%%%%%%%%%%%%%%%%%%%%%%%

\subsection{Experimental results}

The algorithms in this section were implemented using the same
platform as the one in the previous section, using in particular NTL's
built-in implementation of power projection.

In \cref{tbl:comparison_algos}, we give the ratio of the runtime of
{\sf Block\-Parametrization\-WithSplitting} to that of our first
algorithm, $\mainalgoname$, for each input: numbers less than $1$
indicate a speed-up.  The last column shows the number of points in
$V_\mathfrak{A}$, that is, $D_\mathfrak{A}$, compared to the total
degree of $I$, which is $D=D_\mathfrak{A}+D_\mathfrak{B}$. The inputs,
the machine used for timings and the prime field are the same as in
\cref{section:ex}.

The performance of $\mathsf{BlockParametrizationWithSplitting}$
depends on the density of $\mM_1$ and the number of points
$D_\mathfrak{A}$.  For generic inputs, with no multiplicity,
$D_\mathfrak{A}=D$, so we will actually not need to spend any time
computing correction matrices, or running $\mainalgoname{\sf
Residual}$. On the other hand, in the worst case, if
$D_\mathfrak{A}=0$, so that $D_\mathfrak{B}=D$, then Algorithm
$\mathsf{BlockParametrizationWithSplitting}$ may take more than twice as
long as $\mainalgoname$ (due to the two calls to respectively
$\mathsf{BlockParametrizationX}_1$ and
$\mathsf{BlockParametrizationResidual}$, together with the overhead induced 
by power projection).

This unlucky case was seldomly seen in our experiments, since the
systems with multiplicities generated randomly had few multiple
points, and thus were favorable to us. An unfavourable case is system
sot1, where $V_\mathfrak{A}$ only accounts for $1012$ points out of
$7682$ points in the variety. 


\begin{table}[H]
  \centering
  \def\arraystretch{1.2}
  \setlength\tabcolsep{6pt}
  \caption{Comparison of $\mathsf{BlockParametrizationWithSplitting}$ and $\mainalgoname$}
  \label{tbl:comparison_algos}
  \begin{tabular}{c|c|c|c|c|c|c}
    \textbf{name}& $\bm{n}$ & $\bm{D}$ & $\bm{m = 1}$ & $\bm{m = 3}$ & $\bm{m = 6}$&$D_\mathfrak{A}/D$\\
    \hline
    rand1-26&3 &17576&0.453&0.384&0.65&17576/17576 \\
    rand1-28&3 &21952&0.438&0.435&0.562& 21952/21952\\
    rand1-30&3 &27000&0.429&0.577&0.608&27000/27000 \\
    rand2-10&4 &10000&0.437&0.462&0.49& 10000/10000\\
    rand2-11&4 &14641&0.423&0.566&0.435&14641/14641 \\
    rand2-12&4 &20736&0.431&0.437&0.399&20736/20736 \\
    mixed1-22&3 &10864&0.49&0.568&0.791& 10648/10675\\
    mixed1-23&3 &12383&0.477&0.546&0.655& 12167/12194\\
    mixed1-24&3 &14040&0.463&0.514&0.613& 13824/13851\\
    mixed2-10&4 &10256&0.43&0.482&0.626& 10000/10016\\
    mixed2-11&4 &14897&0.414&0.408&0.521& 14641/14657\\
    mixed2-12&4 &20992&0.416&0.438&0.416&20736/20752 \\
    mixed3-12&12 &4109&0.453&0.513&0.664& 4096/4097\\
    mixed3-13&13 &8206&0.435&0.454&0.471& 8192/8193\\
    eco12&12 &1024&0.446&0.572&0.602& 1024/1024\\
    sot1&5 &8694&1.31&1.84&2.37& 1012/8694\\
    W1-6-5-2&5 &18000&0.462&0.471&0.472& 18000/18000\\
    W1-4-6-2&6 &6480&0.452&0.474&0.57& 6480/6480\\
    katsura10&11 &1024&0.557&0.661&0.652& 1024/1024
  \end{tabular}
\end{table}


\bibliographystyle{elsarticle-harv} 
\bibliography{biblio}

\end{document}
