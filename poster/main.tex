%%%%%%%%%%%%%%%%%%%%%%%%%%%%%%%%%%%%%%%%%
% Jacobs Landscape Poster
% LaTeX Template
% Version 1.1 (14/06/14)
%
% Created by:
% Computational Physics and Biophysics Group, Jacobs University
% https://teamwork.jacobs-university.de:8443/confluence/display/CoPandBiG/LaTeX+Poster
% 
% Further modified by:
% Nathaniel Johnston (nathaniel@njohnston.ca)
%
% This template has been downloaded from:
% http://www.LaTeXTemplates.com
%
% License:
% CC BY-NC-SA 3.0 (http://creativecommons.org/licenses/by-nc-sa/3.0/)
%
%%%%%%%%%%%%%%%%%%%%%%%%%%%%%%%%%%%%%%%%%

%----------------------------------------------------------------------------------------
%	PACKAGES AND OTHER DOCUMENT CONFIGURATIONS
%----------------------------------------------------------------------------------------

\documentclass[final]{beamer}

\usepackage{mathtools}
\usepackage[scale=1.24]{beamerposter} % Use the beamerposter package for laying out the poster

\usetheme{confposter} % Use the confposter theme supplied with this template

\setbeamercolor{block title}{fg=ngreen,bg=white} % Colors of the block titles
\setbeamercolor{block body}{fg=black,bg=white} % Colors of the body of blocks
\setbeamercolor{block alerted title}{fg=white,bg=dblue!70} % Colors of the highlighted block titles
\setbeamercolor{block alerted body}{fg=black,bg=dblue!10} % Colors of the body of highlighted blocks
% Many more colors are available for use in beamerthemeconfposter.sty

%-----------------------------------------------------------
% Define the column widths and overall poster size
% To set effective sepwid, onecolwid and twocolwid values, first choose how many columns you want and how much separation you want between columns
% In this template, the separation width chosen is 0.024 of the paper width and a 4-column layout
% onecolwid should therefore be (1-(# of columns+1)*sepwid)/# of columns e.g. (1-(4+1)*0.024)/4 = 0.22
% Set twocolwid to be (2*onecolwid)+sepwid = 0.464
% Set threecolwid to be (3*onecolwid)+2*sepwid = 0.708

\newlength{\sepwid}
\newlength{\onecolwid}
\newlength{\twocolwid}
\newlength{\threecolwid}
\setlength{\paperwidth}{48in} % A0 width: 46.8in
\setlength{\paperheight}{36in} % A0 height: 33.1in
\setlength{\sepwid}{0.024\paperwidth} % Separation width (white space) between columns
\setlength{\onecolwid}{0.22\paperwidth} % Width of one column
\setlength{\twocolwid}{0.464\paperwidth} % Width of two columns
\setlength{\threecolwid}{0.708\paperwidth} % Width of three columns
\setlength{\topmargin}{-0.5in} % Reduce the top margin size
%-----------------------------------------------------------

\usepackage{graphicx}  % Required for including images
\usepackage{multicol}
\usepackage{booktabs} % Top and bottom rules for tables

% Colors
\definecolor{classicrose}{rgb}{0.98, 0.8, 0.91}
\definecolor{antiquefuchsia}{rgb}{0.57, 0.36, 0.51}
\definecolor{blanchedalmond}{rgb}{1.0, 0.92, 0.8}

\setbeamertemplate{itemize/enumerate body begin}{\normalsize}
\setbeamertemplate{itemize/enumerate subbody begin}{\normalsize}

%----------------------------------------------------------------------------------------
%	TITLE SECTION 
%----------------------------------------------------------------------------------------

\title{Sparse 
	$\mathbf{\sqrt{FGLM}}$ using the block Wiedemann algorithm} % Poster title

\author{Seung Gyu Hyun, Vincent Neiger, Hamid Rahkooy, \'Eric Schost} % Author(s)

\institute{University of Waterloo, DTU Compute} % Institution(s)

%----------------------------------------------------------------------------------------

\begin{document}

%\addtobeamertemplate{block end}{}{\vspace*{2ex}} % White space under blocks
\addtobeamertemplate{block alerted end}{}{\vspace*{2ex}} % White space under highlighted (alert) blocks

\setlength{\belowcaptionskip}{2ex} % White space under figures
\setlength\belowdisplayshortskip{2ex} % White space under equations

\begin{frame}[t] % The whole poster is enclosed in one beamer frame

\begin{columns}[t] % The whole poster consists of three major columns, the second of which is split into two columns twice - the [t] option aligns each column's content to the top

\begin{column}{\sepwid}\end{column} % Empty spacer column

\begin{column}{\onecolwid} % The first column

%----------------------------------------------------------------------------------------
%	OBJECTIVES
%----------------------------------------------------------------------------------------

%\begin{alertblock}{Objectives}

%Lorem ipsum dolor sit amet, consectetur, nunc tellus pulvinar tortor, commodo eleifend risus arcu sed odio:
%\begin{itemize}
%\item Mollis dignissim, magna augue tincidunt dolor, interdum vestibulum urna
%\item Sed aliquet luctus lectus, eget aliquet leo ullamcorper consequat. Vivamus eros sem, iaculis %ut euismod non, sollicitudin vel orci.
%\item Nascetur ridiculus mus.  
%\item Euismod non erat. Nam ultricies pellentesque nunc, ultrices volutpat nisl ultrices a.
%\end{itemize}

%\end{alertblock}

%----------------------------------------------------------------------------------------
%	INTRODUCTION
%----------------------------------------------------------------------------------------

\begin{block}{Introduction}

% add a block for sim/dif - compute something weaker, always works
% doesn't require berlekamp-massey-sakata
% base field needs to be higher than D, generic coordinate
% Correctness also follows from 5
% square free R_n
% in example highlight D = ?
% highlight parallel ...

\begin{itemize}
	\item Compute the Gr\"obner basis for degrevlex ordering first (fast) and convert to lex ordering (better structure)
\end{itemize}


\end{block}

\begin{alertblock}{Main Problem}
	\textbf{Input: }
	\begin{itemize}
	\item Zero-dimensional ideal $I \subset
	\mathbb{K}[x_1,\dots,x_n]$ by means of a monomial basis 
	$\mathbb{B} \subset Q$,\\ 
	$Q = \mathbb{K}[x_1,\dots,x_n]/I$
	\item Multiplication matrices  $T_1,\dots, T_n \in 
	\mathbb{K}^{D \times D}$ of $x_1,\dots,x_n$, with
	$D = dim_{\mathbb{K}}(Q)$
	\end{itemize}
	\textbf{Output:}
	\begin{itemize}
		\item Lex Gr\"obner basis of $\sqrt{I}$
	\end{itemize}
\end{alertblock}

\begin{alertblock}{Assumptions}
	\begin{itemize}
		\item Base field is larger than $D$
		\item $x_n$ \textit{separates} the points of $V(I)$
		\item Ensured by a generic change of coordinates
		\item Under assumption, $\sqrt{I}$ is in shape position
		\item $I \subset \mathbb{K}[x_1,\dots,x_n]$ is in \textit{shape position} if its Gr\"obner basis has the form $(x_1-R_1(x_n), 
		\dots, x_{n-1}-R_{n-1}(x_n), R(x_n))$
	\end{itemize}
\end{alertblock}

%\setbeamercolor{block alerted title}{fg=white,bg=antiquefuchsia}
%\setbeamercolor{block alerted body}{fg=black,bg=white}

\begin{alertblock}{Sparse FGLM}
	\begin{itemize}
	\item Sparse FGLM algorithm of \cite{FaMo17} computes lex basis of an ideal $I$ when $I$ is in shape position
	\item Exploits the sparsity of $T_i$'s
	\item Difficult to parallelize
	\end{itemize}
\end{alertblock}

\begin{alertblock}{Differences}
	\begin{itemize}
		\item If $I$ is not in shape position, Sparse FGLM uses 
		Berlekamp-Massey-Sakata to compute the lex basis
		\item We compute lex basis of $\sqrt{I}$ (weaker), which is in shape position by 
		assumption
	\end{itemize}
\end{alertblock}
	

%------------------------------------------------

%----------------------------------------------------------------------------------------

\end{column} % End of the first column

\begin{column}{\sepwid}\end{column} % Empty spacer column

\begin{column}{\twocolwid} % Begin a column which is two columns wide (column 2)

\begin{block}{Block Sparse $\mathbf{\sqrt{FGLM}}$}

\textbf{Key idea:}  use sequences of small matrices, requires less terms than scalar sequences
(cf. Coppersmith's block Wiedemann Algorithm)

\vspace{1em}

	\setbeamercolor{block alerted title}{fg=white,bg=norange}
	\setbeamercolor{block alerted body}{fg=black,bg=white} 
	
	\begin{alertblock}{Algorithm}
		\begin{itemize}
			\item[] Choose $U,V \in \mathbb{K}^{m \times D}$, where $m$ is the number of threads supported
			\item[] $s= (UT_n^iV^t)_{0 \le i < 2d}$, with $d = \frac{D}{m}$
			\item[] $S = {\sf MatrixBerlekampMassey}(s)$, $N = S\sum_{i\ge 0} \frac{s_i}{x^{i+1}}$
			\item[] $A,S,B={\sf SmithForm}(S)$, with invariant factors $I_1,\dots I_d$
                        \item[] $a = \begin{bmatrix}
			\frac{I_db_1}{I_1} & \frac{I_db_2}{I_2}& \cdots&\frac{I_db_{d-1}}{I_{d-1}}&b_d
			\end{bmatrix} A$, where $b_i$ is the $i^{th}$ entry of the last row of $B$
			\item[]  $N^*=$ $(m,1)$-th entry of $aN$
                        \item[]  $R_n={\sf SquareFreePart}(I_d)$
			\item[] For $j = 1 \dots n-1$:
			\begin{itemize}
			\item[] ~~$s_j = (UT_n^i T_j V^t)_{0 \le i < d}$ and $N_j = S\sum_{i\ge 0} \frac{s_{j,i}}{x^{i+1}}$
			\item[] ~~$R_j=a N_j/N^*$ mod $R_n$
			\end{itemize}
		\end{itemize}
	\end{alertblock}

  \textbf{Correctness:} follows from the analysis of Coppersmith's algorithm by [Villard '97]
and~\cite{KaVi04}, and generating series properties in~\cite{BoSaSc03}.
See also [Kaltofen '95],[Kaltofen,
   Yuhasz '06].


\end{block}
\begin{block}{Example}
		\textbf{Input: }$I = \langle f_1^2, f_2^2,f_3 \rangle \subset \mathbb{F}_{9001}[x_1,x_2,x_3]$
                of degree $\boldsymbol{D=32}$,
                with
		$$ f_1 = 4979x_1^2 + 6202x_1x_2 + \dots, \;
		   f_2 = 4682x_1^2 + 8290x_1x_2 + \dots, \;
		   f_3 = 4199x_1^2 + 2325x_1x_2 + \dots$$
		\textcolor{red}{\bf Step 1} with $\boldsymbol{m = 2}$
		$$ U = \begin{bmatrix}
		1898 &6830 &3494 & 169 &7991 &3352 \dots \\
		3161 &8858 &8467 &5882 &8037 &3726 \dots
		\end{bmatrix} \quad
		V = \begin{bmatrix}
		7595 &8416 &2285 &8351 & 550 &7012 \dots \\
		823 &5686 &6539 &7884 &7105 &3427 \dots 
		\end{bmatrix}^t
		$$
		\textcolor{red}{\bf Step 2 \& 3} with $\boldsymbol{d = 16}$
		$$ s = \left(
		\begin{bmatrix}
		31  & 6977\\
		1178& 1695
		\end{bmatrix},
		\begin{bmatrix}
		8212&1663\\
		4811&4837
		\end{bmatrix}
		\dots
		\right)\,
                ~~\xRightarrow{\mathsf{MatrixBerlekampMassey}}~~
                \begin{array}{l}
		S = \begin{bmatrix}
		\boldsymbol{x^{16}} + \dots & 423\boldsymbol{x^{15}} +\dots\\
		6426\boldsymbol{x^{15}}+ \dots & \boldsymbol{x^{16}} + \dots
		\end{bmatrix} \, \\[1em]
		N = \begin{bmatrix}
		6191\boldsymbol{x^{15}} + \dots & 8101\boldsymbol{x^{15}} + \dots\\
		7116\boldsymbol{x^{15}} + \dots & 2129\boldsymbol{x^{15}} + \dots
		\end{bmatrix}
                \end{array}
		$$
		\textcolor{red}{\bf Step 4:}
		$%% I_1 = \boldsymbol{x^{8}} + 6990x^{7}+ \dots$, $I_2 = \boldsymbol{x^{24}} + 2968x^{23}+ \dots
                %% \implies
                a = \begin{bmatrix}
		2575\boldsymbol{x^7}+\dots ~~&~~ \boldsymbol{x^8} + \dots
		\end{bmatrix}$

		\textcolor{red}{\bf Step 5:}
		$  [N^*] = \begin{bmatrix}
		2575\boldsymbol{x^7}+\dots ~~&~~ \boldsymbol{x^8} + \dots
		\end{bmatrix}
                \begin{bmatrix}
		6191\boldsymbol{x^{15}} + \dots\\
		7116\boldsymbol{x^{15}} + \dots
		\end{bmatrix}\,
=                \begin{bmatrix}
  1178\boldsymbol{x^{23}} + 8727x^{22} + \dots
\end{bmatrix}
		%% R_3 = x^8 + 6990x^7 + \dots
		$

		\textcolor{red}{\bf Step 6} for $\boldsymbol{j = 1}$
	$$ s_1 = \left(
		\begin{bmatrix}
		xx  &xx\\
		xx& xx
		\end{bmatrix},
		\begin{bmatrix}
		xx&xx\\
		xx&xx
		\end{bmatrix}
		\dots
		\right)\,
                ~~\xRightarrow{~~~~~~~~~~}~~
		N_1 = \begin{bmatrix}
		xx\boldsymbol{x^{15}} + \dots & xx\boldsymbol{x^{15}} + \dots\\
		xx\boldsymbol{x^{15}} + \dots & xx\boldsymbol{x^{15}} + \dots
		\end{bmatrix}
		$$
	
		\textcolor{red}{\bf Step 7:}
		$  [N_1^*] = \begin{bmatrix}
		2575\boldsymbol{x^7}+\dots ~~&~~ \boldsymbol{x^8} + \dots
		\end{bmatrix}
                \begin{bmatrix}
		cc\boldsymbol{x^{15}} + \dots\\
		cc\boldsymbol{x^{15}} + \dots
		\end{bmatrix}\,
=                \begin{bmatrix}
  cc\boldsymbol{x^{23}} + ccx^{22} + \dots
\end{bmatrix}
		%% R_3 = x^8 + 6990x^7 + \dots
		$

 %% $N^*_1 = 1178x^{23} + 8727x^{22} + \dots$ and
 %%        	$R_1 = 7964x^7 + 4071x^6 + \dots$\\
 %%        	%% \hspace{1cm}\hspace{1cm}\hspace{1cm}\hspace{1cm}\hspace{0.3cm}
 %%        	%% For $j = 2$, $N^*_2 = 6587x^{23}+3987x^{22} + \dots$ and 
 %%        	%% $R_2 = 1443x^7 + 7818x^6 + \dots$
	

\end{block}


\end{column} % End of the second column

\begin{column}{\sepwid}\end{column} % Empty spacer column

\begin{column}{\onecolwid} % The third column

\setbeamercolor{block alerted title}{fg=white,bg=norange}
\setbeamercolor{block alerted body}{fg=black,bg=white} 
\begin{alertblock}{Parallel Computations}
	\begin{itemize}
		\item Bottleneck is computing the sequence $(UT_n^i)_{0 \le i < 2d}$
		\item Can parallelize by computing the sequences $(U_1T_n^i),\dots,(U_mT_n^i)$
		separately, where $U_i$ is the $i^{th}$ row of $U$
		\item When $m=1$, same computation as Sparse FGLM
	\end{itemize}
\end{alertblock}

%----------------------------------------------------------------------------------------
%	CONCLUSION
%----------------------------------------------------------------------------------------

\begin{block}{Conclusion}

\end{block}

%----------------------------------------------------------------------------------------
%	ADDITIONAL INFORMATION
%----------------------------------------------------------------------------------------

%\begin{block}{Additional Information}

%Maecenas ultricies feugiat velit non mattis. Fusce tempus arcu id ligula varius dictum. 
%\begin{itemize}
%\item Curabitur pellentesque dignissim
%\item Eu facilisis est tempus quis
%\item Duis porta consequat lorem
%\end{itemize}

%\end{block}

%----------------------------------------------------------------------------------------
%	REFERENCES
%----------------------------------------------------------------------------------------

\begin{block}{References}

\nocite{*} % Insert publications even if they are not cited in the poster
\small{\bibliographystyle{unsrt}
\bibliography{sample}\vspace{0.75in}}

\end{block}

%----------------------------------------------------------------------------------------
%	ACKNOWLEDGEMENTS
%----------------------------------------------------------------------------------------

%\setbeamercolor{block title}{fg=red,bg=white} % Change the block title color

%\begin{block}{Acknowledgements}

%\small{\rmfamily{Nam mollis tristique neque eu luctus. Suspendisse rutrum congue nisi sed convallis. Aenean id neque dolor. Pellentesque habitant morbi tristique senectus et netus et malesuada fames ac turpis egestas.}} \\

%\end{block}

%----------------------------------------------------------------------------------------
%	CONTACT INFORMATION
%----------------------------------------------------------------------------------------

%----------------------------------------------------------------------------------------

\end{column} % End of the third column

\end{columns} % End of all the columns in the poster

\end{frame} % End of the enclosing frame

\end{document}
