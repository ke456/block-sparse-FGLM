%%%%%%%%%%%%%%%%%%%%%%%%%%%%%%%%%%%%%%%%%
% Jacobs Landscape Poster
% LaTeX Template
% Version 1.1 (14/06/14)
%
% Created by:
% Computational Physics and Biophysics Group, Jacobs University
% https://teamwork.jacobs-university.de:8443/confluence/display/CoPandBiG/LaTeX+Poster
% 
% Further modified by:
% Nathaniel Johnston (nathaniel@njohnston.ca)
%
% This template has been downloaded from:
% http://www.LaTeXTemplates.com
%
% License:
% CC BY-NC-SA 3.0 (http://creativecommons.org/licenses/by-nc-sa/3.0/)
%
%%%%%%%%%%%%%%%%%%%%%%%%%%%%%%%%%%%%%%%%%

%----------------------------------------------------------------------------------------
%	PACKAGES AND OTHER DOCUMENT CONFIGURATIONS
%----------------------------------------------------------------------------------------

\documentclass[final]{beamer}

\usepackage[scale=1.24]{beamerposter} % Use the beamerposter package for laying out the poster

\usetheme{confposter} % Use the confposter theme supplied with this template

\setbeamercolor{block title}{fg=ngreen,bg=white} % Colors of the block titles
\setbeamercolor{block body}{fg=black,bg=white} % Colors of the body of blocks
\setbeamercolor{block alerted title}{fg=white,bg=dblue!70} % Colors of the highlighted block titles
\setbeamercolor{block alerted body}{fg=black,bg=dblue!10} % Colors of the body of highlighted blocks
% Many more colors are available for use in beamerthemeconfposter.sty

%-----------------------------------------------------------
% Define the column widths and overall poster size
% To set effective sepwid, onecolwid and twocolwid values, first choose how many columns you want and how much separation you want between columns
% In this template, the separation width chosen is 0.024 of the paper width and a 4-column layout
% onecolwid should therefore be (1-(# of columns+1)*sepwid)/# of columns e.g. (1-(4+1)*0.024)/4 = 0.22
% Set twocolwid to be (2*onecolwid)+sepwid = 0.464
% Set threecolwid to be (3*onecolwid)+2*sepwid = 0.708

\newlength{\sepwid}
\newlength{\onecolwid}
\newlength{\twocolwid}
\newlength{\threecolwid}
\setlength{\paperwidth}{48in} % A0 width: 46.8in
\setlength{\paperheight}{36in} % A0 height: 33.1in
\setlength{\sepwid}{0.024\paperwidth} % Separation width (white space) between columns
\setlength{\onecolwid}{0.22\paperwidth} % Width of one column
\setlength{\twocolwid}{0.464\paperwidth} % Width of two columns
\setlength{\threecolwid}{0.708\paperwidth} % Width of three columns
\setlength{\topmargin}{-0.5in} % Reduce the top margin size
%-----------------------------------------------------------

\usepackage{graphicx}  % Required for including images

\usepackage{booktabs} % Top and bottom rules for tables
\definecolor{classicrose}{rgb}{0.98, 0.8, 0.91}
\definecolor{blanchedalmond}{rgb}{1.0, 0.92, 0.8}

%----------------------------------------------------------------------------------------
%	TITLE SECTION 
%----------------------------------------------------------------------------------------

\title{Sparse FGLM using the block Wiedemann algorithm} % Poster title

\author{Seung Gyu Hyun, Vincent Neiger, Hamid Rahkooy, \'Eric Schost} % Author(s)

\institute{University of Waterloo, DTU Compute} % Institution(s)

%----------------------------------------------------------------------------------------

\begin{document}

\addtobeamertemplate{block end}{}{\vspace*{2ex}} % White space under blocks
\addtobeamertemplate{block alerted end}{}{\vspace*{2ex}} % White space under highlighted (alert) blocks

\setlength{\belowcaptionskip}{2ex} % White space under figures
\setlength\belowdisplayshortskip{2ex} % White space under equations

\begin{frame}[t] % The whole poster is enclosed in one beamer frame

\begin{columns}[t] % The whole poster consists of three major columns, the second of which is split into two columns twice - the [t] option aligns each column's content to the top

\begin{column}{\sepwid}\end{column} % Empty spacer column

\begin{column}{\onecolwid} % The first column

%----------------------------------------------------------------------------------------
%	OBJECTIVES
%----------------------------------------------------------------------------------------

%\begin{alertblock}{Objectives}

%Lorem ipsum dolor sit amet, consectetur, nunc tellus pulvinar tortor, commodo eleifend risus arcu sed odio:
%\begin{itemize}
%\item Mollis dignissim, magna augue tincidunt dolor, interdum vestibulum urna
%\item Sed aliquet luctus lectus, eget aliquet leo ullamcorper consequat. Vivamus eros sem, iaculis %ut euismod non, sollicitudin vel orci.
%\item Nascetur ridiculus mus.  
%\item Euismod non erat. Nam ultricies pellentesque nunc, ultrices volutpat nisl ultrices a.
%\end{itemize}

%\end{alertblock}

%----------------------------------------------------------------------------------------
%	INTRODUCTION
%----------------------------------------------------------------------------------------

\begin{block}{Introduction}

\begin{itemize}
	\item Gr\"obner basis of an ideal is essential in solving systems of polynomials
	\item Orderings such as degree reverse lexicographical ordering \textit{(degrevlex)} make computing the Gr\"obner basis faster
	\item Orderings such as lexicographical order \textit{(lex)} make finding
	solution coordinates easier
	\item Compute first for degrevlex ordering and convert to lex ordering
	\item Sparse FGLM algorithm of \cite{FaMo17} computes lex basis of a radical
	ideal $I$ when $I$ is in shape position (although they also provide an algorithm
	for non-radical ideals)
	\item $I \subset \mathbb{K}[x_1,\dots,x_n]$ is in \textit{shape position} if
	its Gr\"obner basis has the form $(x_1-R_1(x_n), \dots, x_{n-1}-R_{n-1}(x_n), R(x_n))$
	\item Difficult to parallelize
\end{itemize}


\end{block}

\begin{alertblock}{Main Problem}
	\textbf{Input: }
	\begin{itemize}
	\item Zero-dimensional ideal $I \subset
	\mathbb{K}[x_1,\dots,x_n]$ by means of a monomial basis 
	$\mathbb{B} \subset Q$,\\ 
	$Q = \mathbb{K}[x_1,\dots,x_n]/I$
	\item Multiplication matrices  $T_1,\dots, T_n \in 
	\mathbb{K}^{D \times D}$ of $x_1,\dots,x_n$, with
	$D = dim_{\mathbb{K}}(Q)$
	\end{itemize}
	\textbf{Output:}
	\begin{itemize}
		\item Lex Gr\"obner basis of $\sqrt{I}$
	\end{itemize}
\end{alertblock}

%------------------------------------------------

%----------------------------------------------------------------------------------------

\end{column} % End of the first column

\begin{column}{\sepwid}\end{column} % Empty spacer column

\begin{column}{\twocolwid} % Begin a column which is two columns wide (column 2)

\begin{block}{Block Sparse FGLM}
	\begin{itemize}
		\item Inspired by Coppersmith's block Wiedemann Algorithm \cite{Coppersmith93}
		\item \textbf{Key idea:} Compute sequences of small matrices that require less terms
		than linear sequences
	\end{itemize}
	\setbeamercolor{block alerted title}{fg=white,bg=norange} % Change the alert block title colors
	\setbeamercolor{block alerted body}{fg=black,bg=white} % Change the alert block body colors
	
	\begin{alertblock}{Algorithm}
		\begin{itemize}
			\item[1.] Choose $U,V \in \mathbb{K}^{D \times m}$, where $m$ is the number of threads supported
			\item[2.] Compute $s = (UT_n^iV)_{0 \le i < \frac{2D}{M}}$, $S = MatrixBerlekampMassey(s)$,
			and $N = S\sum s^{(i)}/x^{i+1}$
			\item[3.] Find the Smith form of $S$, $D = ASB$ with invariant factors 
			$I_1,\dots I_D$. Set
			$\tilde{u} = \begin{bmatrix}
			\frac{I_Db_1}{I_1} & \frac{I_Db_2}{I_2}&,&\cdots&,&\frac{I_Db_{D-1}}{I_{D-1}},b_D
			\end{bmatrix} A$, where $b_i$ is the $i^{th}$ entry of the last row of $B$
			\item[4.] Set $n^*$ to be the $(m,1)$-th entry of $\tilde{u}N$ and $R_n$ as $I_D$
			written in $x_n$
			\item[5.] For $j = 1 \dots n-1$:
			\begin{itemize}
				\item[5a.] Compute $s_j = (UT_n^i T_j V)_{0 \le i <
				\frac{D}{M}}$ and 
				$N_j = S\sum s_j^{(i)}/x^{i+1}$
				\item[5b.] Set $n_j = \tilde{u}N_j$ and $R_j$ as $n_j/n^*$ mod $R_n$
				written in $x_n$
			\end{itemize}
			\item[6.] Return $(x_1-R_1, \dots, x_{n-1} - R_{n-1}, R_n)$
		\end{itemize}
	\end{alertblock}
	\begin{itemize}
	\item Analysis for Coppersmith's algorithm has been provided by
	[Kaltofen '95], [Villard '97], [Kaltofen, Villard '04], [Kaltofen, Yuhasz '06]
	\end{itemize}
	\begin{example}
		We will demonstrate our new algorithm by running it on
		$I = \langle f_1, f_2, f_3 \rangle \subset \mathbb{K}[x_1,x_2,x_3]$,
		\begin{align*}
		f_1 &= 3426x_1^2 - 4443x_1x_2 + 2004x_2^2 + 2335x_1x_3 - 74x_2x_3 + 4215x_3^2 - 1405x_1 + 4108x_2 - 1838x_3 - 2741\\
		f_2 &= -4303x_1^2 - 1401x_1x_2 - 2604x_2^2 + 2745x_1x_3 + 3440x_2x_3 + 3331x_3^2 + 2112x_1 - 271x_2 - 2272x_3 - 3090\\
		f_3 &= -4160x_1^2 + 1056x_1x_2 - 252x_2^2 - 2842x_1x_3 - 3643x_2x_3 + 3024x_3^2 + 3353x_1 + 3908x_2 - 426x_3 + 4197
		\end{align*}
		We choose $m = 2$ and two random matrices (with coefficients in [0,\dots 999]
		for this specific example)
		\begin{align*}
		U = 
		\begin{bmatrix}
		568 &651 &852 &933 &279 &835 &446 &135\\
		485 &707 &441 &238 &678 &552  &95 &900
		\end{bmatrix} \quad
		V =
		\begin{bmatrix}
		338 &147 & 24 &526 &549 &806 &741 &966\\
		243 &637 &563 &545 &580 &432 &544 &165
		\end{bmatrix}^t
		\end{align*}
		Computing the minimal generating polynomial of $s = (UT_3^iV)_{0 \le i < 8}$,
		$$S = \begin{bmatrix}
        x^4 + 5848x^3 + 1193x^2 + 5800x + 4050 &5414x^3 + 6409x^2 + 223x + 783\\
        4469x^3 + 7812x^2 + 80x + 554 &x^4 + 3102x^3 + 4076x^2 + 1871x + 3985
		\end{bmatrix}
		$$
		After steps 4 and 5,
		\begin{align*}
		n &= 320x_3^7 + 3312x_3^6 + 38x_3^5 + 763x_3^4 + 5895x_3^3 + 6024x_3^2 + 8927x_3 + 1804
		\\
		R_3 &= x_3^8 + 8950x_3^7 + 8272x_3^6 + 2637x_3^5 + 3062x_3^4 + 7018x_3^3 + 6992x_3^2 + 8980x_3 + 7724
		\end{align*}
		For $j = 1$,
		\begin{align*}
		n_1 &= 7981x_3^7 + 7201x_3^6 + 1005x_3^5 + 1044x_3^4 + 3205x_3^3 + 6671x_3^2 + 7423x_3 + 832\\
		R_1 &= 4200x_3^7 + 5082x_3^6 + 6769x_3^5 + 5671x_3^4 + 4288x_3^3 + 8885x_3^2 + 7423x_3 + 4930
		\end{align*}
		For $j = 2$,
		\begin{align*}
		n_2 &= 4648x_3^7 + 377x_3^6 + 5551x_3^5 + 1738x_3^4 + 2653x_3^3 + 7064x_3^2 + 3229x_3 + 8719\\
		R_2 &= 1168x_3^7 + 5878x_3^6 + 2896x_3^5 + 4452x_3^4 + 3821x_3^3 + 7586x_3^2 + 7952x_3 + 28
		\end{align*}
	\end{example}
\end{block}

\end{column} % End of the second column

\begin{column}{\sepwid}\end{column} % Empty spacer column

\begin{column}{\onecolwid} % The third column

%----------------------------------------------------------------------------------------
%	CONCLUSION
%----------------------------------------------------------------------------------------

\begin{block}{Conclusion}

\end{block}

%----------------------------------------------------------------------------------------
%	ADDITIONAL INFORMATION
%----------------------------------------------------------------------------------------

%\begin{block}{Additional Information}

%Maecenas ultricies feugiat velit non mattis. Fusce tempus arcu id ligula varius dictum. 
%\begin{itemize}
%\item Curabitur pellentesque dignissim
%\item Eu facilisis est tempus quis
%\item Duis porta consequat lorem
%\end{itemize}

%\end{block}

%----------------------------------------------------------------------------------------
%	REFERENCES
%----------------------------------------------------------------------------------------

\begin{block}{References}

\nocite{*} % Insert publications even if they are not cited in the poster
\small{\bibliographystyle{unsrt}
\bibliography{sample}\vspace{0.75in}}

\end{block}

%----------------------------------------------------------------------------------------
%	ACKNOWLEDGEMENTS
%----------------------------------------------------------------------------------------

%\setbeamercolor{block title}{fg=red,bg=white} % Change the block title color

%\begin{block}{Acknowledgements}

%\small{\rmfamily{Nam mollis tristique neque eu luctus. Suspendisse rutrum congue nisi sed convallis. Aenean id neque dolor. Pellentesque habitant morbi tristique senectus et netus et malesuada fames ac turpis egestas.}} \\

%\end{block}

%----------------------------------------------------------------------------------------
%	CONTACT INFORMATION
%----------------------------------------------------------------------------------------

%----------------------------------------------------------------------------------------

\end{column} % End of the third column

\end{columns} % End of all the columns in the poster

\end{frame} % End of the enclosing frame

\end{document}
