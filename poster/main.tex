%%%%%%%%%%%%%%%%%%%%%%%%%%%%%%%%%%%%%%%%%
% Jacobs Landscape Poster
% LaTeX Template
% Version 1.1 (14/06/14)
%
% Created by:
% Computational Physics and Biophysics Group, Jacobs University
% https://teamwork.jacobs-university.de:8443/confluence/display/CoPandBiG/LaTeX+Poster
% 
% Further modified by:
% Nathaniel Johnston (nathaniel@njohnston.ca)
%
% This template has been downloaded from:
% http://www.LaTeXTemplates.com
%
% License:
% CC BY-NC-SA 3.0 (http://creativecommons.org/licenses/by-nc-sa/3.0/)
%
%%%%%%%%%%%%%%%%%%%%%%%%%%%%%%%%%%%%%%%%%

%----------------------------------------------------------------------------------------
%	PACKAGES AND OTHER DOCUMENT CONFIGURATIONS
%----------------------------------------------------------------------------------------

\documentclass[final]{beamer}

\usepackage{mathtools}
\usepackage[scale=1.24]{beamerposter} % Use the beamerposter package for laying out the poster

\usetheme{confposter} % Use the confposter theme supplied with this template

\setbeamercolor{block title}{fg=ngreen,bg=white} % Colors of the block titles
\setbeamercolor{block body}{fg=black,bg=white} % Colors of the body of blocks
\setbeamercolor{block alerted title}{fg=white,bg=dblue!70} % Colors of the highlighted block titles
\setbeamercolor{block alerted body}{fg=black,bg=dblue!10} % Colors of the body of highlighted blocks
% Many more colors are available for use in beamerthemeconfposter.sty

\def\myvdots{\vbox{\baselineskip=10pt \lineskiplimit=0pt 
\kern6pt \hbox{.}\hbox{.}\hbox{.}}} 

%% \node [shift={(-10 cm,-5cm)}] at (current page.north east) {\includegraphics[height=5cm]{DTU-3.pdf}}; 

\addtobeamertemplate{headline}{} 
{
\begin{tikzpicture}[remember picture,overlay] 
\node [shift={(-112 cm,-7cm)}] at (current page.north east) {\includegraphics[height=5cm]{UW.pdf}}; 
\node [shift={(-10 cm,-7cm)}] at (current page.north east) {\includegraphics[height=7cm]{DTU-3.pdf}}; 
\end{tikzpicture} 
}

%-----------------------------------------------------------
% Define the column widths and overall poster size
% To set effective sepwid, onecolwid and twocolwid values, first choose how many columns you want and how much separation you want between columns
% In this template, the separation width chosen is 0.024 of the paper width and a 4-column layout
% onecolwid should therefore be (1-(# of columns+1)*sepwid)/# of columns e.g. (1-(4+1)*0.024)/4 = 0.22
% Set twocolwid to be (2*onecolwid)+sepwid = 0.464
% Set threecolwid to be (3*onecolwid)+2*sepwid = 0.708

\newlength{\sepwid}
\newlength{\onecolwid}
\newlength{\twocolwid}
\newlength{\threecolwid}
\setlength{\paperwidth}{48in} % A0 width: 46.8in
\setlength{\paperheight}{36in} % A0 height: 33.1in
\setlength{\sepwid}{0.024\paperwidth} % Separation width (white space) between columns
\setlength{\onecolwid}{0.22\paperwidth} % Width of one column
\setlength{\twocolwid}{0.464\paperwidth} % Width of two columns
\setlength{\threecolwid}{0.708\paperwidth} % Width of three columns
\setlength{\topmargin}{-0.5in} % Reduce the top margin size
%-----------------------------------------------------------

\usepackage{graphicx}  % Required for including images
\usepackage{multicol}
\usepackage{booktabs} % Top and bottom rules for tables

% Colors
\definecolor{classicrose}{rgb}{0.98, 0.8, 0.91}
\definecolor{antiquefuchsia}{rgb}{0.57, 0.36, 0.51}
\definecolor{blanchedalmond}{rgb}{1.0, 0.92, 0.8}
\definecolor{col3}{rgb}{0.000,1.000,1.000}
\definecolor{col4}{rgb}{1.000,0.000,0.000}
\definecolor{col5}{rgb}{1.000,0.000,1.000}
\definecolor{Yellow}{rgb}{1.000,1.000,0.000}
\definecolor{col7}{rgb}{1.000,1.000,1.000}
\definecolor{DarkBlue}{rgb}{0.000,0.000,0.560}
\definecolor{Blue}{rgb}{0.000,0.000,0.690}
\definecolor{Lightblue}{rgb}{0.000,0.000,0.820}
\definecolor{VeryLightBlue}{rgb}{0.530,0.810,1.000}
\definecolor{VeryLightGreen}{rgb}{0.930,1.0,0.93}
\definecolor{DarkGreen}{rgb}{0.000,0.360,0.000}
\definecolor{Green}{rgb}{0.000,0.690,0.000}
\definecolor{LightGreen}{rgb}{0.000,0.820,0.000}
\definecolor{Aquamarine}{rgb}{0.000,0.690,0.690}
\definecolor{col17}{rgb}{0.000,0.820,0.820}
\definecolor{RedViolet}{rgb}{0.560,0.000,0.000}
\definecolor{RubineRed}{rgb}{0.690,0.000,0.000}
\definecolor{WildStrawberry}{rgb}{0.820,0.000,0.000}
\definecolor{Violet}{rgb}{0.560,0.000,0.560}
\definecolor{col22}{rgb}{0.690,0.000,0.690}
\definecolor{col23}{rgb}{0.820,0.000,0.820}
\definecolor{Brown}{rgb}{0.500,0.190,0.000}
\definecolor{col25}{rgb}{0.630,0.250,0.000}
\definecolor{Bitter}{rgb}{0.750,0.380,0.000}
\definecolor{Pink}{rgb}{1.000,0.500,0.500}
\definecolor{col28}{rgb}{1.000,0.630,0.630}
\definecolor{col30}{rgb}{1.000,0.880,0.880}
\definecolor{Dandelion}{rgb}{1.000,0.840,0.000}

\definecolor{Turquoise}{rgb}{1.00,0.860,0.860}


\setbeamertemplate{itemize/enumerate body begin}{\normalsize}
\setbeamertemplate{itemize/enumerate subbody begin}{\normalsize}

%----------------------------------------------------------------------------------------
%	TITLE SECTION 
%----------------------------------------------------------------------------------------

\title{Sparse 
	$\mathbf{\sqrt{FGLM}}$ using the block Wiedemann algorithm} % Poster title

\author{Seung Gyu Hyun$^\dagger$, Vincent Neiger$^\star$, Hamid Rahkooy$^\dagger$, \'Eric Schost$^\dagger$} % Author(s)

\institute{$^\dagger$ University of Waterloo, $^\star$ DTU Compute} % Institution(s)

%----------------------------------------------------------------------------------------

\begin{document}

%\addtobeamertemplate{block end}{}{\vspace*{2ex}} % White space under blocks
\addtobeamertemplate{block alerted end}{}{\vspace*{2ex}} % White space under highlighted (alert) blocks

\setlength{\belowcaptionskip}{2ex} % White space under figures
\setlength\belowdisplayshortskip{2ex} % White space under equations

\begin{frame}[t] % The whole poster is enclosed in one beamer frame

\begin{columns}[t] % The whole poster consists of three major columns, the second of which is split into two columns twice - the [t] option aligns each column's content to the top

\begin{column}{\sepwid}\end{column} % Empty spacer column

\begin{column}{\onecolwid} % The first column

%----------------------------------------------------------------------------------------
%	OBJECTIVES
%----------------------------------------------------------------------------------------

%\begin{alertblock}{Objectives}

%Lorem ipsum dolor sit amet, consectetur, nunc tellus pulvinar tortor, commodo eleifend risus arcu sed odio:
%\begin{itemize}
%\item Mollis dignissim, magna augue tincidunt dolor, interdum vestibulum urna
%\item Sed aliquet luctus lectus, eget aliquet leo ullamcorper consequat. Vivamus eros sem, iaculis %ut euismod non, sollicitudin vel orci.
%\item Nascetur ridiculus mus.  
%\item Euismod non erat. Nam ultricies pellentesque nunc, ultrices volutpat nisl ultrices a.
%\end{itemize}

%\end{alertblock}

%----------------------------------------------------------------------------------------
%	INTRODUCTION
%----------------------------------------------------------------------------------------

% add a block for sim/dif - compute something weaker, always works
% doesn't require berlekamp-massey-sakata
% base field needs to be higher than D, generic coordinate
% Correctness also follows from 5
% square free R_n
% in example highlight D = ?
% highlight parallel ...

 %% the Gr\"obner basis for degrevlex ordering first (fast) and convert to lex ordering (better structure)

\begin{alertblock}{{\sf Main Problem}}
	\textbf{Input:} {\sf  $I \subset
	\mathbb{K}[x_1,\dots,x_n]$ zero-dimensional}
	\begin{itemize}
        \item {\sf monomial basis of
	$Q = \mathbb{K}[x_1,\dots,x_n]/I$}
	\item {\sf multiplication matrices  $T_1,\dots, T_n \in 
	\mathbb{K}^{D \times D}$ of $x_1,\dots,x_n$, with
	$D = \dim_{\mathbb{K}}(Q)$}
	\end{itemize}
	\textbf{Output:}
	\begin{itemize}
        \item {\sf lex Gr\"obner basis of $\sqrt{I}$}
	\end{itemize}
\end{alertblock}

\begin{alertblock}{{\sf Assumptions}}
	\begin{itemize}
		\item {\sf characteristic of $\mathbb{K}$ larger than $D$}
		\item {\sf $x_n$ generic coordinate}
	\end{itemize}
{\bf Consequence:} {\sf output is}
$$\begin{array}{l}
 x_1-R_1(x_n),   \\
~~~ \myvdots \\
 x_{n-1}-R_{n-1}(x_n),\\
 R(x_n)
\end{array}$$
\end{alertblock}

%\setbeamercolor{block alerted title}{fg=white,bg=antiquefuchsia}
%\setbeamercolor{block alerted body}{fg=black,bg=white}

\begin{alertblock}{{\sf Previous work}}
{\sf [Faug\`ere {\it et al.}'93]} {\bf FGLM}
\begin{itemize}
\item {\sf dense matrix computations}
\end{itemize}

{\sf [Rouillier'99] {\bf RUR}}
\begin{itemize}
\item  {\sf linearly generated sequence using the trace}
\end{itemize}

{\sf [Bostan {\it et al.}'03]}
\begin{itemize}
\item {\sf  trace $\to$ random linear form}
\end{itemize}

{\sf [Faug\`ere, Mou'17]}  {\bf Sparse FGLM}
\begin{itemize}
\item {\sf lex basis of $I$}
\item {\sf uses Berlekamp-Massey-Sakata in some cases}
\end{itemize}
\end{alertblock}

%% 	\begin{itemize}
%% 	\item Sparse FGLM algorithm of \cite{FaMo17} computes lex basis of an ideal $I$ when $I$ is in shape position
%% 	\item Exploits the sparsity of $T_i$'s
%% 	\item Difficult to parallelize
%% 	\end{itemize}
%% \end{alertblock}

\begin{alertblock}{{\sf At a glance}}
{\bf  Cons:}
\begin{itemize}
\item {\sf computes a lex basis of $\sqrt{I}$ (weaker output)}
\end{itemize}
{\bf Pros:}
\begin{itemize}
\item {\sf few assumptions}
\item {\sf simple algorithm}
\end{itemize}
\end{alertblock}

\end{column} % End of the first column	


\begin{column}{\sepwid}\end{column} % Empty spacer column

\begin{column}{\twocolwid} % Begin a column which is two columns wide (column 2)

%% \begin{block}{Block Sparse $\mathbf{\sqrt{FGLM}}$}

\textbf{Key idea:} use sequences of small matrices, requires less
terms than scalar sequences [Coppersmith'93]. \\ \textbf{Correctness}
 from the analysis of Coppersmith's algorithm [Villard'97],
[Kaltofen, Villard'04] and generating series properties
[Bostan {\it et al.}'03].  See also [Kaltofen'95], [Kaltofen, Yuhasz'13].

\vspace{1em}

	\setbeamercolor{block alerted title}{fg=white,bg=norange}
	\setbeamercolor{block alerted body}{fg=black,bg=white} 
	
	\begin{alertblock}{{\sf Algorithm}}
		\begin{itemize}
			\item[]\textcolor{red}{\bf 1.~} {\sf choose $U,V \in \mathbb{K}^{m \times D}$}
			\item[]\textcolor{red}{\bf 2.~} {\sf $s= (UT_n^iV^t)_{0 \le i < 2d}$, with $d = \frac{D}{m}$}
			\item[]\textcolor{red}{\bf 3.~} {\sf $S = {\sf MatrixBerlekampMassey}(s)$ and $N = S\sum_{i\ge 0} \frac{s_i}{x^{i+1}}$}
			\item[]\textcolor{red}{\bf 4.~} {\sf $P=$ largest invariant factor of $S$ and $R_n={\sf SquareFreePart}(P)$}
                        \item[]\textcolor{red}{\bf 5.~} {\sf $a = [0 ~\cdots 0 P] S^{-1}$}
			\item[]\textcolor{red}{\bf 6.~} {\sf $N^*=$ first entry of $aN$}
			\item[]\textcolor{red}{\bf 7.~} {\sf for $j = 1 \dots n-1$:}
			\begin{itemize}
			\item[]\textcolor{red}{\bf 7.1.} ~~{\sf $s_j = (UT_n^i T_j V^t)_{0 \le i < d}$ and $N_j = S\sum_{i\ge 0} \frac{s_{j,i}}{x^{i+1}}$}
                        \item[]\textcolor{red}{\bf 7.2.} ~~{\sf $N^*_j=$ first entry of $aN_j$}
			\item[]\textcolor{red}{\bf 7.3.} ~~{\sf $R_j=N^*_j/N^*$ mod $R_n$}
			\end{itemize}
		\end{itemize}
	\end{alertblock}


        \begin{alertblock}{{\sf Example}}
		\textbf{Input: }$I = \langle f_1^2, f_2^2,f_3 \rangle \subset \mathbb{F}_{9001}[x_1,x_2,x_3]$,
                {\sf non radical
                of degree $\boldsymbol{D=32}$,
                with}
		$$ f_1 = 4979x_1^2 + 6202x_1x_2 + \dots, \;
		   f_2 = 4682x_1^2 + 8290x_1x_2 + \dots, \;
		   f_3 = 4199x_1^2 + 2325x_1x_2 + \dots$$
		\textcolor{red}{\bf Step 1} {\sf with} $\boldsymbol{m = 2}$
		$$ U = \begin{bmatrix}
		1898 &6830 &3494 & 169 &7991 &3352 \dots \\
		3161 &8858 &8467 &5882 &8037 &3726 \dots
		\end{bmatrix} \quad
		V = \begin{bmatrix}
		7595 &8416 &2285 &8351 & 550 &7012 \dots \\
		823 &5686 &6539 &7884 &7105 &3427 \dots 
		\end{bmatrix}^t
		$$
		\textcolor{red}{\bf Step 2 \& 3} {\sf with} $\boldsymbol{d = 16}$
		$$ s = \left(
		\begin{bmatrix}
		31  & 6977\\
		1178& 1695
		\end{bmatrix},
		\begin{bmatrix}
		8212&1663\\
		4811&4837
		\end{bmatrix}
		\dots
		\right)\,
                ~~\xRightarrow{\mathsf{MatrixBerlekampMassey}}~~
                \begin{array}{l}
		S = \begin{bmatrix}
		\boldsymbol{x^{16}} + \dots & 423\boldsymbol{x^{15}} +\dots\\
		6426\boldsymbol{x^{15}}+ \dots & \boldsymbol{x^{16}} + \dots
		\end{bmatrix} \, \\[1em]
		N = \begin{bmatrix}
		6191\boldsymbol{x^{15}} + \dots & 8101\boldsymbol{x^{15}} + \dots\\
		7116\boldsymbol{x^{15}} + \dots & 2129\boldsymbol{x^{15}} + \dots
		\end{bmatrix}
                \end{array}
		$$
		\textcolor{red}{\bf Step 4 \& 5:}
		$        a = \begin{bmatrix}
		2575\boldsymbol{x^7}+\dots ~~&~ \boldsymbol{x^8} + \dots
		\end{bmatrix}$
                {\sf and} 
                $ R_3=\textcolor{DarkGreen}{\boldsymbol{x^{8}} + 6990x^{7}+ \dots}$\\

\vspace{3mm}

		\textcolor{red}{\bf Step 6:}

\vspace{-2em}

		$$  N^* = \begin{bmatrix}
		2575\boldsymbol{x^7}+\dots ~~&~~ \boldsymbol{x^8} + \dots
		\end{bmatrix}
                \begin{bmatrix}
		6191\boldsymbol{x^{15}} + \dots\\
		7116\boldsymbol{x^{15}} + \dots
		\end{bmatrix}\,
=                \begin{bmatrix}
 \textcolor{DarkBlue}{1178\boldsymbol{x^{23}} + 8727x^{22} + \dots}
\end{bmatrix}$$

\vspace{3mm}

\textcolor{red}{\bf Step 7} {\sf for} $\boldsymbol{j = 1}$

\vspace{-0.5em}

	$$ ~~s_1 = \left(
		\begin{bmatrix}
		3029& 8903\\
		1538& 5610
		\end{bmatrix},
		\begin{bmatrix}
		1914&3734\\
		5221&5431
		\end{bmatrix}
		\dots
		\right)\,
                ~~\xRightarrow{~~~~~~~~~~}~~
		N_1 = \begin{bmatrix}
		  1374\boldsymbol{x^{15}} + \dots & 3271\boldsymbol{x^{15}} + \dots\\
		4027\boldsymbol{x^{15}} + \dots & 1575\boldsymbol{x^{15}} + \dots
		\end{bmatrix}
		$$

		$$  N_1^* = \begin{bmatrix}
		2575\boldsymbol{x^7}+\dots ~~&~ \boldsymbol{x^8} + \dots
		\end{bmatrix}
                \begin{bmatrix}
		1374\boldsymbol{x^{15}} + \dots\\
		4027\boldsymbol{x^{15}} + \dots
		\end{bmatrix}\,
=                \begin{bmatrix}
\textcolor{col4}{ 1538\boldsymbol{x^{23}} + 6498x^{22} + \dots}
\end{bmatrix}
		$$

	$$R_1 =\frac{ \textcolor{col4}{  1538\boldsymbol{x^{23}} + 6498x^{22} + \dots }}{  \textcolor{DarkBlue}{1178\boldsymbol{x^{23}} + 8727x^{22} + \dots} } \bmod \textcolor{DarkGreen}{{\boldsymbol{x^{8}} + 6990x^{7}+ \dots }} = 7964\boldsymbol{x^7} + 4071x^6 + \dots$$


        \end{alertblock}


\end{column} % End of the second column

\begin{column}{\sepwid}\end{column} % Empty spacer column

\begin{column}{\onecolwid} % The third column

\setbeamercolor{block alerted title}{fg=white,bg=DarkGreen}
\setbeamercolor{block alerted body}{fg=black,bg=VeryLightGreen} 
\begin{alertblock}{Parallel Computations}
	\begin{itemize}
		\item Bottleneck is computing the sequence $(UT_n^i)_{0 \le i < 2d}$
		\item Can parallelize by computing the sequences $(U_1T_n^i),\dots,(U_mT_n^i)$
		separately, where $U_i$ is the $i^{th}$ row of $U$
		\item When $m=1$, same computation as Sparse FGLM
	\end{itemize}
\end{alertblock}

%----------------------------------------------------------------------------------------
%	ADDITIONAL INFORMATION
%----------------------------------------------------------------------------------------

%\begin{block}{Additional Information}

%Maecenas ultricies feugiat velit non mattis. Fusce tempus arcu id ligula varius dictum. 
%\begin{itemize}
%\item Curabitur pellentesque dignissim
%\item Eu facilisis est tempus quis
%\item Duis porta consequat lorem
%\end{itemize}

%\end{block}

%----------------------------------------------------------------------------------------
%	REFERENCES
%----------------------------------------------------------------------------------------


\begin{alertblock}{{\sf References}}
\nocite{*} % Insert publications even if they are not cited in the poster
\footnotesize{\bibliographystyle{unsrt}
\bibliography{sample}}
\end{alertblock}

%----------------------------------------------------------------------------------------
%	ACKNOWLEDGEMENTS
%----------------------------------------------------------------------------------------

%\setbeamercolor{block title}{fg=red,bg=white} % Change the block title color

%\begin{block}{Acknowledgements}

%\small{\rmfamily{Nam mollis tristique neque eu luctus. Suspendisse rutrum congue nisi sed convallis. Aenean id neque dolor. Pellentesque habitant morbi tristique senectus et netus et malesuada fames ac turpis egestas.}} \\

%\end{block}

%----------------------------------------------------------------------------------------
%	CONTACT INFORMATION
%----------------------------------------------------------------------------------------

%----------------------------------------------------------------------------------------

\end{column} % End of the third column

\end{columns} % End of all the columns in the poster

\end{frame} % End of the enclosing frame

\end{document}
