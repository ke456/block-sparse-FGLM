\documentclass[12pt]{article}
\usepackage{bbm,fullpage}
\usepackage{alltt}
\usepackage{bm}
\usepackage{amsmath,amssymb,amsfonts,epsfig,color,xspace,mathrsfs}
\usepackage{algorithm, pseudocode}
\usepackage[shortlabels]{enumitem}% for enumerate / itemize: define reasonable margins
\setlist{topsep=0.25\baselineskip,partopsep=0pt,itemsep=1pt,parsep=0pt}



% math and theorem names
\usepackage[colorlinks=true,linkcolor=cyan]{hyperref}
\usepackage{amsthm,thmtools}

\declaretheoremstyle[headfont=\normalfont\bfseries,bodyfont=\normalfont]{myremark}


\declaretheorem[style=plain,parent=section]{definition}
\declaretheorem[sibling=definition]{theorem}
\declaretheorem[sibling=definition]{corollary}
\declaretheorem[sibling=definition]{proposition}
\declaretheorem[sibling=definition]{lemma}
\declaretheorem[style=myremark,sibling=definition]{remark}
\declaretheorem[style=remark,sibling=definition,qed={\qedsymbol}]{example}
\usepackage[capitalize]{cleveref}

%% ------------------------------------------------
%% --------- FOR MATRIX BERLEKAMP-MASSEY ----------
%% ------------------------------------------------
%%% 
%%%notation
%misc
\newcommand{\storeArg}{} % aux, not to be used in document!!
\newcounter{notationCounter}
%spaces
\newcommand{\NN}{\mathbb{N}} % nonnegative integers
\newcommand{\var}{T} % variable for univariate polynomials
\newcommand{\field}{\mathbb{K}} % base field
\newcommand{\polRing}{\field[\var]} % polynomial ring
\newcommand{\Pox}{[\mkern-3mu[ \var ]\mkern-3.2mu]}
\newcommand{\Poxi}{[\mkern-3mu[ \var^{-1} ]\mkern-3.2mu]}
\newcommand{\psRing}{\field\Pox}
\newcommand{\matSpace}[1][\rdim]{\renewcommand\storeArg{#1}\matSpaceAux} % polynomial matrix space, 2 opt args
\newcommand{\matSpaceAux}[1][\storeArg]{\field^{\storeArg \times #1}} % not to be used in document
\newcommand{\polMatSpace}[1][\rdim]{\renewcommand\storeArg{#1}\polMatSpaceAux} % polynomial matrix space, 2 opt args
\newcommand{\polMatSpaceAux}[1][\storeArg]{\polRing^{\storeArg \times #1}} % not to be used in document
\newcommand{\psMatSpace}[1][\rdim]{\renewcommand\storeArg{#1}\psMatSpaceAux} % polynomial matrix space, 2 opt args
\newcommand{\psMatSpaceAux}[1][\storeArg]{\psRing^{\storeArg \times #1}} % not to be used in document
\newcommand{\mat}[1]{\bm{\MakeUppercase{#1}}} % for a matrix
\newcommand{\row}[1]{\bm{\MakeLowercase{#1}}} % for a matrix
\newcommand{\col}[1]{\bm{\MakeLowercase{#1}}} % for a matrix
\newcommand{\matCoeff}[1]{\MakeLowercase{#1}} % for a coefficient in a matrix
\newcommand{\rdim}{m} % row dimension
\newcommand{\cdim}{{m'}} % column dimension
\newcommand{\diag}[1]{\mathrm{Diag}(#1)}  % diagonal matrix with diagonal entries #1
\newcommand{\seqelt}[1]{\bm{F}_{#1}} % element of sequence of matrices
\newcommand{\seqeltSpace}{\matSpace[\rdim][\cdim]} % element of sequence of matrices
\newcommand{\seq}{\mat{\mathcal{F}}} % sequence of matrices
\newcommand{\seqLelt}[1]{\bm{L}_{#1}} % element of sequence of matrices
\newcommand{\seqL}{\mat{\mathcal{L}}} % sequence of matrices
\newcommand{\seqpm}{\mat{Z}} % power series matrix from a sequence
\newcommand{\rel}{\col{p}} % linear relation
\newcommand{\relbas}{\mat{P}} % linear relation
\newcommand{\relSpace}{\polMatSpace[1][\rdim]} % space for linear relations
\newcommand{\relbasSpace}{\polMatSpace[\rdim][\rdim]} % space for linear relations
\newcommand{\num}{\row{q}} % numerator for linear recurrence relation
\newcommand{\nummat}{\mat{Q}} % numerator for linear recurrence relation basis
\newcommand{\rem}{\row{r}} % remnant for linear recurrence relation
\newcommand{\remmat}{\mat{R}} % remnant for linear recurrence relation basis
\newcommand{\remSpace}{\polMatSpace[1][\cdim]} % space for linear relations
\newcommand{\degBd}{d} % bound on degree of minimal generator
\newcommand{\degBdr}{d_{r}} % bound on degree of a right minimal generator
\newcommand{\degBdl}{d_{\ell}} % bound on degree of a left minimal generator
\newcommand{\degDet}[1][\seq]{\operatorname{\Delta}(#1)}
\newcommand{\rdeg}[2][]{\mathrm{rdeg}_{{#1}}(#2)} % shifted row degree
\newcommand{\cdeg}[2][]{\mathrm{cdeg}_{{#1}}(#2)} % shifted column degree
\newcommand{\sys}{\mat{F}} % input matrix series to approximant basis
\newcommand{\appMod}[2]{\mathcal{A}(#1,#2)} % module of approximants for #2 at order #1
\newcommand{\basis}{\mathscr{B}}
\newcommand{\trace}{\mathrm{trace}}
\newcommand{\softO}[1]{O{\tilde{~}}(#1)} % module of approximants for #2 at order #1
\newcommand{\todo}[1]{\textcolor{red}{({\bf todo:} #1)}}
\newcommand{\fixme}[1]{\textcolor{blue}{#1}}
\newcommand{\genseries}{Z}
\newcommand{\minpoly}{P}
\newcommand{\mainalgoname}{{\sf BlockParametrization}}
\newcommand{\lf}{X}
\newcommand{\mf}{Y}
\newcommand{\residueI}{\mathscr{Q}}
\newcommand{\sqfree}{Q}


\newcommand{\density}{\rho}

\def\M {\ensuremath{\mathsf{M}}}
\def\Deg{D}
\def\dg{\kappa}

\def\C {\ensuremath{\mathbb{C}}}
\def\Q {\ensuremath{\mathbb{Q}}}
\def\N {\ensuremath{\mathbb{N}}}
\def\R {\ensuremath{\mathbb{R}}}
\def\Z {\ensuremath{\mathbb{Z}}}
\def\F {\ensuremath{\mathbb{F}}}
\def\H {\ensuremath{\mathbb{H}}}
\def\K{\mathbb{K}}
\def\K {\ensuremath{\mathbb{K}}}
\def\Kbar {{\ensuremath{\overline{\mathbb{K}}}}}
\def\A {\ensuremath{\mathbb{A}}}
\def\D {\ensuremath{D}}
\def\m {\ensuremath{\mathfrak{m}}}

\def\scrY{\mathscr{Y}}
\def\scrM {\ensuremath{\mathscr{M}}}
\def\calL {\ensuremath{\mathcal{L}}}
\def\scrP {\ensuremath{\mathscr{P}}}
\def\scrS {\ensuremath{\mathscr{S}}}
\def\scrU {\ensuremath{\mathscr{U}}}
\def\scrV {\ensuremath{\mathscr{V}}}
\def\ann {\ensuremath{\mathrm{ann}}}
\def\rk {\ensuremath{\mathrm{rk}}}

\DeclareBoldMathCommand{\bell}{\ell}
\DeclareBoldMathCommand{\be}{e}
\DeclareBoldMathCommand{\bu}{u}
\DeclareBoldMathCommand{\bv}{v}
\DeclareBoldMathCommand{\bX}{X}
\DeclareBoldMathCommand{\bx}{x}
\DeclareBoldMathCommand{\balpha}{\alpha}
\DeclareBoldMathCommand{\bbeta}{\beta}
\DeclareBoldMathCommand{\mA}{A}
\DeclareBoldMathCommand{\mB}{B}
\DeclareBoldMathCommand{\mD}{D}
\DeclareBoldMathCommand{\mF}{F}
\DeclareBoldMathCommand{\mG}{G}
\DeclareBoldMathCommand{\mI}{I}
\DeclareBoldMathCommand{\mM}{M}
\DeclareBoldMathCommand{\mNs}{N^*}
\DeclareBoldMathCommand{\mN}{N}
\DeclareBoldMathCommand{\mS}{S}
\DeclareBoldMathCommand{\mT}{T}
\DeclareBoldMathCommand{\mU}{U}
\DeclareBoldMathCommand{\mV}{V}
\DeclareBoldMathCommand{\mW}{W}
\DeclareBoldMathCommand{\mX}{X}
\DeclareBoldMathCommand{\mY}{Y}
\DeclareBoldMathCommand{\mZ}{Z}
\DeclareBoldMathCommand{\bell}{\ell}
\DeclareBoldMathCommand{\be}{e}
\DeclareBoldMathCommand{\bu}{u}
\DeclareBoldMathCommand{\bv}{v}
\DeclareBoldMathCommand{\bX}{X}
\DeclareBoldMathCommand{\bx}{x}
\DeclareBoldMathCommand{\balpha}{\alpha}
\DeclareBoldMathCommand{\bbeta}{\beta}



\title{Block Sparse-FGLM Algorithm}

\title{A block version of a sparse-FGLM-like algorithm}

\author{Kevin Hyun, Vincent Neiger, Hamid Rahkooy, \'Eric Schost}

\begin{document}

\maketitle

%%%%%%%%%%%%%%%%%%%%%%%%%%%%%%%%%%%%%%%%%%%%%%%%%%%%%%%%%%%%
%%%%%%%%%%%%%%%%%%%%%%%%%%%%%%%%%%%%%%%%%%%%%%%%%%%%%%%%%%%%
%%%%%%%%%%%%%%%%%%%%%%%%%%%%%%%%%%%%%%%%%%%%%%%%%%%%%%%%%%%%

\section{Introduction}

Computing the Gr\"obner basis of an ideal with respect to a giving
term ordering is an essential step in solving systems of polynomials.
Certain term orderings, such as the degree reverse lexicographical
ordering (\textit{degrevlex}), tend to make the computation of the
Gr\"obner basis faster: this has been observed empirically since the
1980's, and is now supported by theoretical results, at least for some
``nice'' families of inputs, such as complete intersections or certain
determinantal systems~\cite{Faugere02,FaSaSp13,BaFaSa15}.  On the
other hand, other orderings, such as the lexicographical ordering
(\textit{lex}), make it easier to find the coordinates of the
solutions, or to perform arithmetic operations in the corresponding
residue class ring.  For instance, for a zero-dimensional radical
ideal $I$ in generic coordinates in $\K[X_1,\dots,X_n]$, for some field $\K$,
the lexicographic Gr\"obner basis of $I$ for the lexicographic order $X_1 > \cdots > X_n$
has the form
\begin{equation}\label{eq:shapelemma}
 \{ X_1 - R_1(X_n),\dots, X_{n-1}-R_{n-1}(X_n),R_n(X_n)\},
\end{equation}
where $R_n \in \K[X_n]$ is squarefree and all $R_i$'s are in $\K[X_n]$
of degree less than $\deg(R_n)$; this is known as the {\em shape
  lemma}~\cite{GiMo89}. The points in the variety $V(I)$ are then
$$\{ ( R_1(\tau), \dots, R_{n-1}(\tau), \tau ) \mid \forall
 \tau: R_n(\tau)=0\}.$$ As a result, the standard approach to
 solve a zero-dimensional system by means of Gr\"obner basis
 algorithms first computes a Gr\"obner basis for a degree ordering,
 then converts it to a more readable output, such as a lexicographic
 basis. As pointed out in~\cite{FaMo17}, the latter step, while of
 polynomial complexity, can now be a bottleneck in practice. This
 paper will thus focus on this stage; in order to describe our 
 contributions, we first discuss previous work on the question.

 Let $I$ be a zero-dimensional ideal in $\K[X_1,\dots,X_n]$.  As
 input, we assume that we know a monomial basis $\basis$ of
 $\residueI=\K[X_1,\dots,X_n]/I$, together with the multiplication matrices
 $\mM_1,\dots,\mM_n$ of respectively $X_1,\dots,X_n$ in this basis; we
 denote by $D$ the degree of $I$, that is, the vector space dimension
 of $\residueI$. We should point out that knowing a degree Gr\"obner basis of
 $I$, computing the multiplication matrices efficiently is not a
 straightforward task: Faug\`ere {\it et al.} showed how to do it in
 time $O(nD^3)$ in~\cite{FaGiLaMo93} and more recently, algorithms have
 been given with cost bound
 $\softO{nD^\omega}$~\cite{FaGaHuRe13,FaGaHuRe14,Neiger16}, at
 least for some favorable families of inputs. Here, the notation
 $O\tilde{~}$ hides polylogarithmic factors and $\omega$ is a feasible
 exponent for matrix multiplication. While improving these results in
 an interesting question in itself, we will not address it in this
 paper.

Given such an input, the FGLM algorithm~\cite{FaGiLaMo93} computes the
lexicographic Gr\"obner basis of $I$ in $O(nD^3)$ operations in $\K$.
While the algorithm has an obvious relation to linear algebra,
lowering the runtime to $O\tilde{~}(nD^\omega)$ was only recently
achieved~\cite{FaGaHuRe13,FaGaHuRe14,Neiger16}. 

Polynomials as in~\eqref{eq:shapelemma} are a very useful data
structure, but there is no guarantee that the lexicographic Gr\"obner
basis of $I$ possesses such a structure; when it does, we will say
that $I$ is in {\em shape position} (some sufficient conditions are
in~\cite{BeMoMaTr94}). As an alternative, one may use Rouillier's
Rational Univariate Representation algorithm \cite{Rouillier99} (see
also~\cite{AlBeRoWo94,BeWo96} for related considerations). The output
is a description of the zero-set $V(I)$ by means of rational functions
\begin{equation}\label{eq:RUR}
 \{  F(T)=0, \quad X_1 = \frac{G_1(T)}{G(T)}, \dots,X_n = \frac{G_n(T)}{G(T)} \},
\end{equation}
where the multiplicities of a root $\tau$ of $F$ equals that of
$I$ at the corresponding point
$(G_1(\tau)/G(\tau),\dots,G_n(\tau)/G(\tau)) \in V(I)$; the fact that
we use rational functions makes it in particular possible to control
precisely the bit-size of their coefficients, if we work over $\K=\Q$.

The algorithms of~\cite{AlBeRoWo94, BeWo96, Rouillier99} rely on an
idea that will be at the core of our algorithms as well: duality.
Indeed, these algorithms compute sequences of values of the form
$\tau_i=(\trace(\lf^i))_{i \ge 0}$ or 
$\tau_{i,j}=(\trace(\lf^i X_j))_{i \ge 0}$, where $\trace: \residueI \to \K$ is the trace 
form and $\lf$ is typically of the form $\lf=t_1 X_1 + \cdots + t_n X_n$.
From these values, it is then possible to recover the output
in~\eqref{eq:RUR} by means of some structured linear algebra calculation.

A drawback of this approach is that we need to know the trace of all
elements of the basis $\basis$; while feasible in polynomial time,
this is by no means straightforward. In~\cite{BoSaSc03}, Bostan {\it
  et al.} introduced randomization to alleviate this issue. They show
that computing values such as $\ell(\lf^i)$ and $\ell(\lf^i X_j)$, where
$\lf$ is as above and 
$\ell$ is a random $\K$-linear form $\residueI \to \K$, allows one to deduce
a description of $V(I)$ of the form
\begin{equation}\label{eq:BoSaSc03}
 \{  \sqfree(T)=0, \quad X_1 = V_1(T), \dots,X_n = V_n(T) \},
\end{equation}
where $\sqfree$ is a monic squarefree polynomial in $\K[T]$ and $V_i$ is in
$\K[T]$ of degree less than $\deg(\sqfree)$ for all $i$. Precisely, the
output of such algorithms is the tuple $((\sqfree,V_1,\dots,V_n),\lf)$; this
will be called a {\em zero-dimensional parametrization} of $V(I)$. In
particular, this output will in general differ from the description
in~\eqref{eq:RUR}, since the latter keeps trace of the multiplicities
of the solutions (the algorithm in~\cite{BoSaSc03} actually computes
the {\em nil-indices} of the solutions).

The most costly part of such an algorithm is the computation of the
values $\ell(\lf^i)$ and $\ell(\lf^i X_j)$; the rest essentially boils
down to applying the Berlekamp-Massey algorithm and univariate
polynomial arithmetic. In~\cite{FaMo17}, Faug{\`e}re and Mou pointed
out that the multiplication matrices $\mM_1,\dots,\mM_n$ can be
expected to be sparse; they give precise estimates on their expected
sparsity, assuming the validity of a conjecture due to
Moreno-Soc\'ias~\cite{MorenoSocias91}. 
On this basis, they designed several forms of {\em sparse FGLM}
algorithms. For instance, if $I$ is in shape position, the algorithms
in \cite{FaMo17} recover this basis, also by considering values of
a linear form $\ell:\residueI \to \K$. For less favorable inputs, the algorithm
falls back on the Berlekamp-Massey-Sakata algorithm~\cite{Sakata90},
or plain FGLM.

The ideas at play in these algorithms are essentially Krylov space
methods, using projections and Berlekamp-Massey techniques; they are
also widely used in integer factorization or discrete logarithm
calculations~\cite{xxx,yyy}. It is then natural to adapt to our
context the block version of such algorithms, as pioneered by
Coppersmith~\cite{Coppersmith94} in the context of integer
factorization; this makes it possible to easily parallelize the
bottleneck of the algorithm. This was already discussed by Steel
in~\cite{Steel15}, where he showed how to compute the analogue of
polynomial $\sqfree$ in~\eqref{eq:BoSaSc03}. In that reference, one is
only interested in the solutions in the base field $\K$ ($\K$ being a
finite field in that context): the algorithm computes the roots of $\sqfree$
in $\K$ and substitutes them in the input system, before computing a
Gr\"obner basis in $n-1$ variables for each of them.

Our first contribution is to give a block version of the algorithm
in~\cite{BoSaSc03} that extends the approach in~\cite{Steel15} to
compute all polynomials in~\eqref{eq:BoSaSc03} for essentially the
same cost as the computation of $\sqfree$. More precisely, the bottleneck for
the algorithm of~\cite{Steel15} is the computation of a block-Krylov
sequence, and we show that once this sequence has been computed, 
not only $\sqfree$ but also all other polynomials in the zero-dimensional 
parametrization can be computed efficiently
.

\paragraph{Complexity model.}
We measure the cost of our algorithms by counting all operations in
$\K$ at unit cost. Most algorithms are randomized; they involve the
choice of a vector $\gamma \in \K^S$ of field elements, for some $S$
that depends on the size of our input, and success is guaranteed if
the vector $\gamma$ avoid a hypersurface of the parameter space
$\K^S$. 
Given a zero-dimensional parametrization
 computed by one of
these algorithms, one can always evaluate  \todo{finish}

%% We rely Over a field $\mathbb{K}$, recall that we can compute
%% multiplication, division with remainder, extended GCD, and square
%% free part of polynomial of degree at most $n$ in $O^{\tilde{~}}(n)$
%% field operations ($O^{\tilde{~}}$ omits polylogarithmic factors)
%% \cite{GaGe13}.

%%%%%%%%%%%%%%%%%%%%%%%%%%%%%%%%%%%%%%%%%%%%%%%%%%%%%%%%%%%%
%%%%%%%%%%%%%%%%%%%%%%%%%%%%%%%%%%%%%%%%%%%%%%%%%%%%%%%%%%%%
%%%%%%%%%%%%%%%%%%%%%%%%%%%%%%%%%%%%%%%%%%%%%%%%%%%%%%%%%%%%

\section{Linearly recurrent sequences}

This section starts with a review of known facts on linearly recurrent
sequences; we first discuss the scalar case, then show how the ideas
carry over to matrix sequences. 

The main results we will need from the first subsections are
\cref{coro:cost_approx} and \cref{randXY}, which give cost estimates
on the cost of computing a {\em minimal matrix generator} of a
linearly recurrent matrix sequence and a degree bound for such
generators, for matrices obtained from Krylov sequences. These results
are for the most part not new
(see~\cite{Villard97,Villard97a,KalVil01,Turner02}), but the cost
analysis we give uses results not available when those references were
first written.  The last subsection presents a useful result for our
main algorithm, that allows us to compute a ``numerator'' for a scalar
sequence from a similar object obtained from a matrix sequence.

%%%%%%%%%%%%%%%%%%%%%%%%%%%%%%%%%%%%%%%%%%%%%%%%%%%%%%%%%%%%

\subsection{Scalar sequences} \label{section:linseq}

Let $\K$ be a field and consider a sequence $\mathcal{L}=(\ell_s)_{s
  \ge 0} \in \K^\N$. We say that a degree $d$ polynomial $\minpoly =
p_0 + \cdots + p_d T^d \in\K[T]$ {\em cancels} the sequence
$\mathcal{L}$ if $p_0 \ell_s + \cdots + p_d \ell_{s+d}=0$ for all $s
\ge 0$. The sequence $\mathcal{L}$ is {\em linearly recurrent} if
there exists a non-zero polynomial that cancels it.  The {\em minimal
  polynomial} of a linearly recurrent sequence
$\mathcal{L}=(\ell_s)_{s \ge 0}$ is the monic polynomial of lowest
degree that cancels it; the {\em order} of $\mathcal{L}$ is the degree
of this polynomial~$\minpoly$.

One can rephrase these properties in term of the associated generating
series $S=\sum_{s \ge 0} \ell_s T^s \in \K[[T]]$.  Then, $\minpoly$
cancels $\mathcal{L}$ if and only if ${\rm rev}(\minpoly) S$ is a
polynomial (in which case it must have degree less than $d$), with
${\rm rev}(\minpoly)=T^d \minpoly(1/T)$.  The sequence $\mathcal{L}$ 
is linearly recurrent if and only if there
exist polynomials $A,B$ in $\K[T]$ such that $S=A/B$; these
polynomials are unique if we assume $\gcd(A,B)=1$ and $B(0)=1$.  Given
these $A$ and $B$, the minimal polynomial of $\mathcal{L}$ is
$\minpoly = T^{\max(\deg(A)+1,\deg(B))}B(1/T)$. 

It is often easier to work with a closed form that has the actual
minimal polynomial as its denominator, rather than its reverse; this
is done by working with generating series in the variable $1/T$.  Let
indeed $\genseries = \sum_{s\ge0} \ell_s / T^{s+1}$ and $\minpoly$
be any polynomial; then, $\minpoly$ cancels the sequence $\mathcal{L}$
if and only if $Q=\minpoly \genseries $ is a polynomial, in which case it
must have degree less than $\deg(\minpoly)$.  Using generating series in
$1/T$, the minimal polynomial of $\mathcal{L}$ is thus the
monic polynomial $\minpoly$ of minimal degree for which there exists $Q \in
\K[T]$ that satisfies $\genseries=Q/\minpoly$.

In particular, given a sequence $\mathcal{L}=(\ell_s)_{s \ge 0}$ and a
polynomial that cancels it, there is a well-defined notion of
associated numerator.  The next definition makes this idea precise:
\begin{definition}
  \label{def:omega}
  Let $\mathcal{L}=(\ell_s)_{s \ge 0}$ be a sequence and let $P$ be a
  polynomial that cancels $\mathcal{L}$. Then, the {\em numerator} of $\mathcal{L}$
  with respect to $P$ is defined as 
  $$\Omega(\mathcal{L},P) = P \genseries, \quad\text{with}\quad \genseries=\sum_{s \ge 0} \frac
  {\ell_s}{T^{s+1}}.$$ In view of the discussion above, $\Omega(\mathcal{L},P)$
  is a polynomial of degree less than $\deg(P)$.
\end{definition}
\begin{remark}~
  \begin{itemize}
  \item We may as well write this definition using expressions in
    variable $T$ instead of $1/T$: given
    $\mathcal{L}=(\ell_s)_{s\ge0}$ and $P$ as above, we recover
    $\Omega(\mathcal{L},P)$ by considering $S=\sum_{s\ge 0} \ell_s
    T^s$ and computing $A = {\rm rev}(P) S$, with ${\rm rev}(P)=T^d
    P(1/T)$ and $d=\deg(P)$; then
    $$\Omega(\mathcal{L} ,P) = T^{d - 1} A (1/T).$$ 
\item If $\deg(P)=d$, we only need to know the coefficients
  $\ell_0,\dots,\ell_{d-1}$ to compute $\Omega(\mathcal{L}, P)$. Explicitly, we 
  have
  $$\Omega(\mathcal{L},P) = ( P \sum_{s=0}^{d-1} \ell_{d - s} T^s ) {\rm ~div~} T^d.$$
 \end{itemize}
\end{remark}
We give an example of this construction using the Fibonacci sequence $\mathcal{L} =
(1,1,2,3,5,8,\dots)$, which is linearly recurrent, with minimal
polynomial $P=T^2-T-1$. Define $S= \sum_{s\ge 0} \ell_{s} T^s$ and
${\rm rev}(P) = 1-T-T^2$; then, we can write $S$ in closed form as
$$S = \frac{A}{{\rm rev}(P)} = \frac{1}{1-T-T^2},$$
so that $\Omega(\mathcal{L},P)=T^{2-1} \cdot 1=T$. Equivalently, we can define $\genseries = \sum_{s\ge0} \ell_{s}/T^{s+1}$; then, 
we recover 
\begin{align*}
\Omega(\mathcal{L},P) &= (T^2-T-1)\left (\frac 1T +\frac 1{T^2} + \frac 2{T^3} + \frac 3{T^4} + \cdots \right ) \\
&=T.
\end{align*}
The numerators thus defined are related to the Lagrange interpolation
polynomials; this will explain why they play a useful role in our main
algorithm. Explicitly, suppose that $\mathcal{L}$ is given by $\ell_s
= a_1^s + \cdots + a_d^s$ for $s \ge 0$, for some pairwise distinct
elements $(a_1,\dots,a_d)$ of $\K$. Then, its minimal polynomial is
$\minpoly=(T-a_1) \cdots (T-a_s)$, and the numerator
$\Omega(\mathcal{L}, P)$ is $\sum_{i=1}^n \prod_{i'\ne i} (T-a_{i'})$.

In terms of complexity, assuming that $\mathcal{L}$ is known to have
order at most $d$, given the first $2d$ terms of $\mathcal{L}$, we can
recover its minimal polynomial $P$ by means of the Berlekamp-Massey
algorithm, or of Euclid's algorithm applied to rational function
reconstruction; this can be done in $O(\M(d)\log(d))$ operations in
$\K$. Given $P$ and $\ell_0,\dots,\ell_{\deg(P)-1}$, we
can deduce $\Omega(\mathcal{L},P)$ for the cost $\M(\deg(P))$ of one
polynomial multiplication in degree $\deg(P)$.

%%%%%%%%%%%%%%%%%%%%%%%%%%%%%%%%%%%%%%%%%%%%%%%%%%%%%%%%%%%%

\subsection{Linearly recurrent matrix sequences}\label{section:matrix_seq}

Next, we discuss the analogue of these ideas for matrix sequences; our
main goal is to give a cost estimate for the computation of a suitable
{\em matrix generator} for a matrix sequence $\seq$, obtained by means
of recent algorithms for approximant bases. A similar discussion
(without a cost analysis) is in~\cite[Chapter~4]{Turner02}.

We first present the notion of linear recurrence for sequences of
matrices over a field $\field$, which extends the notions we
introduced for scalar sequences; the definition comes
from~\cite[Section~3]{KalVil01} and~\cite[Definition~4.2]{Turner02}.
\begin{definition}
  
  \label{dfn:recurrence_relation}
  Let $\seq = (\seqelt{s})_{s\ge 0} \subset \seqeltSpace$ be a matrix
  sequence.  Then,
  \begin{itemize}
  \item a polynomial $P = p_0 + \cdots + p_\degBd T^\degBd \in \polRing$ is
    a \emph{scalar relation for $\seq$} if $
    p_0 \seqelt{s} + \cdots + p_{\degBd} \seqelt{s+\degBd} = \mat{0}$ holds for all $s \ge 0$;
  \item a polynomial vector 
$\rel = \row{p}_0 + \cdots + \row{p}_\degBd T^\degBd \in
    \relSpace$ is  a \emph{(left, vector) relation for $\seq$} if
$  \row{p}_0 \seqelt{s} + \cdots + \row{p}_{\degBd} \seqelt{s+\degBd} = \row{0}$ holds for all
    $s \ge 0$;
  \item $\seq$ is  \emph{linearly recurrent} if it admits  a
    non-zero scalar relation.
  \end{itemize}
\end{definition}
For designing efficient algorithms, it will be useful to rely on
operations on polynomials, that is, truncated power series; hence the
following characterization of vector relations.

\begin{lemma}
  \label{lem:linearly_recurrent}
  Consider a matrix sequence $\seq = (\seqelt{s})_{s\ge 0} \subset
  \seqeltSpace$ and its generating series $\seqpm = \sum_{s\ge 0} \seqelt{s} /
  \var^{s+1} \in \field\Poxi^{\rdim \times \cdim}$.  Then, $\rel \in \relSpace$
  is a vector relation for $\seq$ if and only if the entries of $\num = \rel
  \seqpm$ are in $\polRing$; furthermore, in this case, $\deg(\num) <
  \deg(\rel)$.
\end{lemma}
\begin{proof}
  Let $\rel = \sum_{0 \le k \le \degBd} \row{p}_k \var^k$. For $s \ge 0$,
  the coefficient of $\num$ of degree $-s-1<0$ is $\sum_{0\le k \le
    \degBd} \row{p}_k \seqelt{s+k}$. Hence the equivalence, by definition of
  a relation.  The degree comparison is clear since $\seqpm$ has only
  terms of negative degree.
\end{proof}

Concerning the algebraic structure of the set of vector relations, we have the
following basic result, which can be found for example in
\cite{Villard97,KalVil01,Turner02}.

\begin{lemma}
  \label{lem:module_rank}
  The sequence $\seq$ is linearly recurrent if and only if the set of left
  vector relations for $\seq$ is a $\polRing$-submodule of $\relSpace$ of rank
  $\rdim$.
\end{lemma}
%% \begin{proof}
%% 	The set of vector relations for $\seq$ is a $\polRing$-submodule of
%% 	$\relSpace$, and hence is free of rank at most $\rdim$
%% 	\cite[Chapter~12]{DumFoo04}.
	
%% 	If $\seq$ is linearly recurrent, let $p \in \polRing$ be a nontrivial scalar
%% 	relation for $\seq$. Then each vector $[0 \; \cdots \; 0 \; p \; 0 \; \cdots
%% 	\; 0]$ with $p$ at index $1 \le i \le \rdim$ is a vector relation for $\seq$,
%% 	hence $\seq$ has rank $\rdim$.  Conversely, if $\seq$ has rank $\rdim$, then
%% 	it has a basis with $\rdim$ vectors, which form a matrix in $\relbasSpace$;
%% 	the determinant of this matrix is a nontrivial scalar relation for $\seq$.
%% \end{proof}

Note however that a matrix sequence may admit nontrivial vector relations and
have no scalar relation (and therefore not be linearly recurrent with the
present definition); in this case the module of vector relations has rank less
than $\rdim$.

\begin{definition}
  \label{dfn:matrix_generator}
  Let $\seq \subset \seqeltSpace$ be linearly recurrent.  A \emph{(left) matrix
    generator} $\mat{P}$ for $\seq$ is a matrix in $\relbasSpace$ whose rows form a basis
  of the module of left vector relations for $\seq$. This basis is said to be
  \begin{itemize}
  \item \emph{minimal} if $\mat{P}$ is row reduced \cite{Wolovich74,Kailath80};
  \item \emph{canonical} if $\mat{P}$ is in Popov form \cite{Popov72,Kailath80}.
  \end{itemize}
\end{definition}

Note that the canonical generator is also a minimal generator;
furthermore, all matrix generators $\relbas \in \relbasSpace$ for
$\seq$ have the same determinantal degree $\deg(\det(\relbas))$, which
we denote by $\degDet$.  


We now point out that matrix generators are denominators in some
irreducible fraction description of the generating series of the
sequence; this is a direct consequence of
\cref{lem:linearly_recurrent,lem:module_rank} and of basic properties
of polynomial matrices. As suggested in the previous subsection,
working with generating series in variable $1/T$ turns out to be the
most convenient choice here.  In what follows, for an $r \times s$
matrix $\mat{P}$ with entries in $\K[T]$, we denote by
$\rdeg{\mat{P}}$ and $\cdeg{\mat{P}}$ respectively its row-degree and
column-degrees, that is, the size-$r$ vector of the degrees of its
rows, respectively the size-$s$ vector of the degrees of its columns.

\begin{corollary}
  A matrix sequence $\seq = (\seqelt{s})_{s\ge 0}
  \subset \seqeltSpace$ is linearly recurrent if and only if its
  generating series $\seqpm = \sum_{s\ge 0} \seqelt{s} / \var^{s+1}
  \in \field\Poxi^{\rdim \times \cdim}$ can be written as a matrix
  fraction $\seqpm = \relbas^{-1} \nummat$ where $\relbas
  \in \relbasSpace$ is nonsingular and $\nummat \in
  \polMatSpace[\rdim][\cdim]$. In this case, we have $\rdeg{\nummat} <
  \rdeg{\relbas}$ termwise and $\deg(\det(\relbas)) \ge \degDet$; $\relbas$
  is a matrix generator of $\seq$ if and only if $\deg(\det(\relbas))
  = \degDet$.%%  or, equivalently, the fraction $\relbas^{-1} \nummat$ is
  %% irreducible (that is, $\relbas \mat{U} + \nummat \mat{V} = \mat{I}$
  %% for some polynomial matrices $\mat{U}$ and $\mat{V}$).
\end{corollary}

Given a nonsingular matrix of relations $\relbas$ for  $\seq$, we will thus
write $\mat{\Omega}(\seq, \relbas)= \relbas \seqpm  \in
\polMatSpace[\rdim][\cdim]$, generalizing \cref{def:omega}.  By the
previous corollary, this is a polynomial matrix, whose $i$th row has
degree less than the $i$th row of $\mat{P}$ for $i=1,\dots,\rdim$.  As
in the scalar case, if $\mat{P}$ has degree $d$, we only need to know
$\seqelt{0},\dots,\seqelt{d-1}$ to recover $\mat{\Omega}(\seq,
\relbas)$.  The cost of computing it then depends on the cost of
rectangular matrix multiplication in sizes $(\rdim,\rdim)$ and
$(\rdim,\cdim)$. For simplicity, we only give the two most useful
cases: if $\cdim=\rdim$, this takes $O(\rdim^\omega \M(d))$ operations
in $\K$; if $\cdim =1$, the cost becomes $O(\rdim^2 \M(d))$.

In all the previous discussion, we remark that we may also consider
vector relations operating on the right: in particular,
\cref{lem:linearly_recurrent} shows that if the sequence is linearly
recurrent then these right relations form a submodule of
$\polMatSpace[\cdim][1]$ of rank $\cdim$. Thus, a linearly recurrent
sequence also admits a right canonical generator.

Now, we focus on our algorithmic problem: given a linearly recurrent
sequence, find a minimal matrix generator.  We assume the availability
of bounds $(\degBdl,\degBdr)$ on the degrees of the left and right
canonical generators, which allow us to control the number of terms of
the sequence we will access during the algorithm.  Since taking the
Popov form of a reduced matrix does not change the degree, any 
minimal left matrix generator $\relbas$ has the same degree $\deg(\relbas)$
as the left canonical generator: thus, $\degBdl$ is also a bound on
the degree of any minimal left generator. The same remark holds for
$\degBdr$ and right minimal generators.  The bounds $\degBdl,\degBdr$
are the same as $\gamma_1,\gamma_2$ in
\cite[Definitions~4.6~and~4.7]{Turner02}; see also $\delta_l,\delta_r$
in \cite[Section~4.2]{Villard97a}; a result similar to the following
lemma is in \cite[Theorem~4.5]{Turner02}.

\begin{lemma}
  \label{lem:finitely_many_terms}
  Let $\seq = (\seqelt{s})_{s \ge 0} \subset \seqeltSpace$ be linearly
  recurrent and let $\degBdr \in \NN$ be such that the right canonical
  matrix generator of $\seq$ has degree at most $\degBdr$.  Then,
  $\rel =\row{p}_0 + \cdots +\row{p}_{\degBd}\var^\degBd \in \relSpace$ is a left
  relation for $\seq$ if and only if $\row{p}_0 \seqelt{s} + \cdots +
  \row{p}_{\degBd} \seqelt{s + \degBd} = \row{0}$ holds for $s \in
  \{0,\ldots,\degBdr-1\}$.
\end{lemma}
\begin{proof}
  Since the right canonical generator $\relbas
  \in \polMatSpace[\cdim]$ is in column Popov form, we have $\relbas =
  \mat{L}\,\diag{\var^{t_1},\ldots,\var^{t_{\cdim}}} - \mat{Q}$ where
  $\cdeg{\mat{Q}} < \cdeg{\relbas} = (t_1,\ldots,t_{\cdim})$
  componentwise and $\mat{L} \in \matSpace[\cdim]$ is unit upper
  triangular. We define the matrix $\mat{U} =
  \diag{\var^{\degBdr-t_1},\ldots,\var^{\degBdr-t_{\cdim}}}
  \mat{L}^{-1}$, which is in $\polMatSpace[\cdim]$ since $\degBdr \ge
  \deg(\relbas) = \max_j t_j$. Then, the columns of the right multiple
  $\relbas \mat{U} = \var^{\degBdr} \mat{I}_\cdim - \mat{Q} \mat{U}$
  are right relations for $\seq$, and we have $\deg(\mat{Q} \mat{U}) <
  \degBdr$. As a consequence, writing $\mat{Q} \mat{U} = \sum_{0 \le k
    < \degBdr} \mat{Q}_k \var^k$, we have $\seqelt{s+\degBdr} =
  \sum_{0 \le k < \degBdr} \seqelt{s+k} \mat{Q}_k$ for all $s \ge 0$.
	
  Assuming that $\row{p}_0 \seqelt{s} + \cdots + \row{p}_{\degBd} \seqelt{s +
    \degBd} = \row{0}$ holds for all $s \in \{0,\ldots,\degBdr-1\}$, we
  prove by induction that this holds for all $s\in\NN$. Let $s \ge
  \degBdr-1$ and assume that this identity holds for all integers up
  to $s$. Then, the identity concluding the previous paragraph implies
  that
  \begin{align*}
    \sum_{0 \le k \le \degBd} \row{p}_{k} \seqelt{s+1 + k} & =
    \sum_{0 \le k \le \degBd} \row{p}_{k} \left(\sum_{0\le j<\degBdr} \seqelt{s+1+k-\degBdr+j} \mat{Q}_j\right) \\
    & = \sum_{0\le j<\degBdr} 
    \underbrace{\left(\sum_{0 \le k \le \degBd} \row{p}_{k} \seqelt{s+1-\degBdr+j+k}\right)}_{=\, 0 \text{ since } s+1-\degBdr+j \le s} \mat{Q}_j = \row{0},
  \end{align*}
  and the proof is complete.
\end{proof}


The fast computation of matrix generators is usually handled via
algorithms for computing minimal approximant bases
\cite{Villard97,Turner02,GioLeb14}. The next result gives the main
idea behind this approach. This result is similar to
\cite[Theorem~4.6]{Turner02} (see also
\cite[Theorems~4.7,~4.8,~4.9,~4.10]{Turner02}), using however the
reversal of the input sequence rather than of the output matrix
generator.

We recall from \cite{BarBul92,BecLab94} that, given a matrix $\sys \in
\polMatSpace[\rdim][\cdim]$ and an integer $d \in \NN$, the set of
\emph{approximants for $\sys$ at order $d$} is defined as
\[
\appMod{\sys}{d} = \{ \rel \in \relSpace \mid \rel \sys = \row{0} \bmod \var^d \}.
\]
Then, the next theorem shows that relations for $\seq$ can be retrieved as
subvectors of approximants at order about $\degBdl+\degBdr$ for a matrix
involving the first $\degBdl+\degBdr$ entries of $\seq$. 

\begin{theorem}
  \label{thm:mingen_via_appbas}
  Let $\seq = (\seqelt{s})_{s \ge 0} \subset \seqeltSpace$ be a
  linearly recurrent sequence and let $(\degBdl,\degBdr) \in \NN^2$ be
  such that the left (resp.~right) canonical matrix generator of
  $\seq$ has degree $\le\degBdl$ (resp.~$\le \degBdr$).  For
  $\degBd>0$, define
  \begin{equation}
    \label{eqn:series_to_approximate}
    \sys =
    \begin{bmatrix}
      \sum_{0\le s < \degBd} \seqelt{s} \var^{\degBd-s-1} \\ - \mat{I}_{\cdim}
    \end{bmatrix} \in \polMatSpace[(\rdim+\cdim)][\cdim].
  \end{equation}
  Suppose that  $\degBd \ge \degBdr+1$ and let $\mat{B} \in \polMatSpace[(\rdim+\cdim)][(\rdim+\cdim)]$
  be a basis of $\appMod{\sys}{\degBdl+\degBdr+1}$. Then,
  \begin{itemize}
  \item if $\mat{B}$ is in Popov form then its $\rdim\times\rdim$ leading
    principal submatrix is the canonical matrix generator for $\seq$;
  \item if $\mat{B}$ is row reduced then it has exactly $\rdim$ rows of
    degree $\le\degBdl$, and the corresponding submatrix $[\relbas \;\;
      \remmat]$ of $\mat{B}$ is such that $\relbas\in\relSpace$ is a minimal
    matrix generator for~$\seq$.
  \end{itemize}
\end{theorem}
\begin{proof}
  For any relation $\rel \in \relSpace$ for $\seq$, there exists $\rem \in
  \remSpace$ such that $\deg(\rem) < \deg(\rel)$ and $[\rel \;\; \rem]
  \in \appMod{\sys}{\degBd}$. Indeed, from
  \cref{lem:linearly_recurrent}, if $\rel$ is a relation for $\seq$
  then $\num = \rel \seqpm$ has polynomial entries, where $\seqpm =
  \sum_{s\ge 0} \seqelt{s} \var^{-s-1}$. Then, the vector $\rem = -
  \rel (\sum_{s \ge \degBd} \seqelt{s} \var^{\degBd-s-1})$ has
  polynomial entries, has degree less than $\deg(\rel)$, and is such
  that $[\rel \;\; \rem] \sys = \num \var^{\degBd}$, hence $[\rel \;\;
    \rem] \in \appMod{\sys}{\degBd}$.

  Conversely, for any vectors $\rel \in \relSpace$ and $\rem
  \in \remSpace$, if $[\rel \;\; \rem] \in\appMod{\sys}{\degBd}$ and
  $\deg([\rel \;\; \rem])\le\degBd-\degBdr-1$, then $\rel$ is a
  relation for $\seq$. Indeed, if $[\rel \;\; \rem]
  \in\appMod{\sys}{\degBd}$ we have $\rel (\sum_{0\le s< \degBd}
  \seqelt{s} \var^{\degBd-s-1}) = \rem \bmod \var^\degBd$. Since
  $\degBd\ge\degBdr+1$ and $\deg([\rel \;\;
    \rem])\le\degBd-\degBdr-1$, this implies that the coefficients of
  degree $\degBd-\degBdr$ to $\degBd-1$ of $\rel(\sum_{0\le s <
    \degBd} \seqelt{s} \var^{\degBd-s-1})$ are zero. Then,
  \cref{lem:finitely_many_terms} shows that $\rel$ is a relation for
  $\seq$.
  
  The two items in the theorem are straightforward consequences.
\end{proof}

Using fast approximant basis algorithms, we obtain the main results
from this section. They are stated in slightly more generality than
needed, since in our main algorithm, we will always use $\cdim =
\rdim$.  However, we believe that the more general form stated here
may find further applications. The proof is a direct consequence of
the previous theorem, using the algorithms of
respectively~\cite{GiJeVi03},~\cite{ZhoLab12} and~\cite{JeNeScVi16}.
\begin{corollary}\label{coro:cost_approx}
  Let $\seq \subset \seqeltSpace$ be a linearly recurrent sequence and
  let $\degBd = \degBdl+\degBdr+1$, where $(\degBdl,\degBdr) \in
  \NN^2$ are such that the left (resp.~right) canonical matrix
  generator of $\seq$ has degree $\le\degBdl$ (resp.~$\le \degBdr$).
  Then, given $\seqelt{0},\dots,\seqelt{d-1}$,
  \begin{itemize}
  \item if $\cdim \in \Omega(\rdim)$, a  minimal left matrix generator
    for $\seq$ can be computed in $O(\cdim^\omega \M(\degBd)
    \log(\degBd))$ operations in $\field$;
  \item if $\cdim \in O(\rdim)$, a  minimal left matrix generator for
    $\seq$ can be computed in $O(\rdim^\omega \M(\cdim\degBd/\rdim)
    \log(\cdim\degBd))$ operations in $\field$;
  \item the left canonical matrix generator for $\seq$ can be computed
    in $O((\rdim+\cdim)^{\omega-1} \M(\cdim\degBd)
    \log(\cdim\degBd)^3)$ operations in $\field$.
  \end{itemize}
\end{corollary}
In particular, when $\cdim = \rdim$, we can find the left canonical
matrix generator of $\seq$ in $O(\rdim^\omega \M(\degBd)
\log(\degBd))$ operations in $\field$.

%%%%%%%%%%%%%%%%%%%%%%%%%%%%%%%%%%%%%%%%%%%%%%%%%%%%%%%%%%%%

\subsection{Application to the block Wiedemann algorithm}\label{ssec:appliW}

Finally, we apply the results seen above to a particular class of
matrix sequences, namely the Krylov sequences used in the block
Wiedemann algorithm. 

Let $\mM$ be in $\mathbb{K}^{D \times D}$ and $\mU,\mV \in
\mathbb{K}^{D \times m}$ be two blocking matrices; we can then define
the Krylov sequence $\seq_{\mU,\mV}=(\seqelt{s,\mU,\mV})_{s \ge 0}$ by
$$\seqelt{s,\mU,\mV} = \mU^{\perp} \mM^s \mV, \quad s \ge 0.$$ This
sequence is linearly recurrent, since the minimal polynomial of $\mM$
is a scalar relation for it. The following theorem states some useful
properties of a minimal left generator of $\seq_{\mU,\mV}$, with in
particular a bound on its degree, for generic choices of $\mU$ and
$\mV$; we also state properties of the invariant factors of such a
generator.  These results are not new, as these statements can all be
found between~\cite{Villard97a} and~\cite{KaVi04}; we chose to give a
close-to-self-contained presentation, that refers to the above
references for the key properties.

We let $s_1, \dots, s_r$ be the invariant factors of $T\mI_D
- \mM$, ordered in such a way that $s_r | s_{r-1}| \dots | s_1$, and
let $d_i = $ deg$(s_i)$ for all $i$; for $i > r$, we let $s_i$ = 1,
with $d_i = 0$.  We define $\nu = d_1 + \cdots + d_m \le D$ and
$\delta = \lceil \nu / m \rceil \le \lceil D / m \rceil$.

%% \todo{unify subscripts  $\mat{P}^{\mU,\mV}$ or   $\seq^{\mU,\mV}$}

%% First, we present a proof that one can compute the minimal polynomial
%% of $\mM$ from the output of Matrix Berlekamp-Massey if we choose the
%% blocking matrices $\mU$ and $\mV$ generically.

\begin{theorem}
  \label{randXY}
  For a generic choice of $\mU$ and $\mV$ in $\K^{D \times m}$, the
  following holds.  Let $\mat{P}_{\mU,\mV}$ be a minimal left
  generator for $\seq_{\mU,\mV}$ and denote by $\sigma_1, \dots,
  \sigma_k$ the invariant factors of $\mat{P}_{\mU,\mV}$, for some $k
  \le m$, ordered as above; write $\sigma_i=1$ for $i > k$. Then,
  \begin{itemize}
  \item $\mat{P}_{\mU,\mV}$ is a minimal left generator for the
    sequence $\seqL_{\mU} = (\mU^\perp \mM^s)_{s \ge 0}$;
  \item $\mat{P}_{\mU,\mV}$ has degree $\delta$;
  \item $s_i = \sigma_i$ for $1 \le i \le m$.
  \end{itemize}
\end{theorem}
\begin{proof}
  Without loss of generality, we can assume that $\mat{P}_{\mU,\mV}$
  is the canonical left generator. Then, define the matrix sequence
  $\seqL_{\mU} = (\mU^\perp\mM^s)_{s \ge 0}$; this sequence is
  linearly generated as well, and we let $\mat{P}_{\mU}$ be the
  canonical left generator for it.  We denote by $\langle \mU \rangle$
  the vector space generated by the rows of $\mU^\perp,\mU^\perp \mM,
  \mU^\perp{\mM^2},\dots$, and we write $D_\mU=\dim(\langle \mU
  \rangle)$.
	
  First, we prove that for any $\mU$ in $\K^{D \times m}$, for a
  generic $\mV$ in $\K^{D\times m}$,
  $\mat{P}_{\mU}=\mat{P}_{\mU,\mV}$.  Indeed,
  by~\cite[Lemma~4.2]{Villard97a} (which considers right-generators),
  there exist matrices $\mat{Q}_\mU$ in $\K^{D\times D_\mU}$ and
  $\mN_\mU \in \K^{D_\mU \times D_\mU}$, with $\mat{Q}_\mU$ of full
  rank $D_\mU$, and where $\mN_\mU$ is a matrix of the restriction of
  $\mM^\perp$ to $\langle \mU \rangle$, such that
  $\mat{P}_{\mU,\mV}=\mat{P}_{\mU}$ if and only if the dimension of
  the vector space generated by the rows of the span of
  $\mV^\perp\mat{Q}_\mU, \mV^\perp \mat{Q}_\mU \mN_\mU, \mV^\perp
  \mat{Q}_\mU \mN_\mU^2,\dots$ is equal to $D_\mU$.
%% , with $\mB_\mV=\mM^\perp_\mV$
  %% and $\mZ=\mat{P}_\mV^\perp \mU \in \K^{D_\mV \times m}$.
	
  By construction, one can find a basis of $\langle \mU \rangle$ in
  which the matrix of $\mN_\mU$ is block-companion, with $\mu \le m$
  blocks (take the $\mN_\mU$-span of the first column of $\mU$, then of
  the second column, working modulo the previous vector space, etc.)
  Thus, $\mN^\perp_\mU$ is similar to a block-companion matrix with $\mu$
  blocks as well; since $\mat{Q}_\mU^\perp \mV$ has $m$ columns, the span of
  the columns of
  $\mat{Q}_\mU^\perp \mV,  \mN^\perp_\mU\mat{Q}_\mU^\perp \mV,\dots$
 has full dimension $D_\mU$ for a
  generic $\mat{Q}_\mU^\perp \mV$, and thus for a generic $\mV$, since $\mat{Q}_\mU$ has rank
  $D_\mU$. As a result, as claimed, for generic choices of $\mU$ and $\mV$,
  $\mat{P}_{\mU,\mV}=\mat{P}_{\mU}$.
	
  Let us next introduce a matrix $\scrU$ of indeterminates of size $D
  \times m$, and let $\mat{P}_{\scrU}$ be the canonical generating
  polynomial of the ``generic'' sequence $(\scrU^\perp \mM^s)_{s \ge
    0}$. The notation $\langle \scrU \rangle$ and $D_\scrU$ are
  defined as above.  In particular,
  by~\cite[Proposition~6.1]{Villard97a}, the canonical generating
  polynomial $\mat{P}_{\scrU}$ has degree $\delta$ and determinantal
  degree $\nu$.
  
%%  Indeed,
%%   $\langle \scrV\rangle$ is the span of $K_\scrV=[\scrV ~ \mM \scrV ~
%%     \cdots ~ \mM^{N-1} \scrV]$, whereas $\langle \mV \rangle$ is the
%%   span of $[\mV ~ \mM \mV ~ \cdots ~\mM^{N-1} \mV]$.
%% Take a
%%  maximal non-zero minor $\mu$ of $K_\scrV$; as soon as $\mu(Y)\ne 0$,
%%  we have equality of the dimensions.
  Now, for a generic $\mU$ in $\K^{D\times m}$, $D_\mU=D_\scrU$. On
  the other hand, by~\cite[Lemma~4.3]{Villard97a}, for any $\mU$
  (including $\scrU$), the degree of $\mat{P}_{\mU}$ is equal to the
  first index $d$ such that the dimension of the span of the rows of
  $\mU^\perp,\mU^\perp \mM,\dots, \mU^\perp \mM^{d-1}$ is $D_\mU$. As
  a result, for generic $\mU$, $\mat{P}_{\mU}$ and $\mat{P}_{\scrU}$
  have the same degree, that is, $\delta$.  Taking the first item into
  account, we see that the second item is proved as well.
	
  We conclude by proving that for generic $\mU,\mV$, the invariant
  factors $\sigma_1,\dots,\sigma_m$ of $\mat{P}_{\mU,\mV}$ are
  $s_1,\dots,s_m$.  By~\cite[Theorem~2.12]{KaVi04} (which considers 
  right-generators as well), for any $\mU$ and
  $\mV$ in $\K^{D\times m}$, for $i=1,\dots,m$, the $i$-th invariant
  factor $\sigma_i$ of $\mat{P}_{\mU,\mV}$ divides $s_i$, so that
  $\deg(\det(\mat{P}_{\mU,\mV}))\le\nu$, with equality if and only
  if $\sigma_i=s_i$ for all $i \le m$.
	
  For $\mU$ as above and any integers $e,d$, we let ${\rm
    Hk}_{e,d}(\mU)$ be the block Hankel matrix
  $$ {\rm Hk}_{d,e}(\mU) =
  \begin{bmatrix}
    \mU^\perp \\     \mU^\perp \mM \\     \mU^\perp \mM^2 \\ \vdots  \\      \mU^\perp\mM^{d-1}
  \end{bmatrix}
  \begin{bmatrix}
    \mI_D & \mM & \mM^2 &\cdots& \mM^{e-1}
  \end{bmatrix}
  $$ By~\cite[Eq.~(2.6)]{KaVi04}, ${\rm rank}({\rm Hk}_{d,e}(\mU)) =
  \deg(\det(\mat{P}_{\mU}))$ for $d \ge \deg(\mat{P}_{\mU})$
  and $e \ge D$.  We take $e=D$, so that ${\rm rank}({\rm
    Hk}_{d,D}(\mU)) = \deg(\det(\mat{P}_{\mU}))$ for $d \ge
  \deg(\mat{P}_{\mU})$. On the other hand, the sequence ${\rm
    rank}({\rm Hk}_{d,D}(\mU))$ is constant for $d \ge D$; as a
  result, ${\rm rank}({\rm Hk}_{D,D}(\mU)) =
  \deg(\det(\mat{P}_{\mU}))$. For the same reason, we also have ${\rm
    rank}({\rm Hk}_{D,D}(\scrU)) = \deg(\det(\mat{P}_{\scrU}))$, so that
  for a generic $\mU$, $\mat{P}_{\mU}$ and $\mat{P}_{\scrU}$
  have the same determinantal degree, that is, $\nu$.  As a result,
  for generic $\mU$ and $\mV$, we also have
  $\deg(\det(\mat{P}_{\mU,\mV}))=\deg(\det(\mat{P}_{\mU}))=\nu$, and the conclusion follows.
\end{proof} 

In particular, combining \cref{coro:cost_approx} and \cref{randXY}, we
deduce that for generic $\mU,\mV$, given the first $2 \lceil D/m
\rceil$ terms of the sequence $\seq_{\mU,\mV}$, we can recover a
minimal matrix generator  $\mat{P}_{\mU,\mV}$ of it using $O(m^\omega \M(D/m) \log(D/m))
\subset \softO{m^{\omega-1} D}$ operations in $\K$.

Besides, the theorem shows that for generic $\mU$ and $\mV$, the
largest invariant factor $\sigma_1$ of $\mat{P}_{\mU,\mV}$ is the
minimal polynomial $\minpoly$ of $\mM$.  Given $\mat{P}_{\mU,\mV}$,
$\minpoly$ can thus be computed by solving a linear system
$\mat{P}_{\mU,\mV} \col{x} = \col{y}$, where $\col{y}$ is a vector of
$m$ randomly chosen elements in $\K$: for a generic choice of
$\col{y}$, the least common multiple of the denominators of the
entries of $\col{x}$ is $\minpoly$.  It can thus be computed using the
high-order lifting algorithm of \cite[Algorithm~5]{Stor03}; by
\cite[Corollary~16]{Stor03}, the cost of this is $O(m^{\omega} \M(D/m)
\log(D/m) \log(m)) \subset \softO{m^{\omega-1}D}$ operations in $\K$;
the latter algorithm is randomized, since we chose a random point in
$\K$ to compute (parts of) the expansion of the inverse of
$\mat{P}_{\mU,\mV}$.

An alternative would be to compute the Smith form of
$\mat{P}_{\mU,\mV}$, but using an algorithm such as
Storjohann's~\cite{Stor03}, the cost is slightly higher (by a
logarithmic factor).

%%%%%%%%%%%%%%%%%%%%%%%%%%%%%%%%%%%%%%%%%%%%%%%%%%%%%%%%%%%%

\subsection{Computing a scalar numerator}\label{ssec:scalar_numer}

Let us keep the notation of the previous subsection.  The main
advantage of the block Wiedemann algorithm is that it allows one to
distribute the bulk of the computation in a straightforward manner: on
a platform with $m$ processors (or cores, \dots), one would typically
compute the $D \times m$ matrices $\mat{L}_s=\mU^\perp\mM^s$, for
$s=0,\dots,2\lceil D/m \rceil$, by having each processor compute a
sequence $\row{u}_i \mM^s$, where $\row{u}_i$ is the $i$th row of
$\mU^\perp$. From these values, we then deduce the matrices
$\seqelt{s,\mU,\mV}=\mU^\perp \mM^s \mV = \mat{L}_s\mV$, for
$s=0,\dots,2\lceil D/m \rceil$. \todo{Let us assume for simplicity that 
memory is not limited}

However, our main algorithm will need to compute scalar numerators of
the form \sloppy $\Omega((\row{u}_i \mM^s \col{w})_{s \ge 0},
\minpoly)$, where $\row{w}$ is a given vector in $\K^{D \times 1}$ and
$\minpoly$ is the minimal polynomial of $\mM$. Since $\minpoly$ may
have degree $D$, and $\Omega((\row{u}_i \mM^s \col{w})_{s \ge 0},
\minpoly)$ itself may have degree $D-1$,
the definition we gave of $\Omega$ suggests that we
may need to compute up to $D$ terms of the sequence $\row{u}_i \mM^s
\col{w}$, which we would of course like to avoid. We now present an
alternative solution, that involves solving a polynomial linear system
and computing a matrix numerator, but only uses the sequence elements
$\mat{L}_s= \mU^\perp \mM^s$, for $s=0,\dots,\lceil D/m \rceil-1$,
which were previously computed.

Fix $i$ in $1,\dots,m$ and let $\row{a}_i$ be the row vector defined
by $$\row{a}_i =[0~\cdots~0~\minpoly~0~\cdots~0]  (\mat{P}_{\mU,\mV})^{-1} ,$$
where the minimal polynomial $\minpoly$ appears at the $i$th entry  of the
left-hand row vector. 
\begin{lemma}\label{utilde}
  For generic $\mU,\mV$, the row vector $\row{a}_i$ has polynomial
  entries of degree at most~$D$.
  %% Let $\textbf{a}$ be defined as line 7 of \cref{algo:block-sparse-fglm}, 
  %% then $\textbf{a}$ has polynomial entries.
\end{lemma}
\begin{proof}
  For generic $\mU,\mV$, we saw that $\minpoly$ is the largest
  invariant factor of $ (\mat{P}_{\mU,\mV})^{-1}$; thus, the product
  $\minpoly\ (\mat{P}_{\mU,\mV})^{-1}$ has polynomial entries. Since $\row{a}_i$ 
  the $i$th row of this matrix, our first claim follows. Since
  $\mat{P}_{\mU,\mV}$ has determinantal degree at most $D$~\cite{xxx}, the bound
  on the degree of the entries of $\row{a}_i$ follows from Cramer's
  formulas.
\end{proof}
To compute $\row{a}_i$, we use again Storjohann's high-order lifting;
as before, the cost is $O(m^{\omega} \M(D/m) \log(D/m) \log(m))
\subset \softO{m^{\omega-1}D}$ operations in $\K$.
Once $\row{a}_i$ is known, we can recover the scalar numerator
$\Omega((\row{u}_i \mM^s \col{w})_{s \ge 0}, \minpoly)$ as
a dot product. \todo{define matrix generator}
\begin{lemma}\label{lemma:omegaOmega}
  For a generic choice of $\mU$ and $\mV$, and for any $\col{w}$ in
 $\K^{D \times 1}$, $ \mat{P}_{\mU,\mV}$ is a matrix generator 
  for the sequence $\mat{\mathcal{E}} =
 (\col{e}_s)_{s\ge0}$, with
  $\col{e}_s=\mat{U}^\perp \mM^s\, \col{w}$
 for all $s$,
  and we have 
\begin{align}\label{eq:OmegaOmega}
[~\Omega((\row{u}_i \mM^s \col{w})_{s \ge 0}, \minpoly)~] = \row{a}_i\cdot \mat{\Omega}(\mat{\mathcal{E}}, \mat{P}_{\mU,\mV}).
\end{align}
with $\row{a}_i$ in $\K[\var]^{1 \times m}$ and 
$\mat{\Omega}(\mat{\mathcal{E}} , \mat{P}_{\mU,\mV}) \in \K[\var]^{m \times 1}.$
\end{lemma}
\begin{proof}
The first item in \cref{randXY} shows that for a generic choice of
$\mU$ and $\mV$, $\mat{P}_{\mU,\mV}$ cancels the sequence $(\mU^\perp
\mM^s)_{s \ge 0}$, and thus the sequence $\mat{\mathcal{E}}$ as well;
this proves the first point. Then, the equality
in~\eqref{eq:OmegaOmega} follows at once from the definitions:
\begin{align*}
[~ \Omega((\row{u}_i \mM^s \col{w})_{s \ge 0}, \minpoly)~]  &= [~\minpoly~]\ \sum_{s \ge 0} \frac{\row{u}_i \mM^s \col{w}}{T^{s+1}}\\
&=  [~0~\cdots~0~\minpoly~0~\cdots~0~]\  \sum_{s \ge 0} \frac{\mat{U}^\perp \mM^s \col{w}}{T^{s+1}}\\
&=  [~0~\cdots~0~\minpoly~0~\cdots~0~]\ (\mat{P}_{\mU,\mV})^{-1} \mat{P}_{\mU,\mV} \sum_{s \ge 0} \frac{\mat{U}^\perp \mM^s \col{w}}{T^{s+1}}\\
&=  \row{a}_i\cdot \mat{\Omega}(\mat{\mathcal{E}}, \mat{P}_{\mU,\mV}).
\end{align*}
\end{proof}

The algorithm to compute the scalar numerator $\Omega((\row{u}_i \mM^s
\col{w})_{s \ge 0}, \minpoly)$ follows at once; in the algorithm, we
assume that we know $\mat{P}_{\mU,\mV}$, $P$, $\row{a}_i$, and the
matrices $\mat{L}_s=\mU^\perp \mM^s \in \K^{m \times D}$, for
$s=0,\dots,\lceil D/m \rceil-1$.

\begin{algorithm}[H]
	\caption{ScalarNumerator($\mat{P}_{\mU,\mV}, \minpoly, \row{w}, i, \row{a}_i,(\mat{L}_s)_{0 \le s < \lceil D/m\rceil}$)}
	{\bf Input:} \vspace{-0.5em}
	\begin{itemize}
		\item a minimal generator $\mat{P}_{\mU,\mV}$ of $(\mU^\perp \mM^s \mV)_{s \ge 0}$
                \item the minimal polynomial $P$ of $\mM$
                \item $\row{w}$ in $\K^{D \times 1}$
                \item $i$ in $\{1,\dots,m\}$
                \item $\row{a}_i =  [0~\cdots~0~\minpoly~0~\cdots~0]  (\mat{P}_{\mU,\mV})^{-1} $
		\item $\mat{L}_s = \mU^\perp \mM^s$, for $s=0,\dots,\lceil D/m\rceil-1$
	\end{itemize}
	{\bf Output:} the scalar numerator $\Omega((\row{u}_i \mM^s \col{w})_{s \ge 0}, \minpoly) \in \K[T]$
        \begin{enumerate}[{\bf 1.}]
        \item compute $\col{E}_s = \mat{L}_s \col{w}$ for $s=0,\dots,\lceil D/m\rceil-1$
        \item use these values to compute the matrix numerator $ \mat{\Omega}(\mat{\mathcal{E}}, \mat{P}_{\mU,\mV})$ 
        \item return the entry of the $1 \times 1$ matrix $\row{a}_i\cdot  \mat{\Omega}(\mat{\mathcal{E}}, \mat{P}_{\mU,\mV})$ 
        \end{enumerate}
	\label{algo:scalar_numerator}
\end{algorithm}

Computing the first $\lceil D/m \rceil$ values of the sequence
$\mat{\mathcal{E}}=(\col{e}_s)_{s \ge 0}$ is done by using the
equality $\col{E}_s = \mat{L}_s \col{w}$ and takes $O(D^2)$ base field
operations. Then, applying the cost estimate given in
\cref{section:matrix_seq}, we see that we have enough terms to compute
$\mat{\Omega}( (\mat{\mathcal{E}}, \mat{P}_{\mU,\mV}) \in \K[T]^{m
  \times 1}$ and that it takes $O(m^2 \M(D/m))$ operations in
$\K$. Then, the dot product with $\row{a}_i$ takes $O(m \M(D))$
operations, since both vectors have size $m$ and entries of degree at
most $D$. Thus, the runtime is 
$$O(D^2 + m \M(D)) \subset \softO{D^2}$$
operations in $\K$. \todo{$m \le D$}


%%%%%%%%%%%%%%%%%%%%%%%%%%%%%%%%%%%%%%%%%%%%%%%%%%%%%%%%%%%%
%%%%%%%%%%%%%%%%%%%%%%%%%%%%%%%%%%%%%%%%%%%%%%%%%%%%%%%%%%%%
%%%%%%%%%%%%%%%%%%%%%%%%%%%%%%%%%%%%%%%%%%%%%%%%%%%%%%%%%%%%

\section{Sequences associated to a zero-dimensional ideal}

We now focus on our main question: computing a zero-dimensional
parametrization of an algebraic set of the form $V=V(I)$, for some
zero-dimensional ideal $I$ in $\K[X_1,\dots,X_n]$. 

We write $V=\{\balpha_1,\dots,\balpha_\dg\},$ with all $\balpha_i$'s in
$\Kbar^n$, and $\balpha_i=(\alpha_{i,1},\dots,\alpha_{i,n})$ for all
$i$.  We also let $\D$ be the dimension of $\residueI$, so that $\dg \le \D$,
and {\em we assume that ${\rm char}(\K)$ is greater than $D$}. In this
section, we recall and generalize results from the appendix
of~\cite{BoSaSc03}, with the objective of computing a zero-dimensional
parametrization of $V$. At this stage, we do not discuss data
structures or complexity (this is the subject of the following
section); the algorithm in this section simply describes what 
polynomial should be computed in order to obtain a zero-dimensional
parametrization.

%%%%%%%%%%%%%%%%%%%%%%%%%%%%%%%%%%%%%%%%%%%%%%%%%%%%%%%%%%%%

\subsection{The structure of the dual}\label{ssec:dual}

For $i$ in $\{1,\dots,\dg\}$, let $\residueI_i$ be the local algebra at
$\balpha_i$, that is $\residueI_i=\Kbar[X_1,\dots,X_n]/I_i$, with $I_i$ the
$\m_{\balpha_i}$-primary component of $I$. By the Chinese Remainder
Theorem, $\residueI\otimes_\K \Kbar=\Kbar[X_1,\dots,X_n]/I$ is isomorphic to
the direct product $\residueI_1\times \cdots \times \residueI_\dg$.  We let $N_i$ be the
{\em nil-index} of $\residueI_i$, that is, the maximal integer $N$ such that
$\m_{\alpha_i}^N$ is not contained in $I_i$; for instance, $N_i=0$ if
and only if $\residueI_i$ is a field, if and only if $\balpha_i$ is a
non-singular root of $I$. We also let $\D_i=\dim_\Kbar(\residueI_i)$, so that
we have $D_i \ge N_i$ and $\D=\D_1 + \cdots + \D_\dg$.

The sequences we consider below are of the form $(\ell(\lf^s))_{s \ge
  0}$, for $\ell$ a $\K$-linear form $\residueI \to \K$ and $\lf$ in $\residueI$
(we will often write $\ell \in {\rm hom}_\K(\residueI,\K)$). For
such sequences, the following standard result will be useful
(see e.g.~\cite[Proposition~1 \& 2]{BoSaSc03} for a proof).
\begin{lemma}\label{lemma:minpoly}
  Let $\lf$ be in $\residueI$ and let $P \in \K[T]$ be its minimal
  polynomial. For a generic choice of $\ell$ in ${\rm hom}_\K(\residueI,\K)$,
  $P$ is the minimal polynomial of the sequence $(\ell(\lf^s))_{s \ge
    0}$.
\end{lemma}

The following results are classical; they go back
to~\cite{Macaulay16}, and have been used in computational algebra
since the 1990's~\cite{MaMoMo96,Mourrain97}. Fix $i$ in $1,\dots,\dg$.
There exists a basis of the dual ${\rm hom}_\Kbar(\residueI_i,\Kbar)$
consisting of linear forms $(\lambda_{i,j})_{1\le j \le \D_i}$ of the
form
$$\lambda_{i,j}: f \mapsto (\Lambda_{i,j}(f))(\balpha_i),$$
where $\Lambda_{i,j}$ is the operator
$$f \mapsto \Lambda_{i,j}(f) = \sum_{\mu=(\mu_1,\dots,\mu_n) \in
	S_{i,j}} c_{i,j,\mu} \frac{ \partial^{\mu_1 + \cdots + \mu_n} f}
{\partial X_1^{\mu_1} \cdots \partial X_n^{\mu_n}},$$ for some finite
subset $S_{i,j}$ of $\N^n$ and non-zero constants $c_{i,j,\mu}$ in
$\Kbar$. 
%% Let $w_{i,j}$ be the maximum of all $|\mu|$ for $\mu$ in
%% $S_{i,j}$, with $|\mu| = \mu_1 +\cdots + \mu_n $.
%% By~\cite[Lemma~3.3]{Mourrain97}, we have that $\max_j w_{i,j} =N_i$
%% for all $i$.
For instance, when $\balpha_i$ is non-singular, we have $D_i=1$, so
there is only one function $\lambda_{i,j}$, namely $\lambda_{i,1}$, we
write it $\lambda_{i,1}(f) = f(\balpha_i)$.

More generally, we can always take $\lambda_{i,1}$ of the form
$\lambda_{i,1}(f) = f(\balpha_i)$; for $j>1$, we can then also assume
that $S_{i,j}$ does not contain $\mu=(0,\dots,0)$ (that is, all terms
in $\Lambda_{i,j}$ have order $1$ or more). Thus, introducing new
variables $(U_{i,j})_{j =1,\dots,D_i}$, we deduce the existence of
non-zero homogeneous linear forms $P_{i,\mu}$ in
$(U_{i,j})_{j=1,\dots,D_i}$ such that for any $\lambda$ in ${\rm
	hom}_\Kbar(\residueI_i,\Kbar)$, there exist $\bu_i=(u_{i,j}) \in
\Kbar{}^{D_i}$ such that we have
\begin{align}\label{ell_param}
\ell: f \mapsto \ell(f)
&= \sum_{j=1}^{D_i} u_{i,j} \lambda_{i,j}(f)\nonumber\\
&= \sum_{j=1}^{D_i} u_{i,j} \big(\Lambda_{i,j}(f)\big)(\balpha_i)\nonumber\\
&= \sum_{j=1}^{D_i} u_{i,j}
\sum_{\mu=(\mu_1,\dots,\mu_n) \in
	S_{i,j}} c_{i,j,\mu} \frac{ \partial^{\mu_1 + \cdots + \mu_n} f}
{\partial X_1^{\mu_1} \cdots \partial X_n^{\mu_n}}(\balpha_i)\nonumber\\
&= \sum_{\mu=(\mu_1,\dots,\mu_n) \in S_i} P_{i,\mu}(\bu_i)
\frac{ \partial^{\mu_1 + \cdots + \mu_n} f}
{\partial X_1^{\mu_1} \cdots \partial X_n^{\mu_n}}(\balpha_i),
\end{align}
where $S_i$ is  the union of $S_{i,1},\dots,S_{i,D_i}$,
with in particular $P_{i,(0,\dots,0)}=U_{i,1}$ and where $P_{i,\mu}$
depends only on $(U_{i,j})_{j =2,\dots,D_i}$ for all $\mu$ in $S_i$,
$\mu \ne (0,\dots,0)$. Explicily, we can write $P_{i,\mu}=\sum_{j\in
	\{1,\dots,D_i\} \text{~such that~} \mu \in S_{i,j}} c_{i,j,\mu}
U_{i,j}$. 

Fix $\ell$ non-zero in ${\rm hom}_\Kbar(\residueI_i,\Kbar)$, written
as in~\eqref{ell_param}. We can then
define its {\em order} $w$ and {\em symbol} $\pi$. The former is
the maximum of all $|\mu|=\mu_1+\cdots+\mu_n$ for
$\mu=(\mu_1,\dots,\mu_n)$ in $S_i$ such that $P_{i,\mu}(\bu_i)$ is
non-zero; by~\cite[Lemma~3.3]{Mourrain97} we have $w \le
N_i-1$. Then, we let
$$\pi =\sum_{\mu \in S_i,\ |\mu|=w}P_{i,\mu}(\bu_i) X_1^{\mu_1} \cdots
X_n^{\mu_n}$$ be the {\em symbol} of $\ell$; by construction,
this is a non-zero polynomial. In the following paragraphs, we will
need the next easy lemma.

\begin{lemma}\label{lemma:symbol0}
  Fix $i$ in $\{1,\dots,\dg\}$. For a generic choice of $\ell$ in
  ${\rm hom}_\Kbar(\residueI_i,\Kbar)$, and of $t_1,\dots,t_n$ in $\Kbar{}^n$,
  $\pi_i(t_1,\dots,t_n)$ is non-zero.
\end{lemma}
\begin{proof}
  Let $\Omega$ be the maximum of all $|\mu|=\mu_1+\cdots+\mu_n$ for
  $\mu=(\mu_1,\dots,\mu_n)$ in $S_i$, and define 
	$$\Pi =\sum_{\mu \in S_i,\ |\mu|=\Omega}P_{i,\mu} X_1^{\mu_1}
  \cdots X_n^{\mu_n} \in
  \Kbar[U_{i,1},\dots,U_{i,D_i},X_1,\dots,X_n];$$ this is by
  construction a non-zero polynomial.  Thus, for a generic choice of
  $\bu_i=(u_{i,1},\dots,u_{i,D_i})$, that define a linear form
  $\ell$ in ${\rm hom}_\Kbar(\residueI_i,\Kbar)$ as in \cref{ell_param},
  and of $t_1,\dots,t_n$ in $\Kbar{}^n$, the value
  $\Pi(u_{i,1},\dots,u_{i,D_i},t_1,\dots,t_n)$ is non-zero. As a
  result, the symbol of such a linear form $\ell$ is $\pi
  =\sum_{\mu \in S_i,\ |\mu|=\Omega}P_{i,\mu}(\bu_i) X_1^{\mu_1}
  \cdots X_n^{\mu_n},$ and $\pi(t_1,\dots,t_n)$ is then non-zero.
\end{proof}

Finally, we say a word about global objects.  Fix a linear form $\ell:
\residueI \to \K$. By the Chinese Remainder Theorem, there exist unique
$\ell_1,\dots,\ell_\dg$, with $\ell_i$ in ${\rm hom}_\Kbar(\residueI_i,\Kbar)$
for all $i$, such that the extension $\ell_\Kbar: \residueI\otimes_\K \Kbar
\to \Kbar$ decomposes as $\ell_\Kbar = \ell_1 + \cdots + \ell_\dg$. We
call {\em support} of $\ell$ the subset $\mathfrak{S}$ of
$\{1,\dots,\dg\}$ such that $\ell_i$ is non-zero exactly for $i$ in
$\mathfrak{S}$.  As a consequence, for all $f$ in $\residueI$, we have
\begin{align}\label{eq:fui}
\ell(f) &= \ell_1(f) + \cdots + \ell_\dg(f)\nonumber\\
&=  \sum_{i \in \mathfrak{S}} \ell_i(f).
\end{align}
For $i$ in $\mathfrak{S}$, we denote by $w_i$ and $\pi_i$ respectively
the order and the symbol of $\ell_i$. For such a subset
$\mathfrak{S}$, we also write $\residueI_\mathfrak{S}=\prod_{i \in
  \mathfrak{S}} \residueI_i$ and $V_\mathfrak{S}=
\{\balpha_i \mid i \in \mathfrak{S}\}$.

%%%%%%%%%%%%%%%%%%%%%%%%%%%%%%%%%%%%%%%%%%%%%%%%%%%%%%%%%%%%

\subsection{A fundamental formula}  \label{ssec:genseries}

Let $\lf$ be in $\residueI$ and $\ell$ in ${\rm hom}_\K(\residueI,\K)$.  The sequences
$(\ell(\lf^s))_{s\ge 0}$, and more generally $(\ell(v \lf^s))_{s\ge
  0}$, for $v$ in $\residueI$, are the main ingredients in our algorithm.  The
following lemma justifies this by giving an explicit form for a
generating series of the form $\sum_{s \ge 0} \ell(v
\lf^s)/T^{s+1}$. A less precise version of it is in~\cite{BoSaSc03};
the more explicit expression given here will be needed in the last
section of this paper.

\begin{lemma}\label{lemma:formula}
  Let $\ell$ be in ${\rm hom}_\K(\residueI,\K)$, with support $\mathfrak{S}$,
  and let $\{\pi_i \mid i \in \mathfrak{S}\}$ and $\{w_i \mid i \in
  \mathfrak{S}\}$ be the symbols and orders of $\{\ell_i \mid i \in \mathfrak{S}\}$,
  for $\{\ell_i \mid i \in \mathfrak{S}\}$ as in~\eqref{eq:fui}.
	
  Let $\lf=t_1 X_1 + \cdots +t_n X_n$, for some $t_1,\dots,t_n$ in $\K$
  and let $v$ be in $\K[X_1,\dots,X_n]$. Then, we have the equality
  \begin{align}\label{eq:sumgenseries}
    \sum_{s \ge 0} \frac{\ell(v \lf^s)}{T^{s+1}} = \sum_{i \in \mathfrak{S}}
    \frac{ v(\balpha_i)\, w_i!\, \pi_{i}(t_1,\dots,t_n) +
      (T-\lf(\balpha_i))A_{v,i}} {(T-\lf(\balpha_i))^{w_{i}+1}},
  \end{align}
  for some polynomials $\{A_{v,i} \in \Kbar[T] \mid i \in \mathfrak{S}\}$ (that
  depend on the choice of $v$), with $A_{v,i}$ of degree less than $w_i$ for all $i$ in
  $\mathfrak{S}$.
\end{lemma}
\begin{proof}
  Take $v$ and $\lf$ as above. Consider first an operator of the form $f
  \mapsto \frac{ \partial^{|\mu|} f} {\partial X_1^{\mu_1} \cdots
    \partial X_n^{\mu_n}}$, where we write
  $|\mu|=\mu_1+\cdots+\mu_n$. Then, we have the following generating
  series identities, with coefficients in $\K(X_1,\dots,X_n)$:
  \begin{align*}
    \sum_{s \ge 0} 
    \frac{ \partial^{|\mu|} ( v \lf^s )} {\partial X_1^{\mu_1} \cdots
      \partial X_n^{\mu_n}}
    \frac{1}{T^{s+1}} 
    &=  \sum_{s \ge 0} 
    \frac{ \partial^{|\mu|} (v \lf^s/T^{s+1})} {\partial X_1^{\mu_1} \cdots
      \partial X_n^{\mu_n}}\\
    &=  
    \frac{ \partial^{|\mu|} } {\partial X_1^{\mu_1} \cdots
      \partial X_n^{\mu_n}}
    \left (\sum_{s \ge 0} \frac{v \lf^s}{T^{s+1}}\right ) \\
    &= \frac{ \partial^{|\mu|} } {\partial X_1^{\mu_1} \cdots
      \partial X_n^{\mu_n}}
    \left (\frac v{T-\lf} \right ) \\
    &= v\, |\mu|!\, \prod_{1 \le k \le n} 
    \left (\frac{ \partial \lf} {\partial X_k} \right)^{\mu_k}
    \frac {1}{(T-\lf)^{|\mu|+1}} + \frac{P_{|\mu|}}{(T-\lf)^{|\mu|}} + \cdots + \frac{P_{1}}{(T-\lf)}\\
    &= v\, |\mu|!\, \prod_{1 \le k \le n} 
    t_k^{\mu_k}
    \frac {1}{(T-\lf)^{|\mu|+1}} + \frac{\sqfree}{(T-\lf)^{|\mu|}},
  \end{align*}
  for some polynomials $P_1,\dots,P_{|\mu|},\sqfree$ in $\K[X_1,\dots,X_n,T]$ that
  depend on the choices of $\mu$, $v$ and $X$, with $\deg(P_i,T) < i$
  for all $i$ and thus $\deg(P,T) < |\mu|$.
  
  Take now a $\Kbar$-linear combination of such operators, such as $f \mapsto
  \sum_{\mu \in R} c_\mu \frac{ \partial^{|\mu|} f } {\partial
    X_1^{\mu_1} \cdots \partial X_n^{\mu_n}}$. The corresponding
  generating series becomes
  \begin{align*}
    \sum_{s \ge 0} \sum_{\mu \in R} c_\mu \frac{ \partial^{|\mu|} ( v
      \lf^s )} {\partial X_1^{\mu_1} \cdots \partial X_n^{\mu_n}}
    \frac{1}{T^{s+1}} &= v\,\sum_{\mu \in R} c_\mu |\mu|!\, \prod_{1
      \le k \le n} t_k^{\mu_k} \frac {1}{(T-\lf )^{|\mu|+1}} +\sum_{\mu
      \in R} \frac{\sqfree_\mu}{(T-\lf)^{|\mu|}},
  \end{align*}
  where each $\sqfree_\mu \in \Kbar[X_1,\dots,X_n,T]$ has degree less than
  $|\mu|$ in $T$.  Let $w$ be the maximum of all $|\mu|$ for $\mu$ in
  $R$. We can rewrite the above as
  \begin{align*}
    v\, w! 
    \sum_{\mu \in R, |\mu|=w} c_\mu
    \, \prod_{1 \le k \le n} 
    t_k^{\mu_k}
    \frac {1}{(T-\lf )^{w+1}}
	+ \frac{A}{(T-\lf )^{w}},
  \end{align*}
  for some polynomial $A \in \Kbar[X_1,\dots,X_n,T]$ of degre less than $w$ in $T$. If we let 
  $\pi =\sum_{\mu \in R,\ |\mu|=w} c_{\mu} X_1^{\mu_1} \cdots
  X_n^{\mu_n}$, this becomes
  \begin{align*}
    \sum_{s \ge 0} 
    \sum_{\mu \in R} c_\mu \frac{ \partial^{|\mu|} ( v \lf^s )} { X_1^{\mu_1} \cdots
      X_n^{\mu_n}}
    \frac{1}{T^{s+1}} 
    &=
    v\, w! \,  \pi(t_1,\dots,t_n)
    \frac {1}{(T-\lf )^{w+1}}
    + \frac{A}{(T-\lf )^{w}}.
  \end{align*}
  Applying this formula to the sum $\ell=\sum_{i \in
    \mathfrak{S}}\ell_i$ from \cref{eq:fui}, and taking into account
  the expression~\ref{ell_param} for each $\ell_i$, we obtain the
  claim in the lemma.
\end{proof}

\noindent 
The most useful consequence of the previous lemma is the following
interpolation formula involving the operator $\Omega$, which
generalizes a brief comment made in \cref{section:linseq}.

Fix a subset $\mathfrak{S}$ of $\{1,\dots,\dg\}$. The mapping
$\lf:V_\mathfrak{S} \to \Kbar$ defined by $\balpha_i \mapsto
\lf(\balpha_{i})$ plays a special role in the formula in the previous
lemma; this leads us to the following definitions.
\begin{itemize}
\item We consider $\ell$ and $\lf$ as in \cref{lemma:formula}, such
  that $\ell$ has support $\mathfrak{S}$.
\item $\mathfrak{T}$ is the subset of $\mathfrak{S}$ consisting of
  all indices $i$ such that 
  \begin{itemize}
  \item $\lf(\balpha_{i'}) \ne \lf(\balpha_i)$ for $i' \ne i$ in $\mathfrak{S}$;
  \item $\pi_i(t_1,\dots,t_n)$ is non-zero.
  \end{itemize}
\item $\{r_1,\dots,r_c\}$ are the pairwise distinct values taken by $\lf$ on
  $V_\mathfrak{S}$, for some $c \le |\mathfrak{S}|$.
\item $\mathfrak{t}$ is the set of all indices $j$ in
  $\{1,\dots,c\}$ such that
  \begin{itemize}
  \item the fiber $\lf^{-1}(r_j) \subset V_{\mathfrak{S}}$ contains a single
    point, written $\balpha_{\sigma_j}$;
  \item   $\pi_{\sigma_j}(t_1,\dots,t_n)$ is non-zero.
  \end{itemize}
  Remark that $j \mapsto \sigma_j$ induces a one-to-one correspondence
  between  $\mathfrak{t}$ and  $\mathfrak{T}$, and that $\lf(\balpha_{\sigma_j})=r_j$ 
  for all $j$ in $\mathfrak{t}$.
\end{itemize}

\begin{lemma} \label{lemma:anyv}
  Let $\ell$, $\lf$ and all other notation be as above. Let further
  $\minpoly$ be the minimal polynomial of $\lf$ in
  $\residueI_\mathfrak{S}$. Suppose that $\minpoly$ is also the minimal
  polynomial of the sequence $(\ell(\lf^s))_{s \ge 0}$. Then, $\minpoly$
  cancels the sequence $(\ell(v \lf^s))_{s\ge0}$ and
there
  exist non-zero constants $\{c_j \mid j \in \mathfrak{t}\}$ such that
  for $v$ in $\K[X_1,\dots,X_n]$,
  $$\Omega((\ell(v \lf^s))_{s\ge0},\minpoly) = c_{j} v(\balpha_{\sigma_j}) \quad \text{for all $j$ in $\mathfrak{t}$.}$$
\end{lemma}
\begin{proof}
  For $j=1,\dots,c$, we write $T_j$ for the set of all indices $i$ in
  $\mathfrak{S}$ such that $t(\balpha_{i})=r_j$; the sets
  $T_1,\dots,T_c$ form a partition of $\mathfrak{S}$. When $T_j$ has
  cardinality~$1$, we thus have $T_j=\{\sigma_j\}$.
  
  Take an arbitrary $v$ in $\K[X_1,\dots,X_n]$ and let us
  collect terms in \cref{eq:sumgenseries} as
  \begin{align*}
    \sum_{s \ge 0} \frac{\ell(v \lf^s)}{T^{s+1}} =&
    \sum_{j \in \{1,\dots,c\}}
    \sum_{i \in T_j} \frac{
      v(\balpha_{i})   w_{i}!\, \pi_{i}(t_1,\dots,t_n)
      + (T-r_{j} )A_{v,i}}
	{(T-r_{j} )^{w_{i}+1}}\\
	=&
	\sum_{j \in \mathfrak{t}}
	\frac{
	  v(\balpha_{\sigma_{j}})  w_{\sigma_j}!\, \pi_{\sigma_j}(t_1,\dots,t_n)
	  + (T-r_{j}  )A_{v,\sigma_j} }
	     {(T-r_{j} )^{w_{\sigma_j}+1}}\\
	     &+
	     \sum_{j \in \{1,\dots,c\}-\mathfrak{t}}
	     \frac{   \sum_{i \in T_j} \Big( \big[
		 v(\balpha_{i})   w_{i}!\, \pi_{i}(t_1,\dots,t_n)
		 + (T-r_{j}  )A_{v,i} \big ]
	       (T-r_{j} )^{y_j-(w_i+1)}\Big)}
	          {(T-r_{j} )^{y_j}},
  \end{align*}
  where $y_j$ is the maximum of all $w_i$ for $i$ in $T_j$.  Remark
  that for $v=1$, our condition that $\pi_i(t_1,\dots,t_n)$ is
  non-zero for $i$ in $\mathfrak{T}$, together with our assumption on
  the characteristic of $\K$, imply that in the second line, all terms
  in the first sum are non-zero and in reduced form.
  
  After simplyfing terms in the second sum, we can rewrite the
  expression above as
  \begin{align*}
    \sum_{s \ge 0} \frac{\ell(v \lf^s)}{T^{s+1}} =&
    \sum_{j \in \mathfrak{t}} \frac{
      v(\balpha_{\sigma_{j}})   w_{\sigma_{j}}!\, \pi_{{\sigma_{j}}}(t_1,\dots,t_n)
      + (T-r_{j} )A_{v,\sigma_{j}}}
	{(T-r_{j} )^{w_{\sigma_{j}}+1}}  
	+\sum_{j \in  \{1,\dots,c\}-\mathfrak{t}}
	\frac{D_{v,j}}
	     {(T-r_{j})^{z_{v,j}}},
  \end{align*}
  for some positive integers $\{z_{v,j} \mid j\in
  \{1,\dots,c\}-\mathfrak{t}\}$ and polynomials $\{D_{v,j} \mid j\in
  \{1,\dots,c\}-\mathfrak{t}\}$ such that for all $j$ in $
  \{1,\dots,c\}-\mathfrak{t}$, we have $\deg(D_{v,j}) < z_{v,j}$ and
  $\gcd(D_{v,j}, T-r_{j} )=1$. Some of the polynomials $D_{v,j}$ may
  vanish, so we let $\mathfrak{u}_v \subset
  \{1,\dots,c\}-\mathfrak{t}$ be the set of all $j$ for which this is
  not the case.  We then arrive at our final form for this sum, namely
  \begin{align}\label{eq:sumgenseries_collect1}
    \sum_{s \ge 0}  \frac{\ell(v \lf^s)}{T^{s+1}} =&
    \sum_{j \in \mathfrak{t}} \frac{
      v(\balpha_{\sigma_{j}})   w_{\sigma_{j}}!\, \pi_{{\sigma_{j}}}(t_1,\dots,t_n)
      + (T-r_{j}  )A_{v,\sigma_{j}}}
	{(T-r_{j} )^{w_{\sigma_{j}}+1}}  
	+\sum_{j \in  \mathfrak{u}_v}
	\frac{D_{v,j}}
	     {(T-r_{j} )^{z_{v,j}}},
  \end{align}
  where all terms in the second sum are non-zero and in reduced form
  (and similarly for the first sum, for $v=1$).
  This implies that the minimal
  polynomial of the sequence $(\ell(v\lf^s))_{s \ge 0}$ is 
  $$\minpoly_v=\prod_{j \in \mathfrak{t}} (T-r_{j})^{\zeta_j} \prod_{j
    \in \mathfrak{u}_v} (T-r_{j})^{z_{v,j}},$$
  for some integers $\{\zeta_j \le w_{\sigma_{j}}+1 \mid j \in \mathfrak{t}\}$; for $v=1$, 
  we actually have $\zeta_j = w_{\sigma_{j}}+1$ for all such~$j$.
	
  Now, for $v=1$, we assume that the minimal polynomial of the sequence
  $(\ell(\lf^s))_{s \ge 0}$ is the minimal polynomial $\minpoly$ of $t$ in
  $\residueI_\mathfrak{S}$.  Writing $\mathfrak{u}=\mathfrak{u}_1$ and
  $z_k=z_{1,k}$ for all $k$ in $\mathfrak{u}$, we can thus write it as
  $$\minpoly=\prod_{j \in \mathfrak{t}} (T-r_{j})^{w_{\sigma_{j}}+1}
  \prod_{j \in \mathfrak{u}} (T-r_{j})^{z_{j}}.$$ Since it is the
  minimal polynomial of $\lf$ in $\residueI_\mathfrak{S}$, it also cancels the
  sequence $(\ell(v \lf^s))_{s \ge 0}$ for any $v$, so that for all
  $v$, $\minpoly_v$ divides $\minpoly$ (which proves the first point
  in the lemma), and in particular $\mathfrak{u}_v$ is contained in
  $\mathfrak{u}$.
  We can then rewrite the
  sum in \cref{eq:sumgenseries_collect1} as $$ \sum_{s \ge 0} \frac{\ell(v \lf^s)}{T^{s+1}} =\frac{\Omega((\ell(v \lf^s))_{s \ge 0} ,\minpoly)}{\minpoly},$$ with
  \begin{align*}
    \Omega((\ell(v \lf^s))_{s \ge 0} ,\minpoly)=&\sum_{j \in \mathfrak{t}} \Big(\big[
      v(\balpha_{\sigma_{j}})  w_{\sigma_j}!\, \pi_{\sigma_j}(t_1,\dots,t_n)
      + (T-r_j) A_{v,\sigma_j}\big]
    \prod_{\iota \in \mathfrak{t}-\{j\}}(T-r_{\iota} )^{w_{\sigma_{\iota}}+1}
    \Big)\prod_{j \in \mathfrak{u}}(T-r_j )^{z_j}
    \\
    &+
    \Big(\prod_{j \in \mathfrak{t}}(T-r_{j} )^{w_{\sigma_j}+1}\Big)
    \sum_{j \in \mathfrak{u}_v} \Big (
    (T-r_j)^{z_{j}-z_{v,j}} D_{v,j}
    \prod_{\iota \in \mathfrak{u}-\{j\}}(T-r_{\iota} )^{z_{\iota}}\Big).
  \end{align*}
  In particular, 
  for $k$ in $\mathfrak{t}$,
  the value $\Omega((\ell(v \lf^s))_{s \ge 0} ,\minpoly)(r_k)$ is 
  \begin{align*}
    \Omega((\ell(v \lf^s))_{s \ge 0} ,\minpoly)(r_k)&= v(\balpha_{\sigma_{k}}) w_{\sigma_k}!\, \pi_{\sigma_k}(t_1,\dots,t_n)
    \prod_{\iota \in \mathfrak{t}-\{k\}}(r_\iota-r_{k} )^{w_{\sigma_{\iota}}+1}
    \prod_{j \in \mathfrak{u}}(r_j-r_{k} )^{z_{k}}\\
    &= v(\balpha_{\sigma_{k}}) c_{k},
  \end{align*}
  with 
  $$c_{k}=
  w_{\sigma_k}!\, \pi_{\sigma_k}(t_1,\dots,t_n)
  \prod_{\iota \in \mathfrak{t}-\{k\}}(r_\iota-r_{k} )^{w_{\sigma_{\iota}}+1}
  \prod_{j \in \mathfrak{u}}(r_j-r_{k} )^{z_{k}}$$
  for $k$ in $\mathfrak{t}$. This is a non-zero constant, independent 
  of $v$, which finishes the proof of the lemma.
\end{proof}

%%%%%%%%%%%%%%%%%%%%%%%%%%%%%%%%%%%%%%%%%%%%%%%%%%%%%%%%%%%%

\subsection{Computing a zero-dimensional parametrization}  \label{ssec:abstractlago}

As an application, the following algorithm shows how to compute a
zero-dimensional parametrization of $V_{\mathfrak{S}}$; our main usage 
of it will be with $\mathfrak{S}=\{1,\dots,\dg\}$ (so $V_{\mathfrak{S}}=V$),
but in the last section of the paper, we will also work with 
strict subsets. At this stage,
we do not discuss data structures, complexity, or what terms in the
sequences are needed; this is the subject of the following section.

\begin{algorithm}[H]
  \caption{$\mathsf{Parametrization}(\ell,\lf)$}  ~\\
	  {\bf Input:} \vspace{-0.5em}
	  \begin{itemize}\setlength\itemsep{0em}
	  \item  a linear form $\ell$ over $\residueI_\mathfrak{S}$
	  \item $\lf=t_1 X_1 + \cdots + t_n X_n$
	  \end{itemize}
	      {\bf Output:} polynomials $((\sqfree,V_1,\dots,V_n),\lf)$
	      \begin{enumerate}\setlength\itemsep{0em}
	      \item let $\minpoly$ be the minimal polynomial of the sequence $(\ell(\lf^s))_{s \ge 0}$
	      \item let $\sqfree$ be the squarefree part of $\minpoly$
	      \item let $C_1 = \Omega((\ell(\lf^s))_{s\ge0} ,\minpoly)$
	      \item \textbf{for} $i=1,\dots,n$ \textbf{do}
		\begin{enumerate}
		\item let $C_{X_i} = \Omega((\ell(X_i \lf^s))_{s\ge0}, \minpoly)$ 
		\end{enumerate}
	      \item \textbf{return} $((\sqfree, C_{X_1}/ C_1 \bmod \sqfree, \dots, C_{X_n}/ C_{1} \bmod \sqfree),\lf)$
	      \end{enumerate}
	      \label{algo:para2}
\end{algorithm}

\begin{lemma}\label{lemma:para2}
  Suppose that $\ell$ is a generic element of ${\rm
    hom}_{\Kbar}(\residueI_\mathfrak{S},\Kbar)$ and that $\lf$ is a generic
  linear form. Then the output $((\sqfree,V_1,\dots,V_n),\lf)$ of
  $\mathsf{Parametrization}(\ell,\lf)$ is a zero-dimensional
  parametrization of $V_{\mathfrak{S}}$.
\end{lemma}
\begin{proof}
  A generic choice of $\lf$ separates the points of
  $V_{\mathfrak{S}}$, and we saw in \cref{lemma:symbol0} that for a
  generic choice of $\ell$, $\pi_i(t_1,\dots,t_n)$ vanishes for no $i$
  in $\mathfrak{S}$.  As a result, $\mathfrak{T}=\mathfrak{S}$,
  and $\mathfrak{t}$ consists of $|\mathfrak{S}|$ pairwise distinct 
  values.

  Besides, we recall that for a generic $\ell$ in ${\rm
    hom}_{\Kbar}(\residueI_\mathfrak{S},\Kbar)$, the minimal polynomials of
  $(\ell(\lf^s))_{s \ge 0}$ and of $\lf$ are the same
  (\cref{lemma:minpoly}).  Thus, the polynomial $\minpoly$ we compute
  at step 1 is indeed the minimal polynomial of $\lf$ (with roots
  $\{r_j \mid j \in \mathfrak{t}\}$), and we can apply the previous
  lemma; then, for any root $r_j$ of $\minpoly$, and $i=1,\dots,n$, we
  have
  $$\frac{ C_{X_i}(r_j)}{ C_1(r_j)} = \frac{\Omega((\ell(X_i
    \lf^s))_{s\ge0}, \minpoly)(r_j)}{\Omega((\ell(\lf^s))_{s\ge0} ,\minpoly)(r_j)}=
  \frac{c_j \alpha_{\sigma_j,i}}{c_j} = \alpha_{\sigma_j,i},$$ so that
  $ C_{X_i}/ C_1 \bmod P$ is the $i$th polynomial in the
  zero-dimensional parametrization of $V$ corresponding to $\lf$.
\end{proof}

We demonstrate how this algorithm works through a small example. Let 
$$I = \langle (X_1-1)(X_2-2),(X_1-3)(X_2-4)\rangle \subset
\F_{101}[X_1,X_2].$$ Then, $V(I) = \{ (1,4),(3,2) \}$ and $X_1$ separates the
points of $V(I)$. We choose a  linear form 
$$\ell: f \in \F_{101}[X_1,X_2]/I \to \F_{101},\;\ell(f) = 17 f(1,4) + 33 f(3,2).$$
Then, we have
\begin{align*}
\ell(X_1^s) &= 17 \cdot 1^s + 33 \cdot 3^s\\
\ell(X_2X_1^s) &= 17 \cdot 4 \cdot 1^s + 33 \cdot 2 \cdot 3^s.
\end{align*} 
We associate a generating series to each sequence,
\begin{align*}
Z_1 = \sum_{s = 0}^{\infty} \frac{\ell(X^s_1)}{T^{s+1}}
&= \frac{17}{T-1} + \frac{33}{T-3}
= \frac{17(T-3)+33(T-1)}{(T-1)(T-3)} \\
Z_{X_2} = \sum_{s=0}^{\infty} \frac{\ell(X_2X_1^s)}{T^{s+1} }
&= \frac{17\cdot 4}{T-1} + \frac{33 \cdot 2}{t-3}
= \frac{17\cdot 4 (T-3) + 33\cdot 2(T-1)}{(T-1)(T-3)}.
\end{align*}
These generating series $Z_1$ and $Z_2$ have a common denominator $\minpoly = (T-1)(T-3)$,
whose roots are the coordinates of $X_1$ in $V(I)$;
their numerators are respectively
$$C_{1} = \Omega((\ell(X^s_1))_{s\ge 0},\minpoly) = 17 (T-3) + 33(T-1)$$
and
$$C_{X_2} = \Omega((\ell(X_2X^s_1))_{s\ge 0},\minpoly) = 17\cdot 4 (T-3) + 33\cdot 2(T-1).$$
Now, let
\begin{align*}
V_2 
&=\frac{C_{X_2}}{C_1} \mod \minpoly\\
&=\frac{17\cdot 4 (T-3) + 33\cdot 2(T-1)}{17(T-3)+33(T-1)} \mod \minpoly.
\end{align*}
Then, $ V_2(1) = 4$ and $V_2(3) = 2$ as needed.

%%%%%%%%%%%%%%%%%%%%%%%%%%%%%%%%%%%%%%%%%%%%%%%%%%%%%%%%%%%%
%%%%%%%%%%%%%%%%%%%%%%%%%%%%%%%%%%%%%%%%%%%%%%%%%%%%%%%%%%%%
%%%%%%%%%%%%%%%%%%%%%%%%%%%%%%%%%%%%%%%%%%%%%%%%%%%%%%%%%%%%

\section{The main algorithm}

In this section, we will show how extend the algorithm
of~\cite{BoSaSc03} to compute a zero-dimensional parametrization of
$V(I)$, for some zero-dimensional ideal $I$ of $\K[X_1,\dots,X_n]$, by
using blocking methods. As input, we assume that we know a monomial
basis $\basis=(b_1,\dots,b_D)$ of $\residueI=\K[X_1,\dots,X_n]/I$, together with the
multiplication matrices $\mM_1,\dots,\mM_n$ of respectively
$X_1,\dots,X_n$ in this basis; for definiteness, we suppose that the
first basis element in $\basis$ is~$b_1=1$. 

The main feature of the algorithm is that after we compute the Krylov
sequence used to find a minimal matrix generator, we recover the
entries of the output for a minor cost, without computing another
Krylov sequence. We make no assumption on $I$ (radicality or shape
position); however, we assume (as in the previous subsection) that the
characteristic of $\K$ is greater than $D$.


%%%%%%%%%%%%%%%%%%%%%%%%%%%%%%%%%%%%%%%%%%%%%%%%%%%%%%%%%%%%

\subsection{Description, correctness and cost analysis}

We mentioned that Steel's method~\cite{Steel15} already uses the Block
Wiedemann algorithm to compute the minimal polynomial $\minpoly$ of
$\lf=t_1 X_1 + \cdots + t_n X_n$; knowing its roots in $\K$, it uses
an ``evaluation" method for the rest (another Gr\"obner Basis
computation with one variable less).  Our algorithm computes the whole
zero-dimensional parametrization for essentially the same cost as the
computation of the minimal polynomial. In the following pseudo-code,
$\col{\varepsilon}_1$ denotes the size-$D$ column vector whose only
nonzero entry is a $1$ at the first row.

%% Given:
%% \begin{center}
%% 	\begin{tabular}{c c}
%% 		$I \subset \mathbb{K}$:& zero dimensional ideal
%% 		in shape position\\
%% 		$\mathbb{B} \subset \mathbb{K}[X_1,\dots,X_n]/I$:&
%% 		monomial basis of $\mathbb{K}[X_1,\dots,X_n]/I$\\
%% 		$D$: & dimension of $\mathbb{B}$\\
%% 		$\mM_1, \dots,\mM_n$:& multiplication matrices of
%% 		$X_1 ,\dots,X_n$ respectively\\
%% 		$t$:& random linear combination of $X_i$'s\\
%% 		$\mM$:& multiplication matrix of $t$
%% 	\end{tabular}
%% \end{center}
%% we compute a lex Gr\"obner basis that have the same points
%% in its variety as the radical of $I$ 
%% (note that we do not assume that $I$ is radical).
%% This is because we introduce another variable $t$ which,
%% generically, separates the points in the variety.
%% We also assume that the base field $\K$ has characteristic
%% larger than $D$. More precisely, we want to find polynomials
%% $(R,R_1,\dots,R_n)$ such that for all $\alpha$ that is a factor of
%% $R$, $\{ (R_1(\alpha),\dots,R_n(\alpha)) \} = V(I)$
\begin{algorithm}[H]
	\caption{\mainalgoname($\mM_1,\dots,\mM_n,m$)}
	{\bf Input:} \vspace{-0.5em}
	\begin{itemize}
		\item $\mM_1,\dots,\mM_n$ defined as above
		\item dimension of the blocks $m \in \{1,\dots,D\}$
	\end{itemize}
	{\bf Output:} polynomials $((\sqfree,V_1,\dots,V_n),\lf)$
  \begin{enumerate}
  \item\label{mainstep1}  choose $t_1,\dots,t_n$ in $\K$ and let $\mM = t_1 \mM_1 + \cdots + t_n \mM_n$
  \item\label{mainstep2} { choose $\mU,\mV \in \mathbb{K}^{D \times m}$}
  \item\label{mainstep3} { compute $\mat{L}_s = \mU^{\perp}\mM^s$ for $s=0,\dots,2d-1$, with $d = \lceil D/d \rceil$}
  \item\label{mainstep4} { compute $\seqelt{s,\mU,\mV}= \mat{L}_s\mV$ for $s=0,\dots, 2d-1$}
  \item\label{mainstep5} { compute a minimal matrix generator $\mat{P}_{\mU,\mV}$ of $(\seqelt{s,\mU,\mV})_{0 \le s < 2d}$}
  \item\label{mainstep6} { let $\minpoly$ be the largest invariant factor of $\mat{P}_{\mU,\mV}$}
  \item\label{mainstep7} { let $\sqfree$ be  the squarefree part  of $\minpoly$}
  \item\label{mainstep8} { let $\row{a}_1 = [P~0 ~\cdots~ 0] (\mat{P}_{\mU,\mV})^{-1}$}
  \item\label{mainstep9}  let $C_1 = \mathrm{ScalarNumerator}(\mat{P}_{\mU,\mV}, \minpoly, \col{\varepsilon}_1, 1, \row{a}_1, 
    (\mat{L}_s)_{0 \le s < \lceil D/m \rceil})$
  \item\label{mainstep10} \textbf{for} $i=1,\dots,n$ \textbf{do}
    \begin{enumerate}
     \item let $C_{X_i} = \mathrm{ScalarNumerator}(\mat{P}_{\mU,\mV}, \minpoly, \mM_i\col{\varepsilon}_1, 1, \row{a}_1, 
      (\mat{L}_s)_{0 \le s < \lceil D/m \rceil})$
    \end{enumerate}
\item\label{mainstep11}     \textbf{return} $((\sqfree, C_{X_1}/ C_1 \bmod \sqfree, \dots, C_{X_n}/ C_{1} \bmod \sqfree),t_1 X_1 + \cdots + t_n X_n)$
  \end{enumerate}  \label{algo:block-sparse-fglm}
\end{algorithm}

We first prove correctness of the algorithm, for generic choices of
$t_1,\dots,t_n$, $\mU$ and $\mV$. The first step computes the
multiplication matrix $\mM=t_1 \mM_1 + \cdots + t_n \mM_n$ of $\lf=t_1
X_1 + \cdots + t_n X_n$.

Then, we compute the first $2d$ terms of the sequence $\seq_{\mU,\mV}=
(\mU^{\perp}\mM^s\mV)_{s\ge 0}$. For generic choices of these two
matrices, as discussed in \cref{ssec:appliW}, \cref{randXY} shows that
the matrix polynomial $\mat{P}_{\mU,\mV}$ is indeed a minimal left
generator of the sequence $\seq_{\mU,\mV}$. Then, $\minpoly$ is the minimal 
polynomial of $\lf$ and $\sqfree$ its squarefree part.

We find the rest of the polynomials in the output by following
\cref{algo:para2}. The scalar numerators $\Omega(\cdots)$ needed in
this algorithm are computed using the method of
\cref{ssec:scalar_numer}. Algorithm ScalarDenominator
compute $$C_1 = \Omega((\row{u}_i \mM^s \col{\varepsilon}_1)_{s \ge 0}, \minpoly)
\quad\text{and}\quad
C_{X_i} = \Omega((\row{u}_1 \mM^s \mM_i \col{\varepsilon}_1)_{s \ge 0}, \minpoly),\ \ i=1,\dots,n.$$
The key result supporting this is \cref{lemma:omegaOmega} which shows
that for a generic choice of $\mU$ and $\mV$, we have 
$$[~C_1~] = \row{a}_1\cdot \mat{\Omega}((\mU^\perp \mM^s \col{\varepsilon}_1)_{s \ge 0}, \mat{P}_{\mU,\mV})$$
and
$$[~C_{X_i}~] = \row{a}_1\cdot \mat{\Omega}((\mU^\perp \mM^s \mM_i \col{\varepsilon}_1)_{s \ge 0}, \mat{P}_{\mU,\mV}),\ \ i=1,\dots,n;
$$
Algorithm ScalarDenominator precisely computes these quantities.

Let $\ell:\residueI \to \K$ be the linear form $f = \sum_{i \in \basis} f_i b_i \mapsto 
\sum_{1 \le i \le D} f_i u_{i,1}$, where $u_{i,1}$ is the entry at position
$(i,1)$ in $\mU$. Then, the two polynomials above can be rewritten
as 
$$C_1 = \Omega( (\ell(X^s))_{s \ge 0}, \minpoly) \quad\text{and}\quad
C_{X_i} = \Omega( (\ell(X_i X^s))_{s \ge 0}, \minpoly),$$ so they
coincide with the polynomials computed in \cref{algo:para2}.  Then,
\cref{lemma:para2} shows that for generic $\mU$ and $\lf$, the output
is indeed a zero-dimensional parametrization of $V(I)$.  

For the cost analysis, we will mainly focus on a sparse model for
matrices $\mM_1,\dots,\mM_n$, assuming that all of them have density
at most $\density \in [0,1)$, that is, at most $\density D^2$ nonzero entries.  
As a result, $\mM$ has density at most $n\density$, and a matrix-vector product
by $\mM$ can be done in $O(n\density D^2)$ operations in $\K$ (we briefly
discuss variants of the algorithm using dense linear algebra at the
end of this subsection). The cost incurred at Step~\ref{mainstep2}
to compute $\mM$ is $O(n\density D^2)$.

In this context, the main purpose of Coppersmith's blocking strategy
is to allow for easy parallelization. Computing the matrices
$\mat{L}_s=\mU^{\perp}\mM^s$, for $s=0,\dots,2d-1$, is the bottleneck
of the algorithm, but this can be parallelized. This is done by
working row-wise, computing independently the sequences
$(\row{\ell}_{i,s})_{0 \le s < 2d}$ of the $i$th rows of
$(\mat{L}_s)_{0 \le s < 2d}$ as $\row{\ell}_{i,s+1} = \row{u}_{i,s}
\mM$ for all $i,s$, where $\row{u}_i$ is the $i$ row of $\mU^\perp$.
For a fixed $i \in \{1,\dots,m\}$, computing $(\mat{\ell}_{i,s})_{0
  \le i < 2d}$ costs $O(dn \density D^2) = O(n\density D^3/m )$ field operations. If
we are able to compute $m$ vector-matrix products in parallel at once,
the {\em span} of Step~\ref{mainstep3} is thus $O(n\density D^3/m)$, whereas
the total work is $O(n\density D^3)$.

At step~\ref{mainstep4}, we can then compute $\seqelt{s,\mU,\mV}=\mU^{\perp}\mM^s\mV$, for $s=0,\dots,2d-1$ by the
product
$$
\begin{bmatrix}
\mU^{\perp}\\
\mU^{\perp} \mM\\
\mU^{\perp} \mM^2\\
\vdots\\
\mU^{\perp} \mM^{2d-1}\\
\end{bmatrix} \mV
= 
\begin{bmatrix}
\mU^{\perp} \mV\\
\mU^{\perp} \mM \mV\\
\mU^{\perp} \mM^2 \mV\\
\vdots \\
\mU^{\perp} \mM^{2d} \mV\\
\end{bmatrix}
$$
of size $O(D) \times D$ by $D \times m$; since $m \le D$, this  costs $O(m^{\omega-2}D^2)$
base field operations.

Recall from \cref{section:matrix_seq} that we can compute a minimal
generating polynomial $\mat{P}_{\mU,\mV}$ in time $O(m^{\omega}
\M(D/m) \log(D/m))$, which is $O(m^{\omega-1} \M(D) \log(D))$.  In
Section~\ref{ssec:appliW} and~\ref{ssec:scalar_numer}, we saw that the
largest invariant factor $P$ and the vector $\row{a}_1$ can be
computed in time $O(m^{\omega} \M(D/m) \log(D/m) \log(m))$, which is
$O(m^{\omega-1} \M(D) \log(D)\log(m))$.  Computing $\sqfree$ takes time
$O(\M(D) \log(D))$.

In \cref{ssec:scalar_numer}, we saw that each call to ScalarNumerator
takes $O(D^2 + m\M(D))$ operations in $\K$, for a total of $O(nD^2 + n
m\M(D))$; the final modular inverse takes time $O(n\M(D) + \M(D) \log(D))$.
Altogether, the total span, assuming perfect parallelization 
at Step~\ref{mainstep3} is
$$O\left (n\density \frac{D^3}m + m^{\omega-1} \M(D) \log(D) \log(m) + nD^2 + nm\M(D)\right );$$
the total work is
$$O\left (n\density D^3 + m^{\omega-1} \M(D) \log(D) \log(m) + nD^2 +
nm\M(D)\right ).$$ Of course, we could parallelize other steps than
Step~\ref{mainstep3}, but it is simulateously the most costly in
theory and in practice, and the easiest to parallelize. 

\medskip

We conclude this section by a discussion of ``dense'' versions of the
algorithm (to be used when the density $\density$ is close to $1$). In
general, experiments show that choosing $m \in\{1,\dots, D\}$ that
matches the number of cores is optimal (see \cref{section:ex});
however, if we use a dense model for our matrices, our algorithms
should rely on dense matrix multiplication. We will see two possible
approaches, which respectively take $m=1$ and $m=D$; we will not
discuss how they parallelize, simply pointing out that one may just
parallelize dense matrix multiplications throughout the algorithms.

Let us discuss the modifications in the algorithm to apply when
$m=1$. In this case, blocking has no effect. We compute the
row-vectors $\mat{L}_s$, for $s=0,\dots,2D-1$, using the
square-and-multiply technique used in Keller-Gehrig's
algorithm~\cite{Keller85}, for $O(D^\omega \log(D))$ operations in
$\K$. The minimum generating polynomial matrix $\mat{P}_{\mU,\mV}$ is
equal to the minimum polynomial $\minpoly$ of $\mM$, for generic
choices of $\mU$ and $\mV$ (which can be computed by the
Berlekamp-Massey algorithm, or a fast version of it); besides,
$\row{a}_1 = P (\mat{P}_{\mU,\mV})^{-1} = 1$. Computing the scalar
numerators is simply a power series multiplication in degree at most
$D$. Altogether, the runtime is $O(D^{\omega} \log(D) + nD^2)$, 
where the second term gives the cost of computing $\mM$.


When $m = D$, $\mU,\mV \in \mathbb{K}^{D \times D}$ are square
matrices and $d = D/m = 1$. A minimal generating polynomial of
$\seq_{\mU,\mV} = (\mU^\perp, \mU^\perp\mM\mV,\dots)$ is
$\mat{P}_{\mU,\mV} = T \mat{I}_D - \mU^\perp\mM\mU^{-\perp}.$ Its
largest invariant factor $\minpoly$ and $\row{a}_1$ can be computed in
$O(D^\omega \log(D))$ operations in $\K$.  The numerator
$\mat{\Omega}(\mU^\perp \mM \col{\varepsilon}_1, \mat{P}_{\mU,mV})$ is
$\mU^\perp \col{\varepsilon}_1$, that is the first column of
$\mU^\perp$, from which we recover $C_1$ through a dot product with
$\row{a}_1$. Similarly, the numerator $\mat{\Omega}(\mU^\perp \mM
\mM_i \col{\varepsilon}_1, \mat{P}_{\mU,mV})$ is $\mU^\perp \mM_i
\col{\varepsilon}_1$, and gives us $C_{X_i}$. Altogether, 
the runtime is again  $O(D^{\omega} \log(D) + nD^2)$, 
where the second term now gives the cost of computing $\mM$,
as well as $C_1$ and all $C_{X_i}$.



%% and 
%% $$((T - \mU\mM\mU^{-1})(\mU/T + \mU\mM/T^2))_{T^{\ge 0}} = \mU. $$
%% For $\mM_j$, $1 \le j \le n$, we get the sequence $(\mU\mM_j,\mU\mM\mM_j)$ and
%% $$((T - \mU\mM\mU^{-1})(\mU\mM_j/T + \mU\mM\mM_j/T))_{T^{\ge 0}} = \mU\mM_j.$$
%% Thus, we get the following algorithm:

%% \begin{algorithm}[H]
%% 	\caption{Dense Block Sparse-FGLM($\mM,\mM_1,\dots,\mM_n$)}
%% 	{\bf Input:} \vspace{-0.5em}
%% 	\begin{itemize}
%% 		\item $\mM,\mM_1,\dots,\mM_n$ defined as above
%% 	\end{itemize}
%% 	{\bf Output:} polynomials $(R,R_1,\dots,R_n)$
%%   \begin{enumerate}[{\bf 1.}]
%% 		\item choose $\mU \in \mathbb{K}^{D\times D}$
%% 		\item $\mF^{U,M} = T - \mU\mM\mU^{-1}$
%% 		\item $\mN^{*} = \mU$
%% 		\item $P =$ largest invariant factor of $\mF^{U,M}$
%% 		\item $R =$ SquareFreePart$(P)$
%% 		\item {\sf $a = [0 ~\cdots~ 0~ P] (\mF^{\mU,\mV})^{-1}$}
%% 		\item {\sf $N=a\mN^*e$}
%% 		\item {for j = $1 \dots n$:}
%%       \begin{enumerate}[{\bf 8.1.}]
%% 			\item $\mN_j^* = \mU\mM_j$
%% 			\item $N_j = a\mN_j^*e$
%% 			\item $R_j = N_j/N$ mod $R$
%% 		\end{enumerate}
%% 		\item return $((R, R_1,\dots,R_n),t)$
%% 	\end{enumerate}
%% 	\label{algo:dense-block-sparse-fglm}
%% \end{algorithm}
%% Let $\omega$ be an exponent such that matrices of size $n$ can be multiplied
%% in $O(n^{\omega})$. In this case, computing $\mF^{U,M}$ costs $O(D^\omega)$
%% and computing $N,N_1,\dots,N_n$ costs $O(nD^2)$ since entries of $a$ has roughly degree $D$. \todo{We can solve for $a$ by
%% using the high-order lifting algorithm of \cite[Algorithm~5]{Stor03}. 
%% By \cite[Corollary~16]{Stor03}, the cost of this is $O(D^\omega \log(D))$.}
%% Putting everything together, we get the final cost of $O(D^\omega\log(D) + nD^2)$.



%%%%%%%%%%%%%%%%%%%%%%%%%%%%%%%%%%%%%%%%%%%%%%%%%%%%%%%%%%%%

\subsection{Example}

We give an example of our algorithm  the case of a radical ideal. Let
$$I = \langle -16X_1^2 - 15X_1X_2 - 14X_2^2 - 48X_1 + 26, 35X_1X_2 + 47X_1 - 46X_2 - 47 \rangle \subset \F_{101}[X_1,X_2];$$
the correponding residue class ring $\residueI=\F_{101}[X_1,X_2]/I$ has dimension $D=4$.
We choose $\lf = 2X_1 + 53 X_2$, so that $X_1$, $X_2$ and $\lf$ have respective multiplication matrices
in the basis $\basis=(1,X_2,X_2,X_2^2)$ of $\residueI$:
\begin{align*}
\mM_1 = \begin{bmatrix}
  0 & 10 & 68&   1\\
  0 & 85 & 15&  69\\
  1 & 91 & 19& 100\\
  0 &  0 & 37&  85
%%  85&   0&  37&   0\\
%%  69&  85&  15&   0\\
%% 100&  91&  19&   1\\
%%   1&  10&  68&   0
\end{bmatrix},\,
\mM_2 = \begin{bmatrix}
  0 &  0 & 10 & 95\\
  1 &  0 & 85 & 42\\
  0 &  0 & 91 & 51\\
  0 &  1 &  0 & 36
%% 36&  1&  0&  0\\
%% 42&  0& 85&  1\\
%% 51&  0& 91&  0\\
%% 95&  0& 10&  0
\end{bmatrix},\,
\mM = \begin{bmatrix}
 0 & 20&  60&  88\\
53 & 69&  91&  41\\
 2 & 81&  13&  75\\
 0 & 53&  74&  58
%% 58& 53& 74&  0\\
%% 41& 69& 91& 53\\
%% 75& 81& 13&  2\\
%% 88& 20& 60&  0
\end{bmatrix}.
\end{align*}
We choose $m = 2$ and take $\mU,\mV \in \F_{101}^{D\times m}$ with 
entries
$$ \mU = \begin{bmatrix}
84& 38\\
29& 58\\
80& 43\\
 7& 82
\end{bmatrix},\quad
\mV = \begin{bmatrix}
  6&  97\\
 83&  58\\
101&  95\\
 59&  89
\end{bmatrix}.
$$ We compute the first $2d=2\lceil D/m\rceil =4$ terms in the matrix
sequence $\seq_{\mU,\mV} = (\mU^{\perp}\mM^s\mV)_{s\ge0}$ and its
minimum generating matrix polynomial $\mat{P}_{\mU,\mV}$. This is done
by first computing
\begin{align}\label{exampleproducts}
\mU^\perp =
\begin{bmatrix}
  7&  29&  80&  84\\
 82&  58&  43&  38
\end{bmatrix},
& 
\quad \mU^\perp \mM 
=
\begin{bmatrix}
 81&  44&  13&  52\\
 29&  29&  35&  75
\end{bmatrix},
 \\[2mm]
\mU^\perp \mM^2 
=
\begin{bmatrix}
35&  82&  54&  96\\
92&  99&  82&  10
\end{bmatrix},
 &\quad \mU^\perp \mM^3 
=
\begin{bmatrix}
10&  64&  97&   1\\
58&  87&  74&  99
\end{bmatrix}, \nonumber
\end{align}
from which we get, by right-multiplication by $\mV$,
$$ \seq_{\mU,\mV} =
\begin{bmatrix}
92& 75\\  
83& 51
\end{bmatrix},
\begin{bmatrix}
57& 82\\  
23& 16
\end{bmatrix},
\begin{bmatrix}
54& 93\\  
70& 66
\end{bmatrix},
\begin{bmatrix}
50& 77\\
26& 76
\end{bmatrix},\dots
$$
and
$$ \mat{P}_{\mU,\mV} =
\begin{bmatrix}
T^2 + 76T + 8&       87T + 31\\
    100T + 46& T^2 + 87T + 44
\end{bmatrix}.
$$
The largest invariant factor of $\mat{P}_{\mU,\mV}$ is 
$$P = T^4 + 62T^3 + 85T^2 + 69T + 37,$$
and $\sqfree=P$
since $P$ is squarefree. 
Next, we compute the row vector $\row{a}_1$ and the numerator $\mat{\Omega}( (\mU^\perp \mM^s\col{\varepsilon}_1)_{s \ge 0}, \mat{P}_{\mU,\mV})$:
\begin{align*}
\row{a}_1 &= [T + 55, T^2 + 76T+ 8]\\
\mat{\Omega}( (\mU^\perp \mM^s\col{\varepsilon}_1)_{s \ge 0}, \mat{P}_{\mU,\mV}) 
&=\left [\begin{matrix} 7T + 71 \\ 82 + 86   \end{matrix}\right ].
\end{align*}
The former is obtained by solving $\row{a}_1 = [~P~0~] (\mat{P}_{\mU,\mV})^{-1};$
the latter is the product
$$\mat{P}_{\mU,\mV} 
\left (
\begin{bmatrix}
7\\  
82
\end{bmatrix}\cdot \frac 1T+
\begin{bmatrix}
81\\  
29
\end{bmatrix}\cdot \frac 1{T^2}+
 \cdots \right),$$ where the columns are the first columns of the
 matrices in~\eqref{exampleproducts} (we only need $d=2$
 terms in the right-hand side).  From this, we find the scalar
 numerator $C_1 = \Omega((\row{u}_1 \mM^s\col{\varepsilon}_1)_{s \ge
   0}, \minpoly)$ by means of the dot product $[~C_1~] = \row{a}_1\cdot
 \mat{\Omega}( (\mU^\perp \mM^s\col{\varepsilon}_1)_{s \ge 0},
 \mat{P}_{\mU,\mV})$; we get
$$ C_1 = 82T^3 + 63T^2 + 73T + 48.$$
To find $C_{X_1}$, we proceed similarly: we compute  $\mat{\Omega}( (\row{u}_1 \mM^s \mM_1\col{\varepsilon}_1)_{s \ge 0}, \mat{P}_{\mU,\mV})$,
and we obtain
\begin{align*}
\mat{\Omega}( (\row{u}_1 \mM^s \mM_1\col{\varepsilon}_1)_{s \ge 0}, \mat{P}_{\mU,\mV}) &=
\left [\begin{matrix} 80T + 37 \\ 43T+60
  \end{matrix} \right  ]\\
C_{X_1} &= 43T^3 + 75T^2  + 49T+91.
\end{align*}
Thus, the polynomial $V_1 = C_{X_1}/C_1 \bmod \sqfree$ is 
given by $V_1= 61T^3 + 75T^2 + 85T + 23$. 
We compute $V_2= 32T^3 + 41T^2 + 94T + 22$ in the same way,
and our output is 
$$((T^4 + 62T^3 + 85T^2 + 69T + 37, 61T^3 + 75T^2 + 85T + 23, 32T^3 +
41T^2 + 94T + 22), 2X_1 + 53 X_2).$$ As a sanity check, we see that
$V(I)$ has one point in $\F_{101}^2$, namely $(54,79)$;
accordingly, $P$ has one root
in $\F_{101}$, $53$, and we have $V_1(53) = 54$ and $V_2(53) = 79$, as
expected.

%%%%%%%%%%%%%%%%%%%%%%%%%%%%%%%%%%%%%%%%%%%%%%%%%%%%%%%%%%%%

\subsection{Experimental results}\label{section:ex}

In the following table, we give the timings in seconds for different
values of $m$ for \cref{algo:block-sparse-fglm}.  Our
implemention was based on LinBox \cite{LinBox} dense and sparse
matrices as well Shoup's NTL \cite{NTL} polynomials. All timings are
measured on an Intel i7-4790 CPU with 32 GB RAM and 8 CPU's. For each
value of $m$, we create and run $m$ threads in parallel. Beside each
timing, the ratio of the time it took to compute $(\mU^{\perp}
\mM^s)_{s\ge0}$ to the total time is written in brackets.

In all cases, we start from multiplication matrices computed from a
{degrevlex} Gr\"obner basis in Magma. The timings reported here do not
include this precomputation. We refer the reader to~\cite{FaMo17} for
extensive experiments comparing the runtime two stages of the
algorithm (in many cases, computing the degree Gr\"obner basis is
faster). 

All our inputs as well as our source code are available at
\url{https://git.uwaterloo.ca/sghyun/Block-sparse-FGLM.git}. Some systems are well-known (Katsura or Eco
from~\cite{Morgan88}, while we also consider several families of
randomly generated inputs (some are inspired
from~\cite{FaMo17}). Systems rand1-$i$ have $3$ variables and randomly
generated equations (of degree depending on $i$), and rand2-$i$ are
similar, with $4$ variables. These systems are generically radical
(and in shape position). Systems mixed1-$i$ and mixed2-$i$ have
similar number of variables, but some of their solutions are multiple,
so the ideals are not radical (and not in shape position). Systems
mixed3-$i$ have only one multiple root (the others are simple), in
increasing numbers of variables; they are not in shape
position. Systems W1-$d$-$n$-$p$ are determinantal equations that
describe the computation of critical point for the projection $\pi_1:
(\alpha_1,\dots,\alpha_n) \mapsto \alpha_1$ on $V(f_1,\dots,f_p)$,
where $f_1,\dots,f_p$ are $p$ equations of degree $d$ in $n$
variables.

\begin{table}[H]
 \def\arraystretch{1.2}
\setlength\tabcolsep{6pt}
	\caption{Timings (in seconds) for polynomials over $\F_{65537}$}
	\begin{tabular}{c|c|c|c|c|c|c|c}
		\textbf{name}& $\bm{n}$ & $\bm{D}$ & \textbf{Sparsity} & $\bm{m = 1}$ & $\bm{m = 4}$ & $\bm{m = 8}$ & radical \\
		\hline
		rand1-1&3 &5832&0.0371&1780 (0.999)&473 (0.997)&329 (0.994)&yes\\
		rand1-2&3 &8000&0.0333&4128 (0.999)&1099 (0.998)&759 (0.995)&yes\\
		rand1-3&3 &10648&0.0302&8804 (0.999)&2336 (0.998)&1625 (0.996)&yes\\
		rand2-1&4 &4096&0.0756&1283 (0.999)&340 (0.997)&237 (0.995)&yes\\
		rand2-2&4 &6561&0.0667&4615 (0.999)&1236 (0.998)&851 (0.997)&yes\\
		rand2-3&4 &10000&0.0597&14753 (0.999)&3920 (0.999)&2710 (0.998)&yes\\
		
		mixed1-1&3 &2808&0.0521&275 (0.998)&73 (0.993)&51 (0.986)&no\\
		mixed1-2&3 &3591&0.0528&532 (0.998)&140 (0.995)&99 (0.99)&no\\
		mixed1-3&3 &4312&0.0482&864 (0.999)&230 (0.996)&161 (0.992)&no\\
		mixed2-1&4 &1552&0.139&104 (0.997)&28 (0.991)&19 (0.984)&no\\
		mixed2-2&4 &2657&0.105&450 (0.999)&123 (0.996)&84 (0.992)&no\\
		mixed2-3&4 &4352&0.0843&1718 (0.999)&456 (0.998)&316 (0.996)&no\\
		
		mixed3-1&10 &1035&0.233&66 (0.996)&18 (0.989)&12 (0.981)&no\\
		mixed3-2&11 &2060&0.215&488 (0.998)&129 (0.996)&90 (0.993)&no\\
		mixed3-3&12 &4109&0.212&3647 (0.999)&968 (0.998)&671 (0.997)&no\\
		
		eco12&12 &1024&0.131&62 (0.995)&16 (0.987)&12 (0.979)&yes\\
		katsura10&11 &1024&0.258&71 (0.996)&19 (0.989)&13 (0.982)&yes\\
		sot1&5 &8694&0.0002&765 (0.994)&208 (0.983)&143 (0.969)&yes\\
		vor2&6 &574&0.002&10 (0.99)&3 (0.973)&2 (0.955)&yes\\
		W1-2-10-4&10 &1344&0.312&164 (0.998)&43 (0.994)&30 (0.99)&yes\\
		W1-2-11-4&11 &1920&0.319&489 (0.999)&129 (0.997)&90 (0.994)&yes\\
		W1-3-7-3&7 &6480&0.174&10923 (0.999)&2900 (0.999)&2014 (0.999)&yes\\
		W1-4-6-2&6 &6480&0.141&9096 (0.999)&2424 (0.999)&1674 (0.998)&yes
	\end{tabular}
\end{table}
On average, the algorithm performs better by a factor of 3.73 for
$m=4$ and by a factor of 5.39 for $m=8$ compared to $m=1$. Increasing
$m$ has two effects: it (moderately) decreases the effectiveness of
parallel computations, but also increases the time to compute the
output polynomials. From the observed speedup and the ratio in the
table, we can conclude that the bottleneck of the algorithm is
the computation of that sequence and the latter effect is minor
compared to the former.

We can determine the effectiveness of parallel computations by
comparing combined CPU times (that is, total work).  For example, in
test rand1-1 we get $4124, 4350, 6008$ seconds for $m=1,4,8$
respectively; we observed similar patterns for other tests. If the
parallelization was perfect, the combined CPU time should be roughly
equal for all values of $m$ since computing $(\mU^{\perp}
\mM^s)_{s\ge0}$ took more than $99\%$ of the total time for all three
values of $m$ in this test. When $m=4$, the overhead from
parallelization is small whereas when $m=8$, overhead is significant
as it is impossible to run 8 threads in parallel on 8 cores without
switching.

%%%%%%%%%%%%%%%%%%%%%%%%%%%%%%%%%%%%%%%%%%%%%%%%%%%%%%%%%%%%
%%%%%%%%%%%%%%%%%%%%%%%%%%%%%%%%%%%%%%%%%%%%%%%%%%%%%%%%%%%%
%%%%%%%%%%%%%%%%%%%%%%%%%%%%%%%%%%%%%%%%%%%%%%%%%%%%%%%%%%%%

\section{Using the original coordinates}

In this section, we propose a refinment of the algorithm given
previously; the main new feature is that we avoid using a generic
linear form $\lf = t_1 X_1 + \cdots + t_n X_n$ as much as possible.
Indeed, such a linear combination is likely to result in a
multiplication matrix $\mM = t_1 \mM_1 + \cdots + t_n \mM_n$
significantly more dense than $\mM_1,\dots,\mM_n$. In all this
section, we will work under the assumption that $\mM_1$ the sparsest
matrix among $\mM_1,\dots,\mM_n$.

All notation, such as $I$, $\residueI$, \dots, are as in the two previous
sections.  In particular, we write
$V=V(I)=\{\balpha_1,\dots,\balpha_\dg\},$ with all $\balpha_i$'s in
$\Kbar^n$, and $\balpha_i=(\alpha_{i,1},\dots,\alpha_{i,n})$ for all
$i$.

%%%%%%%%%%%%%%%%%%%%%%%%%%%%%%%%%%%%%%%%%%%%%%%%%%%%%%%%%%%%

\subsection{Overview}

The algorithm will decompose $V$ into two parts: for the first part,
we will be able to use $X_1$ as a linear form in our zero-dimensional
parametrization; the remaining points will be dealt with using a
random linear form $\lf$ as above. Throughout, we rely on the
following operations: evaluations of linear forms on successive powers
of a given element in $\residueI$, say $1,X_1,X_2^2,\dots$ or
$1,\lf,\lf^2,\dots$ and (polynomial) linear algebra, as before, as well as
some operations on univariate polynomials related to so-called {\em
  power projection}~\cite{Shoup94,Shoup99}.

The main algorithm is as follows. For the moment, we only describe its
main structure; the details of the subroutines are given in the next
paragraphs. \todo{main algo takes $\ell$, not $\mU$}

\begin{algorithm}[H]
	\caption{$\mathsf{ParametrizationWithSplitting}(\ell,\lf)$} {\bf
		Input:} \vspace{-0.5em}
	\begin{itemize}\setlength\itemsep{0em}
		\item a linear form $\ell$ over $\residueI$
		\item $\lf=t_1 X_1 + \cdots + t_n X_n$
	\end{itemize}
	{\bf Output:} \vspace{-0.5em}
	\begin{itemize}\setlength\itemsep{0em}
		\item polynomials $((\sqfree,V_1,\dots,V_n),\lf)$
	\end{itemize}
	\begin{enumerate}
		\item let $(F,G_1,\dots,G_n,X_1)=\mathsf{ParametrizationX}_1(\ell)$
		\item let $\ell'=\mathsf{Update}(\ell,F,G_1,\dots,G_n,\lf)$
		\item let $(R,W_1,\dots,W_n,\lf)=\mathsf{Parametrization}(\ell',\lf)$
		\item let $((\sqfree,V_1,\dots,V_n),\lf)=\mathsf{Union}(((F,G_1,\dots,G_n),X_1), ((R,W_1,\dots,W_n),\lf))$
		\item \textbf{return} $((\sqfree,V_1,\dots,V_n),\lf)$
	\end{enumerate}
\end{algorithm}

The call to $\mathsf{ParametrizationX}_1(\ell)$ computes a
zero-dimensional parametrization of a subset $V'$ of $V$, with such
that $X_1$ separates the points of $V'$ (that is, takes pairwise
distinct values on the points of $V'$); this is done by using values
of the form $(\ell(X_1^s))_{s \ge 0}$. We then modify $\ell$ (which in
effect removes from $V$ the points we just found) and apply Algorithm
$\mathsf{Parametrization}(\ell',\lf)$ from \cref{ssec:abstractlago},
to obtain a zero-dimensional parametrization of $V''=V-V'$ using
values of the form $(\ell'(\lf^s))_{s \ge 0}$; in those ``lucky''
cases where $V'$ is a large subset of $V$, $V''$ may only contain a
few points, and few values for the latter sequence will be needed.

The last step involves changing coordinates in
$((F,G_1,\dots,G_n),X_1)$ to use $\lf$ as a linear form instead, and
performing the union of the two components $V'$ and $V''$.

%%%%%%%%%%%%%%%%%%%%%%%%%%%%%%%%%%%%%%%%%%%%%%%%%%%%%%%%%%%%

\subsection{Describing a subset of $V$}

In this paragraph, we specialize the discussion of
\cref{ssec:genseries} to the case $\lf=X_1$; we take
$\mathfrak{S}=\{1,\dots,\dg\}$, that is, $V_{\mathfrak{S}}=V$, and we
let $r_1,\dots,r_c$ be the pairwise distinct values taken by $X_1$ on
$V$, for some $c \le \dg$.  For $j=1,\dots,c$, we write $T_j$ for the
set of all indices $i$ in $\{1,\dots,\dg\}$ such that
$\alpha_{i,1}=r_j$; the sets $T_1,\dots,T_c$ form a partition of
$\{1,\dots,\dg\}$. When $T_j$ has cardinality~$1$, we denote it as
$T_j=\{\sigma_j\}$, for some index $\sigma_j$ in $\{1,\dots,\dg\}$, so
that $\alpha_{\sigma_j,1}=r_j$.

For $i=1,\dots,\dg$, let us write $\nu_i$ for the degree of the minimal
polynomial of $X_1$ in $\residueI_i$; thus, this polynomial is
$(T-\alpha_{i,1})^{\nu_i}$. For $j$ in $\{1,\dots,c\}$, we define
$\mu_j$ as the maximum of all $\nu_i$, for $i$ in~$T_j$. As a result, the minimal
polynomial of $X_1$ in $\prod_{j \in T_j} \residueI_j$ is 
$(T-r_j)^{\mu_j}$, and the minimal polynomial of $X_1$ in $\residueI$ is
$M=\prod_{j \in \{1,\dots,c\}} (T-r_j)^{\mu_j}$.

Recall that a linear form $\ell: \residueI \to \Kbar$ can be written uniquely
as $\ell=\sum_{i\in \{1,\dots,\dg\}} \ell_i$, with $\ell_i:\residueI_i \to
\Kbar$; collecting terms, $\ell$ may also be written as $\ell=\sum_{j
	\in \{1,\dots,c\}} \lambda_j$, with $\lambda_j=\sum_{i \in T_j}
\ell_i$.  Given such an $\ell$, we first explain how to compute values
of the form $\lambda_j(1)$. We will do this for some values of $j$
only, namely those $j$ for which $\mu_j=1$. This result is close in 
spirit to \cref{lemma:anyv}, but does not assume that the projection
$X_1: V \to \Kbar$ is one-to-one (that lemma makes such an 
assumption for a more general linear mapping  $\lf:V\to \Kbar$).

\begin{lemma}\label{lemma:valuelambda}
  Let $\ell$ be in ${\rm hom}_\K(\residueI,\K)$ and let $M$ be the minimal
  polynomial of $X_1$ in $\residueI$. Then, the polynomial
  $\Omega((\ell(X_1^s))_{s\ge0},M)$ is well-defined and satisfies
  $$\Omega((\ell(X_1^s))_{s\ge0},M)(r_j) = \lambda_j(1) M'(r_j) \quad \text{for all $j$ in $\{1,\dots,c\}$ such that $\mu_j=1$.}$$
\end{lemma}
\begin{proof}
	Let $\mathfrak{e}$ be the set of all indices $j$ in $\{1,\dots,c\}$
	such that $\mu_j=1$, and let $\mathfrak{f}=\{1,\dots,c\}-\mathfrak{e}$;
	this definition allows us to split the generating series
        of sequence $(\ell(X_1^s))_{s\ge 0}$ as
	\begin{align*}
	\sum_{s \ge 0} \frac{\ell(X_1^s)}{T^{s+1}  }
	&= \sum_{j \in \{1,\dots,c\}}\sum_{i\in T_j} 
	\sum_{s \ge 0} \frac{\ell_i(X_1^s)}{T^{s+1}}  \\
	&=\sum_{j \in \mathfrak{e}}\sum_{i\in T_j}\sum_{s \ge 0}  \frac{\ell_i(X_1^s)}{T^{s+1}} +
	\sum_{j \in \mathfrak{f}}\sum_{i\in T_j}\sum_{s \ge 0}  \frac{\ell_i(X_1^s)}{T^{s+1}}.
	\end{align*}
	Using \cref{lemma:formula} with $\lf=X_1$ and $v=1$, any sum $\sum_{s \ge 0} \ell_i(X_1^s)/T^{s+1}$ 
	in the second summand
	can be rewritten as 
	$$\frac{E_i}{(T-r_j)^{v_i}},$$
	for some integer $v_i$, and for some polynomial $E_i \in \Kbar[T]$ of degree less than
	$v_j$. Next, take $j$ in $\mathfrak{e}$. Since $\mu_j=1$, $\nu_i=1$ for all $i$ in $T_j$,
	so that
	each such $\ell_i$ takes the form 
	$$\ell_i: f \mapsto (\Lambda_{i}(f))(\balpha_i),$$ where $\Lambda_{i}$
	is a differential operator that does not involve $\partial/\partial
	X_1$. Since all terms of positive order in $\Lambda_i$ involve one of
	$\partial/\partial X_2,\dots,\partial/\partial X_n$, they cancel
	$X_1^s$ for $s\ge 0$. Thus, $\ell_i(X_1^s)$ can be rewritten 
	as $\ell_{i,1} \alpha_{i,1}^s$, for some constant $\ell_{i,1}$,
	and the generating series of these terms is 
	$$\frac {\ell_{i,1}}{T-\alpha_{i,1}}=\frac {\ell_{i,1}}{T-r_j}.$$
	Remarking  that we can write $\ell_{i,1}=\ell_i(1)$,
	altogether, the sum in question can be written
	\begin{align*}
	\sum_{s \ge 0} \frac{\ell(X_1^s)}{T^{s+1}  }
	&=\sum_{j \in \mathfrak{e}} 
	\frac{ \sum_{i\in T_j}  \ell_{i}(1) }{T-r_j }
	+ \sum_{j \in \mathfrak{f}} \frac{D_j}{(T-r_j )^{x_j}}\\
	&= \sum_{j \in \mathfrak{e}} 
	\frac{ \lambda_j(1) }{T-r_j }
	+ \sum_{j \in \mathfrak{f}} \frac{D_j}{(T-r_j )^{x_j}}
	\end{align*}
	for some integers $\{x_j \mid j \in \mathfrak{f}\}$ and
        polynomials $\{D_j \mid j \in \mathfrak{f}\}$ such that
        $\deg(D_j) < x_j$ holds, and with $D_j$ and $T-r_j $
        coprime. In particular, the minimal polynomial of  $(\ell(X_1^s))_{s\ge 0}$ is $N=\prod_{j\in
          \mathfrak{e}}(T-r_j) \prod_{j \in  \mathfrak{f}}(T-r_j)^{x_j}$.
	
	%% On the other hand, the minimal polynomial $M$ of $X_1$ can be
        %% rewritten as $M=\prod_{j\in \mathfrak{e}}(T-r_j) \prod_{j \in
        %% \mathfrak{f}}(T-r_j)^{\mu_j}$.  

        The minimal polynomial of the sequence $(\ell(X_1^s))_{s \ge
          0}$ divides $M$, so that $x_j \le \mu_j$ holds for all $j$
        in $\mathfrak{f}$.  As a result,
        $\Omega((\ell(X_1^s))_{s\ge0},M)$ is well-defined and is given by
	\begin{align*}
	\Omega((\ell(X_1^s))_{s\ge0},M)=&
	\sum_{j \in \mathfrak{e}}
	\Big(
	\lambda_j(1) \prod_{\iota \in \mathfrak{e}-\{j\}}(T-r_\iota)\Big)
	\Big(\prod_{j \in \mathfrak{f}}(T-r_j)^{\mu_j} \Big)\\
	&+
	\sum_{j\in \mathfrak{f}}
	\Big(  (T-r_j)^{\mu_j-x_j} D_j
	\prod_{\iota \in \mathfrak{f}-\{j\}}(T-r_j)^{\mu_\iota}\Big)
	\Big(\prod_{j\in \mathfrak{e}} (T-r_j) \Big).
	\end{align*}
	This implies that $$\Omega((\ell(X_1^s))_{s\ge0},M)(r_k) =\lambda_k(1) 
	\prod_{\iota \in \mathfrak{e}-\{k\}}(r_k-r_\iota)
	\prod_{j \in \mathfrak{f}}(r_k-r_j)^{\mu_j} = \lambda_k(1) M'(r_k)$$ 
	holds for all $k$ in $\mathfrak{e}$.
\end{proof}

We now show how to use this result to use sequences of the form
$(\ell(X_1^s))_{s \ge 0}$ to compute a zero-dimensional
parametrization of a subset $V'$ of $V$. Precisely, we characterize
the set $V'$ as follows: for $i$ in $\{1,\dots,\dg\}$, $\balpha_i$ is in
$V'$ if and only if:
\begin{itemize}
	\item for $i'$ in $\{1,\dots,\dg\}$, with $i'\ne i$, $\alpha_{i',1} \ne
	\alpha_{i,1}$;
	\item $\residueI_i$ is a reduced algebra, or equivalently, $I_i$ is radical (see \cref{ssec:dual} 
          for the notation used here).
\end{itemize}
We denote by $\mathfrak{A}\subset \{1,\dots,\dg\}$ the set of
corresponding indices $i$, and we let
$\mathfrak{B}=\{1,\dots,\dg\}-\mathfrak{A}$, so that we have
$V'=V_{\mathfrak{A}}$ and $V''=V_{\mathfrak{B}}$.  Remark that $X_1$
separates the points of $V'$.

Correspondingly, we define $\mathfrak{a}$ as the set of all indices
$j$ in $\{1,\dots,c\}$ such that $\sigma_j$ is in $\mathfrak{A}$. In
other words, $j$ is in $\mathfrak{a}$ if and only if $T_j$ has
cardinality $1$, so that $T_j=\{\sigma_j\}$, and $\residueI_{\sigma_j}$ is reduced.  The algorithm in this
paragraph will compute a zero-dimensional parametrization of
$V_{\mathfrak{A}}$; we use the following lemma to perform this
decomposition of $V$.

\begin{lemma}\label{lemma:acb2}
  Let $j$ be in $\{1,\dots,c\}$ such that $\mu_j=1$, let $\lambda$ be
  a linear form over $\prod_{i \in T_j} \residueI_i$ and let $\mf=t_2
  X_2 + \cdots + t_n X_n$, for some $t_2,\dots,t_n$. Define constants
  $a,b,c$ in $\Kbar$ by
  $$a=\lambda(1),\quad b=\lambda(\mf),\quad c=\lambda(\mf^2).$$
  Then, $j$ is in $\mathfrak{a}$
  if and only if, for a generic choice of $\lambda$ and $\mf$, $ac=b^2$.
\end{lemma}
\begin{proof}
  The assumption that $\mu_j=1$ means that for all $i$ in $T_j$,
  $\nu_i=1$. The linear 
  form $\lambda$ can be uniquely written as a sum $\lambda=\sum_{i \in T_j}
  \ell_i$, where each $\ell_i$ is in ${\rm hom}_\Kbar(\residueI_i,\Kbar)$.
  The fact that all $\nu_i$ are equal to $1$ then implies that each $\ell_i$ takes the form 
  $$\ell_i: f \mapsto (\Lambda_{i}(f))(\balpha_i),$$
  where $\Lambda_{i}$ is a differential operator that does not 
  involve $\partial/\partial X_1$. Thus, as in \cref{ell_param}, we can write a general
  $\Lambda_i$ of this form as
  $$\Lambda_i: f \mapsto u_{i,1} f + \sum_{2 \le r \le n}
  P_{i,r}(u_{i,2},\dots,u_{i,D_i}) \frac{\partial}{\partial X_j} f +
  \sum_{2 \le r \le s \le n} P_{i,r,s}(u_{i,2},\dots,u_{i,D_i})
  \frac{\partial^2}{\partial X_j\partial X_k} f +
  \tilde\Lambda_i(f),$$ where all terms in $\tilde \Lambda_i$ have
  order at least $3$, $\bu_i=(u_{i,1},\dots,u_{i,D_i})$ are parameters and
  $(P_{i,r})_{2 \le r \le n}$ and $(P_{i,r,s})_{2 \le r \le s \le n}$
  are linear forms in $u_{i,2},\dots,u_{i,D_i}$.
  We obtain
  \begin{align*}
    \Lambda_i(1)   &= u_{i,1} \\
    \Lambda_i(\mf)   &= u_{i,1} \mf +\sum_{2 \le r \le n}P_{i,r}(u_{i,2},\dots,u_{i,D_i})t_r \\
    \Lambda_i(\mf^2) &= u_{i,1} \mf^2  +2 \mf \sum_{2 \le r \le n}P_{i,r}(u_{i,2},\dots,u_{i,D_i})t_r + 
    2\sum_{2 \le r \le s \le n} P_{i,r,s}(u_{i,2},\dots,u_{i,D_i})t_rt_s,
  \end{align*}
  which gives
  \begin{align*}
    a  &= \sum_{i\in T_j}u_{i,1} \\
    b  &= \sum_{i\in T_j}u_{i,1} \mf(\balpha_i) +\sum_{i \in T_j, 2 \le r \le n}P_{i,r}(u_{i,2},\dots,u_{i,D_i})t_r \\
    c &= \sum_{i\in T_j}u_{i,1} \mf(\balpha_i)^2  +2  \sum_{i \in T_j, 2 \le r \le n}\mf(\balpha_i) P_{i,r}(u_{i,2},\dots,u_{i,D_i})t_r    
    +2 \sum_{i \in T_j, 2 \le r \le s \le n} P_{i,r,s}(u_{i,2},\dots,u_{i,D_i})t_rt_s.
  \end{align*}
  Suppose first that $j$ is in $\mathfrak{a}$. Then, $T_j=\{\sigma_j\}$, so we 
  have only one term $\Lambda_{\sigma_j}$ to consider, and $\residueI_{\sigma_j}$ 
  is reduced, so that all coefficients $P_{\sigma_j,r}$ and
  $P_{\sigma_j,r,s}$ vanish. Thus, we are left in
  this case with
  $$
  a = u_{\sigma_j,1}, \quad
  b = u_{\sigma_j,1} \mf(\balpha_{\sigma_j}), \quad
  c = u_{\sigma_j,1} \mf(\balpha_{\sigma_j})^2,
  $$ so that we have $ac=b^2$, for {\em any} choice of $\lambda$ and
  $\mf$. Now, we suppose that $j$ is not in $\mathfrak{a}$, and we prove
  that for a generic choice of $\lambda$ and $\mf$, $ac-b^2$ is non-zero.
  The quantity $ac-b^2$ is a polynomial in the parameters
  $(\bu_i)_{i\in T_j}$, and $(t_i)_{i \in \{2,\dots,n\}}$, and we have
  to show that it is not identically zero. We discuss two cases; in both
  of them, we prove that a suitable specialization of $ac-b^2$ is
  non-zero.
  
  Suppose first that for at least one index $\sigma$ in $T_j$,
  $\residueI_\sigma$ is not reduced. In this case, there exists as least one
  index $\rho$ in $\{2,\dots,n\}$ such that
  $P_{\sigma,\rho}(u_{\sigma,2},\dots,u_{\sigma,D_\sigma})$ is not
  identically zero. Let us set all $\bu_{\sigma'}$
  to $0$, for $\sigma'$ in $T_j-\{\sigma\}$, as well as $u_{\sigma,1}$,
  and all $t_r$ for $r\ne \rho$. Then, under this specialization,
  $ac-b^2$ becomes
  $-(P_{\sigma,\rho}(u_{\sigma,2},\dots,u_{\sigma,D_\sigma})t_\rho)^2$,
  which is non-zero, so that $ac-b^2$ itself must be non-zero.
  
  Else, since $j$ is not in $\mathfrak{a}$, we can assume that $T_j$
  has cardinality at least $2$, with $\residueI_\sigma$ reduced for all $\sigma$
  in $T_j$ (so that $P_{\sigma,r}$ and $P_{\sigma,r,s}$ vanish for 
  all such $\sigma$ and all $r,s$). Suppose that $\sigma$ and $\sigma'$ are two indices in
  $T_j$; we set all indices $u_{\sigma'',1}$ to zero, for $\sigma''$
  in $T_j-\{\sigma,\sigma'\}$. We are left with
  $$
  a=u_{\sigma,1}+u_{\sigma',1},\quad
  b=u_{\sigma,1}\mf(\balpha_{\sigma})+u_{\sigma',1}\mf(\balpha_{\sigma'}),\quad
  c=u_{\sigma,1}\mf(\balpha_{\sigma})^2+u_{\sigma',1}\mf(\balpha_{\sigma'})^2.
  $$
  Then, $ac-b^2$ is equal to $2u_{\sigma,1}u_{\sigma',1}(\mf(\balpha_{\sigma})-\mf(\balpha_{\sigma'}))^2$,
  which is non-zero, since $\balpha_\sigma \ne \balpha_{\sigma'}$.
\end{proof}

The previous lemmas allow us to write Algorithm
$\mathsf{ParametrizationX}_1$. After computing $M$, we determine its
factor $P=\prod_{j \in \{1,\dots,c\}, \mu_j=1} (T-r_j)$, which is
obtained by taking the squarefree part of $M$ and dividing it by
$\gcd(M,M')$. We split this polynomial further using the previous
results in order to find $\prod_{j \in \mathfrak{a}} (T-r_j)$, and we
conclude using the same kind of calculations as in Algorithm
$\mathsf{Parametrization}$ of \cref{ssec:abstractlago}.

\begin{algorithm}[H]
	\caption{$\mathsf{ParametrizationX}_1(\ell)$}
	~\\
	{\bf Input:} \vspace{-0.5em}
	\begin{itemize}\setlength\itemsep{0em}
		\item a linear form $\ell$ over $\residueI$
	\end{itemize}
	{\bf Output:} polynomials $((P,V_1,\dots,V_n),X_1)$
	\begin{enumerate}\setlength\itemsep{0em}
		\item let $M$ be the minimal polynomial of the sequence $(\ell(X_1^s))_{s \ge 0}$
		\item let $P$ be the squarefree part of $M$
                \item let $P = P /\gcd(M, M')$
		\item let $\mf$ be a random linear form in $X_2,\dots,X_n$
		\item \textbf{for} $i=0,1,2$ \textbf{do}
		\begin{enumerate}
			\item let $A_i = \Omega((\ell(\mf^i X_1^s))_{s\ge0},M)$
		\end{enumerate}
		\item\label{step:updateP} let $P = \gcd(P, A_0  A_2- A_1^2)$
		\item \textbf{for} $i=2,\dots,n$ \textbf{do}
		\begin{enumerate}
			\item let $A_{X_i} = \Omega((\ell(X_i X_1^s))_{s\ge0},M)$
		\end{enumerate}
		\item \textbf{return} $((P,T,A_{X_2}/ A_1 \bmod P, \dots,A_{X_n}/A_{1} \bmod P),X_1)$
	\end{enumerate}
	\label{algo:para}
\end{algorithm}

\begin{lemma}
	Suppose that $\ell$ is a generic element of ${\rm
		hom}_{\Kbar}(\residueI,\Kbar)$ and that $t$ is a generic linear form. Then
	the output $((P,V_1,\dots,V_n),X_1)$ of
	$\mathsf{ParametrizationX}_1(\ell,t)$ is a zero-dimensional
	parametrization of $V_{\mathfrak{A}}$.
\end{lemma}
\begin{proof}
	\cref{lemma:minpoly} shows that for a generic choice of $\ell$,
	$M$ is the minimal polynomial of $X_1$, so that we indeed have
	$P=\prod_{j \in \{1,\dots,c\}, \mu_j=1} (T-r_j)$. Let then $r_j$ be
	one of these roots; by \cref{lemma:valuelambda}, for $i=0,1,2$
	we have $ A_i(r_j) = M'(r_j) (t^i \cdot \lambda_j)(1)$, where
	$\lambda_j =\sum_{i \in T_j} \ell_i$, and the $\ell_i$'s are the
	components of $\ell$. 
	
	As a result, the value of $ A_0  A_2 -  A_1^2$ at
	$r_j$ is (up to the non-zero factor $M'(r_j)^2$) equal to the
	quantity $ac-b^2$ defined in \cref{lemma:acb2}, so for a
	generic choice of $\ell$ and $t$, it vanishes if and only if $j$ is
	in $\mathfrak{a}$. Thus, after Step~\ref{step:updateP}, 
	$P$ is equal to $\prod_{j \in \mathfrak{a}} (T-r_j)$.
	
	The last step is to compute the zero-dimensional parametrization of
	$V_{\mathfrak{A}}$. This is done using again
	\cref{lemma:valuelambda}. Indeed, for $j$ in $\mathfrak{a}$, 
	$T_j$ is simply equal to $\{\sigma_j\}$, so that we have, for $i=2,\dots,n$,
	$$ A_1(r_j)=M'(r_j) \lambda_j(1) \quad\text{and}\quad 
	A_{X_i}(r_j) = M'(r_j) (X_i \cdot \lambda_j)(1) = M'(r_j) \lambda_j(X_i).$$ Now, since $j$
	is in $\mathfrak{a}$, $\residueI_{\sigma_j}$ is reduced, so that there
	exists a constant $\lambda_{j,1}$ such that for all $f$ in
	$\Kbar[X_1,\dots,X_n]$, $\lambda_j(f)$ takes the form $\lambda_{j,1}
	f(\balpha_{\sigma_j})$. This shows that, as claimed,
	$$\frac{ A_{X_j}(r_j)}{ A_1 (r_j)} = 
	\frac {\Omega((\ell(X_j X_1^s)_{s\ge0},M)(r_j)}{\Omega((\ell(t X_1^s))_{s\ge0},M)(r_j)}=
	\frac
	{M'(r_j) \lambda_{j,1} \alpha_{j,i}}{M'(r_j) \lambda_{j,1}} = \alpha_{j,i}.$$
	For $i=1$, since we use $X_1$ as a separating variable for $V_{\mathfrak{A}}$, 
	we simply add the polynomial $T$ to our list.
\end{proof}

\begin{center}
\begin{tabular}{c|c|c|c|c|c|c}
\textbf{name}& $\bm{n}$ & $\bm{D}$ & $\bm{m = 1}$ & $\bm{m = 4}$ & $\bm{m = 8}$&\\

rand1-1&3 &5832&0.435&0.439&0.455&5832/5832\\
rand1-20&3 &8000&0.437&0.439&0.448&8000/8000\\
rand1-3&3 &10648&0.437&0.444&0.444&10648/10648\\
rand2-1&4 &4096&0.427&0.429&0.445&4096/4096\\
rand2-2&4 &6561&0.429&0.425&0.438&6561/6561\\
rand2-3&4 &10000&0.426&0.426&0.431&10000/10000\\
mixed1-1&3 &2808&0.467&0.576&0.545& 2744/2752\\
mixed1-2&3 &3591&0.545&0.573&0.601& 3375/3402\\
mixed1-3&3 &4312&0.521&0.541&0.597& 4096/4123\\
mixed2-1&4 &1552&0.693&0.743&0.807& 1296/1312\\
mixed2-2&4 &2657&0.557&0.573&0.607& 2041/2417\\
mixed2-3&4 &4352&0.485&0.493&0.51& 4096/4112\\
mixed3-1&10 &1035&0.43&0.449&0.532& 1024/1025\\
mixed3-2&11 &2060&0.413&0.42&0.445& 2048/2049\\
mixed3-3&12 &4109&0.428&0.434&0.441& 4096/4097\\
eco12&12 &1024&0.223&0.245&0.307&1024/1024\\
katsura10&11 &1024&0.411&0.427&0.483&1024/1024\\
sot1&5 &8694&1.07&1.09&1.19 & 1012/7682\\
vor2&6 &574&0.49&0.533&0.675&574/574\\
W1-2-10-4&10 &1344&0.487&0.496&0.527&1344/1344\\
W1-2-11-4&11 &1920&0.486&0.491&0.505&1920/1920\\
W1-3-7-3&7 &6480&0.455&0.456&0.468&6480/6480\\
W1-4-6-2&6 &6480&0.445&0.448&0.451&6480/6480\\
\end{tabular}
\end{center}

\newpage
\bibliographystyle{plain}
\bibliography{thesis}
\end{document}
